\chapterdrop

\begin{center}
\headersanskrit{अथ द्वितीयोऽध्यायः}

\headerspace
\headertransliteration{Atha Dvitīyo'dhyāyaḥ}

\section{Chapter 2}

\headerspace
\headersanskrit{सांख्ययोगः}

\headerspace
\headertransliteration{Sāṅkhya Yogah}

\headerspace
\headertranslation{The Yoga of Knowledge}

\headerspace
\end{center}

\begin{gitaverse}
सञ्जय उवाच \\
तं तथा कृपयाविष्टमश्रुपूर्णाकुलेक्षणम् । \\
विषीदन्तमिदं वाक्यमुवाच मधुसूदनः ॥१॥
\end{gitaverse}

\begin{transliteration}
sañjaya uvāca \\
taṁ tathā kṛpayāviṣṭamaśrupūrṇākulekṣaṇam, \\
viṣīdantamidaṁ vākyamuvāca madhusūdanaḥ.
\end{transliteration}

Sanjaya said: \\
1. To him who was thus overcome with pity and despondency, with eyes full of
tears and agitated, Madhusudana spoke these words.

\begin{gitaverse}
श्रीभगवानुवाच \\
कुतस्त्वा कश्मलमिदं विषमे समुपस्थितम् । \\
अनार्यजुष्टमस्वर्ग्यमकीर्तिकरमर्जुन ॥२॥
\end{gitaverse}

\begin{transliteration}
śrībhagavānuvāca \\
kutastvā kaśmalamidaṁ viṣame samupasthitam, \\
anāryajuṣṭamasvargyamakīrtikaramarjuna.
\end{transliteration}

The Blessed Lord said: \\
2. Whence is this perilous condition come upon thee, this dejection,
un-Aryan-like, heaven-excluding, disgraceful, O Arjuna?

\begin{gitaverse}
क्लैब्यं मा स्म गमः पार्थ नैतत्त्वय्युपपद्यते । \\
क्षुद्रं हृदयदौर्बल्यं त्यक्त्वोत्तिष्ठ परन्तप ॥३॥
\end{gitaverse}

\begin{transliteration}
klaibyaṁ mā sma gamaḥ pārtha naitattvayyupapadyate, \\
kṣudraṁ hṛdayadaurbalyaṁ tyaktvottiṣṭha parantapa.
\end{transliteration}

3. Yield not to impotence, O Partha! It does not befit thee. Cast off this mean
weakness of heart! Stand up, O Parantapa (O scorcher of foes)!

\begin{gitaverse}
अर्जुन उवाच \\
कथं भीष्ममहं सङ्ख्ये द्रोणं च मधुसूदन । \\
इषुभिः प्रतियोत्स्यामि पूजार्हावरिसूदन ॥४॥
\end{gitaverse}

\begin{transliteration}
arjuna uvāca \\
kathaṁ bhīṣmamahaṁ saṅkhye droṇaṁ ca madhusūdana, \\
iṣubhiḥ pratiyotsyāmi pūjārhāvarisūdana.
\end{transliteration}

Arjuna said: \\
4. How, O Madhusudana, shall I in battle fight with arrows against Bhishma and
Drona, who are fit to be worshipped, O destroyer of enemies!

\begin{gitaverse}
गुरूनहत्वा हि महानुभावान् \\
\tab श्रेयो भोक्तुं भैक्ष्यमपीह लोके । \\
हत्वार्थकामांस्तु गुरूनिहैव \\
\tab भुञ्जीय भोगान् रुधिरप्रदिग्धान् ॥५॥
\end{gitaverse}

\begin{transliteration}
gurūnahatvā hi mahānubhāvān \\
\tab śreyo bhoktuṁ bhaikṣyamapīha loke, \\
hatvārthakāmāṁstu gurūnihaiva \\
\tab bhuñjīya bhogān rudhirapradigdhān.
\end{transliteration}

5. Better indeed in this world, is to eat even the bread of `beggary', than to
slay the most noble of teachers. But, if I kill them, even in this world all my
enjoyments of wealth and desires will be stained with blood.

\begin{gitaverse}
न चैतद्विद्मः कतरन्नो गरीयो- \\
\tab जयेम यदि वा नो जयेयुः । \\
यानेव हत्वा न जिजीविषाम- \\
\tab प्रमुखे धार्तराष्ट्राः ॥६॥
\end{gitaverse}

\begin{transliteration}
na caitadvidmaḥ kataranno garīyo- \\
\tab yadvā jayema yadi vā no jayeyuḥ, \\
yāneva hatvā na jijīviṣāma- \\
\tab ste'vasthitāḥ pramukhe dhārtarāṣṭrāḥ.
\end{transliteration}

6. I can scarcely say which will be better; that we should conquer them or that
they should conquer us. Even the sons of Dhritarashtra, after slaying whom we
do not wish to live, stand facing us.

\begin{gitaverse}
कार्पण्यदोषोपहतस्वभावः \\
\tab पृच्छामि त्वां धर्मसम्मूढचेताः । \\
यच्छ्रेयः स्यान्निश्चितं ब्रूहि तन्मे \\
\tab शिष्यस्तेऽहं शाधि मां त्वां प्रपन्नम् ॥७॥
\end{gitaverse}

\begin{transliteration}
kārpaṇyadoṣopahatasvabhāvaḥ \\
\tab pṛcchāmi tvāṁ dharmasammūḍhacetāḥ, \\
yacchreyaḥ syānniścitaṁ brūhi tanme \\
\tab śiṣyaste'haṁ śādhi māṁ tvāṁ prapannam.
\end{transliteration}

7. My heart is overpowered by the taint of pity; my mind is confused as to
duty. I ask Thee. Tell me decisively what is good for me. I am Thy disciple.
Instruct me, who has taken refuge in Thee.

\begin{gitaverse}
न हि प्रपश्यामि ममापनुद्याद् \\
\tab यच्छोकमुच्छोषणमिन्द्रियाणाम् । \\
अवाप्य भूमावसपत्नमृद्धं- \\
\tab राज्यं सुराणामपि चाधिपत्यम् ॥८॥
\end{gitaverse}

\begin{transliteration}
na hi prapaśyāmi mamāpanudyād \\
\tab yacchokamucchoṣaṇamindriyāṇām, \\
avāpya bhūmāvasapatnamṛddhaṁ- \\
\tab rājyaṁ surāṇāmapi cādhipatyam.
\end{transliteration}

8. I do not see that it would remove this sorrow that burns up my senses, even
if I should attain prosperous and unrivalled dominion on earth, or even
Lordship over the gods.

\begin{gitaverse}
सञ्जय उवाच \\
एवमुक्त्वा हृषीकेशं गुडाकेशः परन्तप । \\
न योत्स्य इति गोविन्दमुक्त्वा तूष्णीं बभूव ह ॥९॥
\end{gitaverse}

\begin{transliteration}
sañjaya uvāca \\
evamuktvā hṛṣīkeśaṁ gudākeśaḥ parantapa, \\
na yotsya iti govindamuktvā tūṣṇīṁ babhūva ha.
\end{transliteration}

Sanjaya said: \\
9. Having spoken thus to Hrishikesha, Gudakesha, the destroyer of foes, said to
Govinda: ``I will not fight''; and became silent.

\begin{gitaverse}
तमुवाच हृषीकेशः प्रहसन्निव भारत । \\
सेनयोरुभयोर्मध्ये विषीदन्तमिदं वचः ॥१०॥
\end{gitaverse}

\begin{transliteration}
tamuvāca hṛṣīkeśaḥ prahasanniva bhārata, \\
senayorubhayormadhye viṣīdantamidaṁ vacaḥ.
\end{transliteration}

10. To him who was despondent in the midst of the two armies, Hrishikesha as if
smiling, `O Bharata', spoke these words.

\begin{gitaverse}
श्रीभगवानुवाच \\
अशोच्यानन्वशोचस्त्वं प्रज्ञावादांश्च भाषसे । \\
गतासूनगतासूंश्च नानुशोचन्ति पण्डिताः ॥११॥
\end{gitaverse}

\begin{transliteration}
śrī bhagavānuvāca \\
aśocyānanvaśocastvaṁ prajñāvādāṁśca bhāṣase, \\
gatāsūnagatāsūṁśca nānuśocanti paṇḍitāḥ.
\end{transliteration}

The Blessed Lord said: \\
11. You have grieved for those that should not be grieved for; yet, you speak
words of wisdom. The wise grieve neither for the living nor for the dead.

\begin{gitaverse}
न त्वेवाहं जातु नासं न त्वं नेमे जनाधिपाः । \\
न चैव न भविष्यामः सर्वे वयमतः परम् ॥१२॥
\end{gitaverse}

\begin{transliteration}
na tvevāhaṁ jātu nāsaṁ na tvaṁ neme janādhipāḥ, \\
na caiva na bhaviṣyāmaḥ sarve vayamataḥ param.
\end{transliteration}

12. It is not that at any time (in the past), indeed was I not, nor were you,
nor these rulers of men. Nor verily, shall we all ever cease to be hereafter.

\begin{gitaverse}
देहिनोऽस्मिन्यथा देहे कौमारं यौवनं जरा । \\
तथा देहान्तरप्राप्तिर्धीरस्तत्र न मुह्यति ॥१३॥
\end{gitaverse}

\begin{transliteration}
dehino'sminyathā dehe kaumāraṁ yauvnaṁ jarā, \\
tathā dehāntaraprāptirdhīrastatra na muhyati.
\end{transliteration}

13. Just as in this body the embodied (soul) passes into childhood, youth and
old age, so also does he pass into another body; the firm man does not grieve
at it.

\begin{gitaverse}
मात्रास्पर्शास्तु कौन्तेय शीतोष्णसुखदुःखदाः । \\
आगमापायिनोऽनित्यास्तांस्तितिक्षस्व भारत ॥१४॥
\end{gitaverse}

\begin{transliteration}
mātrāsparśāstu kaunteya śītoṣṇasukhaduḥkhadāḥ, \\
āgamāpāyino'nityāstāṁstitikṣasva bhārata.
\end{transliteration}

14. The contacts of senses with objects, O son of Kunti, which cause heat and
cold, pleasure and pain, have a beginning and an end; they are impermanent;
endure them bravely, O descendant of Bharata.

\begin{gitaverse}
यं हि न व्यथयन्त्येते पुरुषं पुरुषर्षभ । \\
समदुःखसुखं धीरं सोऽमृतत्वाय कल्पते ॥१५॥
\end{gitaverse}

\begin{transliteration}
yaṁ hi na vyathayantyete puruṣaṁ puruṣarṣabha, \\
samaduḥkhasukhaṁ dhīraṁ so'mṛtatvāya kalpate.
\end{transliteration}

15. That firm man to whom, surely these afflict not, O chief among men, to whom
pleasure and pain are the same, is fit for realising the Immortality of the
Self.

\begin{gitaverse}
नासतो विद्यते भावो नाभावो विद्यते सतः । \\
उभयोरपि दृष्टोऽन्तस्त्वनयोस्तत्त्वदर्शिभिः ॥१६॥
\end{gitaverse}

\begin{transliteration}
nāsato vidyate bhāvo nābhāvo vidyate sataḥ, \\
ubhayorapi dṛṣṭo'ntastvanayostattvadarśibhiḥ.
\end{transliteration}

16. The unreal has no being; there is no non-being of the Real; the truth about
both these has been seen by the Knowers of the Truth (or the Seers of the
Essence).

\begin{gitaverse}
अविनाशि तु तद्विद्धि येन सर्वमिदं ततम् । \\
विनाशमव्ययस्यास्य न कश्चित्कर्तुमर्हति ॥१७॥
\end{gitaverse}

\begin{transliteration}
avināśi tu tadviddhi yena sarvamidaṁ tatam, \\
vināśamavyayasyāsya na kaścitkartumarhati.
\end{transliteration}

17. Know That to be Indestructible by which all this is pervaded. None can
cause the destruction of That---the Imperishable.

\begin{gitaverse}
अन्तवन्त इमे देहा नित्यस्योक्ताः शरीरिणः । \\
अनाशिनोऽप्रमेयस्य तस्माद्युध्यस्व भारत ॥१८॥
\end{gitaverse}

\begin{transliteration}
antavanta ime dehā nityasyoktāḥ śarīriṇaḥ, \\
anāśino'prameyasya tasmādyudhyasva bhārata.
\end{transliteration}

18. They have an end, it is said, these bodies of the embodied-Self. The Self
is Eternal, Indestructible, Incomprehensible. Therefore fight, O Bharata.

\begin{gitaverse}
य एनं वेत्ति हन्तारं यश्चैनं मन्यते हतम् । \\
उभौ तौ न विजानीतो नायं हन्ति न हन्यते ॥१९॥
\end{gitaverse}

\begin{transliteration}
ya enaṁ vetti hantāraṁ yaścainaṁ manyate hatam, \\
ubhau tau na vijānīto nāyaṁ hanti na hanyate.
\end{transliteration}

19. He who takes the Self to be the slayer and he who thinks He is slain;
neither of them knows. He slays not, nor is He slain.

\begin{gitaverse}
न जायते म्रियते वा कदाचि- \\
\tab न्नायं भूत्वा भविता वा न भूयः । \\
अजो नित्यः शाश्वतोऽयं पुराणो- \\
\tab न हन्यते हन्यमाने शरीरे ॥२०॥
\end{gitaverse}

\begin{transliteration}
na jāyate mriyate vā kadāci- \\
\tab nnāyaṁ bhūtvā bhavitā vā na bhūyaḥ, \\
ajo nityaḥ śāśvato'yam purāṇo- \\
\tab na hanyate hanyamāne śarīre.
\end{transliteration}

20. He is not born, nor does He ever die; after having been, He again ceases
not to be; Unborn, Eternal, Changeless and Ancient, He is not killed when the
body is killed.

\begin{gitaverse}
वेदाविनाशिनं नित्यं य एनमजमव्ययम् । \\
कथं स पुरुषः पार्थ कं घातयति हन्ति कम् ॥२१॥
\end{gitaverse}

\begin{transliteration}
vedāvināśinaṁ nityaṁ ya enamajamavyayam, \\
kathaṁ sa puruṣaḥ pārtha kaṁ ghātayati hanti kam.
\end{transliteration}

21. Whosoever knows Him to be Indestructible, Eternal, Unborn, and
Inexhaustible, how can that man slay O Partha, or cause others to be slain?

\begin{gitaverse}
वासांसि जीर्णानि यथा विहाय \\
\tab नवानि गृह्णाति नरोऽपराणि । \\
तथा शरीराणि विहाय जीर्णा- \\
\tab न्यन्यानि संयाति नवानि देही ॥२२॥
\end{gitaverse}

\begin{transliteration}
vāsāṁsi jīrṇāni yathā vihāya \\
\tab navāni gṛhṇāti naro'parāṇi, \\
tathā śarīrāṇi vihāya jīrṇā- \\
\tab nyanyāni saṁyāti navāni dehī.
\end{transliteration}

22. Just as a man casts off his worn out clothes and puts on new ones, so also
the embodied-Self casts off its worn out bodies and enters other which are new.

\begin{gitaverse}
नैनं छिन्दन्ति शस्त्राणि नैनं दहति पावकः । \\
न चैनं क्लेदयन्त्यापो न शोषयति मारुतः ॥२३॥
\end{gitaverse}

\begin{transliteration}
nainaṁ chindanti śastrāṇi nainaṁ dahati pāvakaḥ, \\
na cainaṁ kledayantyāpo na śoṣayati mārutaḥ.
\end{transliteration}

23. Weapons cleave It not, fire burns It not, water moistens It not, wind dries
It not.

\begin{gitaverse}
अच्छेद्योऽयमदाह्योऽयमक्लेद्योऽशोष्य एव च । \\
नित्यः सर्वगतः स्थाणुरचलोऽयं सनातनः ॥२४॥
\end{gitaverse}

\begin{transliteration}
acchedyo'yamadāhyo'yamakledyo'śoṣya eva ca, \\
nityaḥ sarvagataḥ sthāṇuracalo'yaṁ sanātanaḥ.
\end{transliteration}

24. This Self cannot be cut, nor burnt, nor moistened, nor dried up. It is
eternal, all-pervading, stable, immovable and ancient.

\begin{gitaverse}
अव्यक्तोऽयमचिन्त्योऽयमविकार्योऽयमुच्यते । \\
तस्मादेवं विदित्वैनं नानुशोचितुमर्हसि ॥२५॥
\end{gitaverse}

\begin{transliteration}
avyakto'yamacintyo'yamavikāryo'yamucyate, \\
tasmādevaṁ viditvainaṁ nānuśocitumarhasi.
\end{transliteration}

25. This (Self) is said to be Unmanifest, Unthinkable and Unchangeable.
Therefore, knowing This to be such, you should not grieve.

\begin{gitaverse}
अथ चैनं नित्यजातं नित्यं वा मन्यसे मृतम् । \\
तथापि त्वं महाबाहो नैवं शोचितुमर्हसि ॥२६॥
\end{gitaverse}

\begin{transliteration}
atha cainam nityajātaṁ nityaṁ vā manyase mṛtam, \\
tathāpi tvaṁ mahābāho naivaṁ śocitumarhasi.
\end{transliteration}

26. But even if you think of Him as being constantly born and constantly dying,
even then, O mighty-armed, you should not grieve.

\begin{gitaverse}
जातस्य हि ध्रुवो मृत्युर्ध्रुवं जन्म मृतस्य च । \\
तस्मादपरिहार्येऽर्थे न त्वं शोचितुमर्हसि ॥२७॥
\end{gitaverse}

\begin{transliteration}
jātasya hi dhruvo mṛtyurdhruvaṁ janma mṛtasya ca, \\
tasmādaparihārye'rthe na tvaṁ śocitumarhasi.
\end{transliteration}

27. Indeed, certain is death for the born, and certain is birth for the dead;
therefore, over the inevitable, you should not grieve.

\begin{gitaverse}
अव्यक्तादीनि भूतानि व्यक्तमध्यानि भारत । \\
अव्यक्तनिधनान्येव तत्र का परिदेवना ॥२८॥
\end{gitaverse}

\begin{transliteration}
avyaktādīni bhūtāni vyaktamadhyāni bhārata, \\
avyaktanidhanānyeva tatra kā paridevanā.
\end{transliteration}

28. Beings unmanifest in the beginning, and unmanifest again in their end seem
to be manifest in the middle, O Bharata. What then is there to grieve about?

\begin{gitaverse}
आश्चर्यवत्पश्यति कश्चिदेन- \\
\tab माश्चर्यवद्वदति तथैव चान्यः । \\
आश्चर्यवच्चैनमन्यः श्रृणोति \\
\tab श्रुत्वाप्येनं वेद न चैव कश्चित् ॥२९॥
\end{gitaverse}

\begin{transliteration}
āscaryavatpaśyati kaścidena- \\
\tab māścaryavadvadati tathaiva cānyaḥ, \\
āścaryavaccainamanyaḥ śṛṇoti \\
\tab śrutvāpyenaṁ veda na caiva kaścit.
\end{transliteration}

29. One sees This as a wonder; another speaks of This as a wonder; another
hears of This as a wonder; yet, having heard none understands This at all!

\begin{gitaverse}
देही नित्यमवध्योऽयं देहे सर्वस्य भारत । \\
तस्मात्सर्वाणि भूतानि न त्वं शोचितुमर्हसि ॥३०॥
\end{gitaverse}

\begin{transliteration}
dehī nityamavadhyo'yaṁ dehe sarvasya bhārata, \\
tasmātsarvāṇi bhūtāni na tvaṁ śocitumarhasi.
\end{transliteration}

30. This---the Indweller in the body of everyone is ever indestructible, O
Bharata; and, therefore, you should not grieve for any creature.

\begin{gitaverse}
स्वधर्ममपि चावेक्ष्य न विकम्पितुमर्हसि । \\
धर्म्याद्धि युद्धाच्छ्रेयोऽन्यत्क्षत्रियस्य न विद्यते ॥३१॥
\end{gitaverse}

\begin{transliteration}
svadharmamapi cāvekṣya na vikampitumarhasi, \\
dharmyāddhi yuddhācchreyo'nyatkṣatriyasya na vidyate.
\end{transliteration}

31. Further, looking at thine own duty thou ought not to waver, for there is
nothing higher for a KSHATRIYA than a righteous war.

\begin{gitaverse}
यदृच्छया चोपपन्नं स्वर्गद्वारमपावृतम् । \\
सुखिनः क्षत्रियाः पार्थ लभन्ते युद्धमीदृशम् ॥३२॥
\end{gitaverse}

\begin{transliteration}
yadṛcchayā copapannaṁ svargadvāramapāvṛtaṁ, \\
sukhinaḥ kṣatriyāḥ pārtha labhante yuddhamīdṛśam.
\end{transliteration}

32. Happy indeed are the KSHATRIYAS, O Partha, who are called to fight in such
a battle, that comes of itself as an open door to heaven.

\begin{gitaverse}
अथ चेत्त्वमिमं धर्म्यं सङ्ग्रामं न करिष्यसि । \\
ततः स्वधर्मं कीर्तिं च हित्वा पापमवाप्स्यसि ॥३३॥
\end{gitaverse}

\begin{transliteration}
atha cettvamimaṁ dharmyaṁ saṅgrāmaṁ na kariṣyasi, \\
tataḥ svadharmaṁ kīrtiṁ ca hitvā pāpamavāpṣyasi.
\end{transliteration}

33. But if you will not fight this righteous war, then having abandoned your
own duty and fame, you shall incur sin.

\begin{gitaverse}
अकीर्तिं चापि भूतानि कथयिष्यन्ति तेऽव्ययाम् । \\
सम्भावितस्य चाकीर्तिर्मरणादतिरिच्यते ॥३४॥
\end{gitaverse}

\begin{transliteration}
akīrtiṁ cāpi bhūtāni kathayiṣyanti te'vyayām, \\
sambhāvitasya cākīrtirmaraṇādatiricyate.
\end{transliteration}

34. People too, will recount your everlasting dishonour and to the one who has
been honoured, dishonour is more than death.

\begin{gitaverse}
भयाद्रणादुपरतं मंस्यन्ते त्वां महारथाः । \\
येषां च त्वं बहुमतो भूत्वा यास्यसि लाघवम् ॥३५॥
\end{gitaverse}

\begin{transliteration}
bhayādraṇāduparataṁ maṁsyante tvāṁ mahārathāḥ, \\
yeṣāṁ ca tvaṁ bahumato bhūtvā yāsyasi lāghavam.
\end{transliteration}

35. The great battalion commanders will think that you have withdrawn from the
battle through fear; and you will be looked down upon by those who had thought
much of you and your heroism in the past.

\begin{gitaverse}
अवाच्यवादांश्च बहून्वदिष्यन्ति तवाहिताः । \\
निन्दन्तस्तव सामर्थ्यं ततो दुःखतरं नु किम् ॥३६॥
\end{gitaverse}

\begin{transliteration}
avācyavādāṁśca bahūnvadiṣyanti tavāhitāḥ, \\
nindantastava sāmarthyaṁ tato duḥkhataraṁ nu kim.
\end{transliteration}

36. And many unspeakable words will your enemies speak caviling about your
powers. What can be more painful than this?

\begin{gitaverse}
हतो वा प्राप्स्यसि स्वर्गं जित्वा वा भोक्ष्यसे महीम् । \\
तस्मादुत्तिष्ठ कौन्तेय युद्धाय कृतनिश्चयः ॥३७॥
\end{gitaverse}

\begin{transliteration}
hato vā pṛāpsyasi svargaṁ jitvā vā bhokṣyase mahīm, \\
tasmāduttiṣṭha kaunteya yuddhāya kṛtaniścayaḥ.
\end{transliteration}

37. Slain, you will obtain heaven; victorious, you will enjoy the earth;
therefore, stand up O son of Kunti, determined to fight.

\begin{gitaverse}
सुखदुःखे समे कृत्वा लाभालाभौ जयाजयौ । \\
ततो युद्धाय युज्यस्व नैवं पापमवाप्स्यसि ॥३८॥
\end{gitaverse}

\begin{transliteration}
sukhaduḥkhe same kṛtvā lābhālābhau jayājayau, \\
tato yuddhāya yujyasva naivaṁ pāpamavāpsyasi.
\end{transliteration}

38. Having made pleasure and pain, gain and loss, victory and defeat the same,
engage in battle for the sake of battle; thus, you shall not incur sin.

\begin{gitaverse}
एषा तेऽभिहिता साङ्ख्ये बुद्धिर्योगे त्विमांश्रुणु । \\
बुद्ध्या युक्तो यया पार्थ कर्मबन्धं प्रहास्यसि ॥३९॥
\end{gitaverse}

\begin{transliteration}
eṣā te'bhihitā sāṅkhye buddhiryoge tvimāṁ śṛṇu, \\
buddhyā yukto yayā pārtha karmabandhaṁ prahāsyasi.
\end{transliteration}

39. This, which has been taught to thee, is wisdom concerning SANKHYA.\@ Now
listen to the wisdom concerning YOGA, having known which, O Partha, you shall
cast off the bonds-of-action.

\begin{gitaverse}
नेहाभिक्रमनाशोऽस्ति प्रत्यवायो न विद्यते । \\
स्वल्पमप्यस्य धर्मस्य त्रायते महतो भयात् ॥४०॥
\end{gitaverse}

\begin{transliteration}
nehābhikramanāśo'sti pratyavāyo na vidyate, \\
svalpamapyasya dharmasya trāyate mahato bhayāt.
\end{transliteration}

40. In this there is no loss of effort, nor is there any harm (production of
contrary results). Even a little of this knowledge, even a little practice of
the YOGA, protects one from the great fear.

\begin{gitaverse}
व्यवसायात्मिका बुद्धिरेकेह कुरुनन्दन । \\
बहुशाखा ह्यनन्ताश्च बुद्धयोऽव्यवसायिनाम् ॥४१॥
\end{gitaverse}

\begin{transliteration}
vyavasāyātmikā buddhirekeha kurunandana, \\
bahuśākhā hyanantāśca buddhayo'vyavasāyinām.
\end{transliteration}

41. Here, O Joy of the Kurus, Kurunandana, there is but a singlepointed
determination; many-branched and endless are the thoughts of the irresolute.

\begin{gitaverse}
यामिमां पुष्पितां वाचं प्रवदन्त्यविपश्चितः । \\
वेदवादरताः पार्थ नान्यदस्तीति वादिनः ॥४२॥
\end{gitaverse}

\begin{transliteration}
yāmimāṁ puṣpitāṁ vācaṁ pravadantyavipaścitaḥ, \\
vedavādaratāḥ pārtha nānyadastīti vādinaḥ.
\end{transliteration}

42. Flowery speech is uttered by the unwise, taking pleasure in the eulogising
words of VEDAS, O Partha, saying, ``There is nothing else''.

\begin{gitaverse}
कामात्मानः स्वर्गपरा जन्मकर्मफलप्रदाम् । \\
क्रियाविशेषबहुलां भोगैश्वर्यगतिं प्रति ॥४३॥
\end{gitaverse}

\begin{transliteration}
kāmātmānaḥ svargaparā janmakarmaphalapradām, \\
kriyāviśeṣabahulāṁ bhogaiśvaryagatiṁ prati.
\end{transliteration}

43. Full of desires, having heaven as their goal, they utter flowery words,
which promise new birth as the reward of their actions, and prescribe various
specific actions for the attainment of pleasure and Lordship.

\begin{gitaverse}
भोगैश्वर्यप्रसक्तानां तयापहृतचेतसाम् । \\
व्यवसायात्मिका बुद्धिः समाधौ न विधीयते ॥४४॥
\end{gitaverse}

\begin{transliteration}
bhogaiśvaryaprasaktānāṁ tayāpahṛtacetasām, \\
vyavasāyātmikā buddhiḥ samādhau na vidhīyate.
\end{transliteration}

44. For, those who cling to joy and Lordship, whose minds are drawn away by
such teaching, are neither determinate and resolute nor are they fit for steady
meditation and SAMADHI.\@

\begin{gitaverse}
त्रैगुण्यविषया वेदा निस्त्रैगुण्यो भवार्जुन । \\
निर्द्वन्द्वो नित्यसत्त्वस्थो निर्योगक्षेम आत्मवान् ॥४५॥
\end{gitaverse}

\begin{transliteration}
traiguṇyaviṣayā vedā nistraiguṇyo bhavārjuna, \\
nirdvāndvo nityasattvastho niryogakṣema ātmavān.
\end{transliteration}

45. The VEDAS deal with the three attributes; be you above these three
attributes (GUNAS) O Arjuna, free yourself from the pairs-of-opposites, and ever
remain in the SATTVA (goodness), freed from all thoughts of acquisition and
preservation, and be established in the Self.

\begin{gitaverse}
यावानर्थ उदपाने सर्वतः सम्प्लुतोदके । \\
तावान्सर्वेषु वेदेषु ब्राह्मणस्य विजानतः ॥४६॥
\end{gitaverse}

\begin{transliteration}
yāvānartha udapāne sarvataḥ samplutodake, \\
tāvānsarveṣu vedeṣu brāhmaṇasya vijānataḥ.
\end{transliteration}

46. To the BRAHMANA who has known the Self, all the VEDAS are of so much use,
as is a reservoir of water in a place where there is flood everywhere.

\begin{gitaverse}
कर्मण्येवाधिकारस्ते मा फलेषु कदाचन । \\
मा कर्मफलहेतुर्भूर्मा ते सङ्गोऽस्त्वकर्मणि ॥४७॥
\end{gitaverse}

\begin{transliteration}
karmaṇyevādhikāraste mā phaleṣu kadācana, \\
mā karmaphalaheturbhūrmā te saṅgo'stvakarmaṇi.
\end{transliteration}

47. Thy right is to work only, but never to its fruits; let the fruit-of-action
be not thy motive, nor let thy attachment be to inaction.

\begin{gitaverse}
योगस्थः कुरु कर्माणि सङ्गं त्यक्त्वा धनञ्जय । \\
सिद्ध्यसिद्ध्योः समो भूत्वा समत्वं योग उच्यते ॥४८॥
\end{gitaverse}

\begin{transliteration}
yogasthaḥ kuru karmāṇi saṅgaṁ tyaktvā dhanañjaya, \\
siddhyasiddhyoḥ samo bhūtvā samatvaṁ yoga ucyate.
\end{transliteration}

48. Perform action O Dhananjaya, abandoning attachment, being steadfast in
YOGA, and balanced in success and failure. Evenness of mind is called YOGA.\@

\begin{gitaverse}
दूरेण ह्यवरं कर्म बुद्धियोगाद्धनञ्जय । \\
बुद्धौ शरणमन्विच्छ कृपणाः फलहेतवः ॥४९॥
\end{gitaverse}

\begin{transliteration}
dūreṇa hyavaraṁ karma buddhiyogāddhanañjaya, \\
buddhau śaraṇamanviccha kṛpaṇāḥ phalahetavaḥ.
\end{transliteration}

49. Far lower than the YOGA-of-wisdom is action, O Dhananjaya. Seek thou refuge
in wisdom; wretched are they whose motive is the `fruit'.

\begin{gitaverse}
बुद्धियुक्तो जहातीह उभे सुकृतदुष्कृते । \\
तस्माद्योगाय युज्यस्व योगः कर्मसु कौशलम् ॥५०॥
\end{gitaverse}

\begin{transliteration}
buddhiyukto jahātīha ubhe sukṛtaduṣkṛte, \\
tasmādyogāya yujyasva yogaḥ karmasu kauśalam.
\end{transliteration}

50. Endowed with the Wisdom of evenness-of-mind, one casts off in this life
both good deeds and evil deeds; therefore, devote yourself to YOGA, Skill in
action is YOGA.\@

\begin{gitaverse}
कर्मजं बुद्धियुक्ता हि फलं त्यक्त्वा मनीषिणः । \\
जन्मबन्धविनिर्मुक्ताः पदं गच्छन्त्यनामयम् ॥५१॥
\end{gitaverse}

\begin{transliteration}
karmajaṁ buddhiyuktā hi phalaṁ tyaktvā manīṣiṇaḥ, \\
janmabandhavinirmuktāḥ padaṁ gacchantyanāmayam.
\end{transliteration}

51. The wise, possessed of knowledge, having abandoned the fruits of their
actions, freed from the fetters of birth, go to the State which is beyond all
evil.

\begin{gitaverse}
यदा ते मोहकलिलं बुद्धिर्व्यतितरिष्यति । \\
तदा गन्तासि निर्वेदं श्रोतव्यस्य श्रुतस्य च ॥५२॥
\end{gitaverse}

\begin{transliteration}
yadā te mohakalilaṁ buddhirvyatitariṣyati, \\
tadā gantāsi nirvedaṁ śrotavyasya śrutasya ca.
\end{transliteration}

52. When your intellect crosses beyond the mire of delusion, then you shall
attain to indifference as to what has been heard and what is yet to be heard.

\begin{gitaverse}
श्रुतिविप्रतिपन्ना ते यदा स्थास्यति निश्चला । \\
समाधावचला बुद्धिस्तदा योगमवाप्स्यसि ॥५३॥
\end{gitaverse}

\begin{transliteration}
śrutivipratipannā te yadā sthāsyati niścalā, \\
samādhāvacalā buddhistadā yogamavāpsyasi.
\end{transliteration}

53. When your intellect, though perplexed by what you have heard, shall stand
immovable and steady in the Self, then you shall attain Self-realisation.

\begin{gitaverse}
अर्जुन उवाच \\
स्थितप्रज्ञस्य का भाषा समाधिस्थस्य केशव । \\
स्थितधीः किं प्रभाषेत किमासीत व्रजेत किम् ॥५४॥
\end{gitaverse}

\begin{transliteration}
Arjuna uvāca \\
sthitaprajñasya kā bhāṣā samādhisthasya keśava, \\
sthitadhīḥ kiṁ prabhāṣeta kimāsīta vrajeta kim.
\end{transliteration}

Arjuna said: \\
54. What, O Keshava, is the description of him who has steady Wisdom and who is
merged in the Superconscious state? How does one of steady Wisdom speak, how
does he sit, how does he walk?

\begin{gitaverse}
श्रीभगवानुवाच \\
प्रजहाति यदा कामान्सर्वान्पार्थ मनोगतान् । \\
आत्मन्येवात्मना तुष्टः स्थितप्रज्ञस्तदोच्यते ॥५५॥
\end{gitaverse}

\begin{transliteration}
śrībhagavānuvāca \\
prajahāti yadā kāmānsarvānpārtha manogatān, \\
ātmanyevātmanā tuṣṭaḥ sthitaprajñastadocyate.
\end{transliteration}

The Blessed Lord said: \\
55. When a man completely casts off O Partha, all the desires of the mind, and
is satisfied in the Self by the Self, then he is said to be the one of
steady-Wisdom.

\begin{gitaverse}
दुःखेष्वनुद्विग्नमनाः सुखेषु विगतस्पृहः । \\
वीतरागभयक्रोधः स्थितधीर्मुनिरुच्यते ॥५६॥
\end{gitaverse}

\begin{transliteration}
duḥkheṣvanudvignamanāḥ sukheṣu vigataspṛhaḥ, \\
vītarāgabhayakrodhaḥ sthitadhīrmunirucyate.
\end{transliteration}

56. He whose mind is not shaken up by adversity, and who in prosperity does not
hanker after pleasures, who is free from attachment, fear, and anger is called
a Sage of Steady-Wisdom?

\begin{gitaverse}
यः सर्वत्रानभिस्नेहस्तत्तत्प्राप्य शुभाशुभम् । \\
नाभिनन्दति न द्वेष्टि तस्य प्रज्ञा प्रतिष्ठिता ॥५७॥
\end{gitaverse}

\begin{transliteration}
yaḥ sarvatrānabhisnehastattatprāpya śubhāśubham, \\
nābhinandati na dveṣṭi tasya prajñā pratiṣṭhitā.
\end{transliteration}

57. He who is everywhere without attachment, on meeting with anything good or
bad, who neither rejoices nor hates, his Wisdom is fixed.

\begin{gitaverse}
यदा संहरते चायं कूर्मोऽङ्गानीव सर्वशः । \\
इन्द्रियाणीन्द्रियार्थेभ्यस्तस्य प्रज्ञा प्रतिष्ठिता ॥५८॥
\end{gitaverse}

\begin{transliteration}
yadā saṁharate cāyaṁ kūrmo'ṅgānīva sarvaśaḥ, \\
indriyāṇīndriyārthebhyastasya prajñā pratiṣṭhitā.
\end{transliteration}

58. When like the tortoise which withdraws its limbs from all sides, He
withdraws His senses from the sense-objects then His Wisdom becomes steady.

\begin{gitaverse}
विषया विनिवर्तन्ते निराहारस्य देहिनः । \\
रसवर्जं रसोऽप्यस्य परं दृष्ट्वा निवर्तते ॥५९॥
\end{gitaverse}

\begin{transliteration}
viṣayā vinivartante nirāhārasya dehinaḥ, \\
rasavarjaṁ raso'pyasya paraṁ dṛṣṭvā nivartate.
\end{transliteration}

59. The objects of the senses turn away from the abstinent man leaving the
longing (behind); but his longing also leaves him upon seeing the Supreme?

\begin{gitaverse}
यततो ह्यपि कौन्तेय पुरुषस्य विपश्चितः । \\
इन्द्रियाणि प्रमाथीनि हरन्ति प्रसभं मनः ॥६०॥
\end{gitaverse}

\begin{transliteration}
yatato hyapi kaunteya puruṣasya vipaścitaḥ, \\
indriyāṇi pramāthīni haranti prasabhaṁ manaḥ.
\end{transliteration}

60. The turbulent senses, O son of Kunti, do violently carry away the mind of a
wise-man, though he be striving (to control them).

\begin{gitaverse}
तानि सर्वाणि संयम्य युक्त आसीत मत्परः । \\
वशे हि यस्येन्द्रियाणि तस्य प्रज्ञा प्रतिष्ठिता ॥६१॥
\end{gitaverse}

\begin{transliteration}
tāni sarvāṇi saṁyamya yukta āsīta matparaḥ, \\
vaśe hi yasyendriyāṇi tasya prajñā pratiṣṭhitā.
\end{transliteration}

61. Having restrained them all, He should sit steadfast, intent on Me; His
Wisdom is steady, whose senses are under control?

\begin{gitaverse}
ध्यायतो विषयान्पुंसः सङ्गस्तेषूपजायते । \\
सङ्गात्सञ्जायते कामः कामात्क्रोधोऽभिजायते ॥६२॥
\end{gitaverse}

\begin{transliteration}
dhyāyato viṣayānpuṁsaḥ saṅgasteṣūpajāyate, \\
saṅgātsañjāyate kāmaḥ kāmātkrodho'bhijāyate.
\end{transliteration}

62. When a man thinks of objects, `attachment' for them arises; from attachment
`desire' is born; from desire arises `anger'\ldots

\begin{gitaverse}
क्रोधाद्भवति सम्मोहः सम्मोहात्स्मृतिविभ्रमः । \\
स्मृतिभ्रंशाद् बुद्धिनाशो बुद्धिनाशात्प्रणश्यति ॥६३॥
\end{gitaverse}

\begin{transliteration}
krodhādbhavati sammohaḥ sammohātsmṛtivibhramaḥ, \\
smṛtibhraṁśād buddhināśo buddhināśātpraṇaśyati.
\end{transliteration}

63. From anger comes `delusion'; from delusion `loss of memory'; from loss of
memory the `destruction of discrimination'; from destruction of discrimination,
he `perishes'.

\begin{gitaverse}
रागद्वेषवियुक्तैस्तु विषयानिन्द्रियैश्चरन् \\
आत्मवश्यैर्विधेयात्मा प्रसादमधिगच्छति ॥६४॥
\end{gitaverse}

\begin{transliteration}
rāgadveṣaviyuktaistu viṣayānindriyaiścaran, \\
ātmavaśyairvidheyātmā prasādamadhigacchati.
\end{transliteration}

64. But the self-controlled man moving among objects, with his senses under
restraint, and free from both attraction and repulsion, attains peace.

\begin{gitaverse}
प्रसादे सर्वदुःखानां हानिरस्योपजायते । \\
प्रसन्नचेतसो ह्याशु बुद्धिः पर्यवतिष्ठते ॥६५॥
\end{gitaverse}

\begin{transliteration}
prasāde sarvaduḥkhānāṁ hānirasyopajāyate, \\
prasannacetaso hyāśu buddhiḥ paryavatiṣṭhate.
\end{transliteration}

65. In that peace all pains are destroyed; for, the intellect of the
tranquil-minded soon becomes steady.

\begin{gitaverse}
नास्ति बुद्धिरयुक्तस्य न चायुक्तस्य भावना । \\
न चाभावयतः शान्तिरशान्तस्य कुतः सुखम् ॥६६॥
\end{gitaverse}

\begin{transliteration}
nāsti buddhirayuktasya na cāyuktasya bhāvanā, \\
na cābhāvayataḥ śantiraśāntasya kutaḥ sukham.
\end{transliteration}

66. There is no knowledge (of the Self) to the unsteady; and to the unsteady no
meditation; and to the unmeditative no peace; to the peaceless, how can there
be happiness?

\begin{gitaverse}
इन्द्रियाणां हि चरतां यन्मनोऽनुविधीयते । \\
तदस्य हरति प्रज्ञां वायुर्नावमिवाम्भसि ॥६७॥
\end{gitaverse}

\begin{transliteration}
indriyāṇāṁ hi caratāṁ yanmano'nuvidhīyate, \\
tadasya harati prajñāṁ vāyurnāvamivāmbhasi.
\end{transliteration}

67. For, the mind which follows in the wake of the wandering senses, carries
away his discrimination, as the wind carries away a boat on the waters.

\begin{gitaverse}
तस्माद्यस्य महाबाहो निगृहीतानि सर्वशः । \\
इन्द्रियाणीन्द्रियार्थेभ्यस्तस्य प्रज्ञा प्रतिष्ठिता ॥६८॥
\end{gitaverse}

\begin{transliteration}
tasmādyasya mahābāho nigṛhītāni sarvaśaḥ, \\
indriyāṇīndriyārthebhyastasya prajñā pratiṣṭhitā.
\end{transliteration}

68. Therefore, O Mighty-armed, his knowledge is steady whose senses are
completely restrained from sense-objects.

\begin{gitaverse}
या निशा सर्वभूतानां तस्यां जागर्ति संयमी । \\
यस्यां जाग्रति भूतानि सा निशा पश्यतो मुनेः ॥६९॥
\end{gitaverse}

\begin{transliteration}
yā niśā sarvabhūtānāṁ tasyāṁ jāgarti saṁyamī, \\
yasyāṁ jāgrati bhūtāni sā niśā paśyato muneḥ.
\end{transliteration}

69. That which is night to all beings, in that the self-controlled man keeps
awake; where all beings are awake, that is the night for the Sage (MUNI) who
sees.

\begin{gitaverse}
आपूर्यमाणमचलप्रतिष्ठं- \\
\tab समुद्रमापः प्रविशन्ति यद्वत् । \\
तद्वत्कामा यं प्रविशन्ति सर्वे \\
\tab स शान्तिमाप्नोति न कामकामी ॥७०॥
\end{gitaverse}

\begin{transliteration}
āpūryamāṇamacalapratiṣṭhaṁ \\
\tab samudramāpaḥ praviśanti yadvat, \\
tadvatkāmā yaṁ praviśanti sarve \\
\tab sa śāntimāpnoti na kāmakāmī.
\end{transliteration}

70. He attains Peace into whom all desires enter, as waters enter the ocean,
which filled from all sides, remains unmoved but not the `desirer of desires'.

\begin{gitaverse}
विहाय कामान्यः सर्वान्पुमांश्चरति निःस्पृहः । \\
निर्ममो निरहङ्कारः स शान्तिमधिगच्छति ॥७१॥
\end{gitaverse}

\begin{transliteration}
vihāya kāmānyaḥ sarvānpumāṁścarati niḥspṛhaḥ, \\
nirmamo nirahaṅkāraḥ sa śāntimadhigacchati.
\end{transliteration}

71. That man attains peace who, abandoning all desires, moves about without
longing, without the sense of `I-ness' and `My-ness'.

\begin{gitaverse}
एषा ब्राह्मी स्थितिः पार्थ नैनां प्राप्य विमुह्यति । \\
स्थित्वास्यामन्तकालेऽपि ब्रह्मनिर्वाणमृच्छति ॥७२॥
\end{gitaverse}

\begin{transliteration}
eṣā brāhmī sthitiḥ pārtha naināṁ prāpya vimuhyati, \\
sthitvāsyāmantakāle'pi brahmanirvāṇamṛcchati.
\end{transliteration}

72. This is the BRAHMIC-state, O Son of Pritha. Attaining this, none is
deluded. Being established therein, even at the end of life, one attains to
oneness with BRAHMAN.\@

\begin{gitaverse}
ॐ तत्सदिति श्रीमद् भगवद् गीतासूपनिषत्सु ब्रह्मविद्यायां \\
योगशास्त्रे श्रीकृष्णार्जुनसंवादे सांख्ययोगो नाम \\
द्वितीयोऽध्यायः
\end{gitaverse}

\begin{transliteration}
oṁ tatsaditi śrīmad bhagavad gītāsūpaniṣatsu brahmavidyāyāṁ \\
yogaśāstre śrī kṛṣṇārjuna saṁvāde sāṅkhyayogo nāma \\
dvitīyo'dhyāyaḥ.
\end{transliteration}

Thus, in the UPANISHADS of the glorious Bhagawad Geeta, in the Science of the
Eternal, in the Scripture of YOGA, in the dialogue between Sri Krishna and
Arjuna, the second discourse ends entitled: The Yoga of Knowledge
