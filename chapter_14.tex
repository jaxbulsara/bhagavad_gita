\chapterdrop

\begin{center}
\headersanskrit{अथ चतुर्दशोऽध्यायः}

\headerspace
\headertransliteration{Atha Caturdaśo'dhyāyaḥ}

\section{Chapter 14}

\headerspace
\headersanskrit{गुणत्रयविभागयोगः}

\headerspace
\headertransliteration{Guṇatraya Vibhāga Yogah}

\headerspace
\headertranslation{The Yoga of Gunas}

\headerspace
\end{center}

\begin{gitaverse}
श्रीभगवानुवाच \\
परं भूयः प्रवक्ष्यामि ज्ञानानां ज्ञानमुत्तमम् । \\
यज्ज्ञात्वा मुनयः सर्वे परां सिद्धिमितो गताः ॥१॥
\end{gitaverse}

\begin{transliteration}
śrībhagavānuvāca \\
paraṁ bhūyaḥ pravakṣyāmi jñānānāṁ jñānamuttamam, \\
yajjñātvā munyaḥ sarve parāṁ siddhimito gatāḥ.
\end{transliteration}

The Blessed Lord said: \\
1. I will again declare (to you) that Supreme Knowledge, the best of all
knowledges, having known which, all the sages have attained Supreme Perfection
after this life.

\begin{gitaverse}
इदं ज्ञानमुपाश्रित्य मम साधर्म्यमागताः । \\
सर्गेऽपि नोपजायन्ते प्रलये न व्यथन्ति च ॥२॥
\end{gitaverse}

\begin{transliteration}
idaṁ jñānam-upāśritya mama sādharmyam-āgatāḥ, \\
sarge'pi nopajāyante pralaye na vyathanti ca.
\end{transliteration}

2. They who, having refuge in this ``Knowledge'' have attained to My Being, are
neither born at the time of Creation, nor are they disturbed at the time of
dissolution.

\begin{gitaverse}
मम योनिर्महद्ब्रह्म तस्मिन्गर्भं दधाम्यहम् । \\
सम्भवः सर्वभूतानां ततो भवति भारत ॥३॥
\end{gitaverse}

\begin{transliteration}
mama yonir-mahad-brahma tasmin-garbhaṁ dadhāmyaham, \\
sambhavaḥ sarva-bhūtānāṁ tato bhavati bhārata.
\end{transliteration}

3. My womb is the great BRAHMAN (MULA PRAKRITI); in that I place the germ; from
which, O Bharata, is the birth of all beings.

\begin{gitaverse}
सर्वयोनिषु कौन्तेय मूर्तयः सम्भवन्ति याः । \\
तासां ब्रह्म महद्योनिरहं बीजप्रदः पिता ॥४॥
\end{gitaverse}

\begin{transliteration}
sarvayoniṣu kaunteya mūrtayaḥ sambhavanti yāḥ, \\
tāsāṁ brahma mahadyonir-ahaṁ bījapradaḥ pitā.
\end{transliteration}

4. Whatever forms are produced, O Kaunteya, in all the wombs whatsoever, the
great BRAHMA (MULA PRAKRITI) is their womb, and I the seed-giving Father.

\begin{gitaverse}
सत्त्वं रजस्तम इति गुणाः प्रकृतिसम्भवाः । \\
निबध्नन्ति महाबाहो देहे देहिनमव्ययम् ॥५॥
\end{gitaverse}

\begin{transliteration}
sattvaṁ rajas-tama iti guṇāḥ prakṛti-sambhavāḥ, \\
nibadhnanti mahābāho dehe dehinam-avyayam.
\end{transliteration}

5. Purity, passion, and inertia---these qualities (GUNAS), O!\@ Mighty-armed,
born of ``PRAKRITI'' bind, the Indestructible, Embodied one, fast in the body.

\begin{gitaverse}
तत्र सत्त्वं निर्मलत्वात्प्रकाशकमनामयम् । \\
सुखसङ्गेन बध्नाति ज्ञानसङ्गेन चानघ ॥६॥
\end{gitaverse}

\begin{transliteration}
tatra sattvaṁ nirmalatvāt-prakāśakam-anāmayam, \\
sukhasaṅgena badhnāti jñānasaṅgena cānagha.
\end{transliteration}

6. Of these, ``SATTVA'' which because of its stainlessness, is luminous and
healthy, (unobstructive). It binds by (creating) attachment to `happiness' and
attachment to `knowledge,' O sinless one.

\begin{gitaverse}
रजो रागात्मकं विद्धि तृष्णासङ्गसमुद्भवम् । \\
तन्निबध्नाति कौन्तेय कर्मसङ्गेन देहिनम् ॥७॥
\end{gitaverse}

\begin{transliteration}
rajo rāgātmakaṁ viddhi tṛṣṇā-saṅga-samudbhavam, \\
tannibadhnāti kaunteya karmasaṅgena dehinam.
\end{transliteration}

7. Know thou `RAJAS' (to be) of the nature of passion, the source of thirst and
attachment; it binds fast, O Kaunteya, the embodied one, by attachment to
action.

\begin{gitaverse}
तमस्त्वज्ञानजं विद्धि मोहनं सर्व देहिनाम् । \\
प्रमादालस्यनिद्राभिस्तन्निबध्नाति भारत ॥८॥
\end{gitaverse}

\begin{transliteration}
tamastvajñānajaṁ viddhi mohanaṁ sarvadehinām, \\
pramādālasya-nidrābhistan-nibadhnāti bhārata.
\end{transliteration}

8. But, know thou TAMAS is born of ignorance, deluding all embodied beings, it
binds fast, O Bharata, by heedlessness, indolence and sleep.

\begin{gitaverse}
सत्त्वं सुखे सञ्जयति रजः कर्मणि भारत । \\
ज्ञानमावृत्य तु तमः प्रमादे सञ्जयत्युत ॥९॥
\end{gitaverse}

\begin{transliteration}
sattvaṁ sukhe sañjayati rajaḥ karmaṇi bhārata, \\
jñānam-āvṛtya tu tamaḥ pramāde sañjayatyuta.
\end{transliteration}

9. SATTVA attaches to happiness, RAJAS to action, O Bharata, while TAMAS,
verily, shrouding knowledge, attaches to heedlessness.

\begin{gitaverse}
रजस्तमश्चाभिभूय सत्त्वं भवति भारत । \\
रजः सत्त्वं तमश्चैव तमः सत्त्वं रजस्तथा ॥१०॥
\end{gitaverse}

\begin{transliteration}
rajas-tamaścābhibhūya sattvaṁ bhavati bhārata, \\
rajaḥ sattvaṁ tamaścaiva tamaḥ sattvaṁ rajas-tathā.
\end{transliteration}

10. Now SATTVA rises (prevails), O Bharata, having over-powered RAJAS and
inertia (TAMAS); now RAJAS, having over-powered SATTVA and inertia; and inertia
(TAMAS), having over-powered SATTVA and RAJAS.\@

\begin{gitaverse}
सर्वद्वारेषु देहेऽस्मिन्प्रकाश उपजायते । \\
ज्ञानं यदा तदा विद्याद्विवृद्धं सत्त्वमित्युत ॥११॥
\end{gitaverse}

\begin{transliteration}
sarvadvāreṣu dehe'smin-prakāśa upajāyate, \\
jñānaṁ yadā tadā vidyād-vivṛddhaṁ sattvamityuta.
\end{transliteration}

11. When, through every gate (sense) in this body, the light-ofintelligence
shines, then it may be known that `SATTVA' is predominant.

\begin{gitaverse}
लोभः प्रवृत्तिरारम्भः कर्मणामशमः स्पृहा । \\
रजस्येतानि जायन्ते विवृद्धे भरतर्षभ ॥१२॥
\end{gitaverse}

\begin{transliteration}
lobhaḥ pravṛttir-ārambhaḥ karmaṇām-aśamaḥ spṛhā, \\
rajasyetāni jāyante vivṛddhe bharatarṣabha.
\end{transliteration}

12. Greed, activity, undertaking of actions, restlessness longing---these arise
when RAJAS is predominant, O best in the Bharata family.

\begin{gitaverse}
अप्रकाशोऽप्रवृत्तिश्च प्रमादो मोह एव च । \\
तमस्येतानि जायन्ते विवृद्धे कुरुनन्दन ॥१३॥
\end{gitaverse}

\begin{transliteration}
aprakāśo'pravṛttiśca pramādo moha eva ca, \\
tamasyetāni jāyante vivṛddhe kurunandana.
\end{transliteration}

13. Darkness, inertness, heedlessness and delusion---these arise when TAMAS is
predominant, O descendant-of-Kuru.

\begin{gitaverse}
यदा सत्त्वे प्रवृद्धे तु प्रलयं याति देहभृत् । \\
तदोत्तमविदां लोकानमलान्प्रतिपद्यते ॥१४॥
\end{gitaverse}

\begin{transliteration}
yadā sattve pravṛddhe tu pralayaṁ yāti dehabhṛt, \\
tadottamavidāṁ lokān-amalān-pratipadyate.
\end{transliteration}

14. If the embodied one meets with death when SATTVA is predominant, then he
attains to the spotless worlds of the ``Knowers of the Highest''.

\begin{gitaverse}
रजसि प्रलयं गत्वा कर्मसङ्गिषु जायते । \\
तथा प्रलीनस्तमसि मूढयोनिषु जायते ॥१५॥
\end{gitaverse}

\begin{transliteration}
rajasi pralayaṁ gatvā karmasaṅgiṣu jāyate, \\
tathā pralīnstamasi mūḍhayoniṣu jāyate.
\end{transliteration}

15. Meeting death in RAJAS, he is born among those attached to action; and
dying in TAMAS, he is born in the womb of the senseless.

\begin{gitaverse}
कर्मणः सुकृतस्याहुः सात्त्विकं निर्मलं फलम् । \\
रजसस्तु फलं दुःखमज्ञानं तमसः फलम् ॥१६॥
\end{gitaverse}

\begin{transliteration}
karmaṇaḥ sukṛtasyāhuḥ sātvikaṁ nirmalaṁ phalam, \\
rajasastu phalaṁ duḥkam-ajñānaṁ tamasaḥ phalam.
\end{transliteration}

16. The fruit of good action, they say, is SATTVIC and pure; verily, the fruit
of RAJAS is pain, and the fruit of TAMAS is ignorance.

\begin{gitaverse}
सत्त्वात्सञ्जायते ज्ञानं रजसो लोभ एव च । \\
प्रमादमोहौ तमसो भवतोऽज्ञानमेव च ॥१७॥
\end{gitaverse}

\begin{transliteration}
sattvāt-sañjāyate jñānaṁ rajaso lobha eva ca, \\
pramāda-mohau tamaso bhavato'jñānam-eva ca.
\end{transliteration}

17. Knowledge arises from SATTVA, greed from RAJAS, heedlessness, delusion and
also ignorance arise from TAMAS.\@

\begin{gitaverse}
ऊर्ध्वं गच्छन्ति सत्त्वस्था मध्ये तिष्ठन्ति राजसाः । \\
जघन्यगुणवृत्तिस्था अधो गच्छन्ति तामसाः ॥१८॥
\end{gitaverse}

\begin{transliteration}
ūrdhvaṁ gacchanti sattvasthā madhye tiṣṭhanti rājasāḥ, \\
jaghanya-guṇa-vṛttisthā adho gacchanti tāmasāḥ.
\end{transliteration}

18. Those who are abiding in SATTVA go upwards; the RAJASIC dwell in the
middle; and the TAMASIC, abiding in the function of the lowest GUNA, go
downwards.

\begin{gitaverse}
नान्यं गुणेभ्यः कर्तारं यदा द्रष्टानुपश्यति । \\
गुणेभ्यश्च परं वेत्ति मद्भावं सोऽधिगच्छति ॥१९॥
\end{gitaverse}

\begin{transliteration}
nānyaṁ guṇebhyaḥ kartāraṁ yadā draṣṭānupaśyati, \\
guṇebhyaśca paraṁ vetti madbhāvaṁ so'dhigacchati.
\end{transliteration}

19. When the Seer beholds no agent other than the GUNAS and knows him who is
higher than the GUNAS, he attains to My Being.

\begin{gitaverse}
गुणानेतानतीत्य त्रीन्देही देहसमुद्भवान् । \\
जन्ममृत्युजरादुःखैर्विमुक्तोऽमृतमश्नुते ॥२०॥
\end{gitaverse}

\begin{transliteration}
guṇān-etān-atītya trīn-dehī deha-samudbhavān, \\
janma-mṛtyu-jarā-duḥkhair-vimukto'mṛtam-aśnute.
\end{transliteration}

20. The embodied-one having crossed beyond these three GUNAS out of which the
body is evolved, is freed from birth, death, decay, and pain, and attains to
Immortality.

\begin{gitaverse}
अर्जुन उवाच \\
कैर्लि ङ्गैस्त्रीन्गुणानेतानतीतो भवति प्रभो । \\
किमाचारः कथं चैतांस्त्रीन्गुणानतिवर्तते ॥२१॥
\end{gitaverse}

\begin{transliteration}
arjuna uvāca \\
kair-liṅgais-trīnguṇān-etān-atīto bhavati prabho, \\
kim-ācāraḥ kathaṁ caitān-strīn-guṇān-ativartate.
\end{transliteration}

Arjuna said: \\
21. What are the marks of him who has crossed over the three GUNAS, O Lord?
What is his conduct, and how does he go beyond these three GUNAS?\@

\begin{gitaverse}
श्रीभगवानुवाच \\
प्रकाशं च प्रवृत्तिं च मोहमेव च पाण्डव । \\
न द्वेष्टि सम्प्रवृत्तानि न निवृत्तानि काङ्क्षति ॥२२॥
\end{gitaverse}

\begin{transliteration}
śrībhagavānuvāca \\
prakāśaṁ ca pravṛttim ca mohameva ca pāṇḍava, \\
na dveṣṭi sampravṛttāni na nivṛttāni kāṅkṣati.
\end{transliteration}

The Blessed Lord said: \\
22. Light, activity, and delusion, when present, O Pandava, he hates not, nor
longs for them when absent.

\begin{gitaverse}
उदासीनवदासीनो गुणैर्यो न विचाल्यते । \\
गुणा वर्तन्त इत्येव योऽवतिष्ठति नेङ्गते ॥२३॥
\end{gitaverse}

\begin{transliteration}
udāsīnavad-āsīno guṇairyo na vicālyate, \\
guṇā vartanta ityeva yo'vatiṣṭhati neṅgate.
\end{transliteration}

23. He who, seated like one unconcerned, is not moved by the `GUNAS,' who,
knowing that the `GUNAS' operate, is selfcentred and swerves not\ldots

\begin{gitaverse}
समदुःखसुखः स्वस्थः समलोष्टाश्मकाञ्चनः । \\
तुल्यप्रियाप्रियो धीरस्तुल्यनिन्दात्मसंस्तुतिः ॥२४॥
\end{gitaverse}

\begin{transliteration}
sama-duḥkha-sukhaḥ svasthaḥ sama-loṣṭāśma-kāñcanaḥ, \\
tulya-priyāpriyo dhīras-tulya-nindātma-saṁstutiḥ.
\end{transliteration}

24. Alike in pleasure and pain; who dwells in the Self; to whom a clod of
earth, a precious stone, and gold are alike; to whom the dear and the not-dear
are the same; firm; the same in censure and self-praise.\ldots

\begin{gitaverse}
मानापमानयोस्तुल्यस्तुल्यो मित्रारिपक्षयोः । \\
सर्वारम्भपरित्यागी गुणातीतः स उच्यते ॥२५॥
\end{gitaverse}

\begin{transliteration}
mānāpamānayos-tulyastulyo mitrāri-pakṣayoḥ, \\
sarvārambha-parityāgī guṇātītaḥ sa ucyate.
\end{transliteration}

25. The same in honour and dishonour; the same to friend and foe; abandoning
all undertakings---he is said to have crossed beyond the GUNAS.\@

\begin{gitaverse}
मां च योऽव्यभिचारेण भक्तियोगेन सेवते । \\
स गुणान्समतीत्यैतान्ब्रह्मभूयाय कल्पते ॥२६॥
\end{gitaverse}

\begin{transliteration}
māṁ ca yo'vyabhicāreṇa bhaktiyogena sevate, \\
sa guṇān-samatītyaitān-brahma-bhūyāya kalpate.
\end{transliteration}

26. And he, serving Me with unswerving devotion, and crossing beyond the GUNAS,
is fit to become BRAHMAN.\@

\begin{gitaverse}
ब्रह्मणो हि प्रतिष्ठाहममृतस्याव्ययस्य च । \\
शाश्वतस्य च धर्मस्य सुखस्यैकान्तिकस्य च ॥२७॥
\end{gitaverse}

\begin{transliteration}
brahmaṇo hi pratiṣṭhāham-amṛtasyāvyayasya ca, \\
śāśvtasya ca dharmasya sukhasyaikāntikasya ca.
\end{transliteration}

27. For I am the Abode of BRAHMAN, the Immortal and the Immutable, of
everlasting DHARMA and of Absolute Bliss.

\begin{gitaverse}
ॐ तत्सदिति श्रीमद् भगवद् गीतासूपनिषत्सु ब्रह्मविद्यायां \\
योगशास्त्रे श्रीकृष्णार्जुनसंवादे गुणत्रयविभागयोगो नाम \\
चतुर्दशोऽध्यायः
\end{gitaverse}

\begin{transliteration}
oṁ tatsaditi śrīmad bhagavad gītāsūpaniṣatsu brahmavidyāyāṁ \\
yogaśāstre śrīkṛṣṇārjunasaṁvāde guṇatrayavibhāgayogo nāma \\
caturdaśo'dhyāyaḥ
\end{transliteration}

Thus, in the UPANISHADS of the glorious Bhagawad Geeta, in the Science of the
Eternal, in the scripture of YOGA, in the dialogue between Sri Krishna and
Arjuna, the fourteenth discourse ends entitled: The Yoga of Gunas
