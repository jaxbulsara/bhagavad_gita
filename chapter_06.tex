\chapterdrop

\begin{center}
\headersanskrit{अथ षष्ठोऽध्यायः}

\headerspace
\headertransliteration{Atha Ṣaṣṭho'dhyāyaḥ}

\section{Chapter 6}

\headerspace
\headersanskrit{आत्मसंयमयोगः}

\headerspace
\headertransliteration{Ātma Saṁyama Yogah}

\headerspace
\headertranslation{The Yoga of Meditation}

\headerspace
\end{center}

\begin{gitaverse}
श्रीभगवानुवाच \\
अनाश्रितः कर्मफलं कार्यं कर्म करोति यः । \\
स सन्न्यासी च योगी च न निरग्निर्न चाक्रियः ॥१॥
\end{gitaverse}

\begin{transliteration}
śrībhagavānuvāca \\
anāśritaḥ karmaphalaṁ kāryaṁ karma karoti yaḥ, \\
sa sannyāsī ca yogī ca na niragnir-na cākriyaḥ.
\end{transliteration}

The Blessed Lord said: \\
1. He who performs his bounden duty without depending on the
fruits-of-actions---he is a SANNYASIN and a YOGI;\@ not he who (has renounced)
is without fire and without action.

\begin{gitaverse}
यं सन्न्यासमिति प्राहुर्योगं तं विद्धि पाण्डव । \\
न ह्यसन्न्यस्तसङ्कल्पो योगी भवति कश्चन ॥२॥
\end{gitaverse}

\begin{transliteration}
yaṁ sannyāsam-iti prāhur-yogaṁ taṁ viddhi pāṇḍava, \\
na hyasannyastasaṅkalpo yogī bhavati kaścana.
\end{transliteration}

2. O Pandava, please know YOGA to be that which they call renunciation; no one
verily becomes a YOGI who has not renounced thoughts.

\begin{gitaverse}
आरुरुक्षोर्मुनेर्योगं कर्म कारणमुच्यते । \\
योगारूढस्य तस्यैव शमः कारणमुच्यते ॥३॥
\end{gitaverse}

\begin{transliteration}
ārurukṣor-muner-yogaṁ karma kāraṇam-ucyate, \\
yogārūḍhasya tasyaiva śamaḥ kāraṇam-ucyate.
\end{transliteration}

3. For a MUNI or sage who `wishes to attune to YOGA', action is said to be the
means; for the same sage who has `attuned to YOGA', inaction (quiescence) is
said to be the means.

\begin{gitaverse}
यदा हि नेन्द्रियार्थेषु न कर्मस्वनुषज्जते । \\
सर्वसङ्कल्पसन्न्यासी योगारूढस्तदोच्यते ॥४॥
\end{gitaverse}

\begin{transliteration}
yadā hi nendriyārtheṣu na karmasvanuṣajjate, \\
sarvasaṅkalpasannyāsī yogārūḍhas-tadocyate.
\end{transliteration}

4. When a man is not attached to sense-objects or to actions, having renounced
all thoughts, then he is said to have attuned to YOGA.\@

\begin{gitaverse}
उद्धरेदात्मनात्मानं नात्मानमवसादयेत् । \\
आत्मैव ह्यात्मनो बन्धुरात्मैव रिपुरात्मनः ॥५॥
\end{gitaverse}

\begin{transliteration}
uddhared-ātmanātmānaṁ nātmānam-avasādayet, \\
ātmaiva hyātmano bandhur-ātmaiva ripur-ātmanaḥ.
\end{transliteration}

5. Let a man lift himself by his own Self alone, and let him not lower himself;
for, this Self alone is the friend of oneself, and this Self is the enemy of
oneself.

\begin{gitaverse}
बन्धुरात्मात्मनस्तस्य येनात्मैवात्मना जितः । \\
अनात्मनस्तु शत्रुत्वे वर्तेतात्मैव शत्रुवत् ॥६॥
\end{gitaverse}

\begin{transliteration}
bandhur-ātmātmanas-tasya yenātmaivātmanā jitaḥ, \\
anātmanas-tu śatrutve vartetātmaiva śatruvat.
\end{transliteration}

6. The Self is the friend of the self for him who has conquered himself by the
Self, but to the unconquered self, the Self stands in the position of an enemy
like the (external) foe.

\begin{gitaverse}
जितात्मनः प्रशान्तस्य परमात्मा समाहितः । \\
शीतोष्णसुखदुःखेषु तथा मानापमानयोः ॥७॥
\end{gitaverse}

\begin{transliteration}
jitātmanaḥ praśāntasya paramātmā samāhitaḥ, \\
śītoṣṇa-sukhaduḥkheṣu tathā mānāpamānayoḥ.
\end{transliteration}

7. The Supreme Self of him who is self-controlled and peaceful, is balanced in
cold and heat, pleasure and pain, as also in honour and dishonour.

\begin{gitaverse}
ज्ञानविज्ञानतृप्तात्मा कूटस्थो विजितेन्द्रियः । \\
युक्त इत्युच्यते योगी समलोष्टाश्मकाञ्चनः ॥८॥
\end{gitaverse}

\begin{transliteration}
jñāna-vijñāna-tṛptātmā kūṭastho vijitendriyaḥ, \\
yukta ityucyate yogī sama-loṣṭāśma-kāñcanaḥ.
\end{transliteration}

8. The YOGI who is satisfied with knowledge and wisdom, who remains unshaken,
who has conquered the senses, to whom a lump of earth, a stone and gold are the
same, is said to be harmonised (i.e., is said to have attained NIRVIKALPA
SAMADHI).

\begin{gitaverse}
सुहृन्मित्रार्युदासीनमध्यस्थद्वेष्यबन्धुषु । \\
साधुष्वपि च पापेषु समबुद्धिर्विशिष्यते ॥९॥
\end{gitaverse}

\begin{transliteration}
suhṛn-mitrāryudāsīna-madhyastha-dveṣya-bandhuṣu, \\
sādhuṣvapi ca pāpeṣu samabuddhirviśiṣyate.
\end{transliteration}

9. He who is of the same mind to the good-hearted, friends, relatives, enemies,
the indifferent, the neutral, the hateful, the righteous and the unrighteous,
excels.

\begin{gitaverse}
योगी युञ्जीत सततमात्मानं रहसि स्थितः । \\
एकाकी यतचित्तात्मा निराशीरपरिग्रहः ॥१०॥
\end{gitaverse}

\begin{transliteration}
yogī yuñjita satatam-ātmānaṁ rahasi sthitaḥ, \\
ekāki yatacittātmā nirāśīr-aparigrahaḥ.
\end{transliteration}

10. Let the YOGI try constantly to keep the mind steady, remaining in solitude,
alone, with the mind and body controlled, free from hope and greed.

\begin{gitaverse}
शुचौ देशे प्रतिष्ठाप्य स्थिरमासनमात्मनः । \\
नात्युच्छ्रितं नातिनीचं चैलाजिनकुशोत्तरम् ॥११॥
\end{gitaverse}

\begin{transliteration}
śucau deśe pratiṣṭhāpya sthiram-āsanam-ātmanaḥ, \\
nātyucchritaṁ nātinīcaṁ cailājina-kuśottaram.
\end{transliteration}

11. Having, in a clean spot, established a firm seat of his own, neither too
high nor too low, made of a cloth, a skin and KUSHAgrass, one over the
other\ldots

\begin{gitaverse}
तत्रैकाग्रं मनः कृत्वा यतचित्तेन्द्रियक्रियः । \\
उपविश्यासने युञ्ज्याद्योगमात्मविशुद्धये ॥१२॥
\end{gitaverse}

\begin{transliteration}
tatraikāgraṁ manaḥ krtvā yatacittendriyakriyaḥ, \\
upaviśyāsane yuñjyād-yogam-ātmaviśuddhaye.
\end{transliteration}

12. There, having made the mind one-pointed, with the actions of the mind and
the senses controlled, let him, seated on the seat, practise YOGA, for the
purification of the self.

\begin{gitaverse}
समं कायशिरोग्रीवं धारयन्नचलं स्थिरः । \\
सम्प्रेक्ष्य नासिकाग्रं स्वं दिशश्चानवलोकयन् ॥१३॥
\end{gitaverse}

\begin{transliteration}
samaṁ kāya-śiro-grīvaṁ dhārayann-acalaṁ sthiraḥ, \\
samprekṣya nāsikāgraṁ svaṁ diśaścānavalokayan.
\end{transliteration}

13. Let him firmly hold his body, head and neck erect and still, gazing at the
tip of his nose, without looking around.

\begin{gitaverse}
प्रशान्तात्मा विगतभीर्ब्रह्मचारिव्रते स्थितः । \\
मनः संयम्य मच्चित्तो युक्त आसीत मत्परः ॥१४॥
\end{gitaverse}

\begin{transliteration}
praśāntātmā vigatabhīr-brahmacārivrate sthitaḥ, \\
manaḥ saṁyamya maccitto yukta āsīta matparaḥ.
\end{transliteration}

14. Serene-minded, fearless, firm in the vow of BRAHMACHARYA, having controlled
the mind, thinking on Me and balanced, let him sit, having Me as the Supreme
Goal.

\begin{gitaverse}
युञ्जन्नेवं सदात्मानं योगी नियतमानसः । \\
शान्तिं निर्वाणपरमां मत्संस्थामधिगच्छति ॥१५॥
\end{gitaverse}

\begin{transliteration}
yuñjann-evaṁ sadātmānaṁ yogī niyatamānasaḥ, \\
śāntim nirvāṇa-paramāṁ matsaṁsthām-adhigacchati.
\end{transliteration}

15. Thus, always keeping the mind balanced, the YOGI, with his mind controlled,
attains the Peace abiding in Me, which culminates in total liberation (NIRVANA
or MOKSHA).

\begin{gitaverse}
नात्यश्नतस्तु योगोऽस्ति न चैकान्तमनश्नतः । \\
न चाति स्वप्नशीलस्य जाग्रतो नैव चार्जुन ॥१६॥
\end{gitaverse}

\begin{transliteration}
nātyaśnatas-tu yogo'sti na caikāntam-anaśnataḥ, \\
na cāti svapnaśīlasya jāgrato naiva cārjuna.
\end{transliteration}

16. Verily, YOGA is not possible for him who eats too much, nor for him who
does not eat at all; nor for him who sleeps too much, nor for him who is
(always) awake, O Arjuna.

\begin{gitaverse}
युक्ताहारविहारस्य युक्तचेष्टस्य कर्मसु । \\
युक्तस्वप्नावबोधस्य योगो भवति दुःखहा ॥१७॥
\end{gitaverse}

\begin{transliteration}
yuktāhāra-vihārasya yukta-ceṣtasya karmasu \\
yukta-svapnāva-bodhasya yogo bhavati duḥkhahā.
\end{transliteration}

17. YOGA becomes the destroyer of pain for him who is moderate in eating and
recreation, who is moderate in his exertion during his actions, who is moderate
in sleep and wakefulness.

\begin{gitaverse}
यदा विनियतं चित्तमात्मन्येवावतिष्ठते । \\
निःस्पृहः सर्वकामेभ्यो युक्त इत्युच्यते तदा ॥१८॥
\end{gitaverse}

\begin{transliteration}
yadā viniyataṁ cittam-ātmanyevāvatiṣthate, \\
niḥspṛhaḥ sarvakāmebhyo yukta ityucyate tadā.
\end{transliteration}

18. When the perfectly controlled mind rests in the Self only, free from
longing for all (objects of) desire, then it is said: ``he is united''
(YUKTAH).

\begin{gitaverse}
यथा दीपो निवातस्थो नेङ्गते सोपमा स्मृता । \\
योगिनो यतचित्तस्य युञ्जतो योगमात्मनः ॥१९॥
\end{gitaverse}

\begin{transliteration}
yathā dīpo nivātastho neṅgate sopamā smṛtā, \\
yogīno yatacittasya yuñjato yogamātmanaḥ.
\end{transliteration}

19. ``As a lamp placed in a windless place does not flicker''---is a simile
used to describe the YOGI of controlled-mind, practising YOGA of the Self (or
absorbed in the YOGA-of-the-Self).

\begin{gitaverse}
यत्रोपरमते चित्तं निरुद्धं योगसेवया । \\
यत्र चैवात्मनात्मानं पश्यन्नात्मनि तुष्यति ॥२०॥
\end{gitaverse}

\begin{transliteration}
yatroparamate cittaṁ niruddhaṁ yogasevayā, \\
yatra caivātmanātmanāṁ paśyann-ātmani tuṣyati.
\end{transliteration}

20. When the mind, restrained by the practice of YOGA, attains quietude and
when seeing the Self by the self, he is satisfied in his own Self;

\begin{gitaverse}
सुखमात्यन्तिकं यत्तद् बुद्धिग्राह्यमतीन्द्रियम् । \\
वेत्ति यत्र न चैवायं स्थितश्चलति तत्त्वतः ॥२१॥
\end{gitaverse}

\begin{transliteration}
sukham-ātyantikaṁ yattad buddhigrāhyam-atīndriyam, \\
vetti yatra na caivāyaṁ sthitaś-calati tattvataḥ.
\end{transliteration}

21. When he (the YOGI) feels that Infinite Bliss---which can be grasped by the
(pure) intellect and which transcends the senses---wherein established he never
moves from the Reality;

\begin{gitaverse}
यं लब्ध्वा चापरं लाभं मन्यते नाधिकं ततः । \\
यस्मिन्स्थितो न दुःखेन गुरुणापि विचाल्यते ॥२२॥
\end{gitaverse}

\begin{transliteration}
yaṁ labdhvā cāparaṁ lābhaṁ manyate nādhikaṁ tataḥ, \\
yasmin-sthito na duḥkhena guruṇāpi vicālyate.
\end{transliteration}

22. Which, having obtained, he thinks there is no other gain superior to it;
wherein established, he is not moved even by heavy sorrow.

\begin{gitaverse}
तं विद्याद् दुःखसंयोगवियोगं योगसञ्ज्ञितम् । \\
स निश्चयेन योक्तव्यो योगोऽनिर्विण्णचेतसा ॥२३॥
\end{gitaverse}

\begin{transliteration}
taṁ vidyād duḥkha-saṁyoga-viyogaṁ yoga-sañjñitam, \\
sa niścayena yoktavyo yogo'nirviṇṇacetasā.
\end{transliteration}

23. Let it be known: the severance from the union-with-pain is YOGA.\@ This
YOGA should be practised with determination and with a mind steady and
undespairing.

\begin{gitaverse}
सङ्कल्पप्रभवान्कामांस्त्यक्त्वा सर्वानशेषतः । \\
मनसैवेन्द्रियग्रामं विनियम्य समन्ततः ॥२४॥
\end{gitaverse}

\begin{transliteration}
saṅkalpa-prabhavān-kāmāṁs-tyaktvā sarvānaśeṣataḥ, \\
manasaivendriya-grāmaṁ viniyamya samantataḥ.
\end{transliteration}

24. Abandoning without reserve all desires born of SANKALPA, and completely
restraining the whole group of senses by the mind from all sides.

\begin{gitaverse}
शनैः शनैरुपरमेद्बुद्ध्या धृतिगृहीतया । \\
आत्मसंस्थं मनः कृत्वा न किञ्चिदपि चिन्तयेत् ॥२५॥
\end{gitaverse}

\begin{transliteration}
śanaiḥ śanair-uparamed-buddhyā dhṛtigṛhītayā, \\
ātmasaṁsthaṁ manaḥ kṛtvā na kiñcid-api cintayet.
\end{transliteration}

25. Little by little, let him attain quietude by his intellect, held firm;
having made the mind established in the Self, let him not think of anything.

\begin{gitaverse}
यतो यतो निश्चरति मनश्चञ्चलमस्थिरम् । \\
ततस्ततो नियम्यैतदात्मन्येव वशं नयेत् ॥२६॥
\end{gitaverse}

\begin{transliteration}
yato yato niścarati manaś-cañcalam-asthiram, \\
tatastato niyamyaitad-ātmanyeva vaśam nayet.
\end{transliteration}

26. From whatever cause the restless and the unsteady mind wanders away, from
that let him restrain it, and bring it back under the control of the Self
alone.

\begin{gitaverse}
प्रशान्तमनसं ह्येनं योगिनं सुखमुत्तमम् । \\
उपैति शान्तरजसं ब्रह्मभूतमकल्मषम् ॥२७॥
\end{gitaverse}

\begin{transliteration}
praśānta-manasaṁ hyenaṁ yogīnaṁ sukham-uttamam, \\
upaiti śāntarajasaṁ brahmabhūtam-akalmaṣam.
\end{transliteration}

27. Supreme Bliss verily comes to this YOGI, whose mind is quite peaceful,
whose passion is quietened, who is free from sin, and who has become BRAHMAN.\@

\begin{gitaverse}
युञ्जन्नेवं सदात्मानं योगी विगतकल्मषः । \\
सुखेन ब्रह्मसंस्पर्शमत्यन्तं सुखमश्नुते ॥२८॥
\end{gitaverse}

\begin{transliteration}
yuñjann-evaṁ sadātmānaṁ yogī vigatakalmaṣaḥ, \\
sukhena brahma-saṁsparśam-atyantaṁ sukhamaśnute.
\end{transliteration}

28. The YOGI engaging the mind thus (in the practice of YOGA), freed from sins,
easily enjoys the Infinite Bliss of ``BRAHMAN-contact''.

\begin{gitaverse}
सर्वभूतस्थमात्मानं सर्वभूतानि चात्मनि । \\
ईक्षते योगयुक्तात्मा सर्वत्र समदर्शनः ॥२९॥
\end{gitaverse}

\begin{transliteration}
sarvabhūtastham-ātmānaṁ sarvabhūtāni cātmani, \\
īkṣate yogayuktātmā sarvatra samadarśanaḥ.
\end{transliteration}

29. With the mind harmonised by YOGA he sees the Self abiding in all beings,
and all beings in the Self; he sees the same everywhere.

\begin{gitaverse}
यो मां पश्यति सर्वत्र सर्वं च मयि पश्यति । \\
तस्याहं न प्रणश्यामि स च मे न प्रणश्यति ॥३०॥
\end{gitaverse}

\begin{transliteration}
yo māṁ paśyati sarvatra sarvaṁ ca mayi paśyati, \\
tasyāhaṁ na praṇaśyāmi sa ca me na praṇaśyati.
\end{transliteration}

30. He who sees Me everywhere, and sees everything in Me, he never gets
separated from Me, nor do I get separated from him.

\begin{gitaverse}
सर्वभूतस्थितं यो मां भजत्येकत्वमास्थितः । \\
सर्वथा वर्तमानोऽपि स योगी मयि वर्तते ॥३१॥
\end{gitaverse}

\begin{transliteration}
sarvabhūtasthitaṁ yo māṁ bhajatyekatvam-āsthitaḥ, \\
sarvathā vartamāno'pi sa yogī mayi vartate.
\end{transliteration}

31. He who, being established in unity, worships Me, dwelling in all beings,
that YOGI abides in Me, whatever be his mode of living.

\begin{gitaverse}
आत्मौपम्येन सर्वत्र समं पश्यति योऽर्जुन । \\
सुखं वा यदि वा दुःखं स योगी परमो मतः ॥३२॥
\end{gitaverse}

\begin{transliteration}
ātmaupamyena sarvatra samaṁ paśyati yo'rjuna, \\
sukhaṁ vā yadi vā duḥkhaṁ sa yogī paramo mataḥ.
\end{transliteration}

32. He who, through the likeness (sameness) of the Self, O Arjuna, sees
equality everywhere, be it pleasure or pain, he is regarded as the highest
YOGI.\@

\begin{gitaverse}
अर्जुन उवाच \\
योऽयं योगस्त्वया प्रोक्तः साम्येन मधुसूदन । \\
एतस्याहं न पश्यामि चञ्चलत्वात्स्थितिं स्थिराम् ॥३३॥
\end{gitaverse}

\begin{transliteration}
arjuna uvāca \\
yo'yaṁ yogastvayā proktaḥ sāmyena madhusūdana, \\
etasyāhaṁ na paśyāmi cañcalatvāt-sthitiṁ sthirām.
\end{transliteration}

Arjuna said: \\
33. This YOGA of Equanimity, taught by Thee, O slayer of Madhu, I see not its
enduring continuity, because of the restlessness (of the mind).

\begin{gitaverse}
चञ्चलं हि मनः कृष्ण प्रमाथि बलवद्दृढम् । \\
तस्याहं निग्रहं मन्ये वायोरिव सुदुष्करम् ॥३४॥
\end{gitaverse}

\begin{transliteration}
cañcalaṁ hi manaḥ kṛṣṇa pramāthi balavad-dṛḍham, \\
tasyāhaṁ nigrahaṁ manye vāyoriva suduṣkaram.
\end{transliteration}

34. The mind verily is, O Krishna, restless, turbulent, strong and unyielding;
I deem it quite as difficult to control as the wind.

\begin{gitaverse}
श्रीभगवानुवाच \\
असंशयं महाबाहो मनो दुर्निग्रहं चलम् । \\
अभ्यासेन तु कौन्तेय वैराग्येण च गृह्यते ॥३५॥
\end{gitaverse}

\begin{transliteration}
śrī bhagavānuvāca \\
asaṁśayaṁ mahābāho mano durnigrahaṁ calam, \\
abhyāsena tu kaunteya vairāgyeṇa ca gṛhyate.
\end{transliteration}

The Blessed Lord said: \\
35. Undoubtedly, O mighty-armed one, the mind is difficult to control and is
restless; but, by practice, O Son of Kunti, and by dispassion, it is
restrained.

\begin{gitaverse}
असंयतात्मना योगो दुष्प्राप इति मे मतिः । \\
वश्यात्मना तु यतता शक्योऽवाप्तुमुपायतः ॥३६॥
\end{gitaverse}

\begin{transliteration}
asaṁyatātmanā yogo duṣprāpa iti me matiḥ, \\
vaśyātmana tu yatatā śakyo'vāptum-upāyataḥ.
\end{transliteration}

36. YOGA, I think is hard to be attained by one of uncontrolled self; but the
self-controlled, striving, can obtain it by (proper) means.

\begin{gitaverse}
अर्जुन उवाच \\
अयतिः श्रद्धयोपेतो योगाच्चलितमानसः । \\
अप्राप्य योगसंसिद्धिं कां गतिं कृष्ण गच्छति ॥३७॥
\end{gitaverse}

\begin{transliteration}
arjuna uvāca \\
ayatiḥ śraddhayopeto yogāc-calitamānasaḥ, \\
aprāpya yogasaṁsiddhiṁ kāṁ gatiṁ kṛṣṇa gacchati.
\end{transliteration}

Arjuna said: \\
37. When a man, though possessed of faith, is unable to control himself, and
his mind wanders away from YOGA, to what end does he, having failed to attain
perfection in YOGA go, O Krishna?

\begin{gitaverse}
कच्चिन्नोभयविभ्रष्टश्छिन्नाभ्रमिव नश्यति । \\
अप्रतिष्ठो महाबाहो विमूढो ब्रह्मणः पथि ॥३८॥
\end{gitaverse}

\begin{transliteration}
kaccinnobhayavibhraṣṭaś-chinnābhramiva naśyati, \\
apratiṣṭho mahābāho vimūḍho brahmaṇaḥ pathi.
\end{transliteration}

38. Fallen from both, does he not, O mighty-armed, perish like a rent cloud,
supportless and deluded in the path of BRAHMAN?\@

\begin{gitaverse}
एतन्मे संशयं कृष्ण छेत्तुमर्हस्यशेषतः । \\
त्वदन्यः संशयस्यास्य छेत्ता न ह्युपपद्यते ॥३९॥
\end{gitaverse}

\begin{transliteration}
etan-me saṁśayaṁ kṛṣṇa chettum-arhasyaśeṣataḥ, \\
tvadanyaḥ saṁśayasyāsya chettā na hyupapadyate.
\end{transliteration}

39. This doubt of mine, O Krishna, please dispel completely; because it is not
possible for any one but You to dispel this doubt.

\begin{gitaverse}
श्रीभगवानुवाच \\
पार्थ नैवेह नामुत्र विनाशस्तस्य विद्यते । \\
न हि कल्याणकृत्कश्चिद्दुर्गतिं तात गच्छति ॥४०॥
\end{gitaverse}

\begin{transliteration}
śrībhagavānuvāca \\
pārtha naiveha nāmutra vīnāśas-tasya vidyate, \\
na hi kalyāṇakṛt-kaścid-durgatiṁ tāta gacchati.
\end{transliteration}

The Blessed Lord said: \\
40. O Partha, neither in this world, nor in the next world is there destruction
for him; none, verily, who strives to be good, O My son, ever comes to grief.

\begin{gitaverse}
प्राप्य पुण्यकृतां लोकानुषित्वा शाश्वतीः समाः । \\
शुचीनां श्रीमतां गेहे योगभ्रष्टोऽभिजायते ॥४१॥
\end{gitaverse}

\begin{transliteration}
prāpya puṇyakṛtāṁ lokān-uṣitvā śāśvatiḥ samāḥ, \\
śucīnāṁ srīmatāṁ gehe yogabhraṣṭo'bhijāyate.
\end{transliteration}

41. Having attained to the worlds of the righteous, and having dwelt there for
everlasting (long) years, he who had fallen from YOGA is born again in the
house of the pure and the wealthy.

\begin{gitaverse}
अथवा योगिनामेव कुले भवति धीमताम् । \\
एतद्धि दुर्लभतरं लोके जन्म यदीदृशम् ॥४२॥
\end{gitaverse}

\begin{transliteration}
athavā yogīnām-eva kule bhavati dhīmatām, \\
etadd-hi durlabhataraṁ loke janma yad-īdrśam.
\end{transliteration}

42. Or, he is even born in the family of the wise YOGIS;\@ verily, a birth like
this is very difficult to obtain in this world.

\begin{gitaverse}
तत्र तं बुद्धिसंयोगं लभते पौर्वदेहिकम् । \\
यतते च ततो भूयः संसिद्धौ कुरुनन्दन ॥४३॥
\end{gitaverse}

\begin{transliteration}
tatra taṁ buddhisaṁyogaṁ labhate paurvadehikam, \\
yatate ca tato bhūyaḥ saṁsiddhau kurunandana.
\end{transliteration}

43. There he comes to be united with the knowledge acquired in his former body
and strives more than before for Perfection, O son of the Kurus.

\begin{gitaverse}
पूर्वाभ्यासेन तेनैव ह्रियते ह्यवशोऽपि सः । \\
जिज्ञासुरपि योगस्य शब्दब्रह्मातिवर्तते ॥४४॥
\end{gitaverse}

\begin{transliteration}
pūrvābhyāsena tenaiva hriyate hyavaśo'pi saḥ, \\
jijñāsur-api yogasya śabdabrahmātivartate.
\end{transliteration}

44. By that very former practice he is borne on inspite of himself. Even he who
merely wishes to know YOGA goes beyond the SHABDA BRAHMAN.\@

\begin{gitaverse}
प्रयत्नाद्यतमानस्तु योगी संशुद्धकिल्बिषः । \\
अनेकजन्मसंसिद्धस्ततो याति परां गतिम् ॥४५॥
\end{gitaverse}

\begin{transliteration}
prayatnād-yatamānas-tu yogī saṁśuddhakilbiṣaḥ, \\
anekajanmasaṁsiddhas-tato yāti parāṁ gatiṁ.
\end{transliteration}

45. But the YOGI, who strives with assiduity, purified from sins and perfected
(gradually) through many births, he then attains the highest Goal.

\begin{gitaverse}
तपस्विभ्योऽधिको योगी ज्ञानिभ्योऽपि मतोऽधिकः । \\
कर्मिभ्यश्चाधिको योगी तस्माद्योगी भवार्जुन ॥४६॥
\end{gitaverse}

\begin{transliteration}
tapasvibhyo'dhiko yogī jñānibhyo'pi mato'dhikaḥ, \\
karmibhyaścādhiko yogī tasmād-yogī bhavārjuna.
\end{transliteration}

46. The YOGI is thought to be superior to the ascetics, and even superior to
Men-of-Knowledge (mere scholars); he is also superior to Men-of-Action;
therefore (you strive to) be a YOGI, O Arjuna.

\begin{gitaverse}
योगिनामपि सर्वेषां मद्गतेनान्तरात्मना । \\
श्रद्धावान्भजते यो मां स मे युक्ततमो मतः ॥४७॥
\end{gitaverse}

\begin{transliteration}
yogīnām-api saraveṣām madgatenāntarātmanā, \\
śrāddhāvān-bhajate yo māṁ sa me yuktatamo mataḥ.
\end{transliteration}

47. And among all YOGIS, he who, full of faith, with his inner-self merged in
Me, worships Me, is, according to Me, the most devout.

\begin{gitaverse}
ॐ तत्सदिति श्रीमद् भगवद् गीतासूपनिषत्सु ब्रह्मविद्यायां \\
योगशास्त्रे श्रीकृष्णार्जुनसंवादे आत्मसंयमयोगो नाम \\
षष्ठोऽध्यायः
\end{gitaverse}

\begin{transliteration}
oṁ tatsaditi śrīmad bhagavad gītāsūpaniṣatsu brahmavidyāyāṁ \\
yogaśāstre śrī kṛṣṇārjuna saṁvāde ātmasaṁyamayogo nāma \\
ṣaṣṭho'dhyāyaḥ.
\end{transliteration}

Thus, in the UPANISHADS of the glorious Bhagawad-Geeta, in the Science of the
Eternal, in the Scripture of YOGA, in the dialogue between Sri Krishna and
Arjuna, the sixth discourse ends entitled: The Yoga of Meditation
