\chapterdrop

\begin{center}
\headersanskrit{अथ पञ्चदशोऽध्यायः}

\headerspace
\headertransliteration{Atha Pañcadaśo'dhyāyaḥ}

\section{Chapter 15}

\headerspace
\headersanskrit{पुरुषोत्तमयोगः}

\headerspace
\headertransliteration{Puruṣottama Yogah}

\headerspace
\headertranslation{The Yoga of the Supreme Spirit}

\headerspace
\end{center}

\begin{gitaverse}
श्रीभगवानुवाच \\
ऊर्ध्वमूलमधःशाखमश्वत्थं प्राहुरव्ययम् । \\
छन्दांसि यस्य पर्णानि यस्तं वेद स वेदवित् ॥१॥
\end{gitaverse}

\begin{transliteration}
Śrībhagavānuvāca \\
ūrdhvamūlamadhaḥśākhamaśvatthaṁ prāhuravyayam, \\
chandāṁsi yasya parṇāni yastaṁ veda sa vedavit.
\end{transliteration}

The Blessed Lord said: \\
1. They (wise people) speak of the indestructible ASHWATTHA tree as having its
roots above and branches below, whose leaves are the VEDAS;\@ he who knows it,
is alone a Veda-knower.

\begin{gitaverse}
अधश्चोर्ध्वं प्रसृतास्तस्य शाखा \\
\tab गुणप्रवृद्धा विषयप्रवालाः । \\
अधश्च मूलान्यनुसन्ततानि \\
\tab कर्मानुबन्धीनि मनुष्यलोके ॥२॥
\end{gitaverse}

\begin{transliteration}
adhaścordhvaṁ prasṛtāstasya śākhā \\
\tab guṇapravṛddhā viṣayapravālāḥ, \\
adhaśca mūlānyanusantatāni \\
\tab karmānubandhīni manuṣyaloke.
\end{transliteration}

2. Below and above are spread its branches, nourished by the GUNAS;\@
sense-objects are its buds; and below, in the world of men, stretch forth the
roots, originating in action.

\begin{gitaverse}
न रूपमस्येह तथोपलभ्यते \\
\tab नान्तो न चादिर्न च सम्प्रतिष्ठा । \\
अश्वत्थमेनं सुविरूढमूल- \\
\tab मसङ्गशस्त्रेण दृढेन छित्त्वा ॥३॥
\end{gitaverse}

\begin{transliteration}
na rūpamasyeha tathopalabhyate \\
\tab nānto na cādirna ca sampratiṣṭhā, \\
aśvatthamenaṁ suvirūḍhamūla- \\
\tab masaṇgaśastreṇa dṛḍhena chittvā.
\end{transliteration}

3. Its form is not perceived here as such, neither its end, nor its origin, nor
its foundation, nor its resting-place; having cut asunder this firm-rooted
ASHWATTHA-tree with the strong axe of nonattachment\ldots

\begin{gitaverse}
ततः पदं तत्परिमार्गितव्यं- \\
\tab यस्मिन्गता न निवर्तन्ति भूयः । \\
तमेव चाद्यं पुरुषं प्रपद्ये \\
\tab यतः प्रवृत्तिः प्रसृता पुराणी ॥४॥
\end{gitaverse}

\begin{transliteration}
tataḥ padaṁ tatparimārgitavyam- \\
\tab yasmingatā na nivartanti bhūyah, \\
tameva cādyam puruṣaṁ prapadye \\
\tab yataḥ pravṛttiḥ prasṛtā purāṇī.
\end{transliteration}

4. Then that Goal should be sought after, where having gone, none returns
again. I seek refuge in that `primeval PURUSHA from which streamed forth, from
time immemorial, all activity (or energy).

\begin{gitaverse}
निर्मानमोहा जितसङ्गदोषा- \\
\tab अध्यात्मनित्या विनिवृत्तकामाः । \\
द्वन्द्वैर्विमुक्ताः सुखदुःखसंज्ञै- \\
\tab र्गच्छन्त्यमूढाः पदमव्ययं तत् ॥५॥
\end{gitaverse}

\begin{transliteration}
nirmānamohā jitasaṅgadoṣā- \\
\tab adhyātmanityā vinivṛttakāmāḥ, \\
dvandvairvimuktāḥ sukhaduḥkhasañjnai- \\
\tab rgacchantyamūḍhāḥ padamavyayaṁ tat.
\end{transliteration}

5. Free from pride and delusion, victorious over the evil of attachment,
dwelling constantly in the Self, their desires having completely retired, freed
from the pairs-of-opposites---such as pleasure and pain---the undeluded reach
that Goal Eternal.

\begin{gitaverse}
न तद्भासयते सूर्यो न शशाङ्को न पावकः । \\
यद्गत्वा न निवर्तन्ते तद्धाम परमं मम ॥६॥
\end{gitaverse}

\begin{transliteration}
na tadbhāsayate sūryo na śaśāṅko na pāvakaḥ, \\
yadgatvā na nivartante taddhāma paramaṁ mama.
\end{transliteration}

6. Neither does the Sun shine there, nor the moon, nor fire; to which having
gone they return not; that is My Supreme Abode.

\begin{gitaverse}
ममैवांशो जीवलोके जीवभूतः सनातनः । \\
मनःषष्ठानीन्द्रियाणि प्रकृतिस्थानि कर्षति ॥७॥
\end{gitaverse}

\begin{transliteration}
mamaivāṁśo jīvaloke jīvabhūtaḥ sanātanaḥ, \\
manaḥṣaṣṭhānīndriyāṇi prakṛtisthāni karṣati.
\end{transliteration}

7. An eternal portion of Myself, having become a living souls in the world of
life, abiding in PRAKRITI, draws (to itself) the (five) senses, with mind for
the sixth.

\begin{gitaverse}
शरीरं यदवाप्नोति यच्चाप्युत्क्रामतीश्वरः । \\
गृहीत्वैतानि संयाति वायुर्गन्धानिवाशयात् ॥८॥
\end{gitaverse}

\begin{transliteration}
śarīraṁ yadavāpnoti yaccāpyutkrāmatīśvaraḥ, \\
gṛhītvaitāni saṁyāti vāyurgandhānivāśayāt.
\end{transliteration}

8. When the Lord obtains a body, and when He leaves it, He takes these and goes
(with them) as the wind takes the scents from their seats (the flowers).

\begin{gitaverse}
श्रोत्रं चक्षुः स्पर्शनं च रसनं घ्राणमेव च । \\
अधिष्ठाय मनश्चायं विषयानुपसेवते ॥९॥
\end{gitaverse}

\begin{transliteration}
śrotraṁ cakṣuḥ sparśanaṁ ca rasanaṁ ghrāṇameva ca, \\
adhiṣṭhāya manaścāyaṁ viṣayānupasevate.
\end{transliteration}

9. Presiding over the ear, the eye, the touch, the taste and the smell, so also
the mind. He enjoys the sense-objects.

\begin{gitaverse}
उत्क्रामन्तं स्थितं वापि भुञ्जानं वा गुणान्वितम् । \\
विमूढा नानुपश्यन्ति पश्यन्ति ज्ञानचक्षुषः ॥१०॥
\end{gitaverse}

\begin{transliteration}
utkrāmantaṁ sthitaṁ vāpi bhuñjānaṁ vā guṇānvitam, \\
vimūḍhā nānupaśyanti paśyanti jñānacakṣuṣaḥ.
\end{transliteration}

10. Him, who departs, stays and enjoys, who is united with GUNAS, the deluded
do not see; but they do behold Him, who posses the `eye-of-knowledge'.

\begin{gitaverse}
यतन्तो योगिनश्चैनं पश्यन्त्यात्मन्यवस्थितम् । \\
यतन्तोऽप्यकृतात्मानो नैनं पश्यन्त्यचेतसः ॥११॥
\end{gitaverse}

\begin{transliteration}
yatanto yoginaścainaṁ paśyantyātmanyavasthitam, \\
yatanto'pyakṛtātmāno nainaṁ paśyantyacetasaḥ.
\end{transliteration}

11. The seekers striving (for Perfection) behold Him dwelling in the Self; but,
the unrefined and unintelligent, even though striving, see Him not.

\begin{gitaverse}
यदादित्यगतं तेजो जगद्भासयतेऽखिलम् । \\
यच्चन्द्रमसि यच्चाग्नौ तत्तेजो विद्धि मामकम् ॥१२॥
\end{gitaverse}

\begin{transliteration}
yadādityagataṁ tejo jagadbhāsayate'khilam, \\
yaccandramasi yaccāgnau tattejo viddhi māmakam.
\end{transliteration}

12. That Light which is residing in the Sun and which illumines the whole
world, and that which is in the moon and the fire---know that Light to be Mine.

\begin{gitaverse}
गामाविश्य च भूतानि धारयाम्यहमोजसा । \\
पुष्णामि चौषधीः सर्वाः सोमो भूत्वा रसात्मकः ॥१३॥
\end{gitaverse}

\begin{transliteration}
gāmāviśya ca bhūtāni dhārayāmyahamojasā, \\
puṣṇāmi cauṣadhīḥ sarvāḥ somo bhūtvā rasātmakaḥ.
\end{transliteration}

13. Permeating the earth I support all beings by (My) energy: and having become
the liquid moon, I nourish all herbs.

\begin{gitaverse}
अहं वैश्वानरो भूत्वा प्राणिनां देहमाश्रितः । \\
प्राणापानसमायुक्तः पचाम्यन्नं चतुर्विधम् ॥१४॥
\end{gitaverse}

\begin{transliteration}
ahaṁ vaisvānaro bhūtvā prāṇinām dehamāśritaḥ, \\
prāṇāpānasamāyuktaḥ pacāmyannaṁ caturvidham.
\end{transliteration}

14. Having become (the fire) VAISHVAANARA, I abide in the body of beings, and
associated with PRANA and APANA digest the four-hold food.

\begin{gitaverse}
सर्वस्य चाहं हृदि सन्निविष्टो- \\
\tab मत्तः स्मृतिर्ज्ञानमपोहनं च । \\
वेदैश्च सर्वैरहमेव वेद्यो- \\
\tab वेदान्तकृद्वेदविदेव चाहम् ॥१५॥
\end{gitaverse}

\begin{transliteration}
sarvasya cāhaṁ hṛdi sanniviṣṭo- \\
\tab mattaḥ smṛtirjñānamapohanaṁ ca, \\
vedaiśca sarvairahameva vedyo- \\
\tab vedāntakṛdvedavideva cāham.
\end{transliteration}

15. And I am seated in the heart in the hearts of all, from Me are memory,
knowledge, as well as their absence. I am verily that which has to be known in
all the VEDAS;\@ I am indeed the author of VEDANTA, and, the ``knower of the
VEDAS'' am I.\@

\begin{gitaverse}
द्वाविमौ पुरुषौ लोके क्षरश्चाक्षर एव च । \\
क्षरः सर्वाणि भूतानि कूटस्थोऽक्षर उच्यते ॥१६॥
\end{gitaverse}

\begin{transliteration}
dvāvimau puruṣau loke kṣaraścākṣara eva ca, \\
kṣaraḥ sarvāṇi bhūtāni kūṭastho'kṣara ucyate.
\end{transliteration}

16. Two `PURUSHAS' are there in this world, the Perishable and the
Imperishable. All beings are the Perishable and the `KOOTASTHAH' is called the
Imperishable.

\begin{gitaverse}
उत्तमः पुरुषस्त्वन्यः परमात्मेत्युदाहृतः । \\
यो लोकत्रयमाविश्य बिभर्त्यव्यय ईश्वरः ॥१७॥
\end{gitaverse}

\begin{transliteration}
uttamaḥ puruṣastvanyaḥ paramātmetyudāhṛtaḥ, \\
yo lokatrayamāviśya bibhartyavyaya īśvaraḥ.
\end{transliteration}

17. But distinct is the Supreme PURUSHA called the Highest Self, the
Indestructible Lord, who, pervading the three worlds (waking, dream and
deep-sleep), sustains them.

\begin{gitaverse}
यस्मात्क्षरमतीतोऽहमक्षरादपि चोत्तमः । \\
अतोऽस्मि लोके वेदे च प्रथितः पुरुषोत्तमः ॥१८॥
\end{gitaverse}

\begin{transliteration}
yasmātkṣaramatīto'hamakṣarādapi cottamaḥ, \\
ato'smi loke vede ca prathitaḥ puruṣottamaḥ.
\end{transliteration}

18. As I transcend the perishable and I am even Higher than the Imperishable, I
am declared as the PURUSHOTTAMA (the Highest-PURUSHA) in the world and in the
VEDAS.\@

\begin{gitaverse}
यो मामेवमसम्मूढो जानाति पुरुषोत्तमम् । \\
स सर्वविद्भजति मां सर्वभावेन भारत ॥१९॥
\end{gitaverse}

\begin{transliteration}
yo māmevamasammūdho jānāti puruṣottamam, \\
sa sarvavidbhajati māṁ sarvabhāvena bhārata.
\end{transliteration}

19. He who undeluded, thus knows Me, the Supreme PURUSHA, he, all-knowing,
worships Me with his whole being, O Bharata.

\begin{gitaverse}
इति गुह्यतमं शास्त्रमिदमुक्तं मयानघ । \\
एतद्बुद्ध्वा बुद्धिमान्स्यात्कृतकृत्यश्च भारत ॥२०॥
\end{gitaverse}

\begin{transliteration}
iti guhyatamaṁ śāstramidamuktaṁ mayānagha, \\
etadbuddhvā buddhimānsyātkṛtakṛtyaśca bhārata.
\end{transliteration}

20. Thus, this most secret science (teaching), has been taught by Me, O sinless
one. On knowing this (a man) becomes `wise' and all his duties are
accomplished, O Bharata.

\begin{gitaverse}
ॐ तत्सदिति श्रीमद् भगवद् गीतासूपनिषत्सु ब्रह्मविद्यायां \\
योगशास्त्रे श्रीकृष्णार्जुन संवादे पुरुषोत्तमयोगो नाम \\
पञ्चदशोऽध्यायः
\end{gitaverse}

\begin{transliteration}
om tatsaditi śrīmadbhagavadgītāsūpaniṣatsu brahmavidyāyāṁ \\
yogaśāstre śrīkṛṣṇārjunasaṁvāde puruṣottamayogo nāma \\
pañcadaśo'dhyāyaḥ
\end{transliteration}

Thus, in the UPANISHADS of the glorious Bhagawad-Geeta, in the Science of the
Eternal, in the Scripture of YOGA, in the dialogue between Sri Krishna and
Arjuna, the fifteenth discourse ends entitled: The Yoga of the Supreme Spirit
