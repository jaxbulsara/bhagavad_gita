\chapterdrop

\begin{center}
\headersanskrit{अथ दशमोऽध्यायः}

\headerspace
\headertransliteration{Atha Daśamo'dhyāyaḥ}

\section{Chapter 10}

\headerspace
\headersanskrit{विभूतियोगः}

\headerspace
\headertransliteration{Vibhūti Yogah}

\headerspace
\headertranslation{The Yoga of Divine Glories}

\headerspace
\end{center}

\begin{gitaverse}
श्रीभगवानुवाच \\
भूय एव महाबाहो श्रुणु मे परमं वचः । \\
यत्तेऽहं प्रीयमाणाय वक्ष्यामि हितकाम्यया ॥१॥
\end{gitaverse}

\begin{transliteration}
śrī bhagavānuvāca \\
bhūya eva mahābāho śṛṇu me paramaṁ vacaḥ, \\
yatte'haṁ prīyamāṇāya vakṣyāmi hitakāmyayā.
\end{transliteration}

The Blessed Lord said: \\
1. Again, O mighty-armed, listen to My Supreme word; which I, wishing your
welfare, will declare to you, who delight in hearing me.

\begin{gitaverse}
न मे विदुः सुरगणाः प्रभवं न महर्षयः । \\
अहमादिर्हि देवानां महर्षीणां च सर्वशः ॥२॥
\end{gitaverse}

\begin{transliteration}
na me viduḥ suragaṇāḥ prabhavaṁ na maharṣayaḥ, \\
aham-ādirhi devānāṁ maharṣīṇāṁ ca sarvaśaḥ.
\end{transliteration}

2. Neither the hosts of heaven, nor great RISHIS know My origin; for, in every
way, I am the source of all the DEVAS and the RISHIS.\@

\begin{gitaverse}
यो मामजमनादिं च वेत्ति लोकमहेश्वरम् । \\
असम्मूढः स मर्त्येषु सर्वपापैः प्रमुच्यते ॥३॥
\end{gitaverse}

\begin{transliteration}
yo mām-ajam-anādiṁ ca vetti lokamaheśvaram, \\
asammūḍhaḥ sa martyeṣu sarvapāpaiḥ pramucyate.
\end{transliteration}

3. He who amongst the mortals, knows Me as unborn and beginningless, as the
great Lord of the worlds, is undeluded and is liberated from all sins.

\begin{gitaverse}
बुद्धिर्ज्ञानमसम्मोहः क्षमा सत्यं दमः शमः । \\
सुखं दुःखं भवोऽभावो भयं चाभयमेव च ॥४॥
\end{gitaverse}

\begin{transliteration}
buddhirjñānamasammohaḥ kṣamā satyaṁ damaḥ śamaḥ, \\
sukhaṁ duḥkhaṁ bhavo'bhāvo bhayaṁ cā'bhayameva ca.
\end{transliteration}

4. Intellect, wisdom, non-delusion, forgiveness, truth, selfrestraint,
calmness, happiness, pain, birth or death, fear and also fearlessness,

\begin{gitaverse}
अहिंसा समता तुष्टिस्तपो दानं यशोऽयशः । \\
भवन्ति भावा भूतानां मत्त एव पृथग्विधाः ॥५॥
\end{gitaverse}

\begin{transliteration}
ahiṁsā samatā tuṣṭistapo dānaṁ yaśo'yaśaḥ, \\
bhavanti bhāvā bhūtānāṁ matta eva pṛthagvidhāḥ.
\end{transliteration}

5. Non-injury, equanimity, contentment, austerity, beneficence, fame,
infamy---all these different kinds of `qualities-of-beings' arise from Me
alone.

\begin{gitaverse}
महर्षयः सप्त पूर्वे चत्वारो मनवस्तथा । \\
मद्भावा मानसा जाता येषां लोक इमाः प्रजाः ॥६॥
\end{gitaverse}

\begin{transliteration}
maharṣayaḥ sapta pūrve catvāro manavastathā, \\
madbhāvā mānasā jātā yeṣāṁ loka imāḥ prajāḥ.
\end{transliteration}

6. The seven great RISHIS, the ancient four and also the MANUS, possessed of
powers like Me, were born of (My) mind; from them are these creatures in the
world (originated and sustained).

\begin{gitaverse}
एतां विभूतिं योगं च मम यो वेत्ति तत्त्वतः । \\
सोऽविकम्पेन योगेन युज्यते नात्र संशयः ॥७॥
\end{gitaverse}

\begin{transliteration}
etāṁ vibhūtiṁ yogaṁ ca mama yo vetti tattvataḥ, \\
so'vikampena yogena yujyate nātra saṁśayaḥ.
\end{transliteration}

7. He, who in truth knows these manifold manifestations of My being
(Macrocosm), and (this) YOGA-power of Mine (Microcosm), becomes established in
the `tremorless-YOGA'; there is no doubt about it.

\begin{gitaverse}
अहं सर्वस्य प्रभवो मत्तः सर्वं प्रवर्तते । \\
इति मत्वा भजन्ते मां बुधा भावसमन्विताः ॥८॥
\end{gitaverse}

\begin{transliteration}
ahaṁ sarvasya prabhavo mattaḥ sarvaṁ pravartate, \\
iti matvā bhajante māṁ budhā bhāvasamanvitāḥ.
\end{transliteration}

8. I am the Source of All; from Me everything evolves; understanding thus, the
`wise' endowed with `loving consciousness' worship Me.

\begin{gitaverse}
मच्चित्ता मद्गतप्राणा बोधयन्तः परस्परम् । \\
कथयन्तश्च मां नित्यं तुष्यन्ति च रमन्ति च ॥९॥
\end{gitaverse}

\begin{transliteration}
maccittā madgataprāṇā bodhayantaḥ parasparam, \\
kathayantaśca māṁ nityaṁ tuṣyanti ca ramanti ca.
\end{transliteration}

9. With their minds wholly resting in Me, with their senses absorbed in Me,
enlightening one another, and ever speaking of Me, they are satisfied and
delighted.

\begin{gitaverse}
तेषां सततयुक्तानां भजतां प्रीतिपूर्वकम् । \\
ददामि बुद्धियोगं तं येन मामुपयान्ति ते ॥१०॥
\end{gitaverse}

\begin{transliteration}
teṣāṁ satatayuktānāṁ bhajatāṁ prītipūrvakam, \\
dadāmi buddhiyogaṁ taṁ yena māmupayānti te.
\end{transliteration}

10. To the ever-steadfast, worshipping Me with love, I give the BUDDHI-YOGA, by
which they come to Me.

\begin{gitaverse}
तेषामेवानुकम्पार्थमहमज्ञानजं तमः । \\
नाशयाम्यात्मभावस्थो ज्ञानदीपेन भास्वता ॥११॥
\end{gitaverse}

\begin{transliteration}
teṣām-evānukampārtham-aham-ajñānajaṁ tamaḥ, \\
nāśayāmy-ātmabhāvastho jñānadīpena bhāsvatā.
\end{transliteration}

11. Out of mere compassion for them, I, dwelling within their hearts, destroy
the darkness born of ignorance by the luminous Lamp-of-Knowledge.

\begin{gitaverse}
अर्जुन उवाच \\
परं ब्रह्म परं धाम पवित्रं परमं भवान् । \\
पुरुषं शाश्वतं दिव्यमादिदेवमजं विभुम् ॥१२॥
\end{gitaverse}

\begin{transliteration}
arjuna uvāca \\
paraṁ brahma paraṁ dhāma pavitraṁ paramaṁ bhavān, \\
puruṣaṁ śāśvataṁ divyam-ādidevamajaṁ vibhum.
\end{transliteration}

Arjuna said: \\
12. You are the Supreme BRAHMAN, the Supreme Abode, the Supreme Purifier,
Eternal, Divine PURUSHA, the God of all gods, Unborn, Omnipresent.

\begin{gitaverse}
आहुस्त्वामृषयः सर्वे देवर्षिर्नारदस्तथा । \\
असितो देवलो व्यासः स्वयं चैव ब्रवीषि मे ॥१३॥
\end{gitaverse}

\begin{transliteration}
āhustvām-ṛṣayaḥ sarve devarṣir-nāradastathā, \\
asito devalo vyāsaḥ svayaṁ caiva bravīṣi me.
\end{transliteration}

13. All the RISHIS have thus declared You, as also the DEVARISHI Narada, so
also Asita, Devala and Vyasa; and now the same You Yourself say to me.

\begin{gitaverse}
सर्वमेतदृतं मन्ये यन्मां वदसि केशव । \\
न हि ते भगवन्व्यक्तिं विदुर्देवा न दानवाः ॥१४॥
\end{gitaverse}

\begin{transliteration}
sarvam-etadṛtaṁ manye yanmāṁ vadasi keśava, \\
na hi te bhagavan-vyaktiṁ vidurdevā na dānavāḥ.
\end{transliteration}

14. I believe all this that You say to me as true, O Keshava; verily, O Blessed
Lord, neither the DEVAS nor the DANAVAS know Your manifestation (identity).

\begin{gitaverse}
स्वयमेवात्मनात्मानं वेत्थ त्वं पुरुषोत्तम । \\
भूतभावन भूतेश देवदेव जगत्पते ॥१५॥
\end{gitaverse}

\begin{transliteration}
svayam-evātmanātmānaṁ vettha tvaṁ puruṣottama, \\
bhūtabhāvana bhūteśa devadeva jagatpate.
\end{transliteration}

15. Verily, You Yourself know Yourself by Yourself, O PURUSHOTTAMA (Supreme
PURUSHA), O Source of beings, O Lord of beings, O God of gods, O Ruler of the
world.

\begin{gitaverse}
वक्तुमर्हस्यशेषेण दिव्या ह्यात्मविभूतयः । \\
याभिर्विभूतिभिर्लोकानिमांस्त्वं व्याप्य तिष्ठसि ॥१६॥
\end{gitaverse}

\begin{transliteration}
vaktum-arhasyaśeśeṇa divyā hyātmavibhūtayaḥ, \\
yābhir-vibhūtibhir-lokān-imāṁstvaṁ vyāpya tiṣṭhasi.
\end{transliteration}

16. You should indeed, without reserve, tell me of Your Divine glories by which
You exist pervading all these worlds.

\begin{gitaverse}
कथं विद्यामहं योगिंस्त्वां सदा परिचिन्तयन् । \\
केषु केषु च भावेषु चिन्त्योऽसि भगवन्मया ॥१७॥
\end{gitaverse}

\begin{transliteration}
kathaṁ vidyām-ahaṁ yogiṁstvāṁ sadā paricintayan, \\
keṣu keṣu ca bhāveṣu cintyo'si bhagavan-mayā.
\end{transliteration}

17. How shall I, ever-meditating, know You, O YOGIN?\@ In what aspects or
things, O Blessed Lord, are You to be thought of by me?

\begin{gitaverse}
विस्तरेणात्मनो योगं विभूतिं च जनार्दन । \\
भूयः कथय तृप्तिर्हि शृण्वतो नास्ति मेऽमृतम् ॥१८॥
\end{gitaverse}

\begin{transliteration}
vistareṇātmano yogaṁ vibhūtiṁ ca janārdana, \\
bhūyaḥ kathaya tṛptirhi ṣṛṇvato nāsti me'mṛtam.
\end{transliteration}

18. Tell me again, in detail, O Janardana, of your YOGA-power and Immanent
glory; for I do not feel satisfied by hearing your (life-giving and so)
nectar-like speech.

\begin{gitaverse}
श्रीभगवानुवाच \\
हन्त ते कथयिष्यामि दिव्या ह्यात्मविभूतयः । \\
प्राधान्यतः कुरुश्रेष्ठ नास्त्यन्तो विस्तरस्य मे ॥१९॥
\end{gitaverse}

\begin{transliteration}
śrī bhagavānuvāca \\
hanta te kathayiṣyāmi divyā hyātmavibhūtayaḥ, \\
prādhānyataḥ kuruśreṣṭha nāstyanto vistarasya me.
\end{transliteration}

The Blessed Lord said: \\
19. Alas! Now I will declare to you My Divine glories, immanent, in their
prominence; O best of the Kurus, there is no end to the details of My extent.

\begin{gitaverse}
अहमात्मा गुडाकेश सर्वभूताशयस्थितः । \\
अहमादिश्च मध्यं च भूतानामन्त एव च ॥२०॥
\end{gitaverse}

\begin{transliteration}
aham ātmā guḍākeśa sarvabhūtāśaya-sthitaḥ, \\
ahamādiśca madhyaṁ ca bhūtānāmanta eva ca.
\end{transliteration}

20. I am the Self, O Gudakesha, seated in the hearts of all beings; I am the
Beginning, the Middle and also the End of all beings.

\begin{gitaverse}
आदित्यानामहं विष्णुर्ज्योतिषां रविरंशुमान् । \\
मरीचिर्मरुतामस्मि नक्षत्राणामहं शशी ॥२१॥
\end{gitaverse}

\begin{transliteration}
ādityānām-ahaṁ viṣṇurjyotiṣāṁ raviraṁśumān, \\
marīcirmarutāmasmi nakṣatrāṇām-ahaṁ śaśī.
\end{transliteration}

21. Among the (twelve) ADITYAS I am Vishnu; among luminaries, the radiant Sun;
I am Marichi among the MARUTS;\@ among asterisms, the Moon am I.\@

\begin{gitaverse}
वेदानां सामवेदोऽस्मि देवानामस्मि वासवः । \\
इन्द्रियाणां मनश्चास्मि भूतानामस्मि चेतना ॥२२॥
\end{gitaverse}

\begin{transliteration}
vedānāṁ sāmavedo'smi devānām-asmi vāsavaḥ, \\
indriyāṇāṁ manaścāsmi bhūtānām-asmi cetanā.
\end{transliteration}

22. Among the VEDAS, I am the Sama-VEDA;\@ I am Vasava among the gods; among the
senses I am the mind; and I am the intelligence among living beings.

\begin{gitaverse}
रुद्राणां शङ्करश्चास्मि वित्तेशो यक्षरक्षसाम् । \\
वसूनां पावकश्चास्मि मेरुः शिखरिणामहम् ॥२३॥
\end{gitaverse}

\begin{transliteration}
rudrāṇāṁ śaṅkaraścāsmi vitteśo yakṣarakṣasām, \\
vasūnāṁ pāvakaścāsmi meruḥ śikhariṇām-aham.
\end{transliteration}

23. And among the RUDRAS I am Sankara; among the YAKSHAS and RAKSHASAS I am the
Lord of Wealth (KUBERA); among the VASUS I am Pavaka (AGNI); and among the
mountains I am the MERU.\@

\begin{gitaverse}
पुरोधसां च मुख्यं मां विद्धि पार्थ बृहस्पतिम् । \\
सेनानीनामहं स्कन्दः सरसामस्मि सागरः ॥२४॥
\end{gitaverse}

\begin{transliteration}
purodhasāṁ ca mukhyaṁ māṁ viddhi pārtha bṛhaspatim, \\
senānīnām-ahaṁ skandaḥ sarasām-asmi sāgaraḥ.
\end{transliteration}

24. And among the household priests, O Partha, know Me to be the chief,
Brihaspati; among generals, I am Skanda; among lakes, I am the ocean.

\begin{gitaverse}
महर्षीणां भृगुरहं गिरामस्म्येकमक्षरम् । \\
यज्ञानां जपयज्ञोऽस्मि स्थावराणां हिमालयः ॥२५॥
\end{gitaverse}

\begin{transliteration}
maharṣīṇāṁ bhṛgur-ahaṁ girām-asmyekam-akṣaram, \\
yajñānāṁ japayajño'smi sthāvarāṇāṁ himālayaḥ.
\end{transliteration}

25. Among the great RISHIS I am Bhrigu; among words I am the one-syllabled
`OM'; among sacrifices I am the sacrifice of silent repetition (JAPA-YAJNA);
among immovable things, the Himalayas.

\begin{gitaverse}
अश्वत्थः सर्ववृक्षाणां देवर्षीणां च नारदः । \\
गन्धर्वाणां चित्ररथः सिद्धानां कपिलो मुनिः ॥२६॥
\end{gitaverse}

\begin{transliteration}
aśvatthaḥ sarva-vṛkṣāṇāṁ devarṣīṇāṁ ca nāradaḥ, \\
gandharvāṇāṁ citrarathaḥ siddhānāṁ kapilo muniḥ.
\end{transliteration}

26. Among all trees (I am) the ASHWATTHA-tree; among Divine RISHIS, Narada;
among GANDHARVAS, Chitraratha; among Perfected ones, the MUNI Kapila.

\begin{gitaverse}
उच्चैःश्रवसमश्वानां विद्धि माममृतोद्भवम् । \\
ऐरावतं गजेन्द्राणां नराणां च नराधिपम् ॥२७॥
\end{gitaverse}

\begin{transliteration}
uccaiḥśravasam-aśvānāṁ viddhi mām-amṛtodbhavam, \\
airāvataṁ gajendrāṇāṁ narāṇāṁ ca narādhipam.
\end{transliteration}

27. Know Me among horses as `UCHAISHRAVAS', born of AMRITA;\@ among lordly
elephants, the `AIRAVATA' and among men, the King.

\begin{gitaverse}
आयुधानामहं वज्रं धेनूनामस्मि कामधुक् । \\
प्रजनश्चास्मि कन्दर्पः सर्पाणामस्मि वासुकिः ॥२८॥
\end{gitaverse}

\begin{transliteration}
āyudhānām-ahaṁ vajraṁ dhenūnām-asmi kāmadhuk, \\
prajanaścāsmi kandarpaḥ sarpāṇām-asmi vāsukiḥ.
\end{transliteration}

28. Among weapons, I am the `thunderbolt'; among cows I am `KAMADHUK'; I am
`KANDARPA', the cause for offspring; among serpents I am `VASUKI'.

\begin{gitaverse}
अनन्तश्चास्मि नागानां वरुणो यादसामहम् । \\
पितॄणामर्यमा चास्मि यमः संयमतामहम् ॥२९॥
\end{gitaverse}

\begin{transliteration}
anantaścāsmi nāgānāṁ varuṇo yādasām-aham, \\
pitṝṇām-aryamā cāsmi yamaḥ saṁyamatāmaham.
\end{transliteration}

29. I am `Ananta' among NAGAS;\@ I am `Varuna' among water deities; I am
`Aryama' among the ancestors; and I am `Yama' among controllers.

\begin{gitaverse}
प्रह्लादश्चास्मि दैत्यानां कालः कलयतामहम् । \\
मृगाणां च मृगेन्द्रोऽहं वैनतेयश्च पक्षिणाम् ॥३०॥
\end{gitaverse}

\begin{transliteration}
prahlādaścāsmi daityanaṁ kālaḥ kalayatām-aham, \\
mṛgāṇaṁ ca mṛgendro'haṁ vainateyaśca pakṣiṇām.
\end{transliteration}

30. I am `Prahlada' among DAITYAS, `Time' among reckoners, the `Lord-of-beasts'
(Lion) among animals, and `Vainateya' (Garuda) among birds.

\begin{gitaverse}
पवनः पवतामस्मि रामः शस्त्रभृतामहम् । \\
झषाणां मकरश्चास्मि स्रोतसामस्मि जाह्नवी ॥३१॥
\end{gitaverse}

\begin{transliteration}
pavanaḥ pavatām-asmi rāmaḥ śastrabhṛtām-aham, \\
jhaṣāṇāṁ makaraścāsmi srotasām-asmi jāhnavī.
\end{transliteration}

31. Among purifiers, I am the `wind'; among warriors, I am `Rama', among
fishes, I am the `shark'; among rivers, I am the `Ganges'.

\begin{gitaverse}
सर्गाणामादिरन्तश्च मध्यं चैवाहमर्जुन । \\
अध्यात्मविद्या विद्यानां वादः प्रवदतामहम् ॥३२॥
\end{gitaverse}

\begin{transliteration}
sargāṇām-ādirantaśca madhyaṁ caivāham-arjuna, \\
adhyātmavidyā vidyānāṁ vādaḥ pravadatām-aham.
\end{transliteration}

32. Among creations, I am the beginning, the middle and also the end, O Arjuna;
among sciences I am the Science of the Self and I am the logic in all
arguments.

\begin{gitaverse}
अक्षराणामकारोऽस्मि द्वन्द्वः सामासिकस्य च । \\
अहमेवाक्षयः कालो धाताहं विश्वतोमुखः ॥३३॥
\end{gitaverse}

\begin{transliteration}
akṣarāṇām'akārosmi dvandvaḥ sāmāsikasya ca, \\
aham-evāksayaḥ kālo dhātāham viśvatomukhaḥ.
\end{transliteration}

33. Among letters I am the letter `A'; among all compounds I am the dual
(co-ordinates); I am verily, the inexhaustible, or the everlasting time; I am
the (All-faced) dispenser (of fruits-ofactions) having faces in all directions.

\begin{gitaverse}
मृत्युः सर्वहरश्चाहमुद्भवश्च भविष्यताम् । \\
कीर्तिः श्रीर्वाक्च नारीणां स्मृतिर्मेधा धृतिः क्षमा ॥३४॥
\end{gitaverse}

\begin{transliteration}
mṛtyuḥ sarvaharaścāham-udbhavaśca bhaviṣyatām, \\
kīrtiḥ śrīrvākca nārīṇāṁ smṛtirmedhā dhṛtiḥ kṣamā.
\end{transliteration}

34. And I am all-devouring Death, and the prosperity of those who are to be
prosperous; among the feminine qualities (I am) fame, prosperity, speech,
memory, intelligence, firmness and forgiveness.

\begin{gitaverse}
बृहत्साम तथा साम्नां गायत्री छन्दसामहम् । \\
मासानां मार्गशीर्षोऽहमृतूनां कुसुमाकरः ॥३५॥
\end{gitaverse}

\begin{transliteration}
bṛhatsāma tathā sāmnāṁ gāyatrī chandasām-aham, \\
māsānāṁ mārgaśīrṣo'ham-ṛtunāṁ kusumākaraḥ.
\end{transliteration}

35. Among hymns also I am the `BRIHAT-SAMAN'; among metres `GAYATRI' am I;\@
among months I am parts of December-January (MARGA-SHIRSHA); among seasons I am
the `flowery-spring'.

\begin{gitaverse}
द्यूतं छलयतामस्मि तेजस्तेजस्विनामहम् । \\
जयोऽस्मि व्यवसायोऽस्मि सत्त्वं सत्त्ववतामहम् ॥३६॥
\end{gitaverse}

\begin{transliteration}
dyūtaṁ chalayatāmasmi tejas-tejasvinām-aham, \\
jayo'smi vyavasāyo'smi sattvaṁ sattvavatām-aham.
\end{transliteration}

36. I am the gambling of the fraudulent; I am the splendour of the splendid; I
am victory; I am the industry (in those who are determined); I am the goodness
in the good.

\begin{gitaverse}
वृष्णीनां वासुदेवोऽस्मि पाण्डवानां धनञ्जयः । \\
मुनीनामप्यहं व्यासः कवीनामुशना कविः ॥३७॥
\end{gitaverse}

\begin{transliteration}
vṛṣṇīnāṁ vāsudevo'smi pāṇḍavānāṁ dhanañjayaḥ, \\
munīnām-apyahaṁ vyāsaḥ kavīnām-uśanā kaviḥ.
\end{transliteration}

37. Among the VRISHNIS I am `Vaasudeva'; among the PANDAVAS, (I am)
`Dhananjaya'; also among the MUNIS I am `Vyasa'; and among the poets I am
`Ushana', the great Seer.

\begin{gitaverse}
दण्डो दमयतामस्मि नीतिरस्मि जिगीषताम् । \\
मौनं चैवास्मि गुह्यानां ज्ञानं ज्ञानवतामहम् ॥३८॥
\end{gitaverse}

\begin{transliteration}
daṇḍo damayatām-asmi nītirasmi jigīṣatām, \\
maunaṁ caivā'smi guhyānāṁ jñānaṁ jñānavatām-aham.
\end{transliteration}

38. Among punishers I am the `Sceptre'; among those who seek victory, I am
`Statesmanship'; and also among secrets, I am `Silence'; and I am the
`Knowledge' among knowers.

\begin{gitaverse}
यच्चापि सर्वभूतानां बीजं तदहमर्जुन । \\
न तदस्ति विना यत्स्यान्मया भूतं चराचरम् ॥३९॥
\end{gitaverse}

\begin{transliteration}
yaccāpi sarvabhūtānāṁ bījaṁ tad-aham-arjuna, \\
na tadasti vinā yatsyānmayā bhūtaṁ carācaram.
\end{transliteration}

39. And whatsoever is the seed of all beings, that also am I, O Arjuna; there
is no being, whether moving, or unmoving, that can exist without Me.

\begin{gitaverse}
नान्तोऽस्ति मम दिव्यानां विभूतीनां परन्तप । \\
एष तूद्देशतः प्रोक्तो विभूतेर्विस्तरो मया ॥४०॥
\end{gitaverse}

\begin{transliteration}
nānto'sti mama divyānāṁ vibhūtīnāṁ parantapa, \\
eṣa tūddeśataḥ prokto vibhūtervistaro mayā.
\end{transliteration}

40. There is no end to My Divine Glories, O Parantapa; but, this is but a brief
statement by Me of the particulars of My Divine Glories.

\begin{gitaverse}
यद्यद्विभूतिमत्सत्त्वं श्रीमदूर्जितमेव वा । \\
तत्तदेवावगच्छ त्वं मम तेजोंऽशसम्भवम् ॥४१॥
\end{gitaverse}

\begin{transliteration}
yadyad-vibhūtimat-sattvaṁ śrīmad-ūrjitameva vā, \\
tattad-evāvagaccha tvaṁ mama tejoṁ'śasaṁbhavam.
\end{transliteration}

41. Whatever it is that is glorious, prosperous or powerful in any being, know
that to be a manifestation of a part of My splendour.

\begin{gitaverse}
अथवा बहुनैतेन किं ज्ञातेन तवार्जुन । \\
विष्टभ्याहमिदं कृत्स्नमेकांशेन स्थितो जगत् ॥४२॥
\end{gitaverse}

\begin{transliteration}
athavā bahunaitena kim jñātena tavārjuna, \\
viṣṭabhyāham-idaṁ kṛṣṇam-ekāṁśena sthito jagat.
\end{transliteration}

42. But, of what avail to thee is the knowledge of all these details, O Arjuna?
I exist, supporting this whole world by one part of Myself.

\begin{gitaverse}
ॐ तत्सदिति श्रीमद् भगवद् गीतासूपनिषत्सु ब्रह्मविद्यायां \\
योगशास्त्रे श्रीकृष्णार्जुन संवादे विभूतियोगो नाम \\
दशमोऽध्यायः
\end{gitaverse}

\begin{transliteration}
oṁ tatsaditi śrīmad bhagavad gītāsūpaniṣatsu brahmavidyāyāṁ \\
yogaśāstre śrīkṛṣṇārjunasaṁvāde vibhūtiyogo nāma \\
daśamo'dhyāyaḥ.
\end{transliteration}

Thus, in the UPANISHADS of the glorious Bhagawad-Geeta, in the Science of the
Eternal, in the scripture of YOGA, in the dialogue between Sri Krishna and
Arjuna, the tenth discourse ends entitled. The Yoga of Divine Glories
