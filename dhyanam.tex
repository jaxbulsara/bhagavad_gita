\chapterdrop

\begin{center}
\section{\large \textbf{Gita Dhyanam}}

\headerspace
\textit{(Invocatory Meditation on the Gita)}

\headerspace
\end{center}

\begin{gitaverse}
ॐ पार्थाय प्रतिबोधितां भगवता नारायणेन स्वयं \\
\tab व्यासेन ग्रथितां पुराणमुनिना मध्ये महाभारते । \\
अद्वैतामृतवर्षिणीं भगवतीमष्टादशाध्यायिनीं \\
\tab अम्ब त्वामनुसन्दधामि भगवद्गीते भवद्वेषिणीम् ॥१॥
\end{gitaverse}

\begin{transliteration}
oṃ pārthāya pratibodhitāṃ bhagavatā nārāyaṇena svayaṃ \\
\tab vyāsena grathitāṃ purāṇamuninā madhye mahābhārate, \\
advaitāmṛtavarṣiṇīṃ bhagavatīmaṣṭādaśādhyāyinīṃ \\
\tab amba tvāmanusandadhāmi bhagavadgīte bhavadveṣiṇīm.
\end{transliteration}

1. Om. O Mother Bhagavad Gita, with which Partha (Arjuna) was illumined by Lord
Narayana Himself and which was composed within the Mahabharata by the ancient
Sage Vyasa, the showerer of the nectar of Non-duality, the destroyer of the
world-process, consisting of eighteen chapters---upon Thee O Mother Bhagavad
Gita! I meditate!

\begin{gitaverse}
नमोऽस्तु ते व्यास विशालबुद्धे फुल्लारविन्दायातपत्रनेत्रे । \\
येन त्वया भारततैलपूर्णः प्रज्वलितो ज्ञानमयः प्रदीपः ॥२॥
\end{gitaverse}

\begin{transliteration}
namo'stu te vyāsa viśālabuddhe phullāravindāyātapatranetre, \\
yena tvayā bhāratatailapūrṇaḥ prajvalito jñānamayaḥ pradīpaḥ.
\end{transliteration}

2. Salutations unto thee, O Vyasa of broad intellect, whose eyes are like the
petals of full blown lotuses, by whom was lighted the lamp of wisdom, filled
with the oil of the Mahabharata.

\begin{gitaverse}
प्रपन्नपरिजाताय तोत्रवेत्रैकपाणये । \\
ज्ञानमुद्राय कृष्णाय गीतामृतदुहे नमः ॥३॥
\end{gitaverse}

\begin{transliteration}
prapannaparijātāya totravetraikapāṇaye, \\
jñānamudrāya kṛṣṇāya gītāmṛtaduhe namaḥ.
\end{transliteration}

3. Salutations to Krishna, who is the Parijata or the Tree of fulfillment to
all those who take refuge in Him, who holds the whip (symbol of power) in one
hand and Jnanamudra (symbol of knowledge) on the other, who is the milker of
the Gita Nectar.

\begin{gitaverse}
सर्वोपनिषदो गावो दोग्धा गोपालनन्दनः । \\
पार्थो वत्सः सुधीर्भोक्ता दुग्धं गीतामृतं महत् ॥४॥
\end{gitaverse}

\begin{transliteration}
sarvopaniṣado gāvo dogdhā gopālanandanaḥ \\
pārtho vatsaḥ sudhīrbhoktā dugdhaṃ gītāmṛtaṃ mahat
\end{transliteration}

4. All the Upanishads are the cows, while Krishna---the cowherd boy is their
milker, Arjuna is the calf, the men of purified intellect are the drinkers of
the milk, and the milk is the great nectar of the Gita.

\begin{gitaverse}
वसुदेवसुतं देवं कंसचाणूरमर्दनम् । \\
देवकीपरमानन्दं कृष्णं वन्दे जगद्गुरुम् ॥५॥
\end{gitaverse}

\begin{transliteration}
vasudevasutaṃ devaṃ kaṃsacāṇūramardanam, \\
devakīparamānandaṃ kṛṣṇaṃ vande jagadgurum.
\end{transliteration}

5. I offer adorations to Lord Krishna, the preceptor of the universe, the son
of Vasudeva, the destroyer of Kamsa and Chanura (the forces of darkness), the
supreme bliss of Devaki.

\begin{gitaverse}
भीष्मद्रोणतटा जयद्रथजला गान्धारीनीलोत्पला \\
\tab शल्यग्राहवती कृपेण वहिनी कर्णेन वेलाकुल । \\
अश्वत्तामविकर्णघोरमकरा दुर्योधनावर्तिनी \\
\tab सोत्तीर्णा खलु पाण्डवार्णवनदी कैवर्तकः केश्वः ॥६॥
\end{gitaverse}

\begin{transliteration}
bhīṣmadroṇataṭā jayadrathajalā gāndhārīnīlotpalā \\
\tab śalyagrāhavatī kṛpeṇa vahinī karṇena velākula, \\
aśvattāmavikarṇaghoramakarā duryodhanāvartinī \\
\tab sottīrṇā khalu pāṇḍavārṇavanadī kaivartakaḥ keśvaḥ.
\end{transliteration}

6. With Lord Krishna as the ferry-man, indeed was crossed over by the Pandavas
the battle-river whose banks were Bhishma and Drona, whose water was
Jayadratha, whose blue-lotus was the King of Gandhara, whose shark was Shalya,
whose current was Kripacharya, whose billow was Karna, whose terrible
crocodiles were Ashwatthama and Vikarna, whose whirlpool was Duryodhana.

\begin{gitaverse}
पाराशर्यवचः सरोजममलं गीतार्थगन्धोत्कटं \\
\tab नानाख्यानककेशरं हरिकथासम्बोधनाबोधितम् । \\
लोके सज्जनषट्पदैरहरहः पेपीयमानं मुदा \\
\tab भूयाद्भारतपङ्कजं कलिमलप्रध्वंसनं श्रेयसे ॥७॥
\end{gitaverse}

\begin{transliteration}
pārāśaryavacaḥ sarojamamalaṃ gītārthagandhotkaṭaṃ \\
\tab nānākhyānakakeśaraṃ harikathāsambodhanābodhitam, \\
loke sajjanaṣaṭpadairaharahaḥ pepīyamānaṃ mudā \\
\tab bhūyādbhāratapaṅkajaṃ kalimalapradhvaṃsanaṃ śreyase.
\end{transliteration}

7. May this spotless lotus of the Mahabharata, born in the lake of the words of
Sri Parashara's son (Sage Vyasa), sweet with the fragrance of the import of the
Gita, with many narratives as its filaments, fully opened by the discourses on
Hari, the destroyer of the sins of Kali Yuga and drunk joyously day after day
by the bees of good and pure men, become the bestower of good to us!

\begin{gitaverse}
मूकं करोति वाचालं पङ्गुं लङ्घायते गिरिम् । \\
यत्कृपा तमहं वन्दे परमानन्दमाधवम् ॥८॥
\end{gitaverse}

\begin{transliteration}
mūkaṃ karoti vācālaṃ paṅguṃ laṅghāyate girim, \\
yatkṛpā tamahaṃ vande paramānandamādhavam.
\end{transliteration}

8. I salute that Madhava (Krishna), the embodiment of Supreme Bliss, whose
grace turns the mute eloquent, and makes the lame scale mountains.

\begin{gitaverse}
यं ब्रह्मा वरुणेन्द्ररुद्रमरुतः स्तुन्वन्ति दिव्यैः स्तवैर् \\
\tab वेदैः साङ्गपदक्रमोपनिषदैर्गायन्ति यं सामगाः । \\
ध्यानावस्थिततद्गतेन मनसा पश्यन्ति यं योगिनो \\
\tab यस्यान्तं न विदुः सुरासुरगणा देवाय तस्मै नमः ॥९॥
\end{gitaverse}

\begin{transliteration}
yaṃ brahmā varuṇendrarudramarutaḥ stunvanti divyaiḥ stavair \\
\tab vedaiḥ sāṅgapadakramopaniṣadairgāyanti yaṃ sāmagāḥ, \\
dhyānāvasthitatadgatena manasā paśyanti yaṃ yogino \\
\tab yasyāntaṃ na viduḥ surāsuragaṇā devāya tasmai namaḥ.
\end{transliteration}

9. Adorations to that God whom Brahma, Varuna, Indra, Rudra and the Maruts
praise with divine hymns, whom the singers of the Sama hymns invoke by the
Vedas and their branches, in the Pada and Krama methods, and by the Upanishads,
whom the Yogis behold with their minds absorbed in Him through meditation, and
whose limit the hosts of Devas and Asuras know not!
