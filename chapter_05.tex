\chapterdrop

\begin{center}
\headersanskrit{पञ्चमोऽध्यायः}

\headerspace
\headertransliteration{Atha Pañchamo'dhyāyaḥ}

\section{Chapter 5}

\headerspace
\headersanskrit{कर्मसन्न्यासयोगः}

\headerspace
\headertransliteration{Karma Sannyāsa Yogah}

\headerspace
\headertranslation{Yoga of True Renunciation}

\headerspace
\end{center}

\begin{gitaverse}
अर्जुन उवाच \\
सन्न्यासं कर्मणां कृष्ण पुनर्योगं च शंससि । \\
यच्छ्रेय एतयोरेकं तन्मे ब्रूहि सुनिश्चितम् ॥१॥
\end{gitaverse}

\begin{transliteration}
Arjuna uvāca \\
sannyāsaṁ karmaṇāṁ kṛṣṇa punaryogaṁ ca śaṁsasi, \\
yacchreya etayorekaṁ tanme brūhi suniścitam.
\end{transliteration}

Arjuna said: \\
1. Renunciation-of-actions, O Krishna, You praise and, again YOGA---performance
of actions. Tell me conclusively that which is the better of the two.

\begin{gitaverse}
श्रीभगवानुवाच \\
सन्न्यासः कर्मयोगश्च निःश्रेयसकरावुभौ । \\
तयोस्तु कर्मसन्न्यासात्कर्मयोगो विशिष्यते ॥२॥
\end{gitaverse}

\begin{transliteration}
śrī bhagavānuvāca \\
sannyāsaḥ karmayogaśca niḥśreyasakarāvubhau, \\
tayostu karmasannyāsāt-karmayogo viśiṣyate.
\end{transliteration}

The Blessed Lord said: \\
2. Renunciation-of-action and YOGA-of-action both lead to the highest bliss;
but of the two, YOGA-of-action is superior to the renunciation-of-action.

\begin{gitaverse}
ज्ञेयः स नित्यसन्न्यासी यो न द्वेष्टि न काङ्क्षति । \\
निर्द्वन्द्वो हि महाबाहो सुखं बन्धात्प्रमुच्यते ॥३॥
\end{gitaverse}

\begin{transliteration}
jñeyaḥ sa nityasannyāsī yo na dveṣṭi na kāṅkṣati, \\
nirdvandvo hi mahābāho sukhaṁ bandhātpramucyate.
\end{transliteration}

3. He should be known as a perpetual SANNYASI who neither hates nor desires;
for, free from the pairs-of-opposites, O Mighty-armed, he is easily set free
from bondage.

\begin{gitaverse}
साङ्ख्ययोगौ पृथग्बालाः प्रवदन्ति न पण्डिताः । \\
एकमप्यास्थितः सम्यगुभयोर्विन्दते फलम् ॥४॥
\end{gitaverse}

\begin{transliteration}
sāṅkhyayogau pṛthagbālāḥ pravadanti na paṇḍitāḥ, \\
ekamapyāsthitaḥ samyagubhayorvindate phalam.
\end{transliteration}

4. Children, not the wise, speak of SANKHYA (Knowledge) and YOGA
(YOGA-of-action) as distinct; he who is truly established even in one, obtains
the fruits of both.

\begin{gitaverse}
यत्साङ्ख्यैः प्राप्यते स्थानं तद्योगैरपि गम्यते । \\
एकं साङ्ख्यं च योगं च यः पश्यति स पश्यति ॥५॥
\end{gitaverse}

\begin{transliteration}
yatsāṅkhyaiḥ prāpyate sthānaṁ tadyogairapi gamyate, \\
ekaṁ sāṅkhyaṁ ca yogaṁ ca yaḥ paśyati sa paśyati.
\end{transliteration}

5. That place which is reached by the SANKHYAS (JNANIS) is also reached by the
YOGINS (KARMA-YOGINS). He `sees' who `sees' SANKHYA and YOGA as one.

\begin{gitaverse}
सन्न्यासस्तु महाबाहो दुःखमाप्तुमयोगतः । \\
योगयुक्तो मुनिर्ब्रह्म नचिरेणाधिगच्छति ॥६॥
\end{gitaverse}

\begin{transliteration}
sannyāsastu mahābāho duḥkhamāptumayogataḥ, \\
yogayukto munirbrahma nacireṇādhigacchati.
\end{transliteration}

6. But renunciation, O mighty-armed, is hard to attain without YOGA;\@ the
YOGA-harmonised man of (steady) contemplation quickly goes to BRAHMAN.\@

\begin{gitaverse}
योगयुक्तो विशुद्धात्मा विजितात्मा जितेन्द्रियः । \\
सर्वभूतात्मभूतात्मा कुर्वन्नपि न लिप्यते ॥७॥
\end{gitaverse}

\begin{transliteration}
yogayukto viśuddhātmā vijitātmā jitendriyaḥ, \\
sarvabhūtātmabhūtātmā kurvannapi na lipyate.
\end{transliteration}

7. He who is devoted to the Path-of-action, whose mind is quite pure, who has
conquered the self, who has subdued his senses, who realises his Self as the
Self in all beings, though acting, is not tainted.

\begin{gitaverse}
नैव किञ्चित्करोमीति युक्तो मन्येत तत्त्ववित् । \\
पश्यञ्शृण्वन्स्पृशञ्जिघ्रन्नश्नन्गच्छन्स्वपञ्श्वसन् ॥८॥
\end{gitaverse}

\begin{transliteration}
naiva kiñcitkaromīti yukto manyeta tattvavit, \\
paśyañśṛṇvanspṛśañjighrannaśnangacchansvapañśvasan.
\end{transliteration}

8. ``I do nothing at all'', thus would the harmonised knower of Truth think
seeing, hearing, touching, smelling, eating, going, sleeping, breathing,

\begin{gitaverse}
प्रलपन्विसृजन्गृह्णन्नुन्मिषन्निमिषन्नपि । \\
इन्द्रियाणीन्द्रियार्थेषु वर्तन्त इति धारयन् ॥९॥
\end{gitaverse}

\begin{transliteration}
pralapanvisṛjangṛhṇannunmiṣannimiṣannapi, \\
indriyāṇīndriyārtheṣu vartanta iti dhārayan.
\end{transliteration}

9. Speaking, letting go, seizing, opening and closing the eyes; convinced that
the senses move among the sense-objects.

\begin{gitaverse}
ब्रह्मण्याधाय कर्माणि सङ्गं त्यक्त्वा करोति यः । \\
लिप्यते न स पापेन पद्मपत्रमिवाम्भसा ॥१०॥
\end{gitaverse}

\begin{transliteration}
brahmaṇyādhāya karmāṇi saṅgaṁ tyaktvā karoti yaḥ, \\
lipyate na sa pāpena padmapatramivāmbhasā.
\end{transliteration}

10. He who does actions, offering them to BRAHMAN, abandoning attachment, is
not tainted by sin, just as a lotus leaf remains unaffected by the water on it.

\begin{gitaverse}
कायेन मनसा बुद्ध्या केवलैरिन्द्रियैरपि । \\
योगिनः कर्म कुर्वन्ति सङ्गं त्यक्त्वात्मशुद्धये ॥११॥
\end{gitaverse}

\begin{transliteration}
kāyena manasā buddhyā kevalairindriyairapi, \\
yoginaḥ karma kurvanti saṅgaṁ tyaktvātmaśuddhaye.
\end{transliteration}

11. YOGIS, having abandoned attachment, perform actions merely by the body,
mind, intellect and senses, for the purification of the self (ego).

\begin{gitaverse}
युक्तः कर्मफलं त्यक्त्वा शान्तिमाप्नोति नैष्ठिकीम् । \\
अयुक्तः कामकारेण फले सक्तो निबध्यते ॥१२॥
\end{gitaverse}

\begin{transliteration}
yuktaḥ karmaphalaṁ tyaktvā śāntimāpnoti naiṣṭhikīm, \\
ayuktaḥ kāmakāreṇa phale sakto nibadhyate.
\end{transliteration}

12. The united one (the well-poised or the harmonised), having abandoned the
fruit of action, attains Eternal Peace; the nonunited (the unsteady or the
unbalanced), impelled by desire and attached to the fruit, is bound.

\begin{gitaverse}
सर्वकर्माणि मनसा सन्न्यस्यास्ते सुखं वशी । \\
नवद्वारे पुरे देही नैव कुर्वन्न कारयन् ॥१३॥
\end{gitaverse}

\begin{transliteration}
sarvakarmāṇi manasā sannyasyāste sukhaṁ vaśī, \\
navadvāre pure dehī naiva kurvanna kārayan.
\end{transliteration}

13. Mentally renouncing all actions and fully self-controlled, the `embodied'
one rests happily in the nine-gate city, neither acting nor causing others
(body and senses) to act.

\begin{gitaverse}
न कर्तृत्वं न कर्माणि लोकस्य सृजति प्रभुः । \\
न कर्मफलसंयोगं स्वभावस्तु प्रवर्तते ॥१४॥
\end{gitaverse}

\begin{transliteration}
na kartṛtvaṁ na karmāṇi lokasya sṛjati prabhuḥ, \\
na karmaphalasaṁyogaṁ svabhāvastu pravartate.
\end{transliteration}

14. Neither agency nor actions does the Lord create for the world, nor union
with the fruits of actions. But it is Nature that acts.

\begin{gitaverse}
नादत्ते कस्यचित्पापं न चैव सुकृतं विभुः । \\
अज्ञानेनावृतं ज्ञानं तेन मुह्यन्ति जन्तवः ॥१५॥
\end{gitaverse}

\begin{transliteration}
nādatte kasyacitpāpaṁ na caiva sukṛtaṁ vibhuḥ, \\
ajñānenāvṛtaṁ jñānaṁ tena muhyanti jantavaḥ.
\end{transliteration}

15. The Lord takes neither the demerit nor even the merit of any; knowledge is
enveloped by ignorance, thereby beings are deluded.

\begin{gitaverse}
ज्ञानेन तु तदज्ञानं येषां नाशितमात्मनः । \\
तेषामादित्यवज्ज्ञानं प्रकाशयति तत्परम् ॥१६॥
\end{gitaverse}

\begin{transliteration}
jñānena tu tadajñānaṁ yeṣāṁ nāśitamātmanaḥ, \\
teṣāmādityavajjñānaṁ prakāśayati tatparam.
\end{transliteration}

16. But to those whose ignorance is destroyed by the Knowledge of the Self,
like the sun, to them Knowledge reveals the Supreme (BRAHMAN).

\begin{gitaverse}
तद्बुद्धयस्तदात्मानस्तन्निष्ठास्तत्परायणाः । \\
गच्छन्त्यपुनरावृत्तिं ज्ञाननिर्धूतकल्मषाः ॥१७॥
\end{gitaverse}

\begin{transliteration}
tadbuddhayastadātmānastanniṣṭhāstatparāyaṇāḥ, \\
gacchantyapunarāvṛttiṁ jñānanirdhūtakalmaṣāḥ.
\end{transliteration}

17. Intellect absorbed in That, their Self being That, established in That,
with That for their Supreme Goal, they go whence there is no return, their sins
dispelled by Knowledge.

\begin{gitaverse}
विद्याविनयसम्पन्ने ब्राह्मणे गवि हस्तिनि । \\
शुनि चैव श्वपाके च पण्डिताः समदर्शिनः ॥१८॥
\end{gitaverse}

\begin{transliteration}
vidyāvinayasampanne brāhmaṇe gavi hastini, \\
śuni caiva śvapāke ca paṇḍitāḥ samadarśinaḥ.
\end{transliteration}

18. Sages look with an equal eye upon a BRAHMANA endowed with learning and
humility, on a cow, on an elephant, and even on a dog and an outcaste.

\begin{gitaverse}
इहैव तैर्जितः सर्गो येषां साम्ये स्थितं मनः । \\
निर्दोषं हि समं ब्रह्म तस्माद्ब्रह्मणि ते स्थिताः ॥१९॥
\end{gitaverse}

\begin{transliteration}
ihaiva tairjitaḥ sargo yeṣāṁ sāmye sthitaṁ manaḥ, \\
nirdoṣaṁ hi samaṁ brahma tasmādbrahmaṇi te sthitāḥ.
\end{transliteration}

19. Even here (in this world), birth (everything) is overcome by those whose
minds rest in equality; BRAHMAN is spotless indeed and equal; therefore they
are established in BRAHMAN.\@

\begin{gitaverse}
न प्रहृष्येत्प्रियं प्राप्य नोद्विजेत्प्राप्य चाप्रियम् । \\
स्थिरबुद्धिरसम्मूढो ब्रह्मविद् ब्रह्मणि स्थितः ॥२०॥
\end{gitaverse}

\begin{transliteration}
na prahṛṣyetpriyaṁ prāpya nodvijetprāpya cāpriyam, \\
sthirabuddhirasammūḍho brahmavid brahmaṇi sthitaḥ.
\end{transliteration}

20. Resting in BRAHMAN, with steady intellect and undeluded, the knower of
BRAHMAN neither rejoices on obtaining what is pleasant, nor grieves on
obtaining what is unpleasant.

\begin{gitaverse}
बाह्यस्पर्शेष्वसक्तात्मा विन्दत्यात्मनि यत्सुखम् । \\
स ब्रह्मयोगयुक्तात्मा सुखमक्षयमश्नुते ॥२१॥
\end{gitaverse}

\begin{transliteration}
bāhyasparśeṣvasaktātmā vindatyātmani yatsukham, \\
sa brahmayogayuktātmā sukhamakṣayamaśnute.
\end{transliteration}

21. With the self unattached to external contacts, he finds happiness in the
Self; with the self engaged in the editation of BRAHMAN, he attains endless
happiness.

\begin{gitaverse}
ये हि संस्पर्शजा भोगा दुःखयोनय एव ते । \\
आद्यन्तवन्तः कौन्तेय न तेषु रमते बुधः ॥२२॥
\end{gitaverse}

\begin{transliteration}
ye hi saṁsparśajā bhogā duḥkhayonaya eva te, \\
ādyantavantaḥ kaunteya na teṣu ramate budhaḥ.
\end{transliteration}

22. The enjoyments that are born of contacts are only generators of pain, for
they have a beginning and an end. O son of Kunti, the wise do not rejoice in
them.

\begin{gitaverse}
शक्नोतीहैव यः सोढुं प्राक्शरीरविमोक्षणात् । \\
कामक्रोधोद्भवं वेगं स युक्तः स सुखी नरः ॥२३॥
\end{gitaverse}

\begin{transliteration}
śaknotīhaiva yaḥ soḍhuṁ prākśarīravimokṣaṇāt, \\
kāmakrodhodbhavaṁ vegaṁ sa yuktaḥ sa sukhī naraḥ.
\end{transliteration}

23. He who is able, while still here (in this world) to withstand, before the
liberation from the body (death), the impulse born out of desire and anger, he
is a YOGI, he is a happy man.

\begin{gitaverse}
योऽन्तःसुखोऽन्तरारामस्तथान्तर्ज्योतिरेव यः । \\
स योगी ब्रह्मनिर्वाणं ब्रह्मभूतोऽधिगच्छति ॥२४॥
\end{gitaverse}

\begin{transliteration}
yo'ntaḥsukho'ntarārāmastathāntarjyotireva yaḥ, \\
sa yogī brahmanirvāṇaṁ brahmabhūto'dhigacchati.
\end{transliteration}

24. He who is happy within, who rejoices within, who is illuminated within,
that YOGI attains Absolute Freedom or MOKSHA, himself becoming BRAHMAN.\@

\begin{gitaverse}
लभन्ते ब्रह्मनिर्वाणमृषयः क्षीणकल्मषाः । \\
छिन्नद्वैधा यतात्मानः सर्वभूतहिते रताः ॥२५॥
\end{gitaverse}

\begin{transliteration}
labhante brahmanirvāṇamṛṣayaḥ kṣīṇakalmaṣāḥ, \\
chinnadvaidhā yatātmānaḥ sarvabhūtahite ratāḥ.
\end{transliteration}

25. Those RISHIS obtain Absolute Freedom or MOKSHA---whose sins have been
destroyed, whose dualities are torn asunder, who are self-controlled and intent
on the welfare of all beings.

\begin{gitaverse}
कामक्रोधवियुक्तानां यतीनां यतचेतसाम् । \\
अभितो ब्रह्मनिर्वाणं वर्तते विदितात्मनाम् ॥२६॥
\end{gitaverse}

\begin{transliteration}
kāmakrodhaviyuktānāṁ yatīnāṁ yatacetasām, \\
abhito brahmanirvāṇaṁ vartate viditātmanām.
\end{transliteration}

26. Absolute Freedom (or BRAHMIC Bliss) exists on all sides for those
self-controlled ascetics, who are free from desire and anger, who have
controlled their thoughts and who have realised the Self.

\begin{gitaverse}
स्पर्शान्कृत्वा बहिर्बाह्यांश्चक्षुश्चैवान्तरे भ्रुवोः । \\
प्राणापानौ समौ कृत्वा नासाभ्यन्तरचारिणौ ॥२७॥
\end{gitaverse}

\begin{transliteration}
sparśānkṛtvā bahirbāhyāṁścakṣuścaivāntare bhruvoḥ, \\
prāṇāpānau samau kṛtvā nāsābhyantaracāriṇau.
\end{transliteration}

27. Shutting out (all) external contacts and fixing the gaze (as though)
between the eyebrows, equalising the outgoing and incoming breath moving within
the nostrils,

\begin{gitaverse}
यतेन्द्रियमनोबुद्धिर्मुनिर्मोक्षपरायणः । \\
विगतेच्छाभयक्रोधो यः सदा मुक्त एव सः ॥२८॥
\end{gitaverse}

\begin{transliteration}
yatendriyamanobuddhirmunirmokṣaparāyaṇaḥ, \\
vigatecchābhayakrodho yaḥ sadā mukta eva saḥ.
\end{transliteration}

28. With senses, mind and intellect (ever) controlled, having liberation as his
Supreme Goal, free from desire, fear and anger---the sage is verily liberated
for ever.

\begin{gitaverse}
भोक्तारं यज्ञतपसां सर्वलोकमहेश्वरम् । \\
सुहृदं सर्वभूतानां ज्ञात्वा मां शान्तिमृच्छति ॥२९॥
\end{gitaverse}

\begin{transliteration}
bhoktāraṁ yajñatapasāṁ sarvalokamaheśvaram, \\
suhṛdaṁ sarvabhūtānāṁ jñātvā māṁ śāntimṛcchati.
\end{transliteration}

29. Knowing Me as Enjoyer of sacrifices and austerities, the Great Lord of all
worlds, the friend of all beings, he attains Peace.

\begin{gitaverse}
ॐ तत्सदिति श्रीमद् भगवद् गीतासूपनिषत्सु ब्रह्मविद्यायां \\
योगशास्त्रे श्रीकृष्णार्जुनसंवादे कर्मसन्न्यासयोगो नाम \\
पञ्चमोऽध्यायः
\end{gitaverse}

\begin{transliteration}
oṁ tatsaditi śrīmad bhagavad gītāsūpaniṣatsu brahmavidyāyāṁ \\
yogaśāstre śrīkṛṣṇārjunasaṁvāde karmasannyāsyogo nāma \\
pañchamo'dhyāyaḥ
\end{transliteration}

Thus in the UPANISHAD of the glorious Bhagawad Geeta, in the Science of the
Eternal, in the Scripture of YOGA, in the dialogue between Sri Krishna and
Arjuna the fifth discourse ends entitled: Yoga of True Renunciation
