\chapterdrop

\begin{center}
\headersanskrit{अथ द्वादशोऽध्यायः}

\headerspace
\headertransliteration{Atha Dvādaśo'dhyāyaḥ}

\section{Chapter 12}

\headerspace
\headersanskrit{भक्तियोगः}

\headerspace
\headertransliteration{Bhakti Yogah}

\headerspace
\headertranslation{The Yoga of Devotion}

\headerspace
\end{center}

\begin{gitaverse}
अर्जुन उवाच \\
एवं सततयुक्ता ये भक्तास्त्वां पर्युपासते । \\
ये चाप्यक्षरमव्यक्तं तेषां के योगवित्तमाः ॥१॥
\end{gitaverse}

\begin{transliteration}
arjuna uvāca \\
evaṁ satatayuktā ye bhaktās-tvāṁ paryupāsate, \\
ye cāpyakṣaram-avyaktaṁ teṣāṁ ke yogavittamāḥ.
\end{transliteration}

Arjuna said: \\
1. Those devotees who, eversteadfast, thus worship you, and also those who
worship the imperishable, the unmanifested---which of them are better versed in
YOGA?\@

\begin{gitaverse}
श्रीभगवानुवाच \\
मय्यावेश्य मनो ये मां नित्ययुक्ता उपासते । \\
श्रद्धया परयोपेतास्ते मे युक्ततमा मताः ॥२॥
\end{gitaverse}

\begin{transliteration}
śrībhagavānuvāca \\
mayyāveśya mano ye māṁ nityayuktā upāsate, \\
śraddhayā parayopetāste me yuktatamā matāḥ.
\end{transliteration}

The Blessed Lord said: \\
2. Those who, fixing their mind on Me, worship Me, ever steadfast and endowed
with supreme faith, these, in my opinion, are the best in YOGA.\@

\begin{gitaverse}
ये त्वक्षरमनिर्देश्यमव्यक्तं पर्युपासते । \\
सर्वत्रगमचिन्त्यं च कूटस्थमचलं ध्रुवम् ॥३॥
\end{gitaverse}

\begin{transliteration}
ye tvakṣaram-anirdeśyam-avyaktaṁ paryupāsate, \\
sarvatragam-acintyaṁ ca kūṭastham-acalaṁ dhruvam.
\end{transliteration}

3. Those who worship the imperishable, the indefinable, the unmanifest, the
omnipresent, the unthinkable, the unchangeable, the immovable and the
eternal\ldots

\begin{gitaverse}
सन्नियम्येन्द्रियग्रामं सर्वत्र समबुद्धयः । \\
ते प्राप्नुवन्ति मामेव सर्वभूतहिते रताः ॥४॥
\end{gitaverse}

\begin{transliteration}
sanniyamyendriyagrāmaṁ sarvatra samabuddhayaḥ, \\
te prāpnuvanti mām-eva sarvabhūtahite ratāḥ.
\end{transliteration}

4. Having restrained all the senses, even-minded everywhere, rejoicing ever in
the welfare of all beings---verily they also come unto Me.

\begin{gitaverse}
क्लेशोऽधिकतरस्तेषामव्यक्तासक्तचेतसाम् । \\
अव्यक्ता हि गतिर्दुःखं देहवद्भिरवाप्यते ॥५॥
\end{gitaverse}

\begin{transliteration}
kleśo'dhikataras-teṣām-avyaktāsakta-cetasām, \\
avyaktā hi gatir-duḥkhaṁ dehavadbhir-avāpyate.
\end{transliteration}

5. Greater is their trouble whose minds are set on the `Unmanifest'; for the
goal, the `Unmanifest', is very hard for the embodied to reach.

\begin{gitaverse}
ये तु सर्वाणि कर्माणि मयि सन्न्यस्य मत्पराः । \\
अनन्येनैव योगेन मां ध्यायन्त उपासते ॥६॥
\end{gitaverse}

\begin{transliteration}
ye tu sarvāṇi karmāṇi mayi sannyasya matparāḥ, \\
ananyenaiva yogena māṁ dhyāyanta upāsate.
\end{transliteration}

6. But those who worship Me, renouncing all actions in Me, regarding Me as the
Supreme Goal, meditating on Me with single-minded devotion (YOGA)\ldots

\begin{gitaverse}
तेषामहं समुद्धर्ता मृत्युसंसारसागरात् । \\
भवामि नचिरात्पार्थ मय्यावेशितचेतसाम् ॥७॥
\end{gitaverse}

\begin{transliteration}
teṣām-ahaṁ samuddhartā mṛtyu-saṁsāra-sāgarāt, \\
bhavāmi nacirāt-pārtha mayyāveśita-cetasām.
\end{transliteration}

7. For them, whose minds are set on Me, verily I become, erelong, O Partha, the
Saviour, (to save them) out of the ocean of finite experiences; the SAMSARA.\@

\begin{gitaverse}
मय्येव मन आधत्स्व मयि बुद्धिं निवेशय । \\
निवसिष्यसि मय्येव अत ऊर्ध्वं न संशयः ॥८॥
\end{gitaverse}

\begin{transliteration}
mayyeva mana ādhatsva mayi buddhiṁ niveśaya, \\
nivasiṣyasi mayyeva ata ūrdhvaṁ na saṁśayaḥ.
\end{transliteration}

8. Fix your mind on Me only, place your intellect in Me; then, (thereafter) you
shall, no doubt, live in Me alone.

\begin{gitaverse}
अथ चित्तं समाधातुं न शक्नोषि मयि स्थिरम् । \\
अभ्यासयोगेन ततो मामिच्छाप्तुं धनञ्जय ॥९॥
\end{gitaverse}

\begin{transliteration}
atha cittaṁ samādhātuṁ na śaknoṣi mayi sthiram, \\
abhyāsayogena tato māmicchāptuṁ dhanañjaya.
\end{transliteration}

9. If you are unable to fix your mind steadily upon Me, then by the YOGA of
constant practice, seek to reach Me, O Dhananjaya.

\begin{gitaverse}
अभ्यासेऽप्यसमर्थोऽसि मत्कर्मपरमो भव । \\
मदर्थमपि कर्माणि कुर्वन्सिद्धिमवाप्स्यसि ॥१०॥
\end{gitaverse}

\begin{transliteration}
abhyāse'pyasamartho'si mat-karma-paramo bhava, \\
mad-artham-api karmāṇi kurvan-siddhim-avāpsyasi.
\end{transliteration}

10. If you are unable even to practise Abhyasa-Yoga, be you intent on
performing actions for My sake; even by doing actions for My sake, you shall
attain perfection.

\begin{gitaverse}
अथैतदप्यशक्तोऽसि कर्तुं मद्योगमाश्रितः । \\
सर्वकर्मफलत्यागं ततः कुरु यतात्मवान् ॥११॥
\end{gitaverse}

\begin{transliteration}
athaitadapyaśakto'si kartuṁ mad-yogam-āśritaḥ, \\
sarva-karma-phala-tyāgaṁ tataḥ kuru yat-ātmavān.
\end{transliteration}

11. If you are unable to do even this, then taking refuge in Me,
self-controlled, renounce the fruits of all actions.

\begin{gitaverse}
श्रेयो हि ज्ञानमभ्यासाज्ज्ञानाद्ध्यानं विशिष्यते । \\
ध्यानात्कर्मफलत्यागस्त्यागाच्छान्तिरनन्तरम् ॥१२॥
\end{gitaverse}

\begin{transliteration}
śreyo hi jñānam-abhyāsājjñānāddhyānaṁ-viśiṣyate, \\
dhyānat-karmaphalatyāgas-tyāgācchāntir-anantaram.
\end{transliteration}

12. `Knowledge' is indeed better than `practice'; `meditation' is better than
`knowledge'; `renunciation of the fruits-of-actions' is better than
`meditation'; peace immediately follows `renunciation'.

\begin{gitaverse}
अद्वेष्टा सर्वभूतानां मैत्रः करुण एव च । \\
निर्ममो निरहङ्कारः समदुःखसुखः क्षमी ॥१३॥
\end{gitaverse}

\begin{transliteration}
adveṣṭā sarvabhūtānāṁ maitraḥ karuṇa eva ca, \\
nirmamo nirahaṅkāraḥ samaduḥkhasukhaḥ kṣamī.
\end{transliteration}

13. He who hates no creature, who is friendly and compassionate to all, who is
free from attachment and egoism, balanced in pleasure and pain, and
forgiving\ldots

\begin{gitaverse}
सन्तुष्टः सततं योगी यतात्मा दृढनिश्चयः । \\
मय्यर्पितमनोबुद्धिर्यो मद्भक्तः स मे प्रियः ॥१४॥
\end{gitaverse}

\begin{transliteration}
santuṣṭaḥ satataṁ yogī yatātmā dṛḍhaniścayaḥ, \\
mayyarpitamanobuddhiryo madbhaktaḥ sa me priyaḥ.
\end{transliteration}

14. Ever content, steady in meditation, self-controlled, possessed of firm
conviction, with mind and intellect dedicated to Me, he, My devotee, is dear to
me.

\begin{gitaverse}
यस्मान्नोद्विजते लोको लोकान्नोद्विजते च यः । \\
हर्षामर्षभयोद्वेगैर्मुक्तो यः स च मे प्रियः ॥१५॥
\end{gitaverse}

\begin{transliteration}
yasmān-nodvijate loko lokān-nodvijate ca yaḥ, \\
harṣāmarṣa-bhayodvegair-mukto yaḥ sa ca me priyaḥ.
\end{transliteration}

15. He by whom the world is not agitated (affected), and who cannot be agitated
by the world, who is freed from joy, envy, fear and anxiety---he is dear to me.

\begin{gitaverse}
अनपेक्षः शुचिर्दक्ष उदासीनो गतव्यथः । \\
सर्वारम्भपरित्यागी यो मद्भक्तः स मे प्रियः ॥१६॥
\end{gitaverse}

\begin{transliteration}
anapekṣaḥ śucir-dakṣa udāsīno gatavyathaḥ, \\
sarvārambha-parityāgī yo madbhaktaḥ sa me priyaḥ.
\end{transliteration}

16. He who is free from wants, pure, alert, unconcerned, untroubled, renouncing
all undertakings (or commencements)---he who is (thus) devoted to Me, is dear
to Me.

\begin{gitaverse}
यो न हृष्यति न द्वेष्टि न शोचति न काङ्क्षति । \\
शुभाशुभपरित्यागी भक्तिमान्यः स मे प्रियः ॥१७॥
\end{gitaverse}

\begin{transliteration}
yo na hṛsyati na dveṣṭi na śocati na kāṅkṣati, \\
śubhāśubha-parityāgī bhaktimān-yaḥ sa me priyaḥ.
\end{transliteration}

17. He who neither rejoices, nor hates, nor grieves, nor desires, renouncing
good and evil, full of devotion, is dear to Me.

\begin{gitaverse}
समः शत्रौ च मित्रे च तथा मानापमानयोः । \\
शीतोष्णसुखदुःखेषु समः सङ्गविवर्जितः ॥१८॥
\end{gitaverse}

\begin{transliteration}
samaḥ śatrau ca mitre ca tathā mānāpamānayoḥ, \\
śitoṣṇa-sukha-duḥkeṣu samaḥ saṅga-vivarjitaḥ.
\end{transliteration}

18. He who is the same to foe and friend, and also in honour and dishonour, who
is the same in cold and heat and in pleasure and pain, who is free from
attachment\ldots

\begin{gitaverse}
तुल्यनिन्दास्तुतिर्मौनी सन्तुष्टो येन केनचित् । \\
अनिकेतः स्थिरमतिर्भक्तिमान्मे प्रियो नरः ॥१९॥
\end{gitaverse}

\begin{transliteration}
tulya-nindā-stutir-maunī santuṣṭo yena kenacit, \\
aniketaḥ sthiramatir-bhaktimān-me priyo naraḥ.
\end{transliteration}

19. To whom censure and praise are equal, who is silent, content with anything,
homeless, steady-minded, full of devotion---that man is dear to Me.

\begin{gitaverse}
ये तु धर्म्यामृतमिदं यथोक्तं पर्युपासते । \\
श्रद्दधाना मत्परमा भक्तास्तेऽतीव मे प्रियाः ॥२०॥
\end{gitaverse}

\begin{transliteration}
ye tu dharmyāmṛtam-idaṁ yathoktaṁ paryupāsate, \\
śraddadhānā matparamā bhaktās-te'tīva me priyāḥ.
\end{transliteration}

20. They indeed, who follow this `Immortal DHARMA' (Law of Life) as described
above, endowed with faith, regarding Me as their Supreme Goal---such devotees
are exceedingly dear to Me.

\begin{gitaverse}
ॐ तत्सदिति श्रीमद् भगवद् गीतासूपनिषत्सु ब्रह्मविद्यायां \\
योगशास्त्रे श्रीकृष्णार्जुनसंवादे भक्तियोगो नाम \\
द्वादशोऽध्यायः
\end{gitaverse}

\begin{transliteration}
oṁ tatsaditi śrīmad bhagavad gītāsūpaniṣatsu brahmavidyāyāṁ \\
yogaśāstre śrīkṛṣṇārjunasaṁvāde bhaktiyogo nāma \\
dvādaśo'dhyāyaḥ
\end{transliteration}

Thus, in the UPANISHADS of the glorious Bhagawad-Geeta, in the Science of the
Eternal, in the scripture of YOGA, in the dialogue between Sri Krishna and
Arjuna, the twelfth discourse ends entitled: The Yoga of Devotion
