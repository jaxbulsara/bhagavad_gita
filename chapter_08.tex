\chapterdrop

\begin{center}
\headersanskrit{अथ अष्टमोऽध्यायः}

\headerspace
\headertransliteration{Atha Aṣṭamo'dhyāyaḥ}

\section{Chapter 8}

\headerspace
\headersanskrit{अक्षरब्रह्मयोगः}

\headerspace
\headertransliteration{Akṣara Brahma Yogah}

\headerspace
\headertranslation{The Yoga of Imperishable Brahman}

\headerspace
\end{center}

\begin{gitaverse}
अर्जुन उवाच \\
किं तद्ब्रह्म किमध्यात्मं किं कर्म पुरुषोत्तम । \\
अधिभूतं च किं प्रोक्तमधिदैवं किमुच्यते ॥१॥
\end{gitaverse}

\begin{transliteration}
Arjuna uvāca \\
kiṁ tadbrahma kimadhyātmaṁ kiṁ karma puruṣottama, \\
adhibhūtaṁ ca kiṁ proktamadhidaivaṁ kimucyate.
\end{transliteration}

Arjuna said: \\
1. What is that BRAHMAN?\@ What is the ADHYATMA?\@ What is `action'? O best
among men, what is declared to be the ADHIBHUTA?\@ And what is ADHIDAIVA said
to be?

\begin{gitaverse}
अधियज्ञः कथं कोऽत्र देहेऽस्मिन्मधुसूदन । \\
प्रयाणकाले च कथं ज्ञेयोऽसि नियतात्मभिः ॥२॥
\end{gitaverse}

\begin{transliteration}
adhiyajñaḥ kathaṁ ko'tra dehe'sminmadhusūdana, \\
prayāṇakāle ca kathaṁ jñeyo'si niyatātmabhiḥ.
\end{transliteration}

2. Who, and how, is ADHIYAJNA here in this body, O destroyer of Madhu? And how,
at the time of death, are you to be known by the self-controlled?

\begin{gitaverse}
श्रीभगवानुवाच \\
अक्षरं ब्रह्म परमं स्वभावोऽध्यात्ममुच्यते । \\
भूतभावोद्भवकरो विसर्गः कर्मसंज्ञितः ॥३॥
\end{gitaverse}

\begin{transliteration}
śrībhagavānuvāca \\
akṣaraṁ brahma paramaṁ svabhāvo'dhyātmamucyate, \\
bhūtabhāvodbhavakaro visargaḥ karmasaṁjñitaḥ.
\end{transliteration}

The Blessed Lord said: \\
3. BRAHMAN is Imperishable, the Supreme; His essential nature is called
Self-knowledge, the creative force that causes beings to spring forth into
manifestation is called `work'.

\begin{gitaverse}
अधिभूतं क्षरो भावः पुरुषश्चाधिदैवतम् । \\
अधियज्ञोऽहमेवात्र देहे देहभृतां वर ॥४॥
\end{gitaverse}

\begin{transliteration}
adhibhūtaṁ kṣaro bhāvaḥ puruṣaścādhidaivatam, \\
adhiyajño'hamevātra dehe dehabhṛtāṁ vara.
\end{transliteration}

4. ADHIBHUTA (or elements) constitutes My perishable nature, and the Indweller
(or the essence) is the ADHIDAIVATA;\@ I alone am the ADHIYAJNA here, in this
body, O best of the embodied.

\begin{gitaverse}
अन्तकाले च मामेव स्मरन्मुक्त्वा कलेवरम् । \\
यः प्रयाति स मद्भावं याति नास्त्यत्र संशयः ॥५॥
\end{gitaverse}

\begin{transliteration}
antakāle ca māmeva smaranmuktvā kalevaram, \\
yaḥ prayāti sa madbhāvaṁ yāti nāstyatra saṁśayaḥ.
\end{transliteration}

5. And whosoever, leaving the body, goes forth remembering Me alone, at the
time of his death, he attains My being; there is no doubt about this.

\begin{gitaverse}
यं यं वापि स्मरन्भावं त्यजत्यन्ते कलेवरम् । \\
तं तमेवैति कौन्तेय सदा तद्भावभावितः ॥६॥
\end{gitaverse}

\begin{transliteration}
yaṁ yaṁ vāpi smaranbhāvaṁ tyajatyante kalevaram, \\
taṁ tamevaiti kaunteya sadā tadbhāvabhāvitaḥ.
\end{transliteration}

6. Whosoever, at the end, leaves the body, thinking of any being, to that being
only he goes, O Kaunteya (O son of Kunti), because of his constant thought of
that being.

\begin{gitaverse}
तस्मात्सर्वेषु कालेषु मामनुस्मर युध्य च । \\
मय्यर्पितमनोबुद्धिर्मामेवैष्यस्यसंशयम् ॥७॥
\end{gitaverse}

\begin{transliteration}
tasmātsarveṣu kāleṣu māmanusmara yudhya ca, \\
mayyarpitamanobuddhirmāmevaiṣyasyasaṁśayam.
\end{transliteration}

7. Therefore, at all times, remember Me, and fight, with mind and intellect
fixed (or absorbed) in Me; you shall doubtless come to Me alone.

\begin{gitaverse}
अभ्यासयोगयुक्तेन चेतसा नान्यगामिना । \\
परमं पुरुषं दिव्यं याति पार्थानुचिन्तयन् ॥८॥
\end{gitaverse}

\begin{transliteration}
abhyāsayogayuktena cetasā nānyagāminā, \\
paramaṁ puruṣaṁ divyaṁ yāti pārthānucintayan.
\end{transliteration}

8. With the mind not moving towards any other thing, made steadfast by the
method of habitual meditation, and constantly meditating on the Supreme
PURUSHA, the Resplendent, O Partha, he goes (to Him).

\begin{gitaverse}
कविं पुराणमनुशासितार- \\
\tab मणोरणीयांसमनुस्मरेद्यः । \\
सर्वस्य धातारमचिन्त्यरूप- \\
\tab मादित्यवर्णं तमसः परस्तात् ॥९॥
\end{gitaverse}

\begin{transliteration}
kaviṁ purāṇamanuśāsitāra- \\
\tab maṇoraṇīyaṁsamanusmaredyaḥ, \\
sarvasya dhātāramacintyarūpa- \\
\tab mādityavarṇaṁ tamasaḥ parastāt.
\end{transliteration}

9. Whosoever, meditates upon the Omniscient, the Ancient, the Ruler (of the
whole world), minuter than the atom, the Supporter of all, of Form
inconceivable, Effulgent like the Sun and beyond the darkness (of ignorance)
\ldots

\begin{gitaverse}
प्रयाणकाले मनसाचलेन \\
\tab भक्त्या युक्तो योगबलेन चैव । \\
भ्रुवोर्मध्ये प्राणमावेश्य सम्यक्- \\
\tab स तं परं पुरुषमुपैति दिव्यम् ॥१०॥
\end{gitaverse}

\begin{transliteration}
prayāṇakāle manasācalena \\
\tab bhaktyā yukto yogabalena caiva, \\
bhruvormadhye prāṇamāveśya samyak- \\
\tab sa taṁ paraṁ puruṣamupaiti divyam.
\end{transliteration}

10. At the time of death, with an unshaken mind full of devotion, by the power
of `YOGA' fixing the whole `PRANA' (breath) between the two eyebrows, he (the
seeker) reaches the Supreme Resplendent `PURUSHA'.

\begin{gitaverse}
यदक्षरं वेदविदो वदन्ति \\
\tab विशन्ति यद्यतयो वीतरागाः । \\
यदिच्छन्तो ब्रह्मचर्यं चरन्ति \\
\tab तत्ते पदं सङ्ग्रहेण प्रवक्ष्ये ॥११॥
\end{gitaverse}

\begin{transliteration}
yadakṣaraṁ vedavido vadanti \\
\tab viśanti yadyatayo vītarāgāḥ, \\
yadicchanto brahmacaryaṁ caranti \\
\tab tatte padaṁ saṅgraheṇa pravakṣye.
\end{transliteration}

11. That which is declared Imperishable by the VEDA-knowers; That into which,
the self-controlled and desire-freed enter; That desiring for which BRAHMCHARYA
is practised---That Goal I will declare to thee in brief.

\begin{gitaverse}
सर्वद्वाराणि संयम्य मनो हृदि निरुध्य च । \\
मूर्ध्न्याधायात्मनः प्राणमास्थितो योगधारणाम् ॥१२॥
\end{gitaverse}

\begin{transliteration}
sarvadvārāṇi saṁyamya mano hṛdi nirudhya ca, \\
mūrdhnyādhāyātmanaḥ prāṇamāsthito yogadhāraṇām.
\end{transliteration}

12. Having closed all the gates, having confined the mind in the heart, having
fixed the life-breath in the `head', engaged in the practice of concentration,

\begin{gitaverse}
ओमित्येकाक्षरं ब्रह्म व्याहरन्मामनुस्मरन् । \\
यः प्रयाति त्यजन्देहं स याति परमां गतिम् ॥१३॥
\end{gitaverse}

\begin{transliteration}
omityekākṣaraṁ brahma vyāharanmāmanusmaran, \\
yaḥ prayāti tyajandehaṁ sa yāti paramāṁ gatim.
\end{transliteration}

13. Uttering the one-syllabled `OM'---the (symbol of) BRAHMAN---and remembering
Me, he who departs, leaving the body, attains the Supreme Goal.

\begin{gitaverse}
अनन्यचेताः सततं यो मां स्मरति नित्यशः । \\
तस्याहं सुलभः पार्थ नित्ययुक्तस्य योगिनः ॥१४॥
\end{gitaverse}

\begin{transliteration}
ananyacetāḥ satataṁ yo māṁ smarati nityaśaḥ, \\
tasyāhaṁ sulabhaḥ pārtha nityayuktasya yoginaḥ.
\end{transliteration}

14. I am easily attainable by that ever-steadfast YOGI who constantly remembers
Me daily, not thinking of anything else, O Partha.

\begin{gitaverse}
मामुपेत्य पुनर्जन्म दुःखालयमशाश्वतम् । \\
नाप्नुवन्ति महात्मानः संसिद्धिं परमां गताः ॥१५॥
\end{gitaverse}

\begin{transliteration}
māmupetya punarjanma duḥkhālayamaśāśvatam, \\
nāpnuvanti mahātmānaḥ saṁsiddhiṁ paramāṁ gatāḥ.
\end{transliteration}

15. Having attained Me, these MAHATMAS (great souls) do not again take birth,
which is the house of pain and is non-eternal, they having reached the Highest
Perfection, MOKSHA.\@

\begin{gitaverse}
आब्रह्मभुवनाल्लोकाः पुनरावर्तिनोऽर्जुन । \\
मामुपेत्य तु कौन्तेय पुनर्जन्म न विद्यते ॥१६॥
\end{gitaverse}

\begin{transliteration}
ābrahmabhuvanāllokāḥ punarāvartino'rjuna, \\
māmupetya tu kaunteya punarjanma na vidyate.
\end{transliteration}

16. Worlds upto the `world-of-BRAHMAJI' are subject to rebirth, O Arjuna; but
he who reaches Me, O Kaunteya, has no birth.

\begin{gitaverse}
सहस्रयुगपर्यन्तमहर्यद्ब्रह्मणो विदुः । \\
रात्रिं युगसहस्रान्तां तेऽहोरात्रविदो जनाः ॥१७॥
\end{gitaverse}

\begin{transliteration}
sahasrayugaparyantamaharyadbrahmaṇo viduḥ, \\
rātriṁ yugasahasrāntāṁ te'horātravido janāḥ.
\end{transliteration}

17. Those people who know (the length of) the day-of-BRAHMA which ends in a
thousand YUGAS (aeons), and the night which (also) ends in a thousand YUGAS
(aeons), they know day-and-night.

\begin{gitaverse}
अव्यक्ताद्व्यक्तयः सर्वाः प्रभवन्त्यहरागमे । \\
रात्र्यागमे प्रलीयन्ते तत्रैवाव्यक्तसञ्ज्ञके ॥१८॥
\end{gitaverse}

\begin{transliteration}
avyaktādvyaktayaḥ sarvāḥ prabhavantyaharāgame, \\
rātryāgame pralīyante tatraivāvyaktasañjñake.
\end{transliteration}

18. From the unmanifested all the manifested proceed at the coming of the
`day'; at the coming of `night' they dissolve verily in that alone which is
called the unmanifest.

\begin{gitaverse}
भूतग्रामः स एवायं भूत्वा भूत्वा प्रलीयते । \\
रात्र्यागमेऽवशः पार्थ प्रभवत्यहरागमे ॥१९॥
\end{gitaverse}

\begin{transliteration}
bhūtagrāmaḥ sa evāyaṁ bhūtvā bhūtvā pralīyate, \\
rātryāgame'vaśaḥ pārtha prabhavatyaharāgame.
\end{transliteration}

19. This same multiple of beings are being born again and again, and are
dissolved (into the unmanifest); helplessly, O Partha, at the coming of
`night,' and they come forth again at the coming of `day'.

\begin{gitaverse}
परस्तस्मात्तु भावोऽन्योऽव्यक्तोऽव्यक्तात्सनातनः । \\
यः स सर्वेषु भूतेषु नश्यत्सु न विनश्यति ॥२०॥
\end{gitaverse}

\begin{transliteration}
parastasmāttu bhāvo'nyo'vyakto'vyaktātsanātanaḥ, \\
yaḥ sa sarveṣu bhūteṣu naśyatsu na vinaśyati.
\end{transliteration}

20. But verily there exists, higher than that unmanifest (AVYAKTA), another
Unmanifested, which is Eternal, which is not destroyed when all beings are
destroyed.

\begin{gitaverse}
अव्यक्तोऽक्षर इत्युक्तस्तमाहुः परमां गतिम् । \\
यं प्राप्य न निवर्तन्ते तद्धाम परमं मम ॥२१॥
\end{gitaverse}

\begin{transliteration}
avyakto'kṣara ityuktastamāhuḥ paramāṁ gatim, \\
yaṁ prāpya na nivartante taddhāma paramaṁ mama.
\end{transliteration}

21. That which is called the Unmanifest and the Imperishable, that, they say is
the Highest Goal (path). They who reach It never again return. That is My
highest abode (state).

\begin{gitaverse}
पुरुषः स परः पार्थ भक्त्या लभ्यस्त्वनन्यया । \\
यस्यान्तःस्थानि भूतानि येन सर्वमिदं ततम् ॥२२॥
\end{gitaverse}

\begin{transliteration}
puruṣaḥ sa paraḥ pārtha bhaktyā labhyastvananyayā, \\
yasyāntaḥsthāni bhūtāni yena sarvamidaṁ tatam.
\end{transliteration}

22. That Highest `PURUSHA', O Partha, is attainable by unswerving devotion to
Him alone, within whom all beings dwell, by whom all this is pervaded.

\begin{gitaverse}
यत्र काले त्वनावृत्तिमावृत्तिं चैव योगिनः । \\
प्रयाता यान्ति तं कालं वक्ष्यामि भरतर्षभ ॥२३॥
\end{gitaverse}

\begin{transliteration}
yatra kāle tvanāvṛttimāvṛttiṁ caiva yoginaḥ, \\
prayātā yānti taṁ kālaṁ vakṣyāmi bharatarṣabha.
\end{transliteration}

23. Now at what time (path) departing, YOGIS go, never to return, as also to
return, that time (path), I will tell you, O Chief of Bharatas.

\begin{gitaverse}
अग्निर्ज्योतिरहः शुक्लः षण्मासा उत्तरायणम् । \\
तत्र प्रयाता गच्छन्ति ब्रह्म ब्रह्मविदो जनाः ॥२४॥
\end{gitaverse}

\begin{transliteration}
agnirjyotirahaḥ śuklaḥ ṣaṇmāsā uttarāyaṇam, \\
tatra prayātā gacchanti brahma brahmavido janāḥ.
\end{transliteration}

24. Fire, light, daytime, the bright fortnight, the six months of the northern
solstice; following this path, men who know BRAHMAN go to BRAHMAN.\@

\begin{gitaverse}
धूमो रात्रिस्तथा कृष्णः षण्मासा दक्षिणायनम् । \\
तत्र चान्द्रमसं ज्योतिर्योगी प्राप्य निवर्तते ॥२५॥
\end{gitaverse}

\begin{transliteration}
dhūmo rātristathā kṛṣṇaḥ ṣaṇmāsā dakṣiṇāyanam, \\
tatra cāndramasaṁ jyotiryogī prāpya nivartate.
\end{transliteration}

25. Smoke, nighttime, the dark fortnight, also six months of the southern
solstice, attaining by these to the Moon, the lunar light, the `YOGI' returns.

\begin{gitaverse}
शुक्लकृष्णे गती ह्येते जगतः शाश्वते मते । \\
एकया यात्यनावृत्तिमन्ययावर्तते पुनः ॥२६॥
\end{gitaverse}

\begin{transliteration}
śuklakṛṣṇe gatī hyete jagataḥ śāśvate mate, \\
ekayā yātyanāvṛttimanyayāvartate punaḥ.
\end{transliteration}

26. The `Path-of-Light' and the `Path-of-Darkness' available for the world are
verily thought to be both eternal; by the one, the `Path-of-Light', a man goes
to return not; by the other, the `Path-of-Darkness', he returns again.

\begin{gitaverse}
नैते सृती पार्थ जानन्योगी मुह्यति कश्चन । \\
तस्मात्सर्वेषु कालेषु योगयुक्तो भवार्जुन ॥२७॥
\end{gitaverse}

\begin{transliteration}
naite sṛtī pārtha jānanyogī muhyati kaścana, \\
tasmātsarveṣu kāleṣu yogayukto bhavārjuna.
\end{transliteration}

27. Knowing these paths, O Partha, no YOGIN is deluded; therefore, at all times
be steadfast in YOGA, O Arjuna.

\begin{gitaverse}
वेदेषु यज्ञेषु तपःसु चैव \\
\tab दानेषु यत्पुण्यफलं प्रदिष्टम् । \\
अत्येति तत्सर्वमिदं विदित्वा \\
\tab योगी परं स्थानमुपैति चाद्यम् ॥२८॥
\end{gitaverse}

\begin{transliteration}
vedeṣu yajñeṣu tapaḥsu caiva \\
\tab dāneṣu yatpuṇyaphalaṁ pradiṣṭam, \\
atyeti tatsarvamidaṁ viditvā \\
\tab yogī paraṁ sthānamupaiti cādyam.
\end{transliteration}

28. Whether the fruit of merit is declared (in the scriptures) as springing up
from study of the VEDAS, from performance of sacrifices, from practice of
austerities, and from charity---beyond all these goes the YOGIN, who having
known this (the two `paths') attains to the Supreme, Primeval (Essence).

\begin{gitaverse}
ॐ तत्सदिति श्रीमद् भगवद् गीतासूपनिषत्सु ब्रह्मविद्यायां \\
योगशास्त्र श्रीकृष्णार्जुनसंवादे अक्षरब्रह्मयोगो नाम \\
अष्टमोऽध्यायः
\end{gitaverse}

\begin{transliteration}
oṁ tatsaditi śrīmad bhagavad gītāsūpaniṣatsu brahmavidyāyāṁ \\
yogaśāstre śrīkṛṣṇārjunasaṁvāde akṣarabrahmayogo nāma \\
aṣṭamo'dhyāyaḥ.
\end{transliteration}

Thus, in the UPANISHADS of the glorious Bhagawad-Geeta, in the Science of the
Eternal, in the scripture of YOGA, in the dialogue between Sri Krishna and
Arjuna, the eighth discourse ends entitled: The Yoga of Imperishable Brahman
