\chapterdrop

\begin{center}
\headersanskrit{अथ एकादशोऽध्यायः}

\headerspace
\headertransliteration{Atha Ekādaśo'dhyāyaḥ}

\section{Chapter 11}

\headerspace
\headersanskrit{विश्वरूपदर्शनयोगः}

\headerspace
\headertransliteration{Viśvarūpa Darśana Yogah}

\headerspace
\headertranslation{The Yoga of the Vision of the Universal Form}

\headerspace
\end{center}

\begin{gitaverse}
अर्जुन उवाच \\
मदनुग्रहाय परमं गुह्यमध्यात्मसञ्ज्ञितम् । \\
यत्त्वयोक्तं वचस्तेन मोहोऽयं विगतो मम ॥१॥
\end{gitaverse}

\begin{transliteration}
arjuna uvāca \\
mad-anugrahāya paramaṁ guhyamadhyātma-sañjñitam, \\
yattvayoktaṁ vacastena moho'yaṁ vigato mama.
\end{transliteration}

Arjuna Said: \\
1. By this word of the highest secret concerning the Self, which You have
spoken out of compassion towards me, my delusion is gone.

\begin{gitaverse}
भवाप्ययौ हि भूतानां श्रुतौ विस्तरशो मया । \\
त्वत्तः कमलपत्राक्ष माहात्म्यमपि चाव्ययम् ॥२॥
\end{gitaverse}

\begin{transliteration}
bhavāpyayau hi bhūtānāṁ śrutau vistaraśo mayā, \\
tvattaḥ kamala-patrākṣa māhātmyam-api cāvyayam.
\end{transliteration}

2. The origin and destruction of beings verily, have been heard by me in detail
from You, O Lotus-eyed Krishna, and also Your inexhaustible greatness.

\begin{gitaverse}
एवमेतद्यथात्थ त्वमात्मानं परमेश्वर । \\
द्रष्टुमिच्छामि ते रूपमैश्वरं पुरुषोत्तम ॥३॥
\end{gitaverse}

\begin{transliteration}
evam-etadyathāttha tvam-ātmānaṁ parameśvara, \\
draṣṭum-icchāmi te rūpam-aiśvaraṁ puruṣottama.
\end{transliteration}

3. (Now) O Supreme Lord! As you have thus described Yourself, I wish to see
(actually) Your Form Divine, O PURUSHOTTAMA.\@

\begin{gitaverse}
मन्यसे यदि तच्छक्यं मया द्रष्टुमिति प्रभो । \\
योगेश्वर ततो मे त्वं दर्शयात्मानमव्ययम् ॥४॥
\end{gitaverse}

\begin{transliteration}
manyase yadi tacchakyaṁ mayā draṣṭum-iti prabho, \\
yogeśvara tato me tvaṁ darśayātmānam-avyayam.
\end{transliteration}

4. If you, O Lord, think it possible for me to see It, if You please, then, O
Lord of YOGAS, show me Your Imperishable Selfform.

\begin{gitaverse}
श्रीभगवानुवाच \\
पश्य मे पार्थ रूपाणि शतशोऽथ सहस्रशः । \\
नानाविधानि दिव्यानि नानावर्णाकृतीनि च ॥५॥
\end{gitaverse}

\begin{transliteration}
śrībhagavānuvāca \\
paśya me pārtha rūpāṇi śataśo'tha sahasraśaḥ, \\
nānā-vidhāni divyāni nānā-varṇākṛtīni ca.
\end{transliteration}

The Blessed Lord said: \\
5. Behold, O Partha, forms of Me, by hundreds and thousands, of different sorts
Divine, of various colours and shapes.

\begin{gitaverse}
पश्यादित्यान्वसून्रुद्रानश्विनौ मरुतस्तथा । \\
बहून्यदृष्टपूर्वाणि पश्याश्चर्याणि भारत ॥६॥
\end{gitaverse}

\begin{transliteration}
paśyādityān-vasūn-rudrān-aśvinau marutas-tathā, \\
bahūnyadṛṣṭa-pūrvāṇi paśyāścaryāṇi bhārata.
\end{transliteration}

6. Behold the ADITYAS, the VASUS, the RUDRAS, the (two) ASHWINS and also the
MARUTS;\@ behold many wonders never seen before, O Bharata.

\begin{gitaverse}
इहैकस्थं जगत्कृत्स्नं पश्याद्य सचराचरम् । \\
मम देहे गुडाकेश यच्चान्यद्द्रष्टुमिच्छसि ॥७॥
\end{gitaverse}

\begin{transliteration}
ihaikasthaṁ jagat-kṛtsnaṁ paśyādya sacarācaram, \\
mama dehe guḍākeśa yaccānyad-draṣṭum-icchasi.
\end{transliteration}

7. Now behold, O Gudakesha, in this Body, that the whole universe centres in
One---including, the moving and the unmoving---and whatever else you desire to
see.

\begin{gitaverse}
न तु मां शक्यसे द्रष्टुमनेनैव स्वचक्षुषा । \\
दिव्यं ददामि ते चक्षुः पश्य मे योगमैश्वरम् ॥८॥
\end{gitaverse}

\begin{transliteration}
na tu māṁ śakyase draṣṭum-anenaiva sva-cakṣuṣā, \\
divyaṁ dadāmi te cakṣuḥ paśya me yogam-aiśvaram.
\end{transliteration}

8. But You are not able to behold Me with these Your own eyes; I give You the
divine-eye; behold My Lordly YOGA.\@

\begin{gitaverse}
सञ्जय उवाच \\
एवमुकत्वा ततो राजन्महायोगेश्वरो हरिः । \\
दर्शयामास पार्थाय परमं रूपमैश्वरम् ॥९॥
\end{gitaverse}

\begin{transliteration}
sañjaya uvāca \\
evam-uktvā tato rājan-mahā-yogeśvaro hariḥ, \\
darśayāmāsa pārthāya paramaṁ rūpam-aiśvaram.
\end{transliteration}

Sanjaya said: \\
9. Having thus spoken, O King, the great Lord of YOGA, Hari, showed to Partha
His Supreme Form, as the Lord (of the Universe).

\begin{gitaverse}
अनेकवक्त्रनयनमनेकाद्भुतदर्शनम् । \\
अनेकदिव्याभरणं दिव्यानेकोद्यतायुधम् ॥१०॥
\end{gitaverse}

\begin{transliteration}
aneka-vaktra-nayanam-anekādbhuta-darśanam, \\
aneka-divyābharaṇaṁ divyānekodyat-āyudham.
\end{transliteration}

10. With numerous mouths and eyes, with numerous wonderful sights, with
numerous divine ornaments, with numerous divine weapons uplifted (such a form
He showed).

\begin{gitaverse}
दिव्यमाल्याम्बरधरं दिव्यगन्धानुलेपनम् । \\
सर्वाश्चर्यमयं देवमनन्तं विश्वतोमुखम् ॥११॥
\end{gitaverse}

\begin{transliteration}
divya-mālyāmbara-dharaṁ divya-gandhānulepanam, \\
sarvāścaryamayaṁ devam-anantaṁ viśvatomukham.
\end{transliteration}

11. Wearing divine garlands (necklaces) and apparel, anointed with divine
unguents, the All-wonderful, Resplendent, Endless, facing all sides.

\begin{gitaverse}
दिवि सूर्यसहस्रस्य भवेद्युगपदुत्थिता । \\
यदि भाः सदृशी सा स्याद्भासस्तस्य महात्मनः ॥१२॥
\end{gitaverse}

\begin{transliteration}
divi sūrya-sahasrasya bhaved-yugapad-utthitā, \\
yadi bhāḥ sadṛśī sā syād-bhāsastasya mahātmanaḥ.
\end{transliteration}

12. If the splendour of a thousand Suns was to blaze all at once
(simultaneously) in the sky, that would be like the splendour of that Mighty
Being (great soul).

\begin{gitaverse}
तत्रैकस्थं जगत्कृत्स्नं प्रविभक्तमनेकधा । \\
अपश्यद्देवदेवस्य शरीरे पाण्डवस्तदा ॥१३॥
\end{gitaverse}

\begin{transliteration}
tatraikasthaṁ jagat-kṛtsnaṁ pravibhaktam-anekadhā, \\
apaśyad-devadevasya śarīre pāṇḍavas-tadā.
\end{transliteration}

13. There, in the body of the God of gods, the Pandava (Son of Pandu) then saw
the whole Universe resting in one, with all its infinite parts.

\begin{gitaverse}
ततः स विस्मयाविष्टो हृष्टरोमा धनञ्जयः । \\
प्रणम्य शिरसा देवं कृताञ्जलिरभाषत ॥१४॥
\end{gitaverse}

\begin{transliteration}
tataḥ sa vismayāviṣṭo hṛṣṭaromā dhanañjayaḥ, \\
praṇamya śirasā devaṁ kṛtāñjalir-abhāṣata.
\end{transliteration}

14. Then, Dhananjaya, filled with wonder, with his hair standing on end, bowed
down his head to the God and spoke with joined palms.

\begin{gitaverse}
अर्जुन उवाच \\
पश्यामि देवांस्तव देव देहे \\
\tab सर्वांस्तथा भूतविशेषसङ्घान् । \\
ब्रह्माणमीशं कमलासनस्थमृ- \\
\tab षींश्च सर्वानुरगांश्च दिव्यान् ॥१५॥
\end{gitaverse}

\begin{transliteration}
arjuna uvāca \\
paśyāmi devāṁs-tava deva dehe \\
\tab sarvāṁs-tathā bhūta-viśeṣa-saṅghān, \\
brahmāṇam-īśaṁ kamalāsanastham- \\
\tab ṛṣīṁśca sarvān-uragāṁśca divyān.
\end{transliteration}

Arjuna said: \\
15. I see all the gods, O God, in Your body, and (also) hosts of various
classes of beings. BRAHMA, the Lord of Creation, seated on the Lotus, all the
RISHIS and celestial serpents.

\begin{gitaverse}
अनेकबाहूदरवक्त्रनेत्रं- \\
\tab पश्यामि त्वां सर्वतोऽनन्तरूपम् । \\
नान्तं न मध्यं न पुनस्तवादिं- \\
\tab पश्यामि विश्वेश्वर विश्वरूप ॥१६॥
\end{gitaverse}

\begin{transliteration}
aneka-bāhūdara-vaktra-netraṁ- \\
\tab paśyāmi tvāṁ sarvato'nanta-rūpam, \\
nāntaṁ na madhyaṁ na punastavādiṁ \\
\tab paśyāmi viśveśvara viśvarūpa.
\end{transliteration}

16. I see Thee of boundless form on every side, with manifold arms, stomachs,
mouths and eyes; neither the end, nor the middle, nor also the beginning do I
see; O, Lord of the Universe, O, Cosmic-Form.

\begin{gitaverse}
किरीटिनं गदिनं चक्रिणं च \\
\tab तेजोराशिं सर्वतो दीप्तिमन्तम् । \\
पश्यामि त्वां दुर्निरीक्ष्यं समन्ताद् \\
\tab दीप्तानलार्कद्युतिमप्रमेयम् ॥१७॥
\end{gitaverse}

\begin{transliteration}
kirīṭinaṁ gadinaṁ cakriṇaṁ ca \\
\tab tejo-rāśiṁ sarvato dīpti-mantam, \\
paśyāmi tvāṁ durnirīkṣyaṁ samantād \\
\tab dīptānalārka-dyutim-aprameyam.
\end{transliteration}

17. I see Thee with Crown, Club, and Discus; a mass of radiance shining
everywhere, very hard to look at, all round blazing like burning fire and Sun,
and incomprehensible.

\begin{gitaverse}
त्वमक्षरं परमं वेदितव्यं- \\
\tab त्वमस्य विश्वस्य परं निधानम् । \\
त्वमव्ययः शाश्वतधर्मगोप्ता \\
\tab सनातनस्त्वं पुरुषो मतो मे ॥१८॥
\end{gitaverse}

\begin{transliteration}
tvam-akṣaraṁ paramaṁ veditavyaṁ- \\
\tab tvam-asya viśvasya paraṁ nidhānam, \\
tvam-avyayaḥ śāśvata-dharma-goptā \\
\tab sanātanas-tvaṁ puruṣo mato me.
\end{transliteration}

18. You are the Imperishable, the Supreme Being worthy to be known. You are the
great treasure-house of this Universe. You are the imperishable Protector of
the Eternal DHARMA.\@ In my opinion, You are the Ancient PURUSHA.\@

\begin{gitaverse}
अनादिमध्यान्तमनन्तवीर्यम्- \\
\tab अनन्तबाहुं शशिसूर्यनेत्रम् । \\
पश्यामि त्वां दीप्तहुताशवक्त्रं- \\
\tab स्वतेजसा विश्वमिदं तपन्तम् ॥१९॥
\end{gitaverse}

\begin{transliteration}
anādi-madhyāntam-anantavīryam- \\
\tab ananta-bāhuṁ śaśi-sūrya-netram, \\
paśyāmi tvāṁ dīpta-hutāśa-vaktraṁ- \\
\tab svatejasā viśvam-idaṁ tapantam.
\end{transliteration}

19. I see You without beginning, middle, or end, infinite in power, of endless
arms, the sun and moon being Your eyes, the burning fire Your mouth, heating
the whole universe with Your radiance.

\begin{gitaverse}
द्यावापृथिव्योरिदमन्तरं हि \\
\tab व्याप्तं त्वयैकेन दिशश्च सर्वाः । \\
दृष्ट्वाद्भुतं रूपमुग्रं तवेदं- \\
\tab लोकत्रयं प्रव्यथितं महात्मन् ॥२०॥
\end{gitaverse}

\begin{transliteration}
dyāvā-pṛthivyor-idam-antaraṁ hi \\
\tab vyāptaṁ tvayaikena diśaśca sarvāḥ, \\
ḍṛṣṭvādbhutaṁ rūpam-ugraṁ tavedaṁ- \\
\tab lokatrayaṁ pravyathitaṁ mahātman.
\end{transliteration}

20. This space between earth and the heavens and all the quarters is filled by
You alone; having seen this, Your wonderful and terrible form, the three worlds
are trembling with fear, O great-souled Being.

\begin{gitaverse}
अमी हि त्वां सुरसङ्घा विशन्ति \\
\tab केचिद्भीताः प्राञ्जलयो गृणन्ति । \\
स्वस्तीत्युक्त्वा महर्षिसिद्धसङ्घाः \\
\tab स्तुवन्ति त्वां स्तुतिभिः पुष्कलाभिः ॥२१॥
\end{gitaverse}

\begin{transliteration}
amī hi tvāṁ surasaṅghā viśanti \\
\tab kecid-bhītāḥ prāñjalayo gṛṇanti, \\
svastītyuktvā maharṣi-siddha-saṅghāḥ \\
\tab stuvanti tvāṁ stutibhiḥ puṣkalābhiḥ.
\end{transliteration}

21. Verily, into You enter these hosts of DEVAS;\@ some extol You in fear with
joined palms; ``May it be well'' thus saying, bands of great RISHIS and SIDDHAS
praise You with hymns sublime.

\begin{gitaverse}
रुद्रादित्या वसवो ये च साध्या- \\
\tab विश्वेऽश्विनौ मरुतश्चोष्मपाश्च । \\
गन्धर्वयक्षासुरसिद्धसङ्घा- \\
\tab वीक्षन्ते त्वां विस्मिताश्चैव सर्वे ॥२२॥
\end{gitaverse}

\begin{transliteration}
rudrādityā vasavo ye ca sādhyā- \\
\tab viśve'śvinau marutaścoṣmapāśca, \\
gandharva-yakṣāsura-siddha-saṅghā- \\
\tab vīkṣante tvāṁ vismitāścaiva sarve.
\end{transliteration}

22. THE RUDRAS, ADITYAS, VASUS, SADHYAS, VISHWEDEVAS, THE TWO ASHWINS, MARUTS,
USHMAPAS AND HOSTS OF GANDHARVAS, YAKSHAS, ASURAS AND SIDDHAS---they are all
looking at you, all quite astonished.

\begin{gitaverse}
रूपं महत्ते बहुवक्त्रनेत्रं- \\
\tab महाबाहो बहुबाहूरुपादम् । \\
बहूदरं बहुदंष्ट्राकरालं- \\
\tab दृष्ट्वा लोकाः प्रव्यथितास्तथाहम् ॥२३॥
\end{gitaverse}

\begin{transliteration}
rūpaṁ mahat-te bahu-vaktra-netraṁ \\
\tab mahābāho bahu-bāhūru-pādam, \\
bahūdaraṁ bahu-daṁṣṭrā-karālaṁ \\
\tab dṛṣṭvā lokāḥ pravyathitāstathāham.
\end{transliteration}

23. Having seen Your immeasurable Form, with many mouths and eyes, O
Mighty-armed, with many arms, thighs, and feet, with many stomachs and fearsome
with many tusks, the worlds are terrified and so too am I.\@

\begin{gitaverse}
नभःस्पृशं दीप्तमनेकवर्णं \\
\tab व्यात्ताननं दीप्तविशालनेत्रम् । \\
दृष्ट्वा हि त्वां प्रव्यथितान्तरात्मा \\
\tab धृतिं न विन्दामि शमं च विष्णो ॥२४॥
\end{gitaverse}

\begin{transliteration}
nabhaḥ-spṛśaṁ dīptam-aneka-varṇaṁ \\
\tab vyāttānanaṁ dīpta-viśāla-netram, \\
dṛṣṭvā hi tvāṁ pravyathitāntarātmā \\
\tab dhṛtiṁ na vindāmi śamaṁ ca viṣṇo.
\end{transliteration}

24. On seeing you, with Your Form touching the sky, flaming in many colours,
with mouths wide open, with large fiery eyes, I am terrified at heart, and I
find neither courage, nor peace, O Vishnu!

\begin{gitaverse}
दंष्ट्राकरालानि च ते मुखानि \\
\tab दृष्ट्वैव कालानलसन्निभानि । \\
दिशो न जाने न लभे च शर्म \\
\tab प्रसीद देवेश जगन्निवास ॥२५॥
\end{gitaverse}

\begin{transliteration}
daṁṣṭrā-karālāni ca te mukhāni \\
\tab dṛṣṭvaiva kālānala-sannibhāni, \\
diśo na jāne na labhe ca śarma \\
\tab prasīda deveśa jagannivāsa.
\end{transliteration}

25. Having seen your mouths fearsome with tusks (blazing) like PRALAYA fires, I
know not the four quarters, nor do I find peace; be gracious, O Lord of the
DEVAS, O Abode of the Universe.

\begin{gitaverse}
अमी च त्वां धृतराष्ट्रस्य पुत्राः \\
\tab सर्वे सहैवावनिपालसङ्घैः । \\
भीष्मो द्रोणः सूतपुत्रस्तथासौ \\
\tab सहास्मदीयैरपि योधमुख्यैः ॥२६॥
\end{gitaverse}

\begin{transliteration}
amī ca tvāṁ dhṛtarāṣṭrasya putrāḥ \\
\tab sarve sahaivāvanipāla-saṅghaiḥ, \\
bhīṣmo droṇaḥ sūtaputrastathāsau \\
\tab sahāsmadīyairapi yodhamukhaiḥ.
\end{transliteration}

26. All the sons of Dhritarashtra with hosts of kings of the earth, Bhishma,
Drona and the son of a charioteer, Karna, with the warrior chieftains of ours;

\begin{gitaverse}
वक्त्राणि ते त्वरमाणा विशन्ति \\
\tab दंष्ट्राकरालानि भयानकानि । \\
केचिद्विलग्ना दशनान्तरेषु \\
\tab सन्दृश्यन्ते चूर्णितैरुत्तमाङ्गैः ॥२७॥
\end{gitaverse}

\begin{transliteration}
vaktrāṇi te tvaramāṇā viśanti \\
\tab daṁṣṭrā-karālāni bhayānakāni, \\
kecid-vilagnā daśanāntareṣu \\
\tab sandṛśyante cūrṇitair-uttamāṅgaiḥ.
\end{transliteration}

27. Into Your mouths, with terrible teeth, and fearful to behold, they
precipitately enter. Some are found sticking in the gaps between the teeth with
their heads crushed into pulp.

\begin{gitaverse}
यथा नदीनां बहवोऽम्बुवेगाः \\
\tab समुद्रमेवाभिमुखा द्रवन्ति । \\
तथा तवामी नरलोकवीरा- \\
\tab विशन्ति वक्त्राण्यभिविज्वलन्ति ॥२८॥
\end{gitaverse}

\begin{transliteration}
yathā nadīnāṁ bahavo'mbu-vegāḥ \\
\tab samudram-evābhimukhā dravanti, \\
tathā tavāmī nara-loka-vīrā- \\
\tab viśanti vaktrāṇyabhivijvalanti.
\end{transliteration}

28. Verily, as many torrents of rivers flow towards the ocean, so these heroes
in the world of men enter Your flaming mouths.

\begin{gitaverse}
यथा प्रदीप्तं ज्वलनं पतङ्गा- \\
\tab विशन्ति नाशाय समृद्धवेगाः । \\
तथैव नाशाय विशन्ति लोकाः- \\
\tab तवापि वक्त्राणि समृद्धवेगाः ॥२९॥
\end{gitaverse}

\begin{transliteration}
yathā pradīptaṁ jvalanaṁ pataṅgā- \\
\tab viśanti nāśāya samṛddha-vegāḥ, \\
tathaiva nāśāya viśanti lokāḥ- \\
\tab tavāpi vaktrāṇi samṛddha-vegāḥ.
\end{transliteration}

29. As moths rush hurriedly into a blazing fire for their own destruction, so
also these creatures hastily rush into Your mouths of destruction.

\begin{gitaverse}
लेलिह्यसे ग्रसमानः समन्तात्- \\
\tab लोकान्समग्रान्वदनैर्ज्वलद्भिः । \\
तेजोभिरापूर्य जगत्समग्रं- \\
\tab भासस्तवोग्राः प्रतपन्ति विष्णो ॥३०॥
\end{gitaverse}

\begin{transliteration}
lelihyase grasamānaḥ samantāt- \\
\tab lokān-samagrān-vadanair-jvaladbhiḥ, \\
tejobhir-āpūrya jagat-samagraṁ- \\
\tab bhāsas-tavogrāḥ pratapanti viṣṇo.
\end{transliteration}

30. Devouring all worlds on every side with Your flaming mouths, You are
licking (in enjoyment). Your fierce rays, filling the whole world with
radiance, are burning, O Vishnu.

\begin{gitaverse}
आख्याहि मे को भवानुग्ररूपो- \\
\tab नमोऽस्तु ते देववर प्रसीद । \\
विज्ञातुमिच्छामि भवन्तमाद्यं- \\
\tab न हि प्रजानामि तव प्रवृत्तिम् ॥३१॥
\end{gitaverse}

\begin{transliteration}
ākhyāhi me ko bhavān-ugrarūpo- \\
\tab namo'stu te devavara prasīda, \\
vijñātum-icchāmi bhavantam-ādyaṁ- \\
\tab na hi prajānāmi tava pravṛttim.
\end{transliteration}

31. Tell me, who You are, so fierce in form? Salutations to You, O God Supreme;
have mercy. I desire to know You, the Original Being, I know not indeed Your
purpose.

\begin{gitaverse}
श्रीभगवानुवाच \\
कालोऽस्मि लोकक्षयकृत्प्रवृद्धो- \\
\tab लोकान् समाहर्तुमिह प्रवृत्तः । \\
ऋतेऽपि त्वां न भविष्यन्ति सर्वे \\
\tab येऽवस्थिताः प्रत्यनीकेषु योधाः ॥३२॥
\end{gitaverse}

\begin{transliteration}
śrī bhagavānuvāca \\
kālo'smi loka-kṣaya-kṛt-pravṛddho- \\
\tab lokān-samāhartum-iha pravṛttaḥ, \\
ṛte'pi tvāṁ na bhaviṣyanti sarve \\
\tab ye'vasthitāḥ pratyanīkeṣu yodhāḥ.
\end{transliteration}

The Blessed Lord said: \\
32. I am the mighty world-destroying Time, now engaged in destroying the
worlds. Even without You, none of the warriors arrayed in hostile armies shall
live.

\begin{gitaverse}
तस्मात्त्वमुत्तिष्ठ यशो लभस्व \\
\tab जित्वा शत्रून् भुङ्क्ष्व राज्यं समृद्धम् । \\
मयैवैते निहताः पूर्वमेव \\
\tab निमित्तमात्रं भव सव्यसाचिन् ॥३३॥
\end{gitaverse}

\begin{transliteration}
tasmāt-tvam-uttiṣṭha yaśo labhasva \\
\tab jitvā śatrūn bhuṅkṣva rājyaṁ samṛddham, \\
mayaivaite nihatāḥ pūrvameva \\
\tab nimitta-mātraṁ bhava savyasācin.
\end{transliteration}

33. Therefore, stand up, and obtain fame. Conquer the enemies and enjoy the
flourishing kingdom. Verily by Myself they have already been slain; be you a
mere instrument, O left-handed archer.

\begin{gitaverse}
द्रोणं च भीष्मं च जयद्रथं च \\
\tab कर्णं तथान्यानपि योधवीरान् । \\
मया हतांस्त्वं जहि मा व्यथिष्ठा- \\
\tab युध्यस्व जेतासि रणे सपत्नान् ॥३४॥
\end{gitaverse}

\begin{transliteration}
droṇaṁ ca bhīṣmaṁ ca jayadrathaṁ ca \\
\tab karṇaṁ tathānyānapi yodha-vīrān, \\
mayā hatāṁs-tvaṁ jahi mā vyathiṣṭhā- \\
\tab yudhyasva jetāsi raṇe sapatnān.
\end{transliteration}

34. Drona, Bhishma, Jayadratha, Karna, and other brave warriors---those have
already been slain by Me; you do kill; be not distressed with fear; fight and
you shall conquer your enemies in battle.

\begin{gitaverse}
सञ्जय उवाच \\
एतच्छ्रुत्वा वचनं केशवस्य \\
\tab कृताञ्जलिर्वेपमानः किरीटी । \\
नमस्कृत्वा भूय एवाह कृष्णं- \\
\tab सगद्गदं भीतभीतः प्रणम्य ॥३५॥
\end{gitaverse}

\begin{transliteration}
sañjaya uvāca \\
etacchrutvā vacanaṁ keśavasya \\
\tab kṛtāñjalir-vepamānaḥ kirīṭī, \\
namaskṛtvā bhūya evāha kṛṣṇaṁ \\
\tab sagad-gadaṁ bhītabhītaḥ praṇamya.
\end{transliteration}

Sanjaya said: \\
35. Having heard that speech of Keshava (Krishna), the crownedone (Arjuna),
with joined palms, trembling and prostrating himself, again addressed Krishna,
in a choked voice, bowing down, overwhelmed with fear.

\begin{gitaverse}
अर्जुन उवाच \\
स्थाने हृषीकेश तव प्रकीर्त्या \\
\tab जगत्प्रहृष्यत्यनुरज्यते च । \\
रक्षांसि भीतानि दिशो द्रवन्ति \\
\tab सर्वे नमस्यन्ति च सिद्धसङ्घाः ॥३६॥
\end{gitaverse}

\begin{transliteration}
arjuna uvāca \\
sthāne hṛṣīkeśa tava prakīrtyā \\
\tab jagat-prahṛṣyatyanurajyate ca, \\
rakṣāṁsi bhītāni diśo dravanti \\
\tab sarve namasyanti ca siddhasaṅghāḥ.
\end{transliteration}

Arjuna said: \\
36. It is but meet, O Hrishikesha (Krishna), that the world delights and
rejoices in Thy praise; RAKSHASAS fly in fear to all quarters, and all hosts of
SIDDHAS bow to Thee.

\begin{gitaverse}
कस्माच्च ते न नमेरन्महात्मन् \\
\tab गरीयसे ब्रह्मणोऽप्यादिकर्त्रे । \\
अनन्त देवेश जगन्निवास \\
\tab त्वमक्षरं सदसत्तत्परं यत् ॥३७॥
\end{gitaverse}

\begin{transliteration}
kasmācca te na nameran-mahātman \\
\tab garīyase brahmaṇo'pyādi-kartre, \\
ananta deveśa jagannivāsa \\
\tab tvam-akṣaraṁ sadasat-tat paraṁ yat.
\end{transliteration}

37. And why should they not, O Great-souled One, bow to Thee, greater (than all
else), the Primal Cause even of Brahma, O Infinite Being, O Lord of Lords, O
Abode of the Universe, You are the Imperishable, that which is beyond both the
Manifest and the Unmanifest.

\begin{gitaverse}
त्वमादिदेवः पुरुषः पुराणस्- \\
\tab त्वमस्य विश्वस्य परं निधानम् । \\
वेत्तासि वेद्यं च परं च धाम \\
\tab त्वया ततं विश्वमनन्तरूप ॥३८॥
\end{gitaverse}

\begin{transliteration}
tvam-ādi-devaḥ puruṣaḥ purāṇas- \\
\tab tvam-asya viśvasya paraṁ nidhānam, \\
vettāsi vedyaṁ ca paraṁ ca dhāma \\
\tab tvayā tataṁ viśvam-anantarūpa.
\end{transliteration}

38. You are the Primal God, the Ancient PURUSHA;\@ You are the Supreme Refuge
of this universe. You are the knower, the knowable, and the Abode-Supreme. By
Thee is the universe pervaded, O Being of Infinite forms.

\begin{gitaverse}
वायुर्यमोऽग्निर्वरुणः शशाङ्कः \\
\tab प्रजापतिस्त्वं प्रपितामहश्च । \\
नमो नमस्तेऽस्तु सहस्रकृत्वः \\
\tab पुनश्च भूयोऽपि नमो नमस्ते ॥३९॥
\end{gitaverse}

\begin{transliteration}
vāyur-yamo'gnir-varuṇaḥ śaśāṅkaḥ \\
\tab prajāpatis-tvaṁ prapitāmahaśca, \\
namo namaste'stu sahasrakṛtvaḥ \\
\tab punaśca bhūyo'pi namo namaste.
\end{transliteration}

39. You are VAYU, YAMA, AGNI, VARUNA, the Moon, PRAJAPATI, and the
great-grandfather of all. Salutations! Salutations unto You a thousand times,
and again salutations unto You!

\begin{gitaverse}
नमः पुरस्तादथ पृष्ठतस्ते \\
\tab नमोऽस्तु ते सर्वत एव सर्व । \\
अनन्तवीर्यामितविक्रमस्त्वं- \\
\tab सर्वं समाप्नोषि ततोऽसि सर्वः ॥४०॥
\end{gitaverse}

\begin{transliteration}
namaḥ purastād-atha pṛṣṭhataste \\
\tab namo'stu te sarvata eva sarva, \\
ananta-vīryāmita-vikramas-tvaṁ- \\
\tab sarvaṁ samāpnoṣi tato'si sarvaḥ.
\end{transliteration}

40. Salutations to You, before and behind! Salutations to You on every side! O
All! You, Infinite in Power, and Infinite in Prowess, pervade all; wherefore
You are the All.

\begin{gitaverse}
सखेति मत्वा प्रसभं यदुक्तं- \\
\tab हे कृष्ण हे यादव हे सखेति । \\
अजानता महिमानं तवेदं- \\
\tab मया प्रमादात्प्रणयेन वापि ॥४१॥
\end{gitaverse}

\begin{transliteration}
sakheti matvā prasabhaṁ yaduktaṁ \\
\tab he kṛṣṇa he yādava he sakheti, \\
ajānatā mahimānaṁ tavedaṁ \\
\tab mayā pramādāt praṇayena vāpi.
\end{transliteration}

41. Whatever I have rashly said from carelessness or love, addressing You as
``O Krishna, O Yadava, O friend'' and regarding You merely as a friend,
unknowing of this greatness of Yours\ldots

\begin{gitaverse}
यच्चावहासार्थमसत्कृतोऽसि \\
\tab विहारशय्यासनभोजनेषु । \\
एकोऽथवाप्यच्युत तत्समक्षं- \\
\tab तत्क्षामये त्वामहमप्रमेयम् ॥४२॥
\end{gitaverse}

\begin{transliteration}
yaccāvahā-sārtham-asatkṛto'si \\
\tab vihāra-śayyāsana-bhojaneṣu, \\
eko'thavāpyacyuta tat-samakṣaṁ \\
\tab tat-kṣāmaye tvā-mahamaprameyam.
\end{transliteration}

42. In whatever way I may have insulted You for the sake of fun, while at play,
reposing or sitting, or at meals, when alone (with You), O Achyuta, or in
company---that, O Immeasurable One, I implore You to forgive.

\begin{gitaverse}
पितासि लोकस्य चराचरस्य \\
\tab त्वमस्य पूज्यश्च गुरुर्गरीयान् । \\
न त्वत्समोऽस्त्यभ्यधिकः कुतोऽन्यो- \\
\tab लोकत्रयेऽप्यप्रतिमप्रभाव ॥४३॥
\end{gitaverse}

\begin{transliteration}
pitāsi lokasya carācarasya \\
\tab tvam-asya pūjyaśca gurur-garīyān, \\
na tvat-samo'sty-abhyadhikaḥ kuto'nyo- \\
\tab lokatraye'pyapratima-prabhāva.
\end{transliteration}

43. You are the Father of this world, moving and unmoving. You are to be adored
by this world. You are the greatest GURU, (for) there exists none who is equal
to You; how can there be then another, superior to You in the three worlds, O
Being of unequalled power?

\begin{gitaverse}
तस्मात्प्रणम्य प्रणिधाय कायं- \\
\tab प्रसादये त्वामहमीशमीड्यम् । \\
पितेव पुत्रस्य सखेव सख्युः \\
\tab प्रियः प्रियायार्हसि देव सोढुम् ॥४४॥
\end{gitaverse}

\begin{transliteration}
tasmāt-praṇamya praṇidhāya kāyaṁ \\
\tab prasādaye tvām-aham-īśam-īḍyam, \\
piteva putrasya sakheva sakhyuḥ \\
\tab priyaḥ priyāyārhasi deva soḍhum.
\end{transliteration}

44. Therefore, bowing down, prostrating my body, I crave your forgiveness,
adorable Lord. As a father forgiveth his son, a friend his friend, a lover his
beloved, even so should You forgive me, O DEVA.\@

\begin{gitaverse}
अदृष्टपूर्वं हृषितोऽस्मि दृष्ट्वा \\
\tab भयेन च प्रव्यथितं मनो मे । \\
तदेव मे दर्शय देवरूपं- \\
\tab प्रसीद देवेश जगन्निवास ॥४५॥
\end{gitaverse}

\begin{transliteration}
adṛṣṭa-pūrvaṁ hṛṣito'smi dṛṣṭvā \\
\tab bhayena ca pravyathitaṁ mano me, \\
tadeva me darśaya devarūpaṁ- \\
\tab prasīda deveśa jagannivāsa.
\end{transliteration}

45. I am delighted, having seen what was never seen before; and (yet) my mind
is distressed with fear. Show me your previous form only, O God; have mercy, O
God of gods, O Abode of the Universe.

\begin{gitaverse}
किरीटिनं गदिनं चक्रहस्तम्- \\
\tab इच्छामि त्वां द्रष्टुमहं तथैव । \\
तेनैव रूपेण चतुर्भुजेन \\
\tab सहस्रबाहो भव विश्वमूर्ते ॥४६॥
\end{gitaverse}

\begin{transliteration}
kirīṭinaṁ gadinaṁ cakra-hastam- \\
\tab icchāmi tvāṁ draṣṭum-ahaṁ tathaiva, \\
tenaiva rūpeṇa caturbhujena \\
\tab sahasra-bāho bhava viśva-mūrte.
\end{transliteration}

46. I desire to see You as before, crowned, bearing a mace, with a discus in
hand, in Your Former Form only, having four arms, O Thousand-armed, O Universal
Form.

\begin{gitaverse}
श्रीभगवानुवाच \\
मया प्रसन्नेन तवार्जुनेदं- \\
\tab रूपं परं दर्शितमात्मयोगात् । \\
तेजोमयं विश्वमनन्तमाद्यं- \\
\tab यन्मे त्वदन्येन न दृष्टपूर्वम् ॥४७॥
\end{gitaverse}

\begin{transliteration}
śrībhagavānuvāca \\
mayā prasannena tavārjunedaṁ- \\
\tab rūpaṁ paraṁ darśitam-ātma-yogāt, \\
tejomayaṁ viśvam-anantam-ādyaṁ- \\
\tab yanme tvadanyena na dṛṣṭa-pūrvam.
\end{transliteration}

The Blessed Lord said: \\
47. Graciously by Me, O Arjuna, this Supreme-Form has been shown to you by My
own YOGA-power---Full of splendour, Primeval, Infinite, this Universal-Form of
Mine has never been seen by any other than yourself.

\begin{gitaverse}
न वेदयज्ञाध्ययनैर्न दानैः- \\
\tab न च क्रियाभिर्न तपोभिरुग्रैः । \\
एवंरूपः शक्य अहं नृलोके \\
\tab द्रष्टुं त्वदन्येन कुरुप्रवीर ॥४८॥
\end{gitaverse}

\begin{transliteration}
na veda-yajñādhyayanairna dānaiḥ- \\
\tab na ca kriyābhirna tapobhirugraiḥ, \\
evaṁ-rūpaḥ śakya ahaṁ nṛloke \\
\tab draṣṭuṁ tvadanyena kurupravīra.
\end{transliteration}

48. Neither by the study of the VEDAS and sacrifices, nor by gifts nor by
rituals, nor by severe austerities, can I be seen in this form in the world of
men by any other than yourself, O great hero among the Kurus.

\begin{gitaverse}
मा ते व्यथा मा च विमूढभावो- \\
\tab दृष्ट्वा रूपं घोरमीदृङ्ममेदम् । \\
व्यपेतभीः प्रीतमनाः पुनस्त्वं- \\
\tab तदेव मे रूपमिदं प्रपश्य ॥४९॥
\end{gitaverse}

\begin{transliteration}
mā te vyathā mā ca vimūḍha-bhāvo- \\
\tab dṛṣṭvā rūpaṁ ghoram-īdṛṅmamedam, \\
vyapetabhīḥ prītamanāḥ punastvaṁ- \\
\tab tadeva me rūpam-idaṁ prapaśya.
\end{transliteration}

49. Be not afraid, nor bewildered on seeing such a terrible-Form of Mine as
this; with your fear dispelled and with gladdened heart, now behold again this
Form of Mine.

\begin{gitaverse}
सञ्जय उवाच \\
इत्यर्जुनं वासुदेवस्तथोक्त्वा \\
\tab स्वकं रूपं दर्शयामास भूयः । \\
आश्वासयामास च भीतमेनं \\
\tab भूत्वा पुनः सौम्यवपुर्महात्मा ॥५०॥
\end{gitaverse}

\begin{transliteration}
sañjaya uvāca \\
ityarjunaṁ vāsudevas-tathoktvā \\
\tab svakaṁ rūpaṁ darśayāmāsa bhūyaḥ, \\
āśvāsayāmāsa ca bhītamenaṁ \\
\tab bhūtvā punaḥ saumyavapur-mahātmā.
\end{transliteration}

Sanjaya said: \\
50. Having thus spoken to Arjuna, Vaasudeva again showed His own Form, and, the
Great-souled One, assuming His gentle Form, consoled him who was so terrified.

\begin{gitaverse}
अर्जुन उवाच \\
दृष्ट्वेदं मानुषं रूपं तव सौम्यं जनार्दन । \\
इदानीमस्मि संवृत्तः सचेताः प्रकृतिं गतः ॥५१॥
\end{gitaverse}

\begin{transliteration}
arjuna uvāca \\
dṛṣṭvedaṁ mānuṣaṁ rūpaṁ tava saumyaṁ janārdana, \\
idānīm-asmi saṁvṛttaḥ sacetāḥ prakṛtiṁ gataḥ.
\end{transliteration}

Arjuna said: \\
51. Having seen this, Thy gentle human-Form, O Janardana, I am now composed and
restored to my own nature.

\begin{gitaverse}
श्रीभगवानुवाच \\
सुदुर्दर्शमिदं रूपं दृष्टवानसि यन्मम । \\
देवा अप्यस्य रूपस्य नित्यं दर्शनकाङ्क्षिणः ॥५२॥
\end{gitaverse}

\begin{transliteration}
śrībhagavānuvāca \\
sudurdarśam-idaṁ rūpaṁ dṛṣṭavānasi yanmama, \\
devā apyasya rūpasya nityaṁ darśana-kāṅkṣiṇaḥ.
\end{transliteration}

The Blessed Lord said: \\
52. Very hard, indeed, it is to see this Form of Mine which you have seen. Even
the gods are ever longing to behold this Form.

\begin{gitaverse}
नाहं वेदैर्न तपसा न दानेन न चेज्यया । \\
शक्य एवंविधो द्रष्टुं दृष्टवानसि मां यथा ॥५३॥
\end{gitaverse}

\begin{transliteration}
nahaṁ vedairna tapasā na dānena na cejyayā, \\
śakya evaṁvidho draṣṭuṁ dṛṣṭavānasi māṁ yathā.
\end{transliteration}

53. Neither by the VEDAS, nor by austerity, nor by gift, nor by sacrifices can
I be seen in this Form as you have seen Me (in your present mental condition).

\begin{gitaverse}
भक्त्या त्वनन्यया शक्य अहमेवंविधोऽर्जुन । \\
ज्ञातुं द्रष्टुं च तत्त्वेन प्रवेष्टुं च परन्तप ॥५४॥
\end{gitaverse}

\begin{transliteration}
bhaktyā tvananyayā śakya aham-evaṁvidho'rjuna, \\
jñātuṁ draṣṭuṁ ca tattvena praveṣṭuṁ ca parantapa.
\end{transliteration}

54. But, by single-minded devotion, can I, of this Form, be `known' and `seen'
in reality, and also `entered' into, O Parantapa (O scorcher of your foes)!

\begin{gitaverse}
मत्कर्मकृन्मत्परमो मद्भक्तः सङ्गवर्जितः । \\
निर्वैरः सर्वभूतेषु यः स मामेति पाण्डव ॥५५॥
\end{gitaverse}

\begin{transliteration}
matkarmakṛnmatparamo madbhaktaḥ saṅgavarjitaḥ, \\
nirvairaḥ sarvabhūteṣu yaḥ sa māmeti pāṇḍava.
\end{transliteration}

55. He who does actions for Me, who looks upon Me as the Supreme, who is
devoted to Me, who is free from attachment, who bears enmity towards none, he
comes to Me, O Pandava.

\begin{gitaverse}
ॐ तत्सदिति श्रीमद् भगवद् गीतासूपनिषत्सु ब्रह्मविद्यायां \\
योगशास्त्रे श्रीकृष्णार्जुनसंवादे विश्वरूपदर्शनयोगो नाम \\
एकादशोऽध्यायः
\end{gitaverse}

\begin{transliteration}
oṁ tatsaditi śrīmad bhagavad gītāsūpaniṣatsu brahmavidyāyāṁ \\
yogaśāstre śrīkṛṣṇārjunasaṁvāde viśvarūpadarśanayogo nāma ekādaśo'dhyāyaḥ
\end{transliteration}

Thus, in the UPANISHADS of the glorious Bhagawad-Geeta, in the Science of the
Eternal, in the scripture of YOGA, in the dialogue between Sri Krishna and
Arjuna, the eleventh discourse ends entitled: The Yoga of the Vision of the
Universal Form
