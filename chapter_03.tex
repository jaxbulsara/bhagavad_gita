\chapterdrop

\begin{center}
\headersanskrit{अथ तृतीयोऽध्यायः}

\headerspace
\headertransliteration{Atha Tṛitīyo'dhyāyaḥ}

\section{Chapter 3}

\headerspace
\headersanskrit{कर्मयोगः}

\headerspace
\headertransliteration{Karma Yogah}

\headerspace
\headertranslation{The Yoga of Action}

\headerspace
\end{center}

\begin{gitaverse}
अर्जुन उवाच \\
ज्यायसी चेत्कर्मणस्ते मता बुद्धिर्जनार्दन । \\
तत्किं कर्मणि घोरे मां नियोजयसि केशव ॥१॥
\end{gitaverse}

\begin{transliteration}
arjuna uvāca \\
jyāyasī cetkarmaṇaste matā buddhirjanārdana, \\
tatkiṁ karmaṇi ghore māṁ niyojayasi keśava.
\end{transliteration}

Arjuna said: \\
1. If it be thought by you that `knowledge' is superior to `action', O
Janardana, why then, do you, O Keshava, engage me in this terrible action?

\begin{gitaverse}
व्यामिश्रेणेव वाक्येन बुद्धिं मोहयसीव मे । \\
तदेकं वद निश्चित्य येन श्रेयोऽहमाप्नुयाम् ॥२॥
\end{gitaverse}

\begin{transliteration}
vyāmiśreṇeva vākyena buddhiṁ mohayasīva me, \\
tadekaṁ vada niścitya yena śreyo'hamāpnuyām.
\end{transliteration}

2. With this apparently perplexing speech you confuse, as it were, my
understanding; therefore, tell me that ONE way by which, I, for certain, may
attain the Highest.

\begin{gitaverse}
श्रीभगवानुवाच \\
लोकेऽस्मिन्द्विविधा निष्ठा पुरा प्रोक्ता मयानघ । \\
ज्ञानयोगेन साङ्ख्यानां कर्मयोगेन योगिनाम् ॥३॥
\end{gitaverse}

\begin{transliteration}
śribhagavānuvāca \\
loke'smindvividhā niṣṭhā purā proktā mayānagha, \\
jñānayogena sāṅkhyānāṁ karmayogena yoginām.
\end{transliteration}

The Blessed Lord said; \\
3. In this world there is a two-fold path, as I said before, O sinless one; the
`Path-of-Knowledge' of the SANKHYANS and the `Path-of-Action' of the YOGINS.\@

\begin{gitaverse}
न कर्मणामनारम्भान्नैष्कर्म्यं पुरुषोऽश्नुते । \\
न च सन्न्यसनादेव सिद्धिं समधिगच्छति ॥४॥
\end{gitaverse}

\begin{transliteration}
na karmaṇāmanārambhānnaiṣkarmyaṁ puruṣo'śnute, \\
na ca sannyasanādeva siddhiṁ samadhigacchati.
\end{transliteration}

4. Not by non-performance of actions does man reach `actionlessness'; nor by
mere renunciation does he attain `Perfection'.

\begin{gitaverse}
न हि कश्चित्क्षणमपि जातु तिष्ठत्यकर्मकृत् । \\
कार्यते ह्यवशः कर्म सर्वः प्रकृतिजैर्गुणैः ॥५॥
\end{gitaverse}

\begin{transliteration}
na hi kaścitkṣaṇamapi jātu tiṣṭhatyakarmakṛt, \\
kāryate hyavaśaḥ karma sarvaḥ prakṛtijairguṇaiḥ.
\end{transliteration}

5. Verily, none can ever remain, even for a moment, without performing action;
for, everyone is made to act helplessly, indeed, by the qualities born of
PRAKRITI.\@

\begin{gitaverse}
कर्मेन्द्रियाणि संयम्य य आस्ते मनसा स्मरन् । \\
इन्द्रियार्थान्विमूढात्मा मिथ्याचारः स उच्यते ॥६॥
\end{gitaverse}

\begin{transliteration}
karmendriyāṇi saṁyamya ya āste manasā smaran, \\
indriyārthānvimūḍhātmā mithyācāraḥ sa ucyate.
\end{transliteration}

6. He who, restraining the organs-of-action, sits thinking in his mind of the
sense-objects, he, of deluded understanding, is called a hypocrite.

\begin{gitaverse}
यस्त्विन्द्रियाणि मनसा नियम्यारभतेऽर्जुन । \\
कर्मेन्द्रियैः कर्मयोगमसक्तः स विशिष्यते ॥७॥
\end{gitaverse}

\begin{transliteration}
yastvindriyāṇi manasā niyamyārabhate'rjuna, \\
karmendriyaiḥ karmayogamasaktaḥ sa viśiṣyate.
\end{transliteration}

7. But, whosoever, controlling the senses by the mind, O Arjuna, engages his
organs-of-action in KARMA YOGA, without attachment, he excels.

\begin{gitaverse}
नियतं कुरु कर्म त्वं कर्म ज्यायो ह्यकर्मणः । \\
शरीरयात्रापि च ते न प्रसिद्ध्येदकर्मणः ॥८॥
\end{gitaverse}

\begin{transliteration}
niyataṁ kuru karma tvaṁ karma jyāyo hyakarmaṇaḥ, \\
śarīrayātrāpi ca te na prasiddhyedakarmaṇaḥ.
\end{transliteration}

8. You perform (your) bounden duty; for, action is superior to inaction. Even
the maintenance of the body would not be possible for you by inaction.

\begin{gitaverse}
यज्ञार्थात्कर्मणोऽन्यत्र लोकोऽयं कर्मबन्धनः । \\
तदर्थं कर्म कौन्तेय मुक्तसङ्गः समाचर ॥९॥
\end{gitaverse}

\begin{transliteration}
yajñārthātkarmaṇo'nyatra loko'yaṁ karmabandhanaḥ, \\
tadarthaṁ karma kaunteya muktasaṅgaḥ samācara.
\end{transliteration}

9. The world is bound by actions other than those performed `for the sake of
sacrifice'; do thou, therefore, O son of Kunti, perform action for that sake
(for YAJNA) alone, free from all attachments.

\begin{gitaverse}
सहयज्ञाः प्रजाः सृष्ट्वा पुरोवाच प्रजापतिः । \\
अनेन प्रसविष्यध्वमेष वोऽस्त्विष्टकामधुक् ॥१०॥
\end{gitaverse}

\begin{transliteration}
sahayajñāḥ prajāḥ sṛṣṭvā purovāca prajāpatiḥ, \\
anena prasaviṣyadhvameṣa vo'stviṣṭakāmadhuk.
\end{transliteration}

10. The PRAJAPATI (the Creator), having in the beginning (of creation) created
mankind together with sacrifices, said, ``by this shall you prosper; let this
be the milch-cow of your desire---KAMADHUK'' (the mythological cow which yields
all desired objects).

\begin{gitaverse}
देवान्भावयतानेन ते देवा भावयन्तु वः । \\
परस्परं भावयन्तः श्रेयः परमवाप्स्यथ ॥११॥
\end{gitaverse}

\begin{transliteration}
devānbhāvayatānena te devā bhāvayantu vaḥ, \\
parasparaṁ bhāvayantaḥ śreyaḥ paramavāpsyatha.
\end{transliteration}

11. With this, you do nourish the gods and may those DEVAS nourish you; thus
nourishing one another, you shall, attain the Highest Good.

\begin{gitaverse}
इष्टान्भोगान्हि वो देवा दास्यन्ते यज्ञभाविताः । \\
तैर्दत्तानप्रदायैभ्यो यो भुङ्क्ते स्तेन एव सः ॥१२॥
\end{gitaverse}

\begin{transliteration}
iṣṭānbhogānhi vo devā dāsyante yajñabhāvitāḥ, \\
tairdattānapradāyaibhyo yo bhuṅkte stena eva saḥ.
\end{transliteration}

12. The DEVAS, nourished by the sacrifice, will give you the desired objects.
Indeed, he who enjoys objects given by the DEVAS, without offering (in return)
to them, is verily a thief.

\begin{gitaverse}
यज्ञशिष्टाशिनः सन्तो मुच्यन्ते सर्वकिल्बिषैः । \\
भुञ्जते ते त्वघं पापा ये पचन्त्यात्मकारणात् ॥१३॥
\end{gitaverse}

\begin{transliteration}
yajñaśiṣṭāśinaḥ santo mucyante sarvakilbiṣaiḥ, \\
bhuñjate te tvaghaṁ pāpā ye pacantyātmakāraṇāt.
\end{transliteration}

13. The righteous, who eat the `remnants of the sacrifices' are freed from all
sins; but those sinful ones, who cook food (only) for their own sake, verily
eat but sin.

\begin{gitaverse}
अन्नाद्भवन्ति भूतानि पर्जन्यादन्नसम्भवः । \\
यज्ञाद्भवति पर्जन्यो यज्ञः कर्मसमुद्भवः ॥१४॥
\end{gitaverse}

\begin{transliteration}
annādbhavanti bhūtāni parjanyādannasambhavaḥ, \\
yajñādbhavati parjanyo yajñaḥ karmasamudbhavaḥ.
\end{transliteration}

14. From food come forth beings; from rain food is produced; from sacrifice
arises rain, and sacrifice is born of action.

\begin{gitaverse}
कर्म ब्रह्मोद्भवं विद्धि ब्रह्माक्षरसमुद्भवम् । \\
तस्मात्सर्वगतं ब्रह्म नित्यं यज्ञे प्रतिष्ठितम् ॥१५॥
\end{gitaverse}

\begin{transliteration}
karma brahmodbhavaṁ viddhi brahmākṣarasamudbhavam, \\
tasmātsarvagataṁ brahma nityaṁ yajñe pratiṣṭhitam.
\end{transliteration}

15. Know you that action comes from BRAHMAJI (the Creator) and BRAHMAJI come
from the Imperishable. Therefore, the all-pervading BRAHMAN (God-principle)
ever rests in sacrifice.

\begin{gitaverse}
एवं प्रवर्तितं चक्रं नानुवर्तयतीह यः । \\
अघायुरिन्द्रियारामो मोघं पार्थ स जीवति ॥१६॥
\end{gitaverse}

\begin{transliteration}
evaṁ pravartitaṁ cakraṁ nānuvartayatīha yaḥ, \\
aghāyurindriyārāmo moghaṁ pārtha sa jīvati.
\end{transliteration}

16. He who does not follow here the wheel thus set revolving, is of a sinful
life, rejoicing in the senses. He lives in vain, O Son of Pritha.

\begin{gitaverse}
यस्त्वात्मरतिरेव स्यादात्मतृप्तश्च मानवः । \\
आत्मन्येव च सन्तुष्टस्तस्य कार्यं न विद्यते ॥१७॥
\end{gitaverse}

\begin{transliteration}
yastvātmaratireva syādātmatṛptaśca mānavaḥ, \\
ātmanyeva ca santuṣṭastasya kāryaṁ na vidyate.
\end{transliteration}

17. But the man who rejoices only in the Self, who is satisfied with the Self,
who is content in the Self alone, for Him verily there is nothing (more) to be
done.

\begin{gitaverse}
नैव तस्य कृतेनार्थो नाकृतेनेह कश्चन । \\
न चास्य सर्वभूतेषु कश्चिदर्थव्यपाश्रयः ॥१८॥
\end{gitaverse}

\begin{transliteration}
naiva tasya kṛtenārtho nākṛteneha kaścana, \\
na cāsya sarvabhūteṣu kaścidarthavyapāśrayaḥ.
\end{transliteration}

18. For him there is here no interest whatever in what is done, or what is not
done; nor does he depend upon any being for any object.

\begin{gitaverse}
तस्मादसक्तः सततं कार्यं कर्म समाचर । \\
असक्तो ह्याचरन्कर्म परमाप्नोति पूरुषः ॥१९॥
\end{gitaverse}

\begin{transliteration}
tasmādasaktaḥ satataṁ kāryaṁ karma samācara, \\
asakto hyācarankarma paramāpnoti pūruṣaḥ.
\end{transliteration}

19. Therefore, always perform actions which should be done, without attachment;
for, by performing action without attachment, man attains the Supreme.

\begin{gitaverse}
कर्मणैव हि संसिद्धिमास्थिता जनकादयः । \\
लोकसङ्ग्रहमेवापि सम्पश्यन्कर्तुमर्हसि ॥२०॥
\end{gitaverse}

\begin{transliteration}
karmaṇaiva hi saṁsiddhimāsthitā janakādayaḥ, \\
lokasaṅgrahamevāpi sampaśyankartumarhasi.
\end{transliteration}

20. Janaka and others attained Perfection verily through action alone; even
with a view to protecting the masses you should perform action.

\begin{gitaverse}
यद्यदाचरति श्रेष्ठस्तत्तदेवेतरो जनः । \\
स यत्प्रमाणं कुरुते लोकस्तदनुवर्तते ॥२१॥
\end{gitaverse}

\begin{transliteration}
yadyadācarati śreṣṭhastattadevetaro janaḥ, \\
sa yatpramāṇaṁ kurute lokastadanuvartate.
\end{transliteration}

21. Whatever a great man does, that other men also do (imitate); whatever he
sets up as the standard, that the world (people) follows.

\begin{gitaverse}
न मे पार्थास्ति कर्तव्यं त्रिषु लोकेषु किञ्चन । \\
नानवाप्तमवाप्तव्यं वर्त एव च कर्मणि ॥२२॥
\end{gitaverse}

\begin{transliteration}
na me pārthāsti kartavyaṁ triṣu lokeṣu kiñcana, \\
nānavāptamavāptavyaṁ varta eva ca karmaṇi.
\end{transliteration}

22. There is nothing in the three worlds, O Partha, that has to be done by Me,
nor is there anything unattained that should be attained by Me; yet, I engage
Myself in action

\begin{gitaverse}
यदि ह्यहं न वर्तेयं जातु कर्मण्यतन्द्रितः । \\
मम वर्त्मानुवर्तन्ते मनुष्याः पार्थ सर्वशः ॥२३॥
\end{gitaverse}

\begin{transliteration}
yadi hyahaṁ na varteyaṁ jātu karmaṇyatandritaḥ, \\
mama vartmānuvartante manuṣyāḥ pārtha sarvaśaḥ.
\end{transliteration}

23. For, should I not ever engage Myself in action, without relaxation, men
would in every way follow My Path, O son of Pritha.

\begin{gitaverse}
उत्सीदेयुरिमे लोका न कुर्यां कर्म चेदहम् । \\
सङ्करस्य च कर्ता स्यामुपहन्यामिमाः प्रजाः ॥२४॥
\end{gitaverse}

\begin{transliteration}
utsīdeyurime lokā na kuryāṁ karma cedaham, \\
saṅkarasya ca kartā syāmupahanyāmimāḥ prajāḥ.
\end{transliteration}

24. These worlds would perish if I did not perform action; I would be the
author of confusion of `castes', and would destroy these beings.

\begin{gitaverse}
सक्ताः कर्मण्यविद्वांसो यथा कुर्वन्ति भारत । \\
कुर्याद्विद्वांस्तथासक्तश्चिकीर्षुर्लोकसङ्ग्रहम् ॥२५॥
\end{gitaverse}

\begin{transliteration}
saktāḥ karmaṇyavidvāṁso yathā kurvanti bhārata, \\
kuryādvidvāṁstathāsaktaścikīrṣurlokasaṅgraham.
\end{transliteration}

25. As the `ignorant' men act from attachment to action, O Bharata, so should
the `wise' men act without attachment, wishing the welfare of the world.

\begin{gitaverse}
न बुद्धिभेदं जनयेदज्ञानां कर्मसङ्गिनाम् । \\
जोषयेत्सर्वकर्माणि विद्वान्युक्तः समाचरन् ॥२६॥
\end{gitaverse}

\begin{transliteration}
na buddhibhedaṁ janayedajñānāṁ karmasaṅginām, \\
joṣayetsarvakarmāṇi vidvānyuktaḥ samācaran.
\end{transliteration}

26. Let no wise man unsettle the minds of ignorant people, who are attached to
action; he should engage them all in actions, himself fulfilling them with
devotion.

\begin{gitaverse}
प्रकृतेः क्रियमाणानि गुणैः कर्माणि सर्वशः । \\
अहङ्कारविमूढात्मा कर्ताहमिति मन्यते ॥२७॥
\end{gitaverse}

\begin{transliteration}
prakṛteḥ kriyamāṇāni guṇaiḥ karmāṇi sarvaśaḥ, \\
ahaṅkāravimūḍhātmā kartāhamiti manyate.
\end{transliteration}

27. All actions are performed, in all cases, merely by the Qualities-in-Nature
(GUNAS). He whose mind is deluded by egoism, thinks ``I am the doer''.

\begin{gitaverse}
तत्त्ववित्तु महाबाहो गुणकर्मविभागयोः । \\
गुणा गुणेषु वर्तन्त इति मत्वा न सज्जते ॥२८॥
\end{gitaverse}

\begin{transliteration}
tattvavittu mahābāho guṇakarmavibhāgayoḥ, \\
guṇā guṇeṣu vartanta iti matvā na sajjate.
\end{transliteration}

28. But he---who knows the Truth, O mighty-armed, about the divisions of the
qualities and (their) functions, and he who knows that GUNAS-as-senses move
amidst GUNAS-asobjects, is not attached.

\begin{gitaverse}
प्रकृतेर्गुणसम्मूढाः सज्जन्ते गुणकर्मसु । \\
तानकृत्स्नविदो मन्दान्कृत्स्नविन्न विचालयेत् ॥२९॥
\end{gitaverse}

\begin{transliteration}
prakṛterguṇasammūḍhāḥ sajjante guṇakarmasu, \\
tānakṛtsnavido mandānkṛtsnavinna vicālayet.
\end{transliteration}

29. Those deluded by the qualities of nature, (GUNAS), are attached to the
functions of the qualities. The man-of-Perfect-Knowledge should not unsettle
the `foolish', who are of imperfect knowledge.

\begin{gitaverse}
मयि सर्वाणि कर्माणि सन्न्यस्याध्यात्मचेतसा । \\
निराशीर्निर्ममो भूत्वा युध्यस्व विगतज्वरः ॥३०॥
\end{gitaverse}

\begin{transliteration}
mayi sarvāṇi karmāṇi sannyasyādhyātmacetasā, \\
nirāśīrnirmamo bhūtvā yudhyasva vigatajvaraḥ.
\end{transliteration}

30. Renouncing all actions in Me, with the mind centered on the Self, free from
hope and egoism (ownership), free from (mental) fever, (you) do fight!

\begin{gitaverse}
ये मे मतमिदं नित्यमनुतिष्ठन्ति मानवाः । \\
श्रद्धावन्तोऽनसूयन्तो मुच्यन्ते तेऽपि कर्मभिः ॥३१॥
\end{gitaverse}

\begin{transliteration}
ye me matamidaṁ nityamanutiṣṭhanti mānavāḥ, \\
śraddhāvanto'nasūyanto mucyante te'pi karmabhiḥ.
\end{transliteration}

31. Those men who constantly practise this teaching of Mine, full of faith and
without cavilling, they too are freed from actions.

\begin{gitaverse}
ये त्वेतदभ्यसूयन्तो नानुतिष्ठन्ति मे मतम् । \\
सर्वज्ञानविमूढांस्तान्विद्धि नष्टानचेतसः ॥३२॥
\end{gitaverse}

\begin{transliteration}
ye tvetadabhyasūyanto nānutiṣṭhanti me matam, \\
sarvajñānavimūḍhāṁstānviddhi naṣṭānacetasaḥ.
\end{transliteration}

32. But those who carp at My teaching and do not practise it, deluded in all
knowledge, and devoid of discrimination, know them to be doomed to destruction.

\begin{gitaverse}
सदृशं चेष्टते स्वस्याः प्रकृतेर्ज्ञानवानपि । \\
प्रकृतिं यान्ति भूतानि निग्रहः किं करिष्यति ॥३३॥
\end{gitaverse}

\begin{transliteration}
sadṛśaṁ ceṣṭate svasyāḥ prakṛterjñānavānapi, \\
prakṛtiṁ yānti bhūtāni nigrahaḥ kiṁ kariṣyati.
\end{transliteration}

33. Even a wise man acts in accordance with his own nature; beings will follow
their own nature; what can restraint do?

\begin{gitaverse}
इन्द्रियस्येन्द्रियस्यार्थे रागद्वेषौ व्यवस्थितौ । \\
तयोर्न वशमागच्छेत्तौ ह्यस्य परिपन्थिनौ ॥३४॥
\end{gitaverse}

\begin{transliteration}
indriyasyendriyasyārthe rāgadveṣau vyavasthitau, \\
tayorna vaśamāgacchettau hyasya paripanthinau.
\end{transliteration}

34. Attachment and aversion for the objects of the senses abide in the senses;
let none come under their sway; for they are his foes.

\begin{gitaverse}
श्रेयान्स्वधर्मो विगुणः परधर्मात्स्वनुष्ठितात् । \\
स्वधर्मे निधनं श्रेयः परधर्मो भयावहः ॥३५॥
\end{gitaverse}

\begin{transliteration}
śreyānsvadharmo viguṇaḥ paradharmātsvanuṣṭhitāt, \\
svadharme nidhanaṁ śreyaḥ paradharmo bhayāvahaḥ.
\end{transliteration}

35. Better is one's own `duty', though devoid of merit, than the `duty' of
another well discharged. Better is death in one's own `duty'; the `duty' of
another is fraught with fear (is productive of positive danger).

\begin{gitaverse}
अर्जुन उवाच \\
अथ केन प्रयुक्तोऽयं पापं चरति पूरुषः । \\
अनिच्छन्नपि वार्ष्णेय बलादिव नियोजितः ॥३६॥
\end{gitaverse}

\begin{transliteration}
arjuna uvāca \\
atha kena prayukto'yaṁ pāpaṁ carati pūruṣaḥ, \\
anicchannapi vārṣṇeya balādiva niyojitaḥ.
\end{transliteration}

Arjuna said: \\
36. But, impelled by what does man commit sin, though against his wishes, O
Varshneya, constrained, as it were, by force?

\begin{gitaverse}
श्रीभगवानुवाच \\
काम एष क्रोध एष रजोगुणसमुद्भवः । \\
महाशनो महापाप्मा विद्ध्येनमिह वैरिणम् ॥३७॥
\end{gitaverse}

\begin{transliteration}
śrībhagavānuvāca \\
kāma eṣa krodha eṣa rajoguṇasamudbhavaḥ, \\
mahāśano mahāpāpmā viddhyenamiha vairiṇam.
\end{transliteration}

The Blessed Lord Said: \\
37. It is desire, it is anger born of the `active', all-devouring, all-sinful;
know this as the foe here (in this world).

\begin{gitaverse}
धूमेनाव्रियते वह्निर्यथादर्शो मलेन च । \\
यथोल्बेनावृतो गर्भस्तथा तेनेदमावृतम् ॥३८॥
\end{gitaverse}

\begin{transliteration}
dhūmenāvriyate vahniryathādarśo malena ca, \\
yatholbenāvṛto garbhastathā tenedamāvṛtam.
\end{transliteration}

38. As fire is enveloped by smoke, as a mirror by dust, as an embryo by the
womb, so this (wisdom) is enveloped by that (desire or anger).

\begin{gitaverse}
आवृतं ज्ञानमेतेन ज्ञानिनो नित्यवैरिणा । \\
कामरूपेण कौन्तेय दुष्पूरेणानलेन च ॥३९॥
\end{gitaverse}

\begin{transliteration}
āvṛtaṁ jñānametena jñānino nityavairiṇā, \\
kāmarūpeṇa kaunteya duṣpūreṇānalena ca.
\end{transliteration}

39. Enveloped, O Son of Kunti, is `wisdom' by this constant enemy of the wise
in the form of `desire', which is difficult to appease.

\begin{gitaverse}
इन्द्रियाणि मनो बुद्धिरस्याधिष्ठानमुच्यते । \\
एतैर्विमोहयत्येष ज्ञानमावृत्य देहिनम् ॥४०॥
\end{gitaverse}

\begin{transliteration}
indriyāṇi mano buddhirasyādhiṣṭhānamucyate, \\
etairvimohayatyeṣa jñānamāvṛtya dehinam.
\end{transliteration}

40. The senses, the mind, and the intellect are said to be its seat; through
these, it deludes the embodied, by veiling his wisdom.

\begin{gitaverse}
तस्मात्त्वमिन्द्रियाण्यादौ नियम्य भरतर्षभ । \\
पाप्मानं प्रजहि ह्येनं ज्ञानविज्ञाननाशनम् ॥४१॥
\end{gitaverse}

\begin{transliteration}
tasmāttvamindriyāṇyādau niyamya bharatarṣabha, \\
pāpmānaṁ prajahi hyenaṁ jñānavijñānanāśanam.
\end{transliteration}

41. Therefore, O best of the Bharatas, controlling first the senses, kill this
sinful thing, the destroyer of knowledge and wisdom.

\begin{gitaverse}
इन्द्रियाणि पराण्याहुरिन्द्रियेभ्यः परं मनः । \\
मनसस्तु परा बुद्धिर्यो बुद्धेः परतस्तु सः ॥४२॥
\end{gitaverse}

\begin{transliteration}
indriyāṇi parāṇyāhurindriyebhyaḥ paraṁ manaḥ, \\
manasastu parā buddhiryo buddheḥ paratastu saḥ.
\end{transliteration}

42. They say that the senses are superior (to the body); superior to the senses
is the mind; superior to the mind is the intellect; one who is even superior to
the intellect is He, (the Atman).

\begin{gitaverse}
एवं बुद्धेः परं बुद्ध्वा संस्तभ्यात्मानमात्मना । \\
जहि शत्रुं महाबाहो कामरूपं दुरासदम् ॥४३॥
\end{gitaverse}

\begin{transliteration}
evaṁ buddheḥ paraṁ buddhvā saṁstabhyātmānamātmanā, \\
jahi śatruṁ mahābāho kāmarūpaṁ durāsadam.
\end{transliteration}

43. Thus knowing Him, who is superior to intellect, and restraining the self by
the Self, slay you, O mighty-armed, the enemy in the form of `desire', no
doubt, hard indeed to conquer.

\begin{gitaverse}
ॐ तत्सदिति श्रीमद् भगवद् गीतासूपनिषत्सु ब्रह्मविद्यायां \\
योगशास्त्रे श्रीकृष्णार्जुनसंवादे कर्मयोगो नाम \\
तृतीयोध्यायः
\end{gitaverse}

\begin{transliteration}
oṁ tatsaditi śrīmad bhagavad gītāsūpaniṣatsu brahmavidyāyāṁ \\
yogaśāstre śrīkṛṣṇārjunasaṁvāde `Karmayogo' nāma \\
tṛitīyo'dhyāyaḥ.
\end{transliteration}

Thus in the UPANISHADS of the glorious Bhagawad Geeta, in the Science of the
Eternal, in the Scripture of YOGA, in the dialogue between Sri Krishna and
Arjuna, the third discourse ends entitled: The Karma Yoga
