\chapterdrop

\begin{center}
\headersanskrit{अथ सप्तदशोऽध्यायः}

\headerspace
\headertransliteration{Atha Saptadaśo'dhyāyaḥ}

\section{Chapter 17}

\headerspace
\headersanskrit{श्रद्धात्रयविभागयोगः}

\headerspace
\headertransliteration{Śraddhā Traya Vibhāga Yogah}

\headerspace
\headertranslation{The Three-Fold Faith}

\headerspace
\end{center}

\begin{gitaverse}
अर्जुन उवाच \\
ये शास्त्रविधिमुत्सृज्य यजन्ते श्रद्धयान्विताः । \\
तेषां निष्ठा तु का कृष्ण सत्त्वमाहो रजस्तमः ॥१॥
\end{gitaverse}

\begin{transliteration}
arjuna uvāca \\
ye śāstravidhim-utsṛjya yajante śraddhayānvitāḥ, \\
teṣāṁ niṣṭhā tu kā kṛṣṇa sattvamāho rajas-tamaḥ.
\end{transliteration}

Arjuna said: \\
1. Those who, setting aside the ordinances of the scriptures, perform sacrifice
with faith, what is their condition, O Krishna? Is it SATTVA, RAJAS or TAMAS?\@

\begin{gitaverse}
श्रीभगवानुवाच \\
त्रिविधा भवति श्रद्धा देहिनां सा स्वभावजा । \\
सात्त्विकी राजसी चैव तामसी चेति तां श्रृणु ॥२॥
\end{gitaverse}

\begin{transliteration}
śrībhagavānuvāca \\
trividhā bhavati śraddhā dehināṁ sā svabhāvajā, \\
sāttvikī rājasī caiva tāmasī ceti tāṁ śṛṇu.
\end{transliteration}

The Blessed Lord said: \\
2. Threefold is the faith of the embodied, which is inherent in their
nature---the SATTVIC (pure), the RAJASIC (passionate) and the TAMASIC (dull,
dark). Thus thou hear of it.

\begin{gitaverse}
सत्त्वानुरूपा सर्वस्य श्रद्धा भवति भारत । \\
श्रद्धामयोऽयं पुरुषो यो यच्छ्रद्धः स एव सः ॥३॥
\end{gitaverse}

\begin{transliteration}
sattvānurūpā sarvasya śraddhā bhavati bhārata, \\
śraddhāmayo'yaṁ puruṣo yo yacchraddhaḥ sa eva saḥ.
\end{transliteration}

3. The faith of each is in accordance with his nature, O Bharata. Man consists
of his faith; as a man's faith is, so is he.

\begin{gitaverse}
यजन्ते सात्त्विका देवान्यक्षरक्षांसि राजसाः । \\
प्रेतान्भूतगणांश्चान्ये यजन्ते तामसा जनाः ॥४॥
\end{gitaverse}

\begin{transliteration}
yajante sāttvikā devānyakṣarakṣāṁsi rājasāḥ, \\
pretānbhūtagaṇāṁscānye yajante tāmasā janāḥ.
\end{transliteration}

4. The SATTVIC, or `pure' men, worship the gods (DEVAS); the RAJASIC or the
`passionate', the YAKSHAS and the RAKSHASAS;\@ the others---TAMASIC people, or
the `dark or dull' folk, worship the ghosts (PRETAS) and the hosts of BHUTAS,
or the nature / spirits.

\begin{gitaverse}
अशास्त्रविहितं घोरं तप्यन्ते ये तपो जनाः । \\
दम्भाहङ्कारसंयुक्ताः कामरागबलान्विताः ॥५॥
\end{gitaverse}

\begin{transliteration}
aśāstravihitaṁ ghoraṁ tapyante ye tapo janāḥ, \\
dambhāhaṅkārasaṁyuktāḥ kāmarāgabalānvitāḥ.
\end{transliteration}

5. Those men who practise terrible austerities, not enjoined by the scriptures,
given to hypocrisy and egoism, impelled by the force of lust and
attachment\ldots

\begin{gitaverse}
कर्शयन्तः शरीरस्थं भूतग्राममचेतसः । \\
मां चैवान्तःशरीरस्थं तान्विद्ध्यासुरनिश्चयान् ॥६॥
\end{gitaverse}

\begin{transliteration}
karśayantaḥ śarīrasthaṁ bhūtagrāmam-acetasaḥ, \\
māṁ caivāntaḥśarīrasthaṁ tānviddhyāsuraniścayān.
\end{transliteration}

6. Senselessly torturing all the elements in the body, and Me also who dwells
within the body---you may know these to be of `demoniacal' resolves.

\begin{gitaverse}
आहारस्त्वपि सर्वस्य त्रिविधो भवति प्रियः । \\
यज्ञस्तपस्तथा दानं तेषां भेदमिमं श्रृणु ॥७॥
\end{gitaverse}

\begin{transliteration}
āhārastvapi sarvasya trividho bhavati priyaḥ, \\
yajñastapastathā dānaṁ teṣāṁ bhedamimaṁ śṛṇu.
\end{transliteration}

7. The food also which is dear to each is threefold, as also sacrifice,
austerity and alms-giving. You may now hear the distinction of these.

\begin{gitaverse}
आयुःसत्त्वबलारोग्यसुखप्रीतिविवर्धनाः । \\
रस्याः स्निग्धाः स्थिरा हृद्या आहाराः सात्त्विकप्रियाः ॥८॥
\end{gitaverse}

\begin{transliteration}
āyuḥ-sattva-balārogya-sukha-prīti-vivardhanāḥ, \\
rasyāḥ snigdhāḥ sthirā hṛdyā āhārāḥ sāttvikapriyāḥ.
\end{transliteration}

8. The foods which increase life, purity, strength, health, joy and
cheerfulness (good appetite), which are savoury and oleaginous, substantial and
agreeable, are dear to the SATTVIC (Pure).

\begin{gitaverse}
कट्वम्ललवणात्युष्णतीक्ष्णरूक्षविदाहिनः । \\
आहारा राजसस्येष्टा दुःखशोकामयप्रदाः ॥९॥
\end{gitaverse}

\begin{transliteration}
kaṭvamla-lavaṇātyuṣṇa-tīkṣṇa-rūkṣa-vidāhinaḥ, \\
āhārā rājasasyeṣṭā duḥkha-śokāmayapradāḥ.
\end{transliteration}

9. The foods that are bitter, sour, saline, excessively hot, pungent, dry and
burning, are liked by the RAJASIC, and are productive of pain, grief and
disease.

\begin{gitaverse}
यातयामं गतरसं पूति पर्युषितं च यत् । \\
उच्छिष्टमपि चामेध्यं भोजनं तामसप्रियम् ॥१०॥
\end{gitaverse}

\begin{transliteration}
yātayāmaṁ gatarasaṁ pūti paryuṣitaṁ ca yat, \\
ucchiṣṭam-api cāmedhyaṁ bhojanaṁ tāmasapriyam.
\end{transliteration}

10. That which is stale, tasteless, putrid and rotten, refuse and impure, is
the food liked by the `TAMASIC'.

\begin{gitaverse}
अफलाकाङ्क्षिभिर्यज्ञो विधिदृष्टो य इज्यते । \\
यष्टव्यमेवेति मनः समाधाय स सात्त्विकः ॥११॥
\end{gitaverse}

\begin{transliteration}
aphalākāṅkṣibhir-yajño vidhidṛṣṭo ya ijyate, \\
yaṣṭavyam-eveti manaḥ samādhāya sa sāttvikaḥ.
\end{transliteration}

11. That sacrifice which is offered by men without desire for fruit, and as
enjoined by ordinance, with a firm faith that sacrifice is a duty, is SATTVIC
or `pure'.

\begin{gitaverse}
अभिसन्धाय तु फलं दम्भार्थमपि चैव यत् । \\
इज्यते भरतश्रेष्ठ तं यज्ञं विद्धि राजसम् ॥१२॥
\end{gitaverse}

\begin{transliteration}
abhisandhāya tu phalaṁ dambhārtham-api caiva yat, \\
ijyate bharataśreṣṭha taṁ yajñaṁ viddhi rājasam.
\end{transliteration}

12. The sacrifice which is offered, O best of the Bharatas, seeking fruit and
for ostentation, you may know that to be a RAJASIC YAJNA.\@

\begin{gitaverse}
विधिहीनमसृष्टान्नं मन्त्रहीनमदक्षिणम् । \\
श्रद्धाविरहितं यज्ञं तामसं परिचक्षते ॥१३॥
\end{gitaverse}

\begin{transliteration}
vidhihīnam-asṛṣṭānnaṁ mantrahīnam-adakṣiṇam, \\
śraddhāvirahitaṁ yajñaṁ tāmasaṁ paricakṣate.
\end{transliteration}

13. They declare that sacrifice to be TAMASIC which is contrary to the
ordinances, in which no food is distributed, which is devoid of MANTRAS and
gifts, and which is devoid of faith.

\begin{gitaverse}
देवद्विजगुरुप्राज्ञपूजनं शौचमार्जवम् । \\
ब्रह्मचर्यमहिंसा च शारीरं तप उच्यते ॥१४॥
\end{gitaverse}

\begin{transliteration}
deva-dvija-guru-prājña-pūjanaṁ śaucam-ārjavam, \\
brahmacaryam-ahiṁsā ca śārīraṁ tapa ucyate.
\end{transliteration}

14. Worship of the gods, the twice-born, the teachers and the `wise'; purity,
straight-forwardness, celibacy, and non-injury; these are called the `austerity
of the body'.

\begin{gitaverse}
अनुद्वेगकरं वाक्यं सत्यं प्रियहितं च यत् । \\
स्वाध्यायाभ्यसनं चैव वाङ्मयं तप उच्यते ॥१५॥
\end{gitaverse}

\begin{transliteration}
anudvegakaraṁ vākyaṁ satyaṁ priyahitaṁ ca yat, \\
svādhyāyābhyasanaṁ caiva vāṅmayaṁ tapa ucyate.
\end{transliteration}

15. Speech which causes no excitement, and is truthful, pleasant and
beneficial, and the practice of the study of the VEDAS, these constitute the
`austerity of speech'.

\begin{gitaverse}
मनःप्रसादः सौम्यत्वं मौनमात्मविनिग्रहः । \\
भावसंशुद्धिरित्येतत्तपो मानसमुच्यते ॥१६॥
\end{gitaverse}

\begin{transliteration}
manaḥ-prasādaḥ saumyatvaṁ maunam-ātmavinigrahaḥ, \\
bhāvasaṁśuddhirityetat-tapo mānasam-ucyate.
\end{transliteration}

16. Serenity of mind, good-heartedness, silence, self-control, purity of
nature---these together are called the `mental austerity'.

\begin{gitaverse}
श्रद्धया परया तप्तं तपस्तत्त्रिविधं नरैः । \\
अफलाकाङ्क्षिभिर्युक्तैः सात्त्विकं परिचक्षते ॥१७॥
\end{gitaverse}

\begin{transliteration}
śraddhayā parayā taptaṁ tapas-tat-trividhaṁ naraiḥ, \\
aphalākāṅkṣibhir-yuktaiḥ sāttvikaṁ paricakṣate.
\end{transliteration}

17. This three-fold austerity, practised by steadfast men, with the utmost
faith, desiring no fruit, they call `SATTVIC'.

\begin{gitaverse}
सत्कारमानपूजार्थं तपो दम्भेन चैव यत् । \\
क्रियते तदिह प्रोक्तं राजसं चलमध्रुवम् ॥१८॥
\end{gitaverse}

\begin{transliteration}
satkāra-māna-pūjārthaṁ tapo dambhena caiva yat, \\
kriyate tadiha proktaṁ rājasaṁ calam-adhruvam.
\end{transliteration}

18. The austerity which is practised with the object of gaining good reception,
honour and worship, and with hypocrisy, is here said to be RAJASIC, unstable,
and transitory.

\begin{gitaverse}
मूढग्राहेणात्मनो यत्पीडया क्रियते तपः । \\
परस्योत्सादनार्थं वा तत्तामसमुदाहृतम् ॥१९॥
\end{gitaverse}

\begin{transliteration}
mūḍhagrāheṇātmano yat-pīḍayā kriyate tapaḥ, \\
parasyotsādanārthaṁ vā tat-tāmasam-udāhṛtam.
\end{transliteration}

19. That austerity which is practised with self-torture, out of some foolish
notion, for the purpose of destroying another, is declared to be TAMASIC.\@

\begin{gitaverse}
दातव्यमिति यद्दानं दीयतेऽनुपकारिणे । \\
देशे काले च पात्रे च तद्दानं सात्त्विकं स्मृतम् ॥२०॥
\end{gitaverse}

\begin{transliteration}
dātavyam-iti yad-dānaṁ dīyate'nupakāriṇe, \\
deśe kāle ca pātre ca tad-dānaṁ sāttvikaṁ smṛtam.
\end{transliteration}

20. That gift which is given, knowing it to be a duty, in a fit time and place,
to a worthy person, from whom we expect nothing in return, is held to be
SATTVIC.\@

\begin{gitaverse}
यत्तु प्रत्युपकारार्थं फलमुद्दिश्य वा पुनः । \\
दीयते च परिक्लिष्टं तद्दानं राजसं स्मृतम् ॥२१॥
\end{gitaverse}

\begin{transliteration}
yat-tu pratyupakārārthaṁ phalam-uddiśya vā punaḥ, \\
dīyate ca parikliṣṭaṁ tad-dānaṁ rājasaṁ smṛtam.
\end{transliteration}

21. And that gift which is given with a view to receiving in return, or looking
for fruit again, or reluctantly, is held to be RAJASIC.\@

\begin{gitaverse}
अदेशकाले यद्दानमपात्रेभ्यश्च दीयते । \\
असत्कृतमवज्ञातं तत्तामसमुदाहृतम् ॥२२॥
\end{gitaverse}

\begin{transliteration}
adeśakāle yad-dānam-apātrebhyaśca dīyate, \\
asatkṛtam-avajñātaṁ tat-tāmasam-udāhṛtam.
\end{transliteration}

22. The gift that is given at a wrong place and time, to unworthy persons,
without respect, or with insult, is declared to be TAMASIC.\@

\begin{gitaverse}
ॐ तत्सदिति निर्देशो ब्रह्मणस्त्रिविधः स्मृतः । \\
ब्राह्मणास्तेन वेदाश्च यज्ञाश्च विहिताः पुरा ॥२३॥
\end{gitaverse}

\begin{transliteration}
om tatsad-iti nirdeśo brahmaṇas-trividhaḥ smṛtaḥ, \\
brāhmaṇās-tena vedāśca yajñāśca vihitāḥ purā.
\end{transliteration}

23. ``OM TAT SAT''---this has been declared to be the triple designation of
BRAHMAN.\@ By that were created formerly, the BRAHMANAS, VEDAS and YAJNAS
(sacrifices).

\begin{gitaverse}
तस्मादोमित्युदाहृत्य यज्ञदानतपःक्रियाः । \\
प्रवर्तन्ते विधानोक्ताः सततं ब्रह्मवादिनाम् ॥२४॥
\end{gitaverse}

\begin{transliteration}
tasmād-omityudāhṛtya yajña-dāna-tapaḥ kriyāḥ, \\
pravartante vidhānoktāḥ satataṁ brahmavādinām.
\end{transliteration}

24. Therefore, with the utterance of `OM' are begun the acts of sacrifice,
gifts and austerity as enjoined in the scriptures, always by the students of
BRAHMAN.\@

\begin{gitaverse}
तदित्यनभिसन्धाय फलं यज्ञतपःक्रियाः । \\
दानक्रियाश्च विविधाः क्रियन्ते मोक्षकाङ्क्षिभिः ॥२५॥
\end{gitaverse}

\begin{transliteration}
tad-ityanabhisandhāya phalaṁ yajña-tapaḥ-kriyāḥ, \\
dāna-kriyāśca vividhāḥ kriyante mokṣa-kāṅkṣibhiḥ.
\end{transliteration}

25. Uttering `TAT' without aiming at the fruits, are the acts of sacrifice and
austerity and the various acts of gift performed by the seekers of liberation.

\begin{gitaverse}
सद्भावे साधुभावे च सदित्येतत्प्रयुज्यते । \\
प्रशस्ते कर्मणि तथा सच्छब्दः पार्थ युज्यते ॥२६॥
\end{gitaverse}

\begin{transliteration}
sadbhāve sādhubhāve ca sad-ityetat-prayujyate, \\
praśaste karmaṇi tathā sacchabdaḥ pārtha yujyate.
\end{transliteration}

26. The word `SAT' is used in the sense of Reality and of Goodness; and also, O
Partha, the word `SAT' is used in the sense of an auspicious act.

\begin{gitaverse}
यज्ञे तपसि दाने च स्थितिः सदिति चोच्यते । \\
कर्म चैव तदर्थीयं सदित्येवाभिधीयते ॥२७॥
\end{gitaverse}

\begin{transliteration}
yajñe tapasi dāne ca sthitiḥ saditi cocyate, \\
karma caiva tadarthīyaṁ sad-ityevābhidhīyate.
\end{transliteration}

27. Steadfastness in sacrifice, austerity and gift is also called `SAT' and
also, action in connection with these (for the sake of the Supreme) is called
`SAT'.

\begin{gitaverse}
अश्रद्धया हुतं दत्तं तपस्तप्तं कृतं च यत् । \\
असदित्युच्यते पार्थ न च तत्प्रेत्य नो इह ॥२८॥
\end{gitaverse}

\begin{transliteration}
aśraddhayā hutaṁ dattaṁ tapas-taptaṁ kṛtaṁ ca yat, \\
asad-ityucyate pārtha na ca tat-pretya no iha.
\end{transliteration}

28. Whatever is sacrificed, given or performed, and whatever austerity is
practised without faith, it is called `A-SAT', O Partha; it is not for here or
hereafter (after death).

\begin{gitaverse}
ॐ तत्सदिति श्रीमद् भगवद् गीतासूपनिषत्सु ब्रह्मविद्यायां \\
योगशास्त्रे श्रीकृष्णार्जुनसंवादे श्रद्धात्रयविभागयोगो नाम \\
सप्तदशोऽध्यायः
\end{gitaverse}

\begin{transliteration}
oṁ tatsaditi śrīmad bhagavad gītāsūpaniṣatsu brahmavidyāyāṁ \\
yogaśāstre śrīkṛṣṇārjunasaṁvāde śraddhātrayavibhāgayogo nāma \\
saptadaśo'dhyāyaḥ
\end{transliteration}

Thus, in the UPANISHADS of the glorious Bhagawad-Geeta, in the Science of the
Eternal, in the scripture of YOGA, in the dialogue between Sri Krishna and
Arjuna, the seventeenth discourse ends entitled: The Three-Fold Faith
