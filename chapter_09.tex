\chapterdrop

\begin{center}
\headersanskrit{अथ नवमोऽध्यायः}

\headerspace
\headertransliteration{Atha Navamo'dhyāyaḥ}

\section{Chapter 9}

\headerspace
\headersanskrit{राजविद्याराजगुह्ययोगः}

\headerspace
\headertransliteration{Rāja Vidyā Rāja Guhya Yogah}

\headerspace
\headertranslation{The Yoga of Royal Knowledge and Royal Secret}

\headerspace
\end{center}

\begin{gitaverse}
इदं तु ते गुह्यतमं प्रवक्ष्याम्यनसूयवे । \\
ज्ञानं विज्ञानसहितं यज्ज्ञात्वा मोक्ष्यसेऽशुभात् ॥१॥
\end{gitaverse}

\begin{transliteration}
idaṁ tu te guhyatamaṁ pravakṣyāmyanasūyave, \\
jñānaṁ vijñānasahitaṁ yajjñātvā mokṣyase'śubhāt.
\end{transliteration}

The Blessed Lord said: \\
1. To you who do not cavil, I shall now declare this, the greatest secret, the
most profound knowledge combined with experience (or realisation); which having
known, you shall be free from the sorrows of life.

\begin{gitaverse}
राजविद्या राजगुह्यं पवित्रमिदमुत्तमम् । \\
प्रत्यक्षावगमं धर्म्यं सुसुखं कर्तुमव्ययम् ॥२॥
\end{gitaverse}

\begin{transliteration}
rājavidyā rājaguhyaṁ pavitramidamuttamam, \\
pratyakṣāvagamaṁ dharmyaṁ susukhaṁ kartumavyayam.
\end{transliteration}

2. Royal Science, Royal Secret, the Supreme purifier is this, realisable by
direct intuitive knowledge, according to the DHARMA, very easy to perform,
imperishable.

\begin{gitaverse}
अश्रद्दधानाः पुरुषा धर्मस्यास्य परन्तप । \\
अप्राप्य मां निवर्तन्ते मृत्युसंसारवर्त्मनि ॥३॥
\end{gitaverse}

\begin{transliteration}
aśraddadhānāḥ puruṣā dharmasyāsya parantapa, \\
aprāpya māṁ nivartante mṛtyusaṁsāravartmani.
\end{transliteration}

3. Persons without faith in this DHARMA (the Divine Self), O Parantapa, without
attaining Me, return to the path of rebirth, fraught with death.

\begin{gitaverse}
मया ततमिदं सर्वं जगदव्यक्तमूर्तिना । \\
मत्स्थानि सर्वभूतानि न चाहं तेष्ववस्थितः ॥४॥
\end{gitaverse}

\begin{transliteration}
mayā tatamidaṁ sarvaṁ jagadavyaktamūrtinā, \\
matsthāni sarvabhūtāni na cāhaṁ teṣvavasthitaḥ.
\end{transliteration}

4. All this world is pervaded by Me in My Unmanifest form (aspect); all beings
exist in Me, but I do not dwell in them.

\begin{gitaverse}
न च मत्स्थानि भूतानि पश्य मे योगमैश्वरम् । \\
भूतभृन्न च भूतस्थो ममात्मा भूतभावनः ॥५॥
\end{gitaverse}

\begin{transliteration}
na ca matsthāni bhūtāni paśya me yogamaiśvaram, \\
bhūtabhṛnna ca bhūtastho mamātmā bhūtabhāvanaḥ.
\end{transliteration}

5. Nor do beings exist (in reality) in Me---behold My Divine YOGA supporting
all beings, but not dwelling in them, I am My Self, the `efficient-cause' of
all beings.

\begin{gitaverse}
यथाकाशस्थितो नित्यं वायुः सर्वत्रगो महान् । \\
तथा सर्वाणि भूतानि मत्स्थानीत्युपधारय ॥६॥
\end{gitaverse}

\begin{transliteration}
yathākāśasthito nityaṁ vāyuḥ sarvatrago mahān, \\
tathā sarvāṇi bhūtāni matsthānītyupadhāraya.
\end{transliteration}

6. As the mighty wind, moving everywhere, rests always in space (the AKASHA),
even so, know you, all beings rest in Me.

\begin{gitaverse}
सर्वभूतानि कौन्तेय प्रकृतिं यान्ति मामिकाम् । \\
कल्पक्षये पुनस्तानि कल्पादौ विसृजाम्यहम् ॥७॥
\end{gitaverse}

\begin{transliteration}
sarvabhūtāni kaunteya prakṛtiṁ yānti māmikām, \\
kalpakṣaye punastāni kalpādau visṛjāmyaham.
\end{transliteration}

7. All beings, O Kaunteya (O Son of Kunti), go into My PRAKRITI (nature) at the
end of a KALPA;\@ I send them forth again at the beginning of (the next)
KALPA.\@

\begin{gitaverse}
प्रकृतिं स्वामवष्टभ्य विसृजामि पुनः पुनः । \\
भूतग्राममिमं कृत्स्नमवशं प्रकृतेर्वशात् ॥८॥
\end{gitaverse}

\begin{transliteration}
prakṛtiṁ svāmavaṣṭabhya visṛjāmi punaḥ punaḥ, \\
bhūtagrāmamimaṁ kṛtsnamavaśaṁ prakṛtervaśāt.
\end{transliteration}

8. Animating My PRAKRITI, I, again and again send forth all this helpless
multitude of beings, by the force of nature (PRAKRITI).

\begin{gitaverse}
न च मां तानि कर्माणि निबध्नन्ति धनञ्जय । \\
उदासीनवदासीनमसक्तं तेषु कर्मसु ॥९॥
\end{gitaverse}

\begin{transliteration}
na ca māṁ tāni karmāṇi nibadhnanti dhanañjaya, \\
udāsīnavadāsīnamasaktaṁ teṣu karmasu.
\end{transliteration}

9. Sitting like one indifferent, and unattached to these acts, Dhananjaya,
these acts do not bind Me.

\begin{gitaverse}
मयाध्यक्षेण प्रकृतिः सूयते सचराचरम् । \\
हेतुनानेन कौन्तेय जगद्विपरिवर्तते ॥१०॥
\end{gitaverse}

\begin{transliteration}
mayādhyakṣeṇa prakṛtiḥ sūyate sacarācaram, \\
hetunānena kaunteya jagadviparivartate.
\end{transliteration}

10. Under Me as her Supervisor, PRAKRITI (nature) produces the moving and the
unmoving; because of this, O Kaunteya, the world revolves.

\begin{gitaverse}
अवजानन्ति मां मूढा मानुषीं तनुमाश्रितम् । \\
परं भावमजानन्तो मम भूतमहेश्वरम् ॥११॥
\end{gitaverse}

\begin{transliteration}
avajānanti māṁ mūḍhā mānuṣīṁ tanumāśritam, \\
paraṁ bhāvamajānanto mama bhūtamaheśvaram.
\end{transliteration}

11. Fools disregard Me when I dwell in human form; they know not My Higher
being as the Great Lord of all beings.

\begin{gitaverse}
मोघाशा मोघकर्माणो मोघज्ञाना विचेतसः । \\
राक्षसीमासुरीं चैव प्रकृतिं मोहिनीं श्रिताः ॥१२॥
\end{gitaverse}

\begin{transliteration}
moghāśā moghakarmāṇo moghajñānā vicetasaḥ, \\
rākṣasīmāsurīṁ caiva prakṛtiṁ mohinīṁ śritāḥ.
\end{transliteration}

12. Of vain hopes, of vain actions, of vain knowledge, and senseless, they
verily are possessed of the delusive nature of RAKSHASAS and ASURAS.\@

\begin{gitaverse}
महात्मानस्तु मां पार्थ दैवीं प्रकृतिमाश्रिताः । \\
भजन्त्यनन्यमनसो ज्ञात्वा भूतादिमव्ययम् ॥१३॥
\end{gitaverse}

\begin{transliteration}
mahātmānastu māṁ pārtha daivīṁ prakṛtimāśritāḥ, \\
bhajantyananyamanaso jñātvā bhūtādimavyayam.
\end{transliteration}

13. But the MAHATMAS (great-souls) O Partha, partaking of My divine nature,
worship Me with a single mind (with a mind devoted to nothing else), knowing Me
as the Imperishable Source of all beings.

\begin{gitaverse}
सततं कीर्तयन्तो मां यतन्तश्च दृढव्रताः । \\
नमस्यन्तश्च मां भक्त्या नित्ययुक्ता उपासते ॥१४॥
\end{gitaverse}

\begin{transliteration}
satataṁ kīrtayanto māṁ yatantaśca dṛḍhavratāḥ, \\
namasyantaśca māṁ bhaktyā nityayuktā upāsate.
\end{transliteration}

14. Always glorifying Me, striving, firm in vows, prostrating before Me, and
always steadfast, they worship Me with devotion.

\begin{gitaverse}
ज्ञानयज्ञेन चाप्यन्ये यजन्तो मामुपासते । \\
एकत्वेन पृथक्त्वेन बहुधा विश्वतोमुखम् ॥१५॥
\end{gitaverse}

\begin{transliteration}
jñānayajñena cāpyanye yajanto māmupāsate \\
ekatvena pṛthaktvena bahudhā viśvatomukham.
\end{transliteration}

15. Others also, offering the `Wisdom-sacrifice' worship Me, regarding Me as
One, as distinct, as manifold---Me, who in all forms, faces everywhere.

\begin{gitaverse}
अहं क्रतुरहं यज्ञः स्वधाहमहमौषधम् । \\
मन्त्रोऽहमहमेवाज्यमहमग्निरहं हुतम् ॥१६॥
\end{gitaverse}

\begin{transliteration}
ahaṁ kraturahaṁ yajñaḥ svadhāhamahamauṣadham, \\
mantro'hamahamevājyamahamagnirahaṁ hutam.
\end{transliteration}

16. I am the KRATU;\@ I am the sacrifice; I am the offering (food) to PITRIS
(or ancestors); I am the medicinal herb, and all plants; I am the MANTRA I am
also the clarified butter; I am the fire; I am the oblation.

\begin{gitaverse}
पिताहमस्य जगतो माता धाता पितामहः । \\
वेद्यं पवित्रमोङ्कार ऋक्साम यजुरेव च ॥१७॥
\end{gitaverse}

\begin{transliteration}
pitāhamasya jagato mātā dhātā pitāmahaḥ, \\
vedyaṁ pavitramoṅkāra ṛksāma yajureva ca.
\end{transliteration}

17. I am the Father of this world, the Mother, the supporter and the grandsire;
the (one) Thing to be known, the Purifier, (the syllable) OM, and also the RIK,
the SAMA and the YAJUH also.

\begin{gitaverse}
गतिर्भर्ता प्रभुः साक्षी निवासः शरणं सुहृत् । \\
प्रभवः प्रलयः स्थानं निधानं बीजमव्ययम् ॥१८॥
\end{gitaverse}

\begin{transliteration}
gatirbhartā prabhuḥ sākṣī nivāsaḥ śaraṇaṁ suhṛt, \\
prabhavaḥ pralayaḥ sthānaṁ nidhānaṁ bījamavyayam.
\end{transliteration}

18. I am the Goal, the Supporter, the Lord, the Witness, the Abode, the
Shelter, the Friend, the Origin, the Dissolution, the Foundation, the
Treasure-house and the Seed Imperishable.

\begin{gitaverse}
तपाम्यहमहं वर्षं निगृह्णाम्युत्सृजामि च । \\
अमृतं चैव मृत्युश्च सदसच्चाहमर्जुन ॥१९॥
\end{gitaverse}

\begin{transliteration}
tapāmyahamahaṁ varṣaṁ nigṛhṇāmyutsṛjāmi ca, \\
amṛtaṁ caiva mṛtyuśca sadasaccāhamarjuna.
\end{transliteration}

19. (As Sun) I give heat; I withhold and send forth the rain; I am Immortality
and also death, both Existence and Non-existence, O Arjuna.

\begin{gitaverse}
त्रैविद्या मां सोमपाः पूतपापा- \\
\tab यज्ञैरिष्ट्वा स्वर्गतिं प्रार्थयन्ते । \\
ते पुण्यमासाद्य सुरेन्द्रलोक- \\
\tab मश्नन्ति दिव्यान्दिवि देवभोगान् ॥२०॥
\end{gitaverse}

\begin{transliteration}
traividyā māṁ somapāḥ pūtapāpā- \\
\tab yajñairiṣṭvā svargatiṁ prārthayante, \\
te puṇyamāsādya surendraloka- \\
\tab maśnanti divyāndivi devabhogān.
\end{transliteration}

20. The Knowers of the three VEDAS, the drinkers of SOMA, purified from sin,
worshipping Me by sacrifices, pray for the way to heaven; they reach the holy
world of the Lord-of-the-gods and enjoy in heaven the Divine pleasures of the
gods.

\begin{gitaverse}
ते तं भुक्त्वा स्वर्गलोकं विशालं- \\
\tab क्षीणे पुण्ये मर्त्यलोकं विशन्ति । \\
एवं त्रयीधर्ममनुप्रपन्ना- \\
\tab गतागतं कामकामा लभन्ते ॥२१॥
\end{gitaverse}

\begin{transliteration}
te taṁ bhuktvā svargalokaṁ viśālaṁ- \\
\tab kṣīṇe puṇye martyalokaṁ viśanti, \\
evaṁ trayīdharmamanuprapannā- \\
\tab gatāgataṁ kāmakāmā labhante.
\end{transliteration}

21. They, having enjoyed the vast heaven-world, when their merits are
exhausted, enter the world-of-the-mortals; thus abiding by the injunctions of
the three (VEDAS), desiring (objects of) desires, they attain to the state of
`going-andreturning ' (SAMSARA).

\begin{gitaverse}
अनन्याश्चिन्तयन्तो मां ये जनाः पर्युपासते । \\
तेषां नित्याभियुक्तानां योगक्षेमं वहाम्यहम् ॥२२॥
\end{gitaverse}

\begin{transliteration}
ananyāścintayanto māṁ ye janāḥ paryupāsate, \\
teṣāṁ nityābhiyuktānāṁ yogakṣemaṁ vahāmyaham.
\end{transliteration}

22. To those men who worship Me alone, thinking of no other, to those ever
self-controlled, I secure for them that which is not already possessed (YOGA)
by them, and preserve for them what they already possess (KSHEMA).

\begin{gitaverse}
येऽप्यन्यदेवता भक्ता यजन्ते श्रद्धयान्विताः । \\
तेऽपि मामेव कौन्तेय यजन्त्यविधिपूर्वकम् ॥२३॥
\end{gitaverse}

\begin{transliteration}
ye'pyanyadevatā bhaktā yajante śraddhayānvitāḥ, \\
te'pi māmeva kaunteya yajantyavidhipūrvakam.
\end{transliteration}

23. Even those devotees, who, endowed with faith worship other gods, worship Me
alone, O son of Kunti, (but) by the wrong method.

\begin{gitaverse}
अहं हि सर्वयज्ञानां भोक्ता च प्रभुरेव च । \\
न तु मामभिजानन्ति तत्त्वेनातश्च्यवन्ति ते ॥२४॥
\end{gitaverse}

\begin{transliteration}
ahaṁ hi sarvayajñānāṁ bhoktā ca prabhureva ca, \\
na tu māmabhijānanti tattvenātaścyavanti te.
\end{transliteration}

24. (For) I alone am the enjoyer in and the Lord of all sacrifices; but they do
not know Me in Essence, and hence they fall (return to this mortal world).

\begin{gitaverse}
यान्ति देवव्रता देवान्पितॄन्यान्ति पितृव्रताः । \\
भूतानि यान्ति भूतेज्या यान्ति मद्याजिनोऽपि माम् ॥२५॥
\end{gitaverse}

\begin{transliteration}
yānti devavratā devānpitṝnyānti pitṛvratāḥ, \\
bhūtāni yānti bhūtejyā yānti madyājino'pi mām.
\end{transliteration}

25. The worshippers of the DEVAS or gods go to the DEVAS;\@ to the PITRIS or
ancestors go the ancestor-worshippers; to the BHUTAS or the elements go
worshippers of the BHUTAS;\@ but My worshippers come unto Me.

\begin{gitaverse}
पत्रं पुष्पं फलं तोयं यो मे भक्त्या प्रयच्छति । \\
तदहं भक्त्युपहृतमश्नामि प्रयतात्मनः ॥२६॥
\end{gitaverse}

\begin{transliteration}
patraṁ puṣpaṁ phalaṁ toyaṁ yo me bhaktyā prayacchati, \\
tadahaṁ bhaktyupahṛtamaśnāmi prayatātmanaḥ.
\end{transliteration}

26. Whoever offers Me with devotion a leaf, a flower, a fruit, water, that I
accept, offered by the pure-minded with devotion.

\begin{gitaverse}
यत्करोषि यदश्नासि यज्जुहोषि ददासि यत् । \\
यत्तपस्यसि कौन्तेय तत्कुरुष्व मदर्पणम् ॥२७॥
\end{gitaverse}

\begin{transliteration}
yatkaroṣi yadaśnāsi yajjuhoṣi dadāsi yat, \\
yattapasyasi kaunteya tatkuruṣva madarpaṇam.
\end{transliteration}

27. Whatever you do, whatever you eat, whatever you offer in sacrifice,
whatever you give in charity, whatever you practise as austerity, O Kaunteya,
do it as an offering to Me.

\begin{gitaverse}
शुभाशुभफलैरेवं मोक्ष्यसे कर्मबन्धनैः । \\
सन्न्यासयोगयुक्तात्मा विमुक्तो मामुपैष्यसि ॥२८॥
\end{gitaverse}

\begin{transliteration}
śubhāśubhaphalairevaṁ mokṣyase karmabandhanaiḥ, \\
sannyāsayogayuktātmā vimukto māmupaiṣyasi.
\end{transliteration}

28. Thus shall you be freed from the bonds-of-actions yielding good and evil
`fruits'; with the mind steadfast in the YOGA of renunciation, and liberated,
you shall come unto Me.

\begin{gitaverse}
समोऽहं सर्वभूतेषु न मे द्वेष्योऽस्ति न प्रियः । \\
ये भजन्ति तु मां भक्त्या मयि ते तेषु चाप्यहम् ॥२९॥
\end{gitaverse}

\begin{transliteration}
samo'haṁ sarvabhūteṣu na me dveṣyo'sti na priyaḥ, \\
ye bhajanti tu māṁ bhaktyā mayi te teṣu cāpyaham.
\end{transliteration}

29. The same am I to all beings, to Me there is none hateful nor dear; but
those who worship Me with devotion, are in Me and I am also in them.

\begin{gitaverse}
अपि चेत्सुदुराचारो भजते मामनन्यभाक् । \\
साधुरेव स मन्तव्यः सम्यग्व्यवसितो हि सः ॥३०॥
\end{gitaverse}

\begin{transliteration}
api cetsudurācāro bhajate māmananyabhāk, \\
sādhureva sa mantavyaḥ samyagvyavasito hi saḥ.
\end{transliteration}

30. Even if the most sinful worships Me, with devotion to none else, (or with
single-pointedness), he too should indeed be regarded as `righteous' for he has
rightly resolved.

\begin{gitaverse}
क्षिप्रं भवति धर्मात्मा शश्वच्छान्तिं निगच्छति । \\
कौन्तेय प्रतिजानीहि न मे भक्तः प्रणश्यति ॥३१॥
\end{gitaverse}

\begin{transliteration}
kṣipraṁ bhavati dharmātmā śaśvacchāntiṁ nigacchati, \\
kaunteya pratijānīhi na me bhaktaḥ praṇaśyati.
\end{transliteration}

31. Soon he becomes righteous and attains Eternal Peace, O Kaunteya, know for
certain that My devotee is never destroyed.

\begin{gitaverse}
मां हि पार्थ व्यपाश्रित्य येऽपि स्युः पापयोनयः । \\
स्त्रियो वैश्यास्तथा शूद्रास्तेऽपि यान्ति परां गतिम् ॥३२॥
\end{gitaverse}

\begin{transliteration}
māṁ hi pārtha vyapāśritya ye'pi syuḥ pāpayonayaḥ, \\
striyo vaiśyāstathā śūdrāste'pi yānti parāṁ gatim.
\end{transliteration}

32. For, taking refuge in Me, they also, who O Partha, may be of a `sinful
birth'---women, VAISHYAS as well as SHUDRAS---even they attain the Supreme
Goal.

\begin{gitaverse}
किं पुनर्ब्राह्मणाः पुण्या भक्ता राजर्षयस्तथा । \\
अनित्यमसुखं लोकमिमं प्राप्य भजस्व माम् ॥३३॥
\end{gitaverse}

\begin{transliteration}
kiṁ punarbrāhmaṇāḥ puṇyā bhaktā rājarṣayastathā, \\
anityamasukhaṁ lokamimaṁ prāpya bhajasva mām.
\end{transliteration}

33. How much more (easily) then the holy BRAHMINS, and devoted Royal saints
(attain the goal). Having reached (obtained) this impermanent and joyless
world, do worship Me devoutly.

\begin{gitaverse}
मन्मना भव मद्भक्तो मद्याजी मां नमस्कुरु । \\
मामेवैष्यसि युक्त्वैवमात्मानं मत्परायणः ॥३४॥
\end{gitaverse}

\begin{transliteration}
manmanā bhava madbhakto madyājī māṁ namaskuru, \\
māmevaiṣyasi yuktvaivamātmānaṁ matparāyaṇaḥ.
\end{transliteration}

34. Fix your mind on Me; be devoted to Me, sacrifice to Me, bow down to Me;
having thus united your (whole) Self with Me, taking me as the Supreme Goal,
you shall come to Me.

\begin{gitaverse}
ॐ तत्सदिति श्रीमद् भगवद् गीतासूपनिषत्सु ब्रह्मविद्यायां \\
योगशास्त्रे श्रीकृष्णार्जुनसंवादे राजविद्याराजगुह्ययोगो नाम \\
नवमोऽध्यायः
\end{gitaverse}

\begin{transliteration}
oṁ tatsaditi śrīmad bhagavad gītāsūpaniṣatsu brahmavidyāyāṁ \\
yogaśāstre śrī kṛṣṇārjuna saṁvāde rājavidyārājaguhyayogo nāma \\
navamo'dhyāyaḥ.
\end{transliteration}

Thus, in the UPANISHADS of the glorious Bhagawad Geeta, in the Science of the
Eternal, in the scripture of YOGA, in the dialogue between Sri Krishna and
Arjuna, the ninth discourse ends entitled: The Yoga of Royal Knowledge and
Royal Secret
