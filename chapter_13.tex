\chapterdrop

\begin{center}
\headersanskrit{अथ त्रयोदशोऽध्यायः}

\headerspace
\headertransliteration{Atha Trayodaśo'dhyāyaḥ}

\section{Chapter 13}

\headerspace
\headersanskrit{क्षेत्रक्षेत्रज्ञविभागयोगः}

\headerspace
\headertransliteration{Kṣetra Kṣetrajña Vibhāga Yogah}

\headerspace
\headertranslation{The Field and the Knower-of-the-Field}

\headerspace
\end{center}

\begin{gitaverse}
अर्जुन उवाच \\
प्रकृतिं पुरुषं चैव क्षेत्रं क्षेत्रज्ञमेव च । \\
एतद्वेदितुमिच्छामि ज्ञानं ज्ञेयं च केशव ॥१॥
\end{gitaverse}

\begin{transliteration}
arjuna uvāca \\
prakṛtiṁ puruṣaṁ caiva kṣetraṁ kṣetrajñameva ca, \\
etad-veditum-icchāmi jñānaṁ jñeyaṁ ca keśava.
\end{transliteration}

Arjuna said: \\
1. Prakriti (Matter) and Purusha (Spirit), also the Kshetra (the Field) and
Kshetrajna (the Knower-of-the-Field), Knowledge and that which ought to be
known---these, I wish to learn, O Keshava.

\begin{gitaverse}
श्रीभगवानुवाच \\
इदं शरीरं कौन्तेय क्षेत्रमित्यभिधीयते । \\
एतद्यो वेत्ति तं प्राहुः क्षेत्रज्ञ इति तद्विदः ॥२॥
\end{gitaverse}

\begin{transliteration}
śrībhagavānuvāca \\
idaṁ śarīraṁ kaunteya kṣetramityabhidhīyate, \\
etadyo vetti taṁ prāhuḥ kṣetrajña iti tadvidaḥ.
\end{transliteration}

The Blessed Lord said: \\
2. This body, O Kaunteya is called Kshetra (the Field) and he who knows it is
called Kshetrajna (the Knower-of-the-Field) by those who know them (Kshetra and
Kshetrajna) i.e., by the sages.

\begin{gitaverse}
क्षेत्रज्ञं चापि मां विद्धि सर्वक्षेत्रेषु भारत । \\
क्षेत्रक्षेत्रज्ञयोर्ज्ञानं यत्तज्ज्ञानं मतं मम ॥३॥
\end{gitaverse}

\begin{transliteration}
kṣetrajñaṁ cāpi māṁ viddhi sarvakṣetreṣu bhārata, \\
kṣetra-kṣetrajñayor-jñānaṁ yattajjñānaṁ mataṁ mama.
\end{transliteration}

3. Know Me as the `Knower-of-the-Field' in all `Fields', O Bharata; Knowledge
of the `Field' as also of the `Knower-of-the-Field' is considered by Me to be
My Knowledge.

\begin{gitaverse}
तत्क्षेत्रं यच्च यादृक्च यद्विकारि यतश्च यत् । \\
स च यो यत्प्रभावश्च तत्समासेन मे श्रुणु ॥४॥
\end{gitaverse}

\begin{transliteration}
tat-kṣetraṁ yacca yādṛkca yadvikāri yataśca yat, \\
sa ca yo yatprabhāvaśca tatsamāsena me śṛṇu.
\end{transliteration}

4. What that Field is; of what nature it is; what are its modifications; whence
it is; and also who He is; and what His powers are---these hear from Me in
brief.

\begin{gitaverse}
ऋषिभिर्बहुधा गीतं छन्दोभिर्विविधैः पृथक् । \\
ब्रह्मसूत्रपदैश्चैव हेतुमद्भिर्विनिश्चितैः ॥५॥
\end{gitaverse}

\begin{transliteration}
ṛṣibhir-bahudhā gītaṁ chandobhir-vividhaiḥ pṛthak, \\
brahmasūtrapadaiś-caiva hetumadbhir-viniścitaiḥ.
\end{transliteration}

5. Rishis have sung (about the `Field' and the `Knower-of-the-Field') in many
ways, in various distinctive chants and also in the suggestive words indicative
of Brahman, full of reason and decision.

\begin{gitaverse}
महाभूतान्यहङ्कारो बुद्धिरव्यक्तमेव च । \\
इन्द्रियाणि दशैकं च पञ्च चेन्द्रियगोचराः ॥६॥
\end{gitaverse}

\begin{transliteration}
mahābhūtānyahaṅkāro buddhiravyaktameva ca, \\
indriyāṇi daśaikaṁ ca pañca cendriyagocarāḥ.
\end{transliteration}

6. The great elements, egoism, intellect, and also the unmanifested
(moola-Prakriti), the ten senses and the one (the mind) and the five
objects-of-the-senses\ldots

\begin{gitaverse}
इच्छा द्वेषः सुखं दुःखं सङ्घातश्चेतना धृतिः । \\
एतत्क्षेत्रं समासेन सविकारमुदाहृतम् ॥७॥
\end{gitaverse}

\begin{transliteration}
icchā dveṣaḥ sukhaṁ duḥkhaṁ saṅghātaścetanā dhṛtiḥ, \\
etatkṣetraṁ samāsena savikāramudāhṛtam.
\end{transliteration}

7. Desire, hatred, pleasure, pain, aggregate (body), intelligence,
fortitude---this Kshetra has been thus briefly described with its
modifications.

\begin{gitaverse}
अमानित्वमदम्भित्वमहिंसा क्षान्तिरार्जवम् । \\
आचार्योपासनं शौचं स्थैर्यमात्मविनिग्रहः ॥८॥
\end{gitaverse}

\begin{transliteration}
amānitvam-adambhitvam-ahiṁsā kṣāntirārjavam, \\
ācāryopāsanaṁ śaucaṁ sthairyamātmavinigrahaḥ.
\end{transliteration}

8. Humility, unpretentiousness, non-injury, forgiveness, uprightness, service
to the teacher, purity, steadfastness, self-control\ldots

\begin{gitaverse}
इन्द्रियार्थेषु वैराग्यमनहङ्कार एव च । \\
जन्ममृत्युजराव्याधिदुःखदोषानुदर्शनम् ॥९॥
\end{gitaverse}

\begin{transliteration}
indriyārtheṣu vairāgyam-anahaṅkāra eva ca, \\
janma-mṛtyu-jarā-vyādhi-duḥkha-doṣānudarśanam.
\end{transliteration}

9. Indifference to the objects of the senses, and also absence of egoism,
perception of (or reflection upon) evils in birth, death, old age, sickness and
pain\ldots

\begin{gitaverse}
असक्तिरनभिष्वङ्गः पुत्रदारगृहादिषु । \\
नित्यं च समचित्तत्वमिष्टानिष्टोपपत्तिषु ॥१०॥
\end{gitaverse}

\begin{transliteration}
asaktir-anabhiṣvaṅgaḥ putra-dāra-gṛhādiṣu, \\
nityaṁ ca samacittatvam-iṣṭāniṣṭopapattiṣu.
\end{transliteration}

10. Non-attachment; non-identification of Self with son, wife, home and the
rest; and constant even-mindedness on the attainment of the desirable and the
undesirable\ldots

\begin{gitaverse}
मयि चानन्ययोगेन भक्तिरव्यभिचारिणी । \\
विविक्तदेशसेवित्वमरतिर्जनसंसदि ॥११॥
\end{gitaverse}

\begin{transliteration}
mayi cānanya-yogena bhaktir-avyabhicāriṇī, \\
viviktadeśa-sevitvam-aratir-janasaṁsadi.
\end{transliteration}

11. Unswerving devotion unto Me by the YOGA of nonseparation, resorting to
solitary places, distaste for the society of men\ldots

\begin{gitaverse}
अध्यात्मज्ञाननित्यत्वं तत्त्वज्ञानार्थदर्शनम् । \\
एतज्ज्ञानमिति प्रोक्तमज्ञानं यदतोऽन्यथा ॥१२॥
\end{gitaverse}

\begin{transliteration}
adhyātma-jñāna-nityatvaṁ tattva-jñānārtha-darśanam, \\
etajjñānam-iti proktam-ajñānaṁ yadato'nyathā.
\end{transliteration}

12. Constancy in Self-knowledge, perception of the end of true knowledge---this
is declared to be `knowledge', and what is opposed to it is `ignorance'.

\begin{gitaverse}
ज्ञेयं यत्तत्प्रवक्ष्यामि यज्ज्ञात्वामृतमश्नुते । \\
अनादिमत्परं ब्रह्म न सत्तन्नासदुच्यते ॥१३॥
\end{gitaverse}

\begin{transliteration}
jñeyaṁ yat-tat-pravakṣyāmi yaj-jñātvāmṛtam-aśnute, \\
anādimat-paraṁ brahma na sat-tannāsad-ucyate.
\end{transliteration}

13. I will declare that which has to be `known' knowing which one attains to
Immortality---the beginningless Supreme Brahman, called neither being nor
non-being.

\begin{gitaverse}
सर्वतःपाणिपादं तत्सर्वतोऽक्षिशिरोमुखम् । \\
सर्वतः श्रुतिमल्लोके सर्वमावृत्य तिष्ठति ॥१४॥
\end{gitaverse}

\begin{transliteration}
sarvataḥ-pāṇi-pādaṁ tat-sarvato'kṣi-śiromukham, \\
sarvataḥ śrutimalloke sarvam-āvṛtya tiṣṭhati.
\end{transliteration}

14. With hands and feet everywhere, with eyes, heads and mouths everywhere,
with ears everywhere, He exists in the world, enveloping all.

\begin{gitaverse}
सर्वेन्द्रियगुणाभासं सर्वेन्द्रियविवर्जितम् । \\
असक्तं सर्वभृच्चैव निर्गुणं गुणभोक्तृ च ॥१५॥
\end{gitaverse}

\begin{transliteration}
sarvendriya-guṇābhāsaṁ sarvendriya-vivarjitam, \\
asaktaṁ sarvabhṛccaiva nirguṇaṁ guṇabhoktṛ ca.
\end{transliteration}

15. Shining by the functions of all the senses, yet without the senses;
unattached, yet supporting all; devoid of qualities, yet their
experiencer\ldots

\begin{gitaverse}
बहिरन्तश्च भूतानामचरं चरमेव च । \\
सूक्ष्मत्वात्तदविज्ञेयं दूरस्थं चान्तिके च तत् ॥१६॥
\end{gitaverse}

\begin{transliteration}
bahir-antaśca bhūtānām-acaraṁ caram-eva ca, \\
sūkṣmatvāt-tad-avijñeyaṁ dūrasthaṁ cāntike ca tat.
\end{transliteration}

16. Without and within (all) beings, the `unmoving' and also the `moving';
because of its subtlety unknowable; and near and far away---is That.

\begin{gitaverse}
अविभक्तं च भूतेषु विभक्तमिव च स्थितम् । \\
भूतभर्तृ च तज्ज्ञेयं ग्रसिष्णु प्रभविष्णु च ॥१७॥
\end{gitaverse}

\begin{transliteration}
avibhaktaṁ ca bhūteṣu vibhaktamiva ca sthitam, \\
bhūtabhartṛ ca tajjñeyaṁ grasiṣṇu prabhaviṣṇu ca.
\end{transliteration}

17. And undivided, yet He exists as if divided in beings; That is to be known
as the Supporter of Beings; He devours and He generates.

\begin{gitaverse}
ज्योतिषामपि तज्ज्योतिस्तमसः परमुच्यते । \\
ज्ञानं ज्ञेयं ज्ञानगम्यं हृदि सर्वस्य विष्ठितम् ॥१८॥
\end{gitaverse}

\begin{transliteration}
jyotiṣām-api tajjyotis-tamasaḥ param-ucyate, \\
jñānaṁ jñeyaṁ jñānagamyaṁ hṛdi sarvasya viṣṭhitam.
\end{transliteration}

18. That (BRAHMAN), the Light-of-all-lights, is said to be beyond darkness; (It
is) Knowledge, the Object-of-Knowledge, seated in the hearts of all, to be
reached by Knowledge.

\begin{gitaverse}
इति क्षेत्रं तथा ज्ञानं ज्ञेयं चोक्तं समासतः । \\
मद्भक्त एतद्विज्ञाय मद्भावायोपपद्यते ॥१९॥
\end{gitaverse}

\begin{transliteration}
iti kṣetraṁ tathā jñānaṁ jñeyaṁ coktaṁ samāsataḥ, \\
madbhakta etadvijñāya madbhāvāyopapadyate.
\end{transliteration}

19. Thus the Field, as well as the knowledge and the knowable have been briefly
stated. Knowing this, My devotee enters into My Being.

\begin{gitaverse}
प्रकृतिं पुरुषं चैव विद्ध्यनादी उभावपि । \\
विकारांश्च गुणांश्चैव विद्धि प्रकृतिसम्भवान् ॥२०॥
\end{gitaverse}

\begin{transliteration}
prakṛtiṁ puruṣaṁ caiva viddhyanādī ubhāvapi, \\
vikārāṁśca guṇaṁścaiva viddhi prakṛtisambhavān.
\end{transliteration}

20. Know you that Matter (PRAKRITI) and Spirit (PURUSHA) are both
beginningless; and know you also that all modifications and qualities are born
of Prakriti.

\begin{gitaverse}
कार्यकरणकर्तृत्वे हेतुः प्रकृतिरुच्यते । \\
पुरुषः सुखदुःखानां भोक्तृत्वे हेतुरुच्यते ॥२१॥
\end{gitaverse}

\begin{transliteration}
kārya-karaṇa-kartṛtve hetuḥ prakṛtir-ucyate, \\
puruṣaḥ sukha-duḥkhānāṁ bhoktṛtve hetur-ucyate.
\end{transliteration}

21. In the production of the effect and the cause, PRAKRITI is said to be the
cause; in the experience of pleasure and pain, PURUSHA is said to be the cause.

\begin{gitaverse}
पुरुषः प्रकृतिस्थो हि भुङ्क्ते प्रकृतिजान्गुणान् । \\
कारणं गुणसङ्गोऽस्य सदसद्योनिजन्मसु ॥२२॥
\end{gitaverse}

\begin{transliteration}
puruṣaḥ prakṛtistho hi bhuṅkte prakṛtijān-guṇān, \\
kāraṇaṁ guṇasaṅgo'sya sad-asad-yoni-janmasu.
\end{transliteration}

22. The Purusha, seated in Prakriti, experiences the qualities born of
Prakriti; attachment to the qualities is the cause of his birth in good and
evil wombs.

\begin{gitaverse}
उपद्रष्टानुमन्ता च भर्ता भोक्ता महेश्वरः । \\
परमात्मेति चाप्युक्तो देहेऽस्मिन्पुरुषः परः ॥२३॥
\end{gitaverse}

\begin{transliteration}
upadraṣṭānumantā ca bhartā bhoktā maheśvaraḥ, \\
paramātmeti cāpyukto dehe'smin-puruṣaḥ paraḥ.
\end{transliteration}

23. The supreme PURUSHA in this body is also called the Spectator, the
Permitter, the Supporter, the Enjoyer, the great Lord and the Supreme Self.

\begin{gitaverse}
य एवं वेत्ति पुरुषं प्रकृतिं च गुणैः सह । \\
सर्वथा वर्तमानोऽपि न स भूयोऽभिजायते ॥२४॥
\end{gitaverse}

\begin{transliteration}
ya evaṁ vetti puruṣaṁ prakṛtiṁ ca guṇaiḥ saha, \\
sarvathā vartamāno'pi na sa bhūyo'bhijāyate.
\end{transliteration}

24. He who thus knows the PURUSHA and PRAKRITI together with the qualities, in
whatsoever condition he may be, he is not born again.

\begin{gitaverse}
ध्यानेनात्मनि पश्यन्ति केचिदात्मानमात्मना । \\
अन्ये साङ्ख्येन योगेन कर्मयोगेन चापरे ॥२५॥
\end{gitaverse}

\begin{transliteration}
dhyānenātmani paśyanti kecidātmānamātmanā, \\
anye sāṅkhyena yogena karmayogena cāpare.
\end{transliteration}

25. Some, by meditation, behold the Self in the Self by the Self; others by the
`YOGA-OF-KNOWLEDGE' (by SANKHYA YOGA); and others by KARMA YOGA.\@

\begin{gitaverse}
अन्ये त्वेवमजानन्तः श्रुत्वान्येभ्य उपासते । \\
तेऽपि चातितरन्त्येव मृत्युं श्रुतिपरायणाः ॥२६॥
\end{gitaverse}

\begin{transliteration}
anye tvevam-ajānantaḥ śrutvānyebhya upāsate, \\
te'pi cātitarantyeva mṛtyuṁ śruti-parāyaṇāḥ.
\end{transliteration}

26. Others also, not knowing this, worship, having heard of it from others;
they too, cross beyond death, if they would regard what they have heard as
their Supreme Refuge.

\begin{gitaverse}
यावत्सञ्जायते किञ्चित्सत्त्वं स्थावरजङ्गमम् । \\
क्षेत्रक्षेत्रज्ञसंयोगात्तद्विद्धि भरतर्षभ ॥२७॥
\end{gitaverse}

\begin{transliteration}
yāvat-sañjāyate kiñcit-sattvaṁ sthāvara-jaṅgamam, \\
kṣetrakṣetrajña-saṁyogāt-tadviddhi bharatarṣabha.
\end{transliteration}

27. Wherever any being is born, the unmoving or the moving, know you, O best of
the Bharatas, that it is from the union between the `Field' and the
`Knower-of-the-Field'.

\begin{gitaverse}
समं सर्वेषु भूतेषु तिष्ठन्तं परमेश्वरम् । \\
विनश्यत्स्वविनश्यन्तं यः पश्यति स पश्यति ॥२८॥
\end{gitaverse}

\begin{transliteration}
samaṁ sarveṣu bhūteṣu tiṣṭhantaṁ parameśvaram, \\
vinaśyatsvavinaśyantaṁ yaḥ paśyati sa paśyati.
\end{transliteration}

28. He sees, who sees the Supreme Lord existing equally in all beings, the
unperishing within the perishing.

\begin{gitaverse}
समं पश्यन्हि सर्वत्र समवस्थितमीश्वरम् । \\
न हिनस्त्यात्मनात्मानं ततो याति परां गतिम् ॥२९॥
\end{gitaverse}

\begin{transliteration}
samaṁ paśyanhi sarvatra sam-avasthitam-īśvaram, \\
na hinasty-ātmanātmānaṁ tato yāti parāṁ gatim.
\end{transliteration}

29. Indeed, he who sees the same Lord everywhere equally dwelling, destroys not
the Self by the Self; therefore, he goes to the Highest Goal.

\begin{gitaverse}
प्रकृत्यैव च कर्माणि क्रियमाणानि सर्वशः । \\
यः पश्यति तथात्मानमकर्तारं स पश्यति ॥३०॥
\end{gitaverse}

\begin{transliteration}
prakṛtyaiva ca karmāṇi kriyamāṇāni sarvaśaḥ, \\
yaḥ paśyati tathātmānam-akartāraṁ sa paśyati.
\end{transliteration}

30. He sees, who sees that all actions are performed by PRAKRITI alone, and
that the Self is actionless.

\begin{gitaverse}
यदा भूतपृथग्भावमेकस्थमनुपश्यति । \\
तत एव च विस्तारं ब्रह्म सम्पद्यते तदा ॥३१॥
\end{gitaverse}

\begin{transliteration}
yadā bhūta-pṛthagbhāvam-ekastham-anupaśyati, \\
tata eva ca vistāraṁ brahma sampadyate tadā.
\end{transliteration}

31. When he (man) sees the whole variety-of-beings, as resting in the One, and
spreading forth from That (One) alone, he then becomes BRAHMAN.\@

\begin{gitaverse}
अनादित्वान्निर्गुणत्वात्परमात्मायमव्ययः । \\
शरीरस्थोऽपि कौन्तेय न करोति न लिप्यते ॥३२॥
\end{gitaverse}

\begin{transliteration}
anāditvān-nirguṇatvāt-paramātmāyam-avyayaḥ, \\
śarīrastho'pi kaunteya na karoti na lipyate.
\end{transliteration}

32. Being without beginning, and being devoid of qualities, the Supreme Self,
the Imperishable, though dwelling in the body, O Kaunteya, neither acts, nor is
tainted.

\begin{gitaverse}
यथा सर्वगतं सौक्ष्म्यादाकाशं नोपलिप्यते । \\
सर्वत्रावस्थितो देहे तथात्मा नोपलिप्यते ॥३३॥
\end{gitaverse}

\begin{transliteration}
yathā sarvagataṁ saukṣmyād-ākāśaṁ nopalipyate, \\
sarvatrāvasthito dehe tathātmā nopalipyate.
\end{transliteration}

33. As the all-pervading ether is not tainted, because of its subtlety, so too
the Self, seated everywhere in the body, is not tainted.

\begin{gitaverse}
यथा प्रकाशयत्येकः कृत्स्नं लोकमिमं रविः । \\
क्षेत्रं क्षेत्री तथा कृत्स्नं प्रकाशयति भारत ॥३४॥
\end{gitaverse}

\begin{transliteration}
yathā prakāśayatyekaḥ kṛtsnaṁ lokam-imaṁ raviḥ, \\
kṣetraṁ kṣetrī tathā kṛtsnaṁ prakāśayati bhārata.
\end{transliteration}

34. Just as the one Sun illumines the whole world, so also the
Lord-of-the-Field (Param-Atman) illumines the whole `Field', O Bharata.

\begin{gitaverse}
क्षेत्रक्षेत्रज्ञयोरेवमन्तरं ज्ञानचक्षुषा । \\
भूतप्रकृतिमोक्षं च ये विदुर्यान्ति ते परम् ॥३५॥
\end{gitaverse}

\begin{transliteration}
kṣetra-kṣetrajñayor-evam-antaraṁ jñāna-cakṣuṣā, \\
bhūta-prakṛti-mokṣaṁ ca ye vidur-yānti te param.
\end{transliteration}

35. They who, with their eye-of-wisdom come to know the distinction between the
`Field' and the `Knower-of-the-Field', and of the liberation from the `PRAKRITI
of the being', go to the Supreme.

\begin{gitaverse}
ॐ तत्सदिति श्रीमद् भगवद् गीतासूपनिषत्सु ब्रह्मविद्यायां \\
योगशास्त्रे श्रीकृष्णार्जुनसंवादे क्षेत्रक्षेत्रज्ञविभागयोगो नाम \\
त्रयोदशोऽध्यायः
\end{gitaverse}

\begin{transliteration}
oṁ tatsaditi śrīmad bhagavad gītāsūpaniṣatsu brahmavidyāyāṁ \\
yogaśāstre śrīkṛṣṇārjunasaṁvāde kṣetrakṣetrajñavibhāgayogo nāma \\
trayodaśo'dhyāyaḥ
\end{transliteration}

Thus, in the UPANISHAD of the glorious Bhagawad Geeta, in the Science of the
Eternal, in the scripture of YOGA, in the dialogue between Sri Krishna and
Arjuna, the thirteenth discourse ends entitled: The Field and the
Knower-of-the-Field
