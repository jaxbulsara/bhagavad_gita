\chapterdrop

\begin{center}
\headersanskrit{अथ चतुर्थोऽध्यायः}

\headerspace
\headertransliteration{Atha Chaturtho'dhyāyaḥ}

\section{Chapter 4}

\headerspace
\headersanskrit{ज्ञानकर्मसंन्यासयोगः}

\headerspace
\headertransliteration{Jñāna Karma Sannyāsa Yogah}

\headerspace
\headertranslation{The Yoga of Renunciation of Action In Knowledge}

\headerspace
\end{center}

\begin{gitaverse}
श्रीभगवानुवाच \\
इमं विवस्वते योगं प्रोक्तवानहमव्ययम् । \\
विवस्वान्मनवे प्राह मनुरिक्ष्वाकवेऽब्रवीत् ॥१॥
\end{gitaverse}

\begin{transliteration}
śrībhagavānuvāca \\
imaṁ vivasvate yogaṁ proktavānahamavyayam, \\
vivasvānmanave prāha manurikṣvākave'bravīt.
\end{transliteration}

The Blessed Lord said: \\
1. I taught this Imperishable YOGA to Vivasvan; Vivasvan taught it to Manu;
Manu taught it to Ikshvaku.

\begin{gitaverse}
एवं परम्पराप्राप्तमिमं राजर्षयो विदुः । \\
स कालेनेह महता योगो नष्टः परन्तप ॥२॥
\end{gitaverse}

\begin{transliteration}
evaṁ paramparāprāptamimaṁ rājarṣayo viduḥ, \\
sa kāleneha mahatā yogo naṣṭaḥ parantapa.
\end{transliteration}

2. This knowledge, handed down thus in regular succession, the royal sages
knew. This YOGA, by long lapse of time, has been lost here, O Parantapa (burner
of the foes).

\begin{gitaverse}
स एवायं मया तेऽद्य योगः प्रोक्तः पुरातनः । \\
भक्तोऽसि मे सखा चेति रहस्यं ह्येतदुत्तमम् ॥३॥
\end{gitaverse}

\begin{transliteration}
sa evāyaṁ mayā te'dya yogaḥ proktaḥ purātanaḥ, \\
bhakto'si me sakhā ceti rahasyaṁ hyetaduttamam.
\end{transliteration}

3. That same ancient `YOGA' has been today taught to you by Me, for you are My
devotee and My friend. This is a Supreme Secret.

\begin{gitaverse}
अर्जुन उवाच \\
अपरं भवतो जन्म परं जन्म विवस्वतः । \\
कथमेतद्विजानीयां त्वमादौ प्रोक्तवानिति ॥४॥
\end{gitaverse}

\begin{transliteration}
arjuna uvāca \\
aparaṁ bhavato janma paraṁ janma vivasvataḥ, \\
kathametadvijānīyāṁ tvamādau proktavāniti.
\end{transliteration}

Arjuna said: \\
4. Later was Your birth, and prior was the birth of Vivasvan (Sun); how am I to
understand that You taught this YOGA in the beginning?

\begin{gitaverse}
श्रीभगवानुवाच \\
बहूनि मे व्यतीतानि जन्मानि तव चार्जुन । \\
तान्यहं वेद सर्वाणि न त्वं वेत्थ परन्तप ॥५॥
\end{gitaverse}

\begin{transliteration}
śrībhagavānuvāca \\
bahūni me vyatītāni janmāni tava cārjuna, \\
tānyahaṁ veda sarvāṇi na tvaṁ vettha parantapa.
\end{transliteration}

The Blessed Lord said: \\
5. Many births of Mine have passed as well as yours, O Arjuna; I know them all
but you know them not, O Parantapa (scorcher of foes).

\begin{gitaverse}
अजोऽपि सन्नव्ययात्मा भूतानामीश्वरोऽपि सन् । \\
प्रकृतिं स्वामधिष्ठाय सम्भवाम्यात्ममायया ॥६॥
\end{gitaverse}

\begin{transliteration}
ajo'pi sannavyayātmā bhūtānāmīśvaro'pi san, \\
prakṛtiṁ svāmadhiṣṭhāya sambhavāmyātmamāyayā.
\end{transliteration}

6. Though I am unborn and am of imperishable nature, and though I am the Lord
of all beings, yet, ruling over My own Nature, I take birth by My own MAYA.\@

\begin{gitaverse}
यदा यदा हि धर्मस्य ग्लानिर्भवति भारत । \\
अभ्युत्थानमधर्मस्य तदात्मानं सृजाम्यहम् ॥७॥
\end{gitaverse}

\begin{transliteration}
yadā yadā hi dharmasya glānirbhavati bhārata, \\
abhyutthānamadharmasya tadātmānaṁ sṛjāmyaham.
\end{transliteration}

7. Whenever there is a decay of righteousness, O Bharata, and a rise of
unrighteousness, then I manifest Myself.

\begin{gitaverse}
परित्राणाय साधूनां विनाशाय च दुष्कृताम् । \\
धर्मसंस्थापनार्थाय सम्भवामि युगे युगे ॥८॥
\end{gitaverse}

\begin{transliteration}
paritrāṇāya sādhūnāṁ vināśāya ca duṣkṛtām, \\
dharmasaṁsthāpanārthāya sambhavāmi yuge yuge.
\end{transliteration}

8. For the protection of the good, for the destruction of the wicked and for
the establishment of righteousness, I am born in every age.

\begin{gitaverse}
जन्म कर्म च मे दिव्यमेवं यो वेत्ति तत्त्वतः । \\
त्यक्त्वा देहं पुनर्जन्म नैति मामेति सोऽर्जुन ॥९॥
\end{gitaverse}

\begin{transliteration}
janma karma ca me divyamevaṁ yo vetti tattvataḥ, \\
tyaktvā dehaṁ punarjanma naiti māmeti so'rjuna.
\end{transliteration}

9. He who thus knows, in true light, My divine birth and action, having
abandoned the body, he is not born again; he comes to Me, O Arjuna.

\begin{gitaverse}
वीतरागभयक्रोधा मन्मया मामुपाश्रिताः । \\
बहवो ज्ञानतपसा पूता मद्भावमागताः ॥१०॥
\end{gitaverse}

\begin{transliteration}
vītarāgabhayakrodhā manmayā māmupāśritāḥ, \\
bahavo jñānatapasā pūtā madbhāvamāgatāḥ.
\end{transliteration}

10. Freed from attachment, fear and anger, absorbed in Me, taking refuge in Me,
purified by the Fire-of-Knowledge, many have attained My Being.

\begin{gitaverse}
ये यथा मां प्रपद्यन्ते तांस्तथैव भजाम्यहम् । \\
मम वर्त्मानुवर्तन्ते मनुष्याः पार्थ सर्वशः ॥११॥
\end{gitaverse}

\begin{transliteration}
ye yathā māṁ prapadyante tāṁstathaiva bhajāmyaham, \\
mama vartmānuvartante manuṣyāḥ pārtha sarvaśaḥ.
\end{transliteration}

11. In whatever way men approach Me, even so do I reward them; My path do men
tread in all ways, O son of Pritha.

\begin{gitaverse}
काङ्क्षन्तः कर्मणां सिद्धिं यजन्त इह देवताः । \\
क्षिप्रं हि मानुषे लोके सिद्धिर्भवति कर्मजा ॥१२॥
\end{gitaverse}

\begin{transliteration}
kāṅkṣantaḥ karmaṇāṁ siddhiṁ yajanta iha devatāḥ, \\
kṣipraṁ hi mānuṣe loke siddhirbhavati karmajā.
\end{transliteration}

12. They who long for satisfaction from actions in this world, make sacrifices
to the gods; because satisfaction is quickly obtained from actions in the
world-of-objects.

\begin{gitaverse}
चातुर्वर्ण्यं मया सृष्टं गुणकर्मविभागशः । \\
तस्य कर्तारमपि मां विद्ध्यकर्तारमव्ययम् ॥१३॥
\end{gitaverse}

\begin{transliteration}
cāturvarṇyaṁ mayā sṛṣṭaṁ guṇakarmavibhāgaśaḥ, \\
tasya kartāramapi māṁ viddhyakartāramavyayam.
\end{transliteration}

13. The fourfold-caste has been created by Me according to the differentiation
of GUNA and KARMA;\@ though I am the author thereof know Me as non-doer and
immutable.

\begin{gitaverse}
न मां कर्माणि लिम्पन्ति न मे कर्मफले स्पृहा । \\
इति मां योऽभिजानाति कर्मभिर्न स बध्यते ॥१४॥
\end{gitaverse}

\begin{transliteration}
na māṁ karmāṇi limpanti na me karmaphale spṛhā, \\
iti māṁ yo'bhijānāti karmabhirna sa badhyate.
\end{transliteration}

14. Actions do not taint Me, nor have I any desire for the fruits-of-actions.
He who knows Me thus is not bound by his actions.

\begin{gitaverse}
एवं ज्ञात्वा कृतं कर्म पूर्वैरपि मुमुक्षुभिः । \\
कुरु कर्मैव तस्मात्त्वं पूर्वैः पूर्वतरं कृतम् ॥१५॥
\end{gitaverse}

\begin{transliteration}
evaṁ jñātvā kṛtaṁ karma pūrvairapi mumukṣubhiḥ, \\
kuru karmaiva tasmāttvaṁ pūrvaiḥ pūrvataraṁ kṛtam.
\end{transliteration}

15. Having known this, the ancient seekers-after-freedom also performed action;
therefore, you too perform action, as did the ancients in the olden times.

\begin{gitaverse}
किं कर्म किमकर्मेति कवयोऽप्यत्र मोहिताः । \\
तत्ते कर्म प्रवक्ष्यामि यज्ज्ञात्वा मोक्ष्यसेऽशुभात् ॥१६॥
\end{gitaverse}

\begin{transliteration}
kiṁ karma kimakarmeti kavayo'pyatra mohitāḥ, \\
tatte karma pravakṣyāmi yajjñātva mokṣyase'śubhāt.
\end{transliteration}

16. What is action? What is inaction? As to this even the `wise' are deluded.
Therefore, I shall teach you `action' (the nature of action and inaction),
knowing which, you shall be liberated from the evil (of SAMSARA---the wheel of
birth and death).

\begin{gitaverse}
कर्मणो ह्यपि बोद्धव्यं बोद्धव्यं च विकर्मणः । \\
अकर्मणश्च बोद्धव्यं गहना कर्मणो गतिः ॥१७॥
\end{gitaverse}

\begin{transliteration}
karmaṇo hyapi boddhavyaṁ boddhavyaṁ ca vikarmaṇaḥ, \\
akarmaṇaśca boddhavyaṁ gahanā karmaṇo gatiḥ.
\end{transliteration}

17. For verily (the true nature) of `right action' should be known; also (that)
of `forbidden (or unlawful) action' and of `inaction'; imponderable is the
nature (path) of action.

\begin{gitaverse}
कर्मण्यकर्म यः पश्येदकर्मणि च कर्म यः । \\
स बुद्धिमान्मनुष्येषु स युक्तः कृत्स्नकर्मकृत् ॥१८॥
\end{gitaverse}

\begin{transliteration}
karmaṇyakarma yaḥ paśyedakarmaṇi ca karma yaḥ, \\
sa buddhimānmanuṣyeṣu sa yuktaḥ kṛtsnakarmakṛt.
\end{transliteration}

18. He who recognises inaction in action and action in inaction is wise among
men; he is a YOGI and a true performer of all actions.

\begin{gitaverse}
यस्य सर्वे समारम्भाः कामसङ्कल्पवर्जिताः । \\
ज्ञानाग्निदग्धकर्माणं तमाहुः पण्डितं बुधाः ॥१९॥
\end{gitaverse}

\begin{transliteration}
yasya sarve samārambhāḥ kāmasaṅkalpavarjitāḥ, \\
jñānāgnidagdhakarmāṇaṁ tamāhuḥ paṇḍitaṁ budhāḥ.
\end{transliteration}

19. Whose undertakings are all devoid of desires and purposes, and whose
actions have been burnt by the Fire-of-Knowledge, him the `wise' call a Sage.

\begin{gitaverse}
त्यक्त्वा कर्मफलासङ्गं नित्यतृप्तो निराश्रयः । \\
कर्मण्यभिप्रवृत्तोऽपि नैव किञ्चित्करोति सः ॥२०॥
\end{gitaverse}

\begin{transliteration}
tyaktvā karmaphalāsaṅgaṁ nityatṛpto nirāśrayaḥ, \\
karmaṇyabhipravṛtto'pi naiva kiñcitkaroti saḥ.
\end{transliteration}

20. Having abandoned attachment to the fruits-of-action, ever content,
depending on nothing, he does not do anything, though engaged in actions.

\begin{gitaverse}
निराशीर्यतचित्तात्मा त्यक्तसर्वपरिग्रहः । \\
शारीरं केवलं कर्म कुर्वन्नाप्नोति किल्बिषम् ॥२१॥
\end{gitaverse}

\begin{transliteration}
nirāśīryatacittātmā tyaktasarvaparigrahaḥ, \\
śārīraṁ kevalaṁ karma kurvannāpnoti kilbiṣam.
\end{transliteration}

21. Without hope, with the mind and Self controlled, having abandoned all
possessions, doing mere bodily action, he incurs no sin.

\begin{gitaverse}
यदृच्छालाभसन्तुष्टो द्वन्द्वातीतो विमत्सरः । \\
समः सिद्धावसिद्धौ च कृत्वापि न निबध्यते ॥२२॥
\end{gitaverse}

\begin{transliteration}
yadṛcchālābhasantuṣṭo dvandvātīto vimatsaraḥ, \\
samaḥ siddhāvasiddhau ca kṛtvāpi na nibadhyate.
\end{transliteration}

22. Content with what comes to him without effort, free from the
pairs-of-opposites and envy, even-minded in success and failure, though acting
he is not bound.

\begin{gitaverse}
गतसङ्गस्य मुक्तस्य ज्ञानावस्थितचेतसः । \\
यज्ञायाचरतः कर्म समग्रं प्रविलीयते ॥२३॥
\end{gitaverse}

\begin{transliteration}
gatasaṅgasya muktasya jñānāvasthitacetasaḥ, \\
yajñāyācarataḥ karma samagraṁ pravilīyate.
\end{transliteration}

23. Of one who is devoid of attachment, who is liberated, whose mind is
established in knowledge, who acts for the sake of sacrifice, all his actions
are dissolved.

\begin{gitaverse}
ब्रह्मार्पणं ब्रह्म हविर्ब्रह्माग्नौ ब्रह्मणा हुतम् । \\
ब्रह्मैव तेन गन्तव्यं ब्रह्मकर्मसमाधिना ॥२४॥
\end{gitaverse}

\begin{transliteration}
brahmārpaṇaṁ brahma havirbrahmāgnau brahmaṇā hutam, \\
brahmaiva tena gantavyaṁ brahmakarmasamādhinā.
\end{transliteration}

24. BRAHMAN is the oblation; BRAHMAN is the clarified butter, etc.,
constituting the offerings; by BRAHMAN is the oblation poured into the fire of
BRAHMAN;\@ BRAHMAN verily shall be reached by him who always sees BRAHMAN in
all actions.

\begin{gitaverse}
दैवमेवापरे यज्ञं योगिनः पर्युपासते । \\
ब्रह्माग्नावपरे यज्ञं यज्ञेनैवोपजुह्वति ॥२५॥
\end{gitaverse}

\begin{transliteration}
daivamevāpare yajñaṁ yoginaḥ paryupāsate, \\
brahmāgnāvapare yajñaṁ yajñenaivopajuhvati.
\end{transliteration}

25. Some YOGIS perform sacrifice to DEVAS alone (DEVAYAJNA); while others offer
`sacrifice' as sacrifice by the Self, in the Fire of BRAHMAN (BRAHMA-YAJNA).

\begin{gitaverse}
श्रोत्रादीनीन्द्रियाण्यन्ये संयमाग्निषु जुह्वति । \\
शब्दादीन्विषयानन्य इन्द्रियाग्निषु जुह्वति ॥२६॥
\end{gitaverse}

\begin{transliteration}
śrotrādīnīndriyāṇyanye saṁyamāgniṣu juhvati, \\
śabdādīnviṣayānanya indriyāgniṣu juhvati.
\end{transliteration}

26. Some again, offer hearing and other senses as sacrifice in the
fires-of-restraint; others offer sound and other objects of sense as sacrifice
in the fires of the senses.

\begin{gitaverse}
सर्वाणीन्द्रियकर्माणि प्राणकर्माणि चापरे । \\
आत्मसंयमयोगाग्नौ जुह्वति ज्ञानदीपिते ॥२७॥
\end{gitaverse}

\begin{transliteration}
sarvāṇīndriyakarmāṇi prāṇakarmāṇi cāpare, \\
ātmasaṁyamayogāgnau juhvati jñānadīpite.
\end{transliteration}

27. Others, again, sacrifice all the functions of the senses and the functions
of the breath (vital energy) in the fire of the YOGA of self-restraint, kindled
by knowledge.

\begin{gitaverse}
द्रव्ययज्ञास्तपोयज्ञा योगयज्ञास्तथापरे । \\
स्वाध्यायज्ञानयज्ञाश्च यतयः संशितव्रताः ॥२८॥
\end{gitaverse}

\begin{transliteration}
dravyayajñāstapoyajñā yogayajñāstathāpare, \\
svādhyāyajñānayajñāśca yatayaḥ saṁśitavratāḥ.
\end{transliteration}

28. Others again offer wealth, austerity and YOGA as sacrifice, while the
ascetics of self-restraint and rigid vows offer study of scriptures and
knowledge as sacrifice.

\begin{gitaverse}
अपाने जुह्वति प्राणं प्राणेऽपानं तथापरे । \\
प्राणापानगती रुद्ध्वा प्राणायामपरायणाः ॥२९॥
\end{gitaverse}

\begin{transliteration}
apāne juhvati prāṇaṁ prāṇe'pānaṁ tathāpare, \\
prāṇāpānagatī ruddhvā prāṇāyāmaparāyaṇāḥ.
\end{transliteration}

29. Others offer as sacrifice the out-going breath in the in-coming, and the
in-coming in the out-going, restraining the courses of the outgoing and
in-coming breaths, solely absorbed in the restraint of breath.

\begin{gitaverse}
अपरे नियताहाराः प्राणान्प्राणेषु जुह्वति । \\
सर्वेऽप्येते यज्ञविदो यज्ञक्षपितकल्मषाः ॥३०॥
\end{gitaverse}

\begin{transliteration}
apare niyatāhārāḥ prāṇānprāṇeṣu juhvati, \\
sarve'pyete yajñavido yajñakṣapitakalmaṣāḥ.
\end{transliteration}

30. Others, with well-regulated diet, offer vital-airs in the Vital-Air. All
these are knowers of sacrifice, whose sins are destroyed by sacrifice.

\begin{gitaverse}
यज्ञशिष्टामृतभुजो यान्ति ब्रह्म सनातनम् । \\
नायं लोकोऽस्त्ययज्ञस्य कुतोऽन्यः कुरुसत्तम ॥३१॥
\end{gitaverse}

\begin{transliteration}
yajñaśiṣṭāmṛtabhujo yānti brahma sanātanam, \\
nāyaṁ loko'styayajñasya kuto'nyaḥ kurusattama.
\end{transliteration}

31. The eaters of the nectar, remnant of the sacrifice, go to the Eternal
BRAHMAN.\@ Even this world is not for the nonperformer of sacrifice; how then
the other (world), O best of the Kurus?

\begin{gitaverse}
एवं बहुविधा यज्ञा वितता ब्रह्मणो मुखे । \\
कर्मजान्विद्धि तान्सर्वानेवं ज्ञात्वा विमोक्ष्यसे ॥३२॥
\end{gitaverse}

\begin{transliteration}
evaṁ bahuvidhā yajñā vitatā brahmaṇo mukhe, \\
karmajānviddhi tānsarvānevaṁ jñātvā vimokṣyase.
\end{transliteration}

32. Thus innumerable sacrifices lie spread out before BRAHMAN---(literally at
the mouth or face of BRAHMAN)---Know them all as born of action, and thus
knowing, you shall be liberated.

\begin{gitaverse}
श्रेयान्द्रव्यमयाद्यज्ञाज्ज्ञानयज्ञः परन्तप । \\
सर्वं कर्माखिलं पार्थ ज्ञाने परिसमाप्यते ॥३३॥
\end{gitaverse}

\begin{transliteration}
śreyāndravyamayādyajñājjñānayajñaḥ parantapa, \\
sarvaṁ karmākhilaṁ pārtha jñāne parisamāpyate.
\end{transliteration}

33. Superior is `knowledge-sacrifice' to `sacrifice-with-objects', O Parantapa.
All actions in their entirety, O Partha, culminate in Knowledge.

\begin{gitaverse}
तद्विद्धि प्रणिपातेन परिप्रश्नेन सेवया । \\
उपदेक्ष्यन्ति ते ज्ञानं ज्ञानिनस्तत्त्वदर्शिनः ॥३४॥
\end{gitaverse}

\begin{transliteration}
tadviddhi praṇipātena paripraśnena sevayā, \\
upadekṣyanti te jñānaṁ jñāninastattvadarśinaḥ.
\end{transliteration}

34. Know that by long prostration, by question, and service, the `wise' who
have realised the Truth will instruct you in (that) `Knowledge'.

\begin{gitaverse}
यज्ज्ञात्वा न पुनर्मोहमेवं यास्यसि पाण्डव । \\
येन भूतान्यशेषेण द्रक्ष्यस्यात्मन्यथो मयि ॥३५॥
\end{gitaverse}

\begin{transliteration}
yajjñātvā na punarmohamevaṁ yāsyasi pāṇḍava, \\
yena bhūtānyaśeṣeṇa drakṣyasyātmanyatho mayi.
\end{transliteration}

35. Knowing that, you shall not, O Pandava, again get deluded like this; and by
that, you shall see all beings in your Self, and also in Me.

\begin{gitaverse}
अपि चेदसि पापेभ्यः सर्वेभ्यः पापकृत्तमः । \\
सर्वं ज्ञानप्लवेनैव वृजिनं सन्तरिष्यसि ॥३६॥
\end{gitaverse}

\begin{transliteration}
api cedasi pāpebhyaḥ sarvebhyaḥ pāpakṛttamaḥ, \\
sarvaṁ jñānaplavenaiva vṛjinaṁ santariṣyasi.
\end{transliteration}

36. Even if you are the most sinful of all sinners, yet you shall verily cross
all sins by the raft of `Knowledge.'

\begin{gitaverse}
यथैधांसि समिद्धोऽग्निर्भस्मसात्कुरुतेऽर्जुन । \\
ज्ञानाग्निः सर्वकर्माणि भस्मसात्कुरुते तथा ॥३७॥
\end{gitaverse}

\begin{transliteration}
yathaidhāṁsi samiddho'gnirbhasmasātkurute'rjuna, \\
jñānāgniḥ sarvakarmāṇi bhasmasātkurute tathā
\end{transliteration}

37. As the blazing fire reduces fuel to ashes, O Arjuna, so does the
Fire-of-Knowledge reduce all actions to ashes.

\begin{gitaverse}
न हि ज्ञानेन सदृशं पवित्रमिह विद्यते । \\
तत्स्वयं योगसंसिद्धः कालेनात्मनि विन्दति ॥३८॥
\end{gitaverse}

\begin{transliteration}
na hi jñānena sadṛśaṁ pavitramiha vidyate, \\
tatsvayaṁ yogasaṁsiddhaḥ kālenātmani vindati.
\end{transliteration}

38. Certainly, there is no purifier in this world like `Knowledge'. He who is
himself perfected in YOGA finds it in the Self in time.

\begin{gitaverse}
श्रद्धावाँल्लभते ज्ञानं तत्परः संयतेन्द्रियः । \\
ज्ञानं लब्ध्वा परां शान्तिमचिरेणाधिगच्छति ॥३९॥
\end{gitaverse}

\begin{transliteration}
śraddhāvāṁllabhate jñānaṁ tatparaḥ saṁyatendriyaḥ, \\
jñānaṁ labdhvā parāṁ śāntimacireṇādhigacchati.
\end{transliteration}

39. The man who is full of faith, who is devoted to It, and who has subdued the
senses, obtains (this) `Knowledge'; and having obtained `Knowledge,' ere long
he goes to the Supreme Peace.

\begin{gitaverse}
अज्ञश्चाश्रद्दधानश्च संशयात्मा विनश्यति । \\
नायं लोकोऽस्ति न परो न सुखं संशयात्मनः ॥४०॥
\end{gitaverse}

\begin{transliteration}
ajñaścāśraddadhānaśca saṁśayātmā vinaśyati, \\
nāyaṁ loko'sti na paro na sukhaṁ saṁśayātmanaḥ.
\end{transliteration}

40. The ignorant, the faithless, the doubting-self goes to destruction; there
is neither this world, nor the other, nor happiness for the doubter.

\begin{gitaverse}
योगसन्न्यस्तकर्माणं ज्ञानसञ्छिन्नसंशयम् । \\
आत्मवन्तं न कर्माणि निबध्नन्ति धनञ्जय ॥४१॥
\end{gitaverse}

\begin{transliteration}
yogasannyastakarmāṇaṁ jñānasañchinnasaṁśayam, \\
ātmavantaṁ na karmāṇi nibadhnanti dhanañjaya.
\end{transliteration}

41. He who has renounced actions by YOGA, whose doubts are rent asunder by
`Knowledge', who is self-possessed, actions do not bind him, O Dhananjaya.

\begin{gitaverse}
तस्मादज्ञानसम्भूतं हृत्स्थं ज्ञानासिनात्मनः । \\
छित्त्वैनं संशयं योगमातिष्ठोत्तिष्ठ भारत ॥४२॥
\end{gitaverse}

\begin{transliteration}
tasmādajñānasambhūtaṁ hṛtsthaṁ jñānāsinātmanaḥ, \\
chittvainaṁ saṁśayaṁ yogamātiṣṭhottiṣṭha bhārata.
\end{transliteration}

42. Therefore with the sword-of-Knowledge, cut asunder the doubt-of-the-Self,
born of `ignorance', residing in your heart, and take refuge in `YOGA'. Arise,
O Bharata.

\begin{gitaverse}
ॐ तत्सदिति श्रीमद् भगवद् गीतासूपनिषत्सु ब्रह्मविद्यायां \\
योगशास्त्रे श्रीकृष्णार्जुनसंवादे ज्ञानकर्मसंन्यासयोगो नाम \\
चतुर्थोऽध्यायः
\end{gitaverse}

\begin{transliteration}
oṁ tatsaditi śrīmad bhagavad gītāsūpaniṣatsu brahmavidyāyāṁ \\
yogaśāstre śrī kṛṣṇārjuna saṁvāde jñānakarmasannyāsayogo nāma \\
caturtho'dhyāyaḥ.
\end{transliteration}

Thus, in the UPANISHADS of the glorious Bhagawad Geeta, in the Science of the
Eternal, in the scripture of YOGA, in the dialogue between Sri Krishna and
Arjuna, the fourth discourse ends entitled: The Yoga of Renunciation of Action
In Knowledge
