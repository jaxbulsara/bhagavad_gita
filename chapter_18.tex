\chapterdrop

\begin{center}
\headersanskrit{अथ अष्टादशोऽध्यायः}

\headerspace
\headertransliteration{Atha Aṣṭādaśo'dhyāyaḥ}

\section{Chapter 18}

\headerspace
\headersanskrit{मोक्षसन्न्यासयोगः}

\headerspace
\headertransliteration{Mokṣa Sannyāsa Yogah}

\headerspace
\headertranslation{The Yoga of Liberation Through Renunciation}

\headerspace
\end{center}

\begin{gitaverse}
अर्जुन उवाच \\
सन्न्यासस्य महाबाहो तत्त्वमिच्छामि वेदितुम् । \\
त्यागस्य च हृषीकेश पृथक्केशिनिषूदन ॥१॥
\end{gitaverse}

\begin{transliteration}
arjuna uvāca \\
sannyāsasya mahābāho tattvam-icchāmi veditum, \\
tyāgasya ca hṛṣīkeśa pṛthak-keśiniṣūdana.
\end{transliteration}

Arjuna said: \\
1. I desire to know severally, O mighty-armed, the essence or truth of
`Renunciation', O Hrishikesa, as also of `Abandonment', O slayer of Keshi
(Krishna).

\begin{gitaverse}
श्रीभगवानुवाच \\
काम्यानां कर्मणां न्यासं सन्न्यासं कवयो विदुः । \\
सर्वकर्मफलत्यागं प्राहुस्त्यागं विचक्षणाः ॥२॥
\end{gitaverse}

\begin{transliteration}
śrībhagavānuvāca \\
kāmyānāṁ karmaṇāṁ nyāsaṁ sannyāsaṁ kavayo viduḥ, \\
sarvakarmaphalatyāgaṁ prāhustyāgaṁ vicakṣaṇāḥ.
\end{transliteration}

The Blessed Lord said: \\
2. The Sages understand SANNYASA to be ``the renunciation of works with
desire''; the wise declare ``the abandonment of the fruits of all actions'' as
TYAGA.\@

\begin{gitaverse}
त्याज्यं दोषवदित्येके कर्म प्राहुर्मनीषिणः । \\
यज्ञदानतपःकर्म न त्याज्यमिति चापरे ॥३॥
\end{gitaverse}

\begin{transliteration}
tyājyaṁ doṣavad-ityeke karma prāhur-manīṣiṇaḥ, \\
yajñadānatapaḥkarma na tyājyam-iti cāpare.
\end{transliteration}

3. All actions should be abandoned as evil, declare some philosophers; while
others (declare) that acts of sacrifice, gift and austerity should not be
relinquished.

\begin{gitaverse}
निश्चयं श्रुणु मे तत्र त्यागे भरतसत्तम । \\
त्यागो हि पुरुषव्याघ्र त्रिविधः सम्प्रकीर्तितः ॥४॥
\end{gitaverse}

\begin{transliteration}
niścayaṁ śṛṇu me tatra tyāge bharatasattama, \\
tyāgo hi puruṣavyāghra trividhaḥ samprakīrtitaḥ.
\end{transliteration}

4. Hear from Me the conclusion or the final truth, about this `abandonment', O
best of the Bharatas; `abandonment', verily, O best of men, has been declared
to be of three kinds.

\begin{gitaverse}
यज्ञदानतपःकर्म न त्याज्यं कार्यमेव तत् । \\
यज्ञो दानं तपश्चैव पावनानि मनीषिणाम् ॥५॥
\end{gitaverse}

\begin{transliteration}
yajñadānatapaḥkarma na tyājyaṁ kāryam-eva tat, \\
yajño dānaṁ tapaścaiva pāvanāni manīṣiṇām.
\end{transliteration}

5. Acts of sacrifice, charity and austerity should not be abandoned, but should
be performed; worship, charity, and also austerity, are the purifiers of even
the `wise'.

\begin{gitaverse}
एतान्यपि तु कर्माणि सङ्गं त्यक्त्वा फलानि च । \\
कर्तव्यानीति मे पार्थ निश्चितं मतमुत्तमम् ॥६॥
\end{gitaverse}

\begin{transliteration}
etānyapi tu karmāṇi saṅgaṁ tyaktvā phalāni ca, \\
kartavyānīti me pārtha niścitaṁ matam-uttamam.
\end{transliteration}

6. But even these actions should be performed leaving aside attachment and the
fruits, O Partha; this is my certain and best belief.

\begin{gitaverse}
नियतस्य तु सन्न्यासः कर्मणो नोपपद्यते । \\
मोहात्तस्य परित्यागस्तामसः परिकीर्तितः ॥७॥
\end{gitaverse}

\begin{transliteration}
niyatasya tu sannayāsaḥ karmaṇo nopapadyate, \\
mohāt-tasya parityāgas-tāmasaḥ parikīrtitaḥ.
\end{transliteration}

7. Verily, the renunciation of `obligatory actions' is not proper; the
abandonment of the same from delusion is declared to be TAMASIC (dull).

\begin{gitaverse}
दुःखमित्येव यत्कर्म कायक्लेशभयात्त्यजेत् । \\
स कृत्वा राजसं त्यागं नैव त्यागफलं लभेत् ॥८॥
\end{gitaverse}

\begin{transliteration}
duḥkhamityeva yatkarma kāya-kleśa-bhayāt-tyajet, \\
sa kṛtvā rājasaṁ tyāgaṁ naiva tyāgaphalaṁ labhet.
\end{transliteration}

8. He who, because of fear of bodily trouble, abandons action because it is
painful, thus performing a RAJASIC (passionate) abandonment, obtains not the
fruit of `abandonment'.

\begin{gitaverse}
कार्यमित्येव यत्कर्म नियतं क्रियतेऽर्जुन । \\
सङ्गं त्यक्त्वा फलं चैव स त्यागः सात्त्विको मतः ॥९॥
\end{gitaverse}

\begin{transliteration}
kāryamityeva yatkarma niyataṁ kriyate'rjuna, \\
saṅgaṁ tyaktvā phalaṁ caiva sa tyāgaḥ sāttviko mataḥ.
\end{transliteration}

9. Whatever `obligatory action' is done, O Arjuna, merely because it ought to
be done, abandoning `attachment and also fruit', that abandonment is regarded
as SATTVIC (pure).

\begin{gitaverse}
न द्वेष्ट्यकुशलं कर्म कुशले नानुषज्जते । \\
त्यागी सत्त्वसमाविष्टो मेधावी छिन्नसंशयः ॥१०॥
\end{gitaverse}

\begin{transliteration}
na dveṣtyakuśalaṁ karma kuśale nānuṣajjate, \\
tyāgī sattvasamāviṣṭo medhāvī chinnasaṁśayaḥ.
\end{transliteration}

10. The abandoner, soaked in purity, being intelligent, with all his doubts cut
asunder, hates not disagreeable action, nor is he attached to an agreeable
action.

\begin{gitaverse}
न हि देहभृता शक्यं त्यक्तुं कर्माण्यशेषतः । \\
यस्तु कर्मफलत्यागी स त्यागीत्यभिधीयते ॥११॥
\end{gitaverse}

\begin{transliteration}
na hi dehabhṛtā śakyaṁ tyaktuṁ karmāṇyaśeṣataḥ, \\
yastu karmaphalatyāgī sa tyāgītyabhidhīyate.
\end{transliteration}

11. Verily, it is not possible for an embodied being to abandon actions
entirely, but he who relinquishes the fruits-of-actions is verily called a
`relinquisher' (Tyagi).

\begin{gitaverse}
अनिष्टमिष्टं मिश्रं च त्रिविधं कर्मणः फलम् । \\
भवत्यत्यागिनां प्रेत्य न तु सन्न्यासिनां क्वचित् ॥१२॥
\end{gitaverse}

\begin{transliteration}
aniṣṭam-iṣṭaṁ miśraṁ ca trividhaṁ karmaṇaḥ phalam, \\
bhavatyatyāgināṁ pretya na tu sannyāsināṁ kvacit.
\end{transliteration}

12. The threefold fruit-of-action, evil, good and mixed-accrues, after death,
only to those who have no spirit of `abandonment'; never to total
relinquishers.

\begin{gitaverse}
पञ्चैतानि महाबाहो कारणानि निबोध मे । \\
साङ्ख्ये कृतान्ते प्रोक्तानि सिद्धये सर्वकर्मणाम् ॥१३॥
\end{gitaverse}

\begin{transliteration}
pañcaitāni mahābāho kāraṇāni nibhodha me, \\
sāṅkhye kṛtānte proktāni siddhaye sarvakarmaṇām.
\end{transliteration}

13. Learn from Me, O mighty-armed, these five causes for the accomplishment of
all actions, as declared in the SANKHYA (UPANISHAD) system, which is the end of
all actions.

\begin{gitaverse}
अधिष्ठानं तथा कर्ता करणं च पृथग्विधम् । \\
विविधाश्च पृथक्चेष्टा दैवं चैवात्र पञ्चमम् ॥१४॥
\end{gitaverse}

\begin{transliteration}
adhiṣṭhānaṁ tathā kartā karaṇaṁ ca pṛthagvidhaṁ, \\
vividhāśca pṛthak-ceṣṭā daivaṁ caivātra pañcamam.
\end{transliteration}

14. The `seat' (body), the doer (ego), the various organs-of-perception, the
different functions of various organs-of-action, and also the presiding deity,
the fifth.

\begin{gitaverse}
शरीरवाङ्मनोभिर्यत्कर्म प्रारभते नरः । \\
न्याय्यं वा विपरीतं वा पञ्चैते तस्य हेतवः ॥१५॥
\end{gitaverse}

\begin{transliteration}
śarīra-vāṅ-manobhir-yat-karma prārabhate naraḥ, \\
nyāyyaṁ vā viparītaṁ vā pañcaite tasya hetavaḥ.
\end{transliteration}

15. Whatever action a man performs by his body, speech and mind---whether
right, or the reverse---these five are its causes.

\begin{gitaverse}
तत्रैवं सति कर्तारमात्मानं केवलं तु यः । \\
पश्यत्यकृतबुद्धित्वान्न स पश्यति दुर्मतिः ॥१६॥
\end{gitaverse}

\begin{transliteration}
tatraivaṁ sati kartāram-ātmānaṁ kevalaṁ tu yaḥ, paśyatyakrtabuddhitvān-na sa paśyati durmatiḥ.
\end{transliteration}

16. Now, such being the case, verily he who---owing to his untrained
understanding---looks upon his Self, which is `alone' (never conditioned by the
`engine'), as the doer, he, of perverted intelligence, sees not.

\begin{gitaverse}
यस्य नाहङ्कृतो भावो बुद्धिर्यस्य न लिप्यते । \\
हत्वापि स इमाँल्लोकान्न हन्ति न निबध्यते ॥१७॥
\end{gitaverse}

\begin{transliteration}
yasya nāhaṅkṛto bhāvo buddhir-yasya na lipyate, \\
hatvāpi sa imāllokān-na hanti na nibadhyate.
\end{transliteration}

17. He who is free from the egoistic notion, whose intelligence is not tainted
(by good or evil), though he slays these people, he slays not, nor is he bound
(by the action).

\begin{gitaverse}
ज्ञानं ज्ञेयं परिज्ञाता त्रिविधा कर्मचोदना । \\
करणं कर्म कर्तेति त्रिविधः कर्मसङ्ग्रहः ॥१८॥
\end{gitaverse}

\begin{transliteration}
jñānaṁ jñeyaṁ parijñātā trividhā karmacodanā, \\
karaṇaṁ karma karteti trividhaḥ karmasaṅgrahaḥ.
\end{transliteration}

18. Knowledge, the known and knower form the threefold `impulse to action'; the
organs, the action, the agent, form the threefold `basis of action'.

\begin{gitaverse}
ज्ञानं कर्म च कर्ता च त्रिधैव गुणभेदतः । \\
प्रोच्यते गुणसङ्ख्याने यथावच्छृणु तान्यपि ॥१९॥
\end{gitaverse}

\begin{transliteration}
jñānaṁ karma ca kartā ca tridhaiva guṇabhedataḥ, \\
procyate guṇasaṅkhyāne yathāvacchṛṇu tānyapi.
\end{transliteration}

19. `Knowledge', `action', and `actor' are declared in the science of
temperaments (gunas) to be of three kinds only, according to the distinctions
of temperaments; hear them also duly.

\begin{gitaverse}
सर्वभूतेषु येनैकं भावमव्ययमीक्षते । \\
अविभक्तं विभक्तेषु तज्ज्ञानं विद्धि सात्त्विकम् ॥२०॥
\end{gitaverse}

\begin{transliteration}
sarvabhūteṣu yenaikaṁ bhāvam-avyayam-īkṣate, \\
avibhaktaṁ vibhakteṣu tajjñānaṁ viddhi sāttvikam.
\end{transliteration}

20. That by which one sees the one indestructible reality in all beings,
undivided in the divided, know that `knowledge' as SATTVIC (Pure).

\begin{gitaverse}
पृथक्त्वेन तु यज्ज्ञानं नानाभावान्पृथग्विधान् । \\
वेत्ति सर्वेषु भूतेषु तज्ज्ञानं विद्धि राजसम् ॥२१॥
\end{gitaverse}

\begin{transliteration}
pṛthaktvena tu yajjñānaṁ nānābhāvān-pṛthagvidhān, \\
vetti sarveṣu bhūteṣu tajjñānaṁ viddhi rājasam.
\end{transliteration}

21. But that `knowledge' which sees in all beings various entities of distinct
kinds, (and) as different from one another, know that knowledge as RAJASIC
(Passionate).

\begin{gitaverse}
यत्तु कृत्स्नवदेकस्मिन्कार्ये सक्तमहैतुकम् । \\
अतत्त्वार्थवदल्पं च तत्तामसमुदाहृतम् ॥२२॥
\end{gitaverse}

\begin{transliteration}
yattu kṛtsnavad-ekasmin-kārye saktam-ahaitukam, \\
atattvārthavad-alpaṁ ca tat-tāmasam-udāhṛtam.
\end{transliteration}

22. But that `knowledge', which clings to one single effect, as if it were the
whole, without reason, without foundation in truth and narrow, that is declared
to be TAMASIC (Dull).

\begin{gitaverse}
नियतं सङ्गरहितमरागद्वेषतः कृतम् । \\
अफलप्रेप्सुना कर्म यत्तत्सात्त्विकमुच्यते ॥२३॥
\end{gitaverse}

\begin{transliteration}
niyataṁ saṅgarahitam-arāgadveṣataḥ kṛtam, \\
aphala-prepsunā karma yat-tat-sāttvikam-ucyate.
\end{transliteration}

23. An `action' which is ordained, which is free from attachment, which is done
without love or hatred, by one who is not desirous of the fruit, that action is
declared to SATTVIC (pure).

\begin{gitaverse}
यत्तु कामेप्सुना कर्म साहङ्कारेण वा पुनः । \\
क्रियते बहुलायासं तद्राजसमुदाहृतम् ॥२४॥
\end{gitaverse}

\begin{transliteration}
yat-tu kāmepsunā karma sāhaṅkāreṇa vā punaḥ, \\
kriyate bahulāyāsaṁ tad-rājasam-udāhṛtam.
\end{transliteration}

24. But that `action' which is done by one, longing for desires, or gain, done
with egoism, or with much effort, is declared to be RAJASIC (Passionate).

\begin{gitaverse}
अनुबन्धं क्षयं हिंसामनवेक्ष्य च पौरुषम् । \\
मोहादारभ्यते कर्म यत्तत्तामसमुच्यते ॥२५॥
\end{gitaverse}

\begin{transliteration}
anubandhaṁ kṣayaṁ hiṁsām-anavekṣya ca pauruṣam, \\
mohād-ārabhyate karma yat-tat-tāmasam-ucyate.
\end{transliteration}

25. That `action' which is undertaken from delusion, without regard for the
consequence, loss, injury, and ability, is declared to be TAMASIC (dull).

\begin{gitaverse}
मुक्तसङ्गोऽनहंवादी धृत्युत्साहसमन्वितः । \\
सिद्ध्यसिद्ध्योर्निर्विकारः कर्ता सात्त्विक उच्यते ॥२६॥
\end{gitaverse}

\begin{transliteration}
muktasaṅgo'nahaṁvādī dhṛtyutsāha-samanvitaḥ, \\
siddhyasiddhyor-nirvikāraḥ kartā sāttvika ucyate.
\end{transliteration}

26. An `agent' who is free from attachment, non-egoistic, endowed with firmness
and enthusiasm, and unaffected by success or failure, is called SATTVIC (pure).

\begin{gitaverse}
रागी कर्मफलप्रेप्सुर्लुब्धो हिंसात्मकोऽशुचिः । \\
हर्षशोकान्वितः कर्ता राजसः परिकीर्तितः ॥२७॥
\end{gitaverse}

\begin{transliteration}
rāgī karmaphalaprepsur-lubdho hiṁsātmako'śuciḥ, \\
harṣaśokānvitaḥ kartā rājasaḥ parikīrtitaḥ.
\end{transliteration}

27. Passionate, desiring to gain the fruits-of-actions, greedy, harmful,
impure, full of delight and grief, such an `agent' is said to be RAJASIC
(passionate).

\begin{gitaverse}
अयुक्तः प्राकृतः स्तब्धः शठोऽनैष्कृतिकोऽलसः । \\
विषादी दीर्घसूत्री च कर्ता तामस उच्यते ॥२८॥
\end{gitaverse}

\begin{transliteration}
ayuktaḥ prākṛtaḥ stabdhaḥ śaṭho'naiṣkṛtiko'lasaḥ, \\
viṣādī dīrghasūtrī ca karta tāmasa ucyate.
\end{transliteration}

28. Unsteady, vulgar, unbending, cheating, malicious, lazy, despondent, and
procrastinating, such an `agent' is said to be TAMASIC (Dull).

\begin{gitaverse}
बुद्धेर्भेदं धृतेश्चैव गुणतस्त्रिविधंश्रुणु । \\
प्रोच्यमानमशेषेण पृथक्त्वेन धनञ्जय ॥२९॥
\end{gitaverse}

\begin{transliteration}
buddher-bhedaṁ dhṛteścaiva guṇatas-trividhaṁ śṛṇu, \\
procyamānam-aśeṣeṇa pṛthaktvena dhanañjaya.
\end{transliteration}

29. Hear (you) the threefold division of `understanding' and `fortitude' (made)
according to the qualities, as I declare them fully and severally, O
Dhananjaya.

\begin{gitaverse}
प्रवृत्तिं च निवृत्तिं च कार्याकार्ये भयाभये । \\
बन्धं मोक्षं च या वेत्ति बुद्धिः सा पार्थ सात्त्विकी ॥३०॥
\end{gitaverse}

\begin{transliteration}
pravṛttiṁ ca nivṛttiṁ ca kāryākārye bhayābhaye, \\
bandhaṁ mokṣaṁ ca yā vetti buddhiḥ sā pārtha sāttvikī.
\end{transliteration}

30. That which knows the paths of work and renunciation, what ought to be done
and what ought not to be done, fear and fearlessness, bondage and liberation,
that `understanding' is SATTVIC (pure), O Partha.

\begin{gitaverse}
यया धर्ममधर्मं च कार्यं चाकार्यमेव च । \\
अयथावत्प्रजानाति बुद्धिः सा पार्थ राजसी ॥३१॥
\end{gitaverse}

\begin{transliteration}
yayā dharmam-adharmaṁ ca kāryaṁ cākāryam-eva ca, \\
ayathāvat-prajānāti buddhiḥ sā pārtha rājasī.
\end{transliteration}

31. That by which one wrongly understands DHARMA and ADHARMA and also what
ought to be done and what ought not to be done, that intellect (understanding),
O Partha, is RAJASIC (passionate).

\begin{gitaverse}
अधर्मं धर्ममिति या मन्यते तमसावृता । \\
सर्वार्थान्विपरीतांश्च बुद्धिः सा पार्थ तामसी ॥३२॥
\end{gitaverse}

\begin{transliteration}
adharmaṁ dharmam-iti ya manyate tamasāvṛtā, \\
sarvārthān-viparītāṁsca buddhiḥ sā pārtha tāmasī.
\end{transliteration}

32. That which, enveloped in darkness, sees ADHARMA as DHARMA, and all things
perverted, that intellect (understanding), O Partha, is TAMASIC (dull).

\begin{gitaverse}
धृत्या यया धारयते मनःप्राणेन्द्रियक्रियाः । \\
योगेनाव्यभिचारिण्या धृतिः सा पार्थ सात्त्विकी ॥३३॥
\end{gitaverse}

\begin{transliteration}
dhṛtyā yayā dhārayate manaḥ-prāṇendriya-kriyāḥ, \\
yogenāvyabhicāriṇyā dhṛtiḥ sā pārtha sāttvikī.
\end{transliteration}

33. The unwavering `fortitude' by which, through YOGA, the functions of the
mind, the PRANA and the senses are restrained, that `fortitude', O Partha, is
SATTVIC (pure).

\begin{gitaverse}
यया तु धर्मकामार्थान्धृत्या धारयतेऽर्जुन । \\
प्रसङ्गेन फलाकाङ्क्षी धृतिः सा पार्थ राजसी ॥३४॥
\end{gitaverse}

\begin{transliteration}
yayā tu dharma-kāmārthān-dhṛtyā dhārayate'rjuna, \\
prasaṅgena phalākāṅkṣī dhṛtiḥ sa pārtha rājasī.
\end{transliteration}

34. But the `fortitude', O Arjuna, by which one holds to duty, pleasure and
wealth, one of attachment and craving for the fruits-of-actions, that
`fortitude', O Partha, is RAJASIC (passionate).

\begin{gitaverse}
यया स्वप्नं भयं शोकं विषादं मदमेव च । \\
न विमुञ्चति दुर्मेधा धृतिः सा पार्थ तामसी ॥३५॥
\end{gitaverse}

\begin{transliteration}
yayā svapnaṁ bhayaṁ śokaṁ viṣadaṁ madam-eva ca, \\
na vimuñcati durmedhā dhṛtiḥ sā pārtha tāmasī.
\end{transliteration}

35. The `constancy' because of which a stupid man does not abandon sleep, fear,
grief, depression, and also arrogance (conceit), that `fortitude', O Partha, is
TAMASIC (dull).

\begin{gitaverse}
सुखं त्विदानीं त्रिविधंश्रुणु मे भरतर्षभ । \\
अभ्यासाद्रमते यत्र दुःखान्तं च निगच्छति ॥३६॥
\end{gitaverse}

\begin{transliteration}
sukhaṁ tvidānīṁ trividhaṁ śṛṇu me bharatarṣabha, \\
abhyāsād-ramate yatra duḥkāntaṁ ca nigacchati.
\end{transliteration}

36. And now hear from me, O best among the Bharatas, of the threefold
`pleasure,' in which one rejoices by practice, and surely comes to the end of
pain.

\begin{gitaverse}
यत्तदग्रे विषमिव परिणामेऽमृतोपमम् । \\
तत्सुखं सात्त्विकं प्रोक्तमात्मबुद्धिप्रसादजम् ॥३७॥
\end{gitaverse}

\begin{transliteration}
yat-tad-agre viṣamiva pariṇāme'mṛtopamam, \\
tatsukhaṁ sāttvikaṁ proktam-ātmabuddhiprasādajam.
\end{transliteration}

37. That which is like poison at first, but in the end like nectar, that
`pleasure' is declared to be SATTVIC (pure), born of the purity of one's own
mind, due to Self realisation.

\begin{gitaverse}
विषयेन्द्रियसंयोगाद्यत्तदग्रेऽमृतोपमम् । \\
परिणामे विषमिव तत्सुखं राजसं स्मृतम् ॥३८॥
\end{gitaverse}

\begin{transliteration}
viṣayendriya-saṁyogād-yat-tad-agre'mṛtopamam, \\
pariṇāme viṣamiva tat-sukhaṁ rājasaṁ smṛtam.
\end{transliteration}

38. That pleasure which arises from the contact of the sense-organs with the
objects, (which is) at first like nectar, (but is) in the end like poison, that
is declared to be RAJASIC (passionate).

\begin{gitaverse}
यदग्रे चानुबन्धे च सुखं मोहनमात्मनः । \\
निद्रालस्यप्रमादोत्थं तत्तामसमुदाहृतम् ॥३९॥
\end{gitaverse}

\begin{transliteration}
yad-agre cānubandhe ca sukhaṁ mohanam-ātmanaḥ, \\
nidrālasya-pramādotthaṁ tat-tāmasam-udahṛtam.
\end{transliteration}

39. The pleasure, which at first and in the sequel deludes the Self, arising
from sleep, indolence and heedlessness, is declared to be TAMASIC (Dull).

\begin{gitaverse}
न तदस्ति पृथिव्यां वा दिवि देवेषु वा पुनः । \\
सत्त्वं प्रकृतिजैर्मुक्तं यदेभिः स्यात्त्रिभिर्गुणैः ॥४०॥
\end{gitaverse}

\begin{transliteration}
na tadasti pṛthivyāṁ vā divi deveṣu vā punaḥ, \\
sattvaṁ prakṛtijairṁuktaṁ yadebhiḥ syāttribhirguṇaih.
\end{transliteration}

40. There is no being on earth, or again in heaven among the ``DEVAS''
(heavenly beings), who is totally liberated from the three qualities, born of
PRAKRITI (matter).

\begin{gitaverse}
ब्राह्मणक्षत्रियविशां शूद्राणां च परन्तप । \\
कर्माणि प्रविभक्तानि स्वभावप्रभवैर्गुणैः ॥४१॥
\end{gitaverse}

\begin{transliteration}
brāhmaṇa-kṣatriya-viśām sūdrāṇām ca parantapa, \\
karmāṇi pravibhaktāni svabhāva-prabhavair-guṇaiḥ.
\end{transliteration}

41. Of scholars (BRAHMANAS), of leaders (KSHATRIYAS) and of traders (VAISHYAS),
as also of workers (SHUDRAS), O Parantapa, the duties are distributed according
to the qualities born of their own nature.

\begin{gitaverse}
शमो दमस्तपः शौचं क्षान्तिरार्जवमेव च । \\
ज्ञानं विज्ञानमास्तिक्यं ब्रह्मकर्म स्वभावजम् ॥४२॥
\end{gitaverse}

\begin{transliteration}
śamo damas-tapaḥ śaucaṁ kṣāntir-ārjavam-eva ca, \\
jñanaṁ vijñānam-āstikyaṁ brahmakarma svabhāvajam.
\end{transliteration}

42. Serenity, self-restraint, austerity, purity, forgiveness and also
uprightness, knowledge, realisation, belief in God---are the duties of the
BRAHMANAS, born of (their own) nature.

\begin{gitaverse}
शौर्यं तेजो धृतिर्दाक्ष्यं युद्धे चाप्यपलायनम् । \\
दानमीश्वरभावश्च क्षात्रं कर्म स्वभावजम् ॥४३॥
\end{gitaverse}

\begin{transliteration}
śauryaṁ tejo dhṛtir-dākṣyaṁ yuddhe cāpyapalāyanam, \\
dānam-īśvarabhāvaśca kṣātraṁ karma svabhāvajam.
\end{transliteration}

43. Prowess, splendour, firmness, dexterity, and also not fleeing from battle,
generosity, lordliness---these are the duties of the KSHATRIYAS, born of (their
own) nature.

\begin{gitaverse}
कृषिगौरक्ष्यवाणिज्यं वैश्यकर्म स्वभावजम् । \\
परिचर्यात्मकं कर्म शूद्रस्यापि स्वभावजम् ॥४४॥
\end{gitaverse}

\begin{transliteration}
kṛṣi-gaurakṣya-vāṇijyaṁ vaiśyakarma svabhāvajam, \\
paricaryātmakaṁ karma śūdrasyāpi svabhāvajam.
\end{transliteration}

44. Agriculture, cattle-rearing and trade are the duties of the VAISHYAS, born
of (their own) nature; and service is the duty of the SHUDRAS, born of (their
own) nature.

\begin{gitaverse}
स्वे स्वे कर्मण्यभिरतः संसिद्धिं लभते नरः । \\
स्वकर्मनिरतः सिद्धिं यथा विन्दति तच्छृणु ॥४५॥
\end{gitaverse}

\begin{transliteration}
sve sve karmaṇyabhirataḥ saṁsiddhiṁ labhate naraḥ, \\
svakarmanirataḥ siddhiṁ yathā vindati tacchṛṇu.
\end{transliteration}

45. Devoted, each to his own duty, man attains Perfection. How, engaged in his
own duty, he attains Perfection, listen.

\begin{gitaverse}
यतः प्रवृत्तिर्भूतानां येन सर्वमिदं ततम् । \\
स्वकर्मणा तमभ्यर्च्य सिद्धिं विन्दति मानवः ॥४६॥
\end{gitaverse}

\begin{transliteration}
yataḥ pravṛttir-bhūtānāṁ yena sarvam-idaṁ tatam, \\
svakarmaṇā tam-abhyarcya siddhiṁ vindati mānavaḥ.
\end{transliteration}

46. From all beings arise, by Whom all this is pervaded, worshipping Him with
one's own duty, man attains Perfection.

\begin{gitaverse}
श्रेयान्स्वधर्मो विगुणः परधर्मात्स्वनुष्ठितात् । \\
स्वभावनियतं कर्म कुर्वन्नाप्नोति किल्बिषम् ॥४७॥
\end{gitaverse}

\begin{transliteration}
śreyān-svadharmo viguṇaḥ paradharmāt-svanuṣṭhitāt, \\
svabhāva-niyataṁ karma kurvannāpnoti kilbiṣam.
\end{transliteration}

47. Better is one's own duty (though) destitute of merits, than the duty of
another well-performed. He who does the duty ordained by his own nature incurs
no sin.

\begin{gitaverse}
सहजं कर्म कौन्तेय सदोषमपि न त्यजेत् । \\
सर्वारम्भा हि दोषेण धूमेनाग्निरिवावृताः ॥४८॥
\end{gitaverse}

\begin{transliteration}
sahajaṁ karma kaunteya sadoṣam-api na tyajet, \\
sarvārambhā hi doṣeṇa dhūmenāgnir-ivāvṛtāḥ.
\end{transliteration}

48. One should not abandon, O Kaunteya, the duty to which one is born, though
faulty; for, are not all undertakings enveloped by evil, as fire by smoke?

\begin{gitaverse}
असक्तबुद्धिः सर्वत्र जितात्मा विगतस्पृहः । \\
नैष्कर्म्यसिद्धिं परमां सन्न्यासेनाधिगच्छति ॥४९॥
\end{gitaverse}

\begin{transliteration}
asaktabuddhiḥ sarvatra jitātmā vigataspṛhaḥ, \\
naiṣkarmyasiddhiṁ paramāṁ sannyāsenādhigacchati.
\end{transliteration}

49. He whose intellect is unattached everywhere, who has subdued his self, from
whom desire has fled, he, through renunciation, attains the Supreme State of
Freedom from action.

\begin{gitaverse}
सिद्धिं प्राप्तो यथा ब्रह्म तथाप्नोति निबोध मे । \\
समासेनैव कौन्तेय निष्ठा ज्ञानस्य या परा ॥५०॥
\end{gitaverse}

\begin{transliteration}
siddhiṁ prāpto yathā brahma tathāpnoti nibodha me, \\
samāsenaiva kaunteya niṣṭhā jñānasya yā parā.
\end{transliteration}

50. How he, who has attained perfection, reaches BRAHMAN (the Eternal), that in
brief do you learn from Me, O!\@ Kaunteya, that Supreme state-of-knowledge.

\begin{gitaverse}
बुद्ध्या विशुद्धया युक्तो धृत्यात्मानं नियम्य च । \\
शब्दादीन्विषयांस्त्यक्तवा रागद्वेषौ व्युदस्य च ॥५१॥
\end{gitaverse}

\begin{transliteration}
buddhyā viśuddhayā yukto dhṛtyātmānaṁ niyamya ca, \\
śabdādīn-viṣayāṁs-tyaktvā rāgadveṣau vyudasya ca.
\end{transliteration}

51. Endowed with a pure intellect; controlling the self by firmness;
relinquishing sound and other objects; and abandoning attraction and hatred;

\begin{gitaverse}
विविक्तसेवी लघवाशी यतवाक्कायमानसः । \\
ध्यानयोगपरो नित्यं वैराग्यं समुपाश्रितः ॥५२॥
\end{gitaverse}

\begin{transliteration}
viviktasevī laghvāśī yata-vāk-kāya-mānasaḥ, \\
dhyānayoga-paro nityaṁ vairāgyaṁ samupāśritaḥ.
\end{transliteration}

52. Dwelling in solitude; eating but little; speech, body and mind subdued;
always engaged in meditation and concentration; taking refuge in dispassion;

\begin{gitaverse}
अहङ्कारं बलं दर्पं कामं क्रोधं परिग्रहम् । \\
विमुच्य निर्ममः शान्तो ब्रह्मभूयाय कल्पते ॥५३॥
\end{gitaverse}

\begin{transliteration}
ahaṅkāraṁ balaṁ darpaṁ kāmaṁ krodhaṁ parigraham, \\
vimucya nirmamaḥ śānto brahmabhūyāya kalpate.
\end{transliteration}

53. Having abandoned egoism, power, arrogance, desire, anger and
aggrandisement, and freed from the notion of `mine', and therefore
peaceful---he is fit to become BRAHMAN.\@

\begin{gitaverse}
ब्रह्मभूतः प्रसन्नात्मा न शोचति न काङ्क्षति । \\
समः सर्वेषु भूतेषु मद्भक्तिं लभते पराम् ॥५४॥
\end{gitaverse}

\begin{transliteration}
brahmabhūtaḥ prasannātmā na śocati na kāṅkṣati, \\
samaḥ sarveṣu bhūteṣu madbhaktiṁ labhate parām.
\end{transliteration}

54. Becoming BRAHMAN, serene in the Self, he neither grieves nor desires; the
same to all beings, he obtains a supreme devotion towards Me.

\begin{gitaverse}
भक्तया मामभिजानाति यावान्यश्चास्मि तत्त्वतः । \\
ततो मां तत्त्वतो ज्ञात्वा विशते तदनन्तरम् ॥५५॥
\end{gitaverse}

\begin{transliteration}
bhaktyā mām-abhijānāti yāvān-yaś-cāsmi tattvataḥ, \\
tato māṁ tattvato jñātvā viśate tad-anantaram.
\end{transliteration}

55. By devotion he knows Me in Essence, what and who I am; then, having known
Me in My Essence, he forthwith enters into the Supreme.

\begin{gitaverse}
सर्वकर्माण्यपि सदा कुर्वाणो मद्व्यपाश्रयः । \\
मत्प्रसादादवाप्नोति शाश्वतं पदमव्ययम् ॥५६॥
\end{gitaverse}

\begin{transliteration}
sarvakarmāṇyapi sadā kurvāṇo mad-vyapāśrayaḥ, \\
mat-prasādād-avāpnoti śāśvataṁ padam-avyayam.
\end{transliteration}

56. Doing all actions, always taking refuge in Me, by My grace he obtains the
Eternal, Indestructible State or Abode.

\begin{gitaverse}
चेतसा सर्वकर्माणि मयि सन्न्यस्य मत्परः । \\
बुद्धियोगमुपाश्रित्य मच्चित्तः सततं भव ॥५७॥
\end{gitaverse}

\begin{transliteration}
cetasā sarvakarmāṇi mayi sannyasya matparaḥ, \\
buddhiyogam-upāśritya maccittaḥ satatam bhava.
\end{transliteration}

57. Mentally renouncing all actions in Me, having Me as the Highest Goal,
resorting to the YOGA-of-discrimination, ever fix your mind in Me.

\begin{gitaverse}
मच्चित्तः सर्वदुर्गाणि मत्प्रसादात्तरिष्यसि । \\
अथ चेत्त्वमहङ्कारान्न श्रोष्यसि विनङ्क्ष्यसि ॥५८॥
\end{gitaverse}

\begin{transliteration}
maccittaḥ sarvadurgāṇi mat-prasādāt-tariṣyasi, \\
atha cet-tvam-ahaṅkārān-na śroṣyasi vinaṅkṣyasi.
\end{transliteration}

58. Fixing your mind upon Me, you shall, by My grace, overcome all obstacles,
but if, from egoism, you will not hear Me, you shall perish.

\begin{gitaverse}
यदहङ्कारमाश्रित्य न योत्स्य इति मन्यसे । \\
मिथ्यैष व्यवसायस्ते प्रकृतिस्त्वां नियोक्ष्यति ॥५९॥
\end{gitaverse}

\begin{transliteration}
yad-ahaṅkāram-āśritya na yotsya iti manyase, \\
mithyaiṣa vyavasāyas-te prakṛtis-tvāṁ niyokṣyati.
\end{transliteration}

59. Filled with egoism, if you think, ``I will not fight'', vain is this, your
resolve; (for) nature will compel you.

\begin{gitaverse}
स्वभावजेन कौन्तेय निबद्धः स्वेन कर्मणा । \\
कर्तुं नेच्छसि यन्मोहात्करिष्यस्यवशोऽपि तत् ॥६०॥
\end{gitaverse}

\begin{transliteration}
svabhāvajena kaunteya nibaddhaḥ svena karmaṇā, \\
kartuṁ necchasi yan-mohāt-kariṣyasyavaśo'pi tat.
\end{transliteration}

60. O son of Kunti, bound by your own KARMA (action) born of your own nature,
that which, through delusion you wish not to do, even that you shall do,
helplessly.

\begin{gitaverse}
ईश्वरः सर्वभूतानां हृद्देशेऽर्जुन तिष्ठति । \\
भ्रामयन्सर्वभूतानि यन्त्रारूढानि मायया ॥६१॥
\end{gitaverse}

\begin{transliteration}
īśvaraḥ sarvabhūtānāṁ hṛddeśe'rjuna tiṣṭhati, \\
bhrāmayan-sarvabhūtāni yantrāruḍhāni māyayā.
\end{transliteration}

61. The Lord dwells in the hearts of all beings, O Arjuna, causing all beings,
by His power of illusions, to revolve, as if mounted on a machine.

\begin{gitaverse}
तमेव शरणं गच्छ सर्वभावेन भारत । \\
तत्प्रसादात्परां शान्तिं स्थानं प्राप्स्यसि शाश्वतम् ॥६२॥
\end{gitaverse}

\begin{transliteration}
tameva śaraṇaṁ gaccha sarvabhāvena bhārata, \\
tat-prasādāt-parāṁ śāntiṁ sthānaṁ prāpsyasi śāśvatam.
\end{transliteration}

62. Fly unto Him for refuge with all your being, O Bharata; by His grace you
shall obtain Supreme Peace (and) the Eternal Abode.

\begin{gitaverse}
इति ते ज्ञानमाख्यातं गुह्याद्गुह्यतरं मया । \\
विमृश्यैतदशेषेण यथेच्छसि तथा कुरु ॥६३॥
\end{gitaverse}

\begin{transliteration}
iti te jñānam-ākhyātaṁ guhyād-guhyataraṁ mayā, \\
vimṛśyaitad-aśeṣeṇa yathecchasi tathā kuru.
\end{transliteration}

63. Thus, the `Wisdom' which is a greater secret than all secrets, has been
declared to you by Me; having reflected upon it fully, you now act as you
choose.

\begin{gitaverse}
सर्वगुह्यतमं भूयःश्रुणु मे परमं वचः । \\
इष्टोऽसि मे दृढमिति ततो वक्ष्यामि ते हितम् ॥६४॥
\end{gitaverse}

\begin{transliteration}
sarva-guhya-tamaṁ bhūyaḥ śṛṇu me paramaṁ vacaḥ, \\
iṣto'si me dṛḍham-iti tato vakṣyāmi te hitam.
\end{transliteration}

64. Hear again My supreme word, most secret of all; because you are My dear
beloved, therefore, I will tell you what is good (for you).

\begin{gitaverse}
मन्मना भव मद्भक्तो मद्याजी मां नमस्कुरु । \\
मामेवैष्यसि सत्यं ते प्रतिजाने प्रियोऽसि मे ॥६५॥
\end{gitaverse}

\begin{transliteration}
manmanā bhava madbhakto madyājī māṁ namaskuru, \\
mām-evaiṣyasi satyaṁ te pratijāne priyo'si me.
\end{transliteration}

65. Fix your mind upon Me; be devoted to Me; sacrifice for Me; bow down to Me;
you shall come, surely then, to Me alone; truly do I promise to you, (for) you
are dear to Me.

\begin{gitaverse}
सर्वधर्मान्परित्यज्य मामेकं शरणं व्रज । \\
अहं त्वा सर्वपापेभ्यो मोक्षयिष्यामि मा शुचः ॥६६॥
\end{gitaverse}

\begin{transliteration}
sarvadharmān-parityajya mām-ekaṁ śaraṇaṁ vraja, \\
ahaṁ tvā sarvapāpebhyo mokṣayiṣyāmi mā śucaḥ.
\end{transliteration}

66. Abandoning all DHARMAS, (of the body, mind, and intellect), take refuge in
Me alone; I will liberate thee from all sins; grieve not.

\begin{gitaverse}
इदं ते नातपस्काय नाभक्ताय कदाचन । \\
न चाशुश्रूषवे वाच्यं न च मां योऽभ्यसूयति ॥६७॥
\end{gitaverse}

\begin{transliteration}
idaṁ te nātapaskāya nābhaktāya kadācana, \\
na cāśuśrūṣave vācyaṁ na ca māṁ yo'bhyasūyati.
\end{transliteration}

67. This is never to be spoken by you to one who is devoid of austerities or
devotion, nor to one who does not render service, nor to one who desires not to
listen, nor to one who cavils at Me.

\begin{gitaverse}
य इमं परमं गुह्यं मद्भक्तेष्वभिधास्यति । \\
भक्तिं मयि परां कृत्वा मामेवैष्यत्यसंशयः ॥६८॥
\end{gitaverse}

\begin{transliteration}
ya imaṁ paramaṁ guhyaṁ madbhakteṣvabhidhāsyati, \\
bhaktiṁ mayi parāṁ kṛtvā mām-evaiṣyatyasaṁśayaḥ.
\end{transliteration}

68. He who, with supreme devotion to Me, will teach this supreme secret to My
devotees, shall doubtless come to Me.

\begin{gitaverse}
न च तस्मान्मनुष्येषु कश्चिन्मे प्रियकृत्तमः । \\
भविता न च मे तस्मादन्यः प्रियतरो भुवि ॥६९॥
\end{gitaverse}

\begin{transliteration}
na ca tasmān-manuṣyeṣu kaścin-me priyakṛttamaḥ, \\
bhavitā na ca me tasmād-anyaḥ priyataro bhuvi.
\end{transliteration}

69. Nor is there any among men who does dearer service to Me, nor shall there
be another on earth dearer to Me than he.

\begin{gitaverse}
अध्येष्यते च य इमं धर्म्यं संवादमावयोः । \\
ज्ञानयज्ञेन तेनाहमिष्टः स्यामिति मे मतिः ॥७०॥
\end{gitaverse}

\begin{transliteration}
adhyeṣyate ca ya imaṁ dharmyaṁ saṁvādam-āvayoḥ, \\
jñānayajñena tenāham-iṣṭaḥ syām-iti me matiḥ.
\end{transliteration}

70. And he who will study this sacred dialogue of ours, by him I shall have
been worshipped by the `sacrifice-of-wisdom', such is My conviction.

\begin{gitaverse}
श्रद्धावाननसूयश्च शृणुयादपि यो नरः । \\
सोऽपि मुक्तः शुभाँल्लोकान्प्राप्नुयात्पुण्यकर्मणाम् ॥७१॥
\end{gitaverse}

\begin{transliteration}
śraddhāvān-anasūyaśca śṛṇuyād-api yo naraḥ, \\
so'pi muktaḥ śubhāmllokān-prāpnuyāt-puṇyakarmaṇām.
\end{transliteration}

71. That man also, who hears this, full of faith and free from malice, he too,
liberated, shall attain to the happy worlds of those righteous deeds.

\begin{gitaverse}
कच्चिदेतच्छ्रुतं पार्थ त्वयैकाग्रेण चेतसा । \\
कच्चिदज्ञानसम्मोहः प्रनष्टस्ते धनञ्जय ॥७२॥
\end{gitaverse}

\begin{transliteration}
kaccid-etacchrutaṁ pārtha tvayaikāgreṇa cetasā, \\
kaccid-ajñānasammohaḥ pranaṣṭas-te dhanañjaya.
\end{transliteration}

72. Has this been heard, O son of Pritha, with single-pointed mind? Has the
distraction, caused by your `ignorance', been dispelled, O Dhananjaya?

\begin{gitaverse}
अर्जुन उवाच \\
नष्टो मोहः स्मृतिर्लब्धा त्वत्प्रसादान्मयाच्युत । \\
स्थितोऽस्मि गतसन्देहः करिष्ये वचनं तव ॥७३॥
\end{gitaverse}

\begin{transliteration}
arjuna uvāca \\
naṣṭo mohaḥ smṛtir-labdhā tvat-prasādān-mayācyuta, \\
sthito'smi gatasandehaḥ kariṣye vacanaṁ tava.
\end{transliteration}

Arjuna said: \\
73. Destroyed is my delusion, as I have now gained my memory (knowledge)
through your grace, O Achyuta. I am firm; my doubts are gone. I will do
according to your word (bidding).

\begin{gitaverse}
सञ्जय उवाच \\
इत्यहं वासुदेवस्य पार्थस्य च महात्मनः । \\
संवादमिममश्रौषमद्भुतं रोमहर्षणम् ॥७४॥
\end{gitaverse}

\begin{transliteration}
sanjaya uvāca \\
ityahaṁ vāsudevasya pārthasya ca mahātmanaḥ, \\
saṁvādam-imam-aśrauṣam-adbhutaṁ romaharṣaṇam.
\end{transliteration}

Sanjaya said: \\
74. Thus have I heard this wonderful dialogue between Vasudeva and the
high-souled Partha, which causes the hair to stand on end.

\begin{gitaverse}
व्यासप्रसादाच्छ्रुतवानेतद्गुह्यमहं परम् । \\
योगं योगेश्वरात्कृष्णात्साक्षात्कथयतः स्वयम् ॥७५॥
\end{gitaverse}

\begin{transliteration}
vyāsaprasādācchrutavān-etad-guhyam-ahaṁ param, \\
yogaṁ yogeśvarāt-kṛṣṇāt-sākṣāt-kathayataḥ svayam.
\end{transliteration}

75. Through the grace of Vyasa I have heard, this supreme and most secret YOGA,
directly from Krishna, the Lord of YOGA, Himself declaring it.

\begin{gitaverse}
राजन्संस्मृत्य संस्मृत्य संवादमिममद्भुतम् । \\
केशवार्जुनयोः पुण्यं हृष्यामि च मुहुर्मुहुः ॥७६॥
\end{gitaverse}

\begin{transliteration}
rājan-saṁsmṛtya saṁsmṛtya saṁvādam-imam-adbhutaṁ, \\
keśavārjunayoḥ puṇyaṁ hṛsyāmi ca muḥur-muhuḥ.
\end{transliteration}

76. O King, remembering this wonderful and holy dialogue between Keshava and
Arjuna, I rejoice again and again.

\begin{gitaverse}
तच्च संस्मृत्य संस्मृत्य रूपमत्यद्भुतं हरेः । \\
विस्मयो मे महान्राजन्हृष्यामि च पुनः पुनः ॥७७॥
\end{gitaverse}

\begin{transliteration}
tacca saṁsmṛtya saṁsmṛtya rūpam-atyadbhutaṁ hareḥ, \\
vismayo me mahān-rājan-hṛṣyāmi ca punaḥ punaḥ.
\end{transliteration}

77. Remembering and again remembering, that most wonderful Form of Hari, great
is my wonder, O king; and I rejoice again and again.

\begin{gitaverse}
यत्र योगेश्वरः कृष्णो यत्र पार्थो धनुर्धरः । \\
तत्र श्रीर्विजयो भूतिर्ध्रुवा नीतिर्मतिर्मम ॥७८॥
\end{gitaverse}

\begin{transliteration}
yatra yogeśvaraḥ kṛṣṇo yatra pārtho dhanurdharaḥ, \\
tatra śrīr-vijayo bhūtir-dhruvā nītir-matir-mama.
\end{transliteration}

78. Wherever is Krishna, the Lord of Yoga, wherever is Partha, the archer,
there are prosperity, victory, happiness and firm (steady or sound) policy;
this is my conviction.

\begin{gitaverse}
ॐ तत्सदिति श्रीमद् भगवद् गीतासूपनिषत्सु ब्रह्मविद्यायां \\
योगशास्त्रे श्रीकृष्णार्जुनसंवादे मोक्षसन्न्यासयोगो नाम \\
अष्टादशोऽध्यायः
\end{gitaverse}

\begin{transliteration}
oṁ tatsaditi śrīmad bhagavad gītāsūpaniṣatsu brahmavidyāyāṁ \\
yogaśāstre śrīkṛṣṇārjunasaṁvāde mokṣasannyāsayogo nāma \\
aṣṭādaśo'dhyāyaḥ
\end{transliteration}

Thus, in the UPANISHADS of the glorious Bhagawad Geeta, in the Science of the
Eternal, in the scripture of YOGA, in the dialogue between Sri Krishna and
Arjuna, the eighteenth discourse ends entitled: The Yoga of Liberation Through
Renunciation
