\chapterdrop

\begin{center}
\sanskrit\fontsize{28pt}{28pt}\selectfont
ॐ

\vspace{0.25em}
\sanskrit\LARGE
श्रीमद्भगवद्गीता

\vspace{0.5em}
\normalfont\normalsize
SRIMAD BHAGAVAD GEETA

\headerspace
\headersanskrit{अथ प्रथमोऽध्यायः}

\headerspace
\headertransliteration{Atha Prathamo'dhyāyaḥ}

\section{Chapter 1}

\headerspace
\headersanskrit{अर्जुनविषादयोगः}

\headerspace
\headertransliteration{Arjuna Viṣāda Yogah}

\headerspace
\headertranslation{The Yoga of the Arjuna-Grief}

\headerspace
\end{center}

\begin{gitaverse}
धृतराष्ट्र उवाच \\
धर्मक्षेत्रे कुरुक्षेत्रे समवेता युयुत्सवः । \\
मामकाः पाण्डवाश्चैव किमकुर्वत सञ्जय ॥१॥
\end{gitaverse}

\begin{transliteration}
dhṛtarāṣṭra uvāca \\
dharmakṣetre kurukṣetre samavetā yuyutsavaḥ, \\
māmakāḥ pāṇḍavāścaiva kimakurvata sañjaya.
\end{transliteration}

Dhritarashtra said: \\
1. What did the sons of Pandu and also my people do, when, desirous to fight,
they assembled together on the holy plain of Kurukshetra, O Sanjaya?

\begin{gitaverse}
सञ्जय उवाच \\
दृष्ट्वा तु पाण्डवानीकं व्यूढं दुर्योधनस्तदा । \\
आचार्यमुपसङ्गम्य राजा वचनमब्रवीत् ॥२॥
\end{gitaverse}

\begin{transliteration}
sañjaya uvāca \\
dṛṣṭvā tu pāṇḍavānīkaṁ vyūḍhaṁ duryodhanastadā, \\
ācāryamupasaṅgamya rājā vacanamabravīt.
\end{transliteration}

Sanjaya said: \\
2. Having seen the army of the Pandavas drawn up in battle array, King
Duryodhana then approached his teacher (Drona) and spoke these words.

\begin{gitaverse}
पश्यैतां पाण्डुपुत्राणामाचार्य महतीं चमूम् । \\
व्यूढां द्रुपदपुत्रेण तव शिष्येण धीमता ॥३॥
\end{gitaverse}

\begin{transliteration}
paśyaitāṁ pāṇḍuputrāṇāmācārya mahatīṁ camūm, \\
vyūḍhāṁ drupadaputreṇa tava śiṣyeṇa dhīmatā.
\end{transliteration}

3. Behold, O Teacher! This mighty army of the sons of Pandu arrayed by the son
of Drupada, thy wise disciple.

\begin{gitaverse}
अत्र शूरा महेष्वासा भीमार्जुनसमा युधि । \\
युयुधानो विराटश्च द्रुपदश्च महारथः ॥४॥
\end{gitaverse}

\begin{transliteration}
atra śūrā maheṣvāsā bhīmārjunasamā yudhi, \\
yuyudhāno virāṭaśca drupadaśca mahārathaḥ.
\end{transliteration}

4. Here are heroes, mighty archers, equal in battle to Bhima and Arjuna,
Yuyudhana, Virata and Drupada, each commanding eleven-thousand archers.

\begin{gitaverse}
धृष्टकेतुश्चेकितानः काशिराजश्च वीर्यवान् । \\
पुरुजित्कुन्तिभोजश्च शैब्यश्च नरपुङ्गवः ॥५॥
\end{gitaverse}

\begin{transliteration}
dhṛṣṭaketuścekitānaḥ kāśirājaśca vīryavān, \\
purujitkuntibhojaśca śaibyaśca narapuṅgavaḥ.
\end{transliteration}

5. Dhrishtaketu, Chekitana, and the valiant king of Kashi, Purujit and
Kuntibhoja and Saibya, the best of men.

\begin{gitaverse}
युधामन्युश्च विक्रान्त उत्तमौजाश्च वीर्यवान् । \\
सौभद्रो द्रौपदेयाश्च सर्व एव महारथाः ॥६॥
\end{gitaverse}

\begin{transliteration}
yudhāmanyuśca vikrānta uttamaujāśca vīryavān, \\
saubhadro draupadeyāśca sarva eva mahārathāḥ.
\end{transliteration}

6. The strong Yudhamanyu and the brave Uttamaujas, the son of Subhadra and the
sons of Draupadi, all of them, divisional commanders.

\begin{gitaverse}
अस्माकं तु विशिष्टा ये तान्निबोध द्विजोत्तम । \\
नायका मम सैन्यस्य सञ्ज्ञार्थं तान्ब्रवीमि ते ॥७॥
\end{gitaverse}

\begin{transliteration}
asmākaṁ tu viśiṣṭā ye tānnibodha dvijottama, \\
nāyakā mama sainyasya sañjñārthaṁ tānbravīmi te.
\end{transliteration}

7. Know also, O best among the twice-born, the names of those who are the most
distinguished amongst ourselves, the leaders of my army; these I name to thee
for thy information.

\begin{gitaverse}
भवान्भीष्मश्च कर्णश्च कृपश्च समितिञ्जयः । \\
अश्वत्थामा विकर्णश्च सौमदत्तिस्तथैव च ॥८॥
\end{gitaverse}

\begin{transliteration}
bhavānbhīṣmaśca karṇaśca kṛpaśca samitiñjayaḥ, \\
aśvatthāmā vikarṇaśca saumadattistathaiva ca.
\end{transliteration}

8. Yourself and Bhishma, and Karna and also Kripa, the victorious in war;
Aswatthama, Vikarna, and so also the son of Somadatta.

\begin{gitaverse}
अन्ये च बहवः शूरा मदर्थे त्यक्तजीविताः । \\
नानाशस्त्रप्रहरणाः सर्वे युद्धविशारदाः ॥९॥
\end{gitaverse}

\begin{transliteration}
anye ca bahavaḥ śūrā madarthe tyaktajīvitāḥ, \\
nānāśastrapraharaṇāḥ sarve yuddhaviśāradāḥ.
\end{transliteration}

9. And many other heroes also who are determined to give up their lives for my
sake, armed with various weapons and missiles, all well-skilled in battle.

\begin{gitaverse}
अपर्याप्तं तदस्माकं बलं भीष्माभिरक्षितम् । \\
पर्याप्तं त्विदमेतेषां बलं भीमाभिरक्षितम् ॥१०॥
\end{gitaverse}

\begin{transliteration}
aparyāptaṁ tadasmākaṁ balaṁ bhīṣmābhirakṣitam, \\
paryāptaṁ tvidameteṣāṁ balaṁ bhīmābhirakṣitam.
\end{transliteration}

10. This army of ours defended by Bhishma is insufficient, whereas that army of
theirs defended by Bhima is sufficient. Or, this army of ours protected by
Bhishma is unlimited, whereas that army of theirs protected by Bhima is
limited.

\begin{gitaverse}
अयनेषु च सर्वेषु यथाभागमवस्थिताः । \\
भीष्ममेवाभिरक्षन्तु भवन्तः सर्व एव हि ॥११॥
\end{gitaverse}

\begin{transliteration}
ayaneṣu ca sarveṣu yathābhāgamavasthitāḥ, \\
bhīṣmamevābhirakṣantu bhavantaḥ sarva eva hi.
\end{transliteration}

11. Therefore do you all, stationed in your respective positions in the several
divisions of the army, protect Bhishma alone.

\begin{gitaverse}
तस्य सञ्जनयन्हर्षं कुरुवृद्धः पितामहः । \\
सिंहनादं विनद्योच्चैः शङ्खं दध्मौ प्रतापवान् ॥१२॥
\end{gitaverse}

\begin{transliteration}
tasya sañjanayanharṣaṁ kuruvṛddhaḥ pitāmahaḥ, \\
siṁhanādaṁ vinadyoccaiḥ śaṅkhaṁ dadhmau pratāpavān.
\end{transliteration}

12. His glorious grandsire (Bhishma), the oldest of the Kauravas, in order to
cheer Duryodhana, now sounded aloud a lion's roar and blew his conch.

\begin{gitaverse}
ततः शङ्खाश्च भेर्यश्च पणवानकगोमुखाः । \\
सहसैवाभ्यहन्यन्त स शब्दस्तुमुलोऽभवत् ॥१३॥
\end{gitaverse}

\begin{transliteration}
tataḥ śaṅkhāśca bheryaśca paṇavānakagomukhāḥ, \\
sahasaivābhyahanyanta sa śabdastumulo'bhavat.
\end{transliteration}

13. Then (following Bhishma), conches and kettledrums, tabors, drums and
cow-horns blared forth quite suddenly and the sound was tremendous.

\begin{gitaverse}
ततः श्वेतैर्हयैर्युक्ते महति स्यन्दने स्थितैा । \\
माधवः पाण्डवश्चैव दिव्यौ शङ्खौ प्रदध्मतुः ॥१४॥
\end{gitaverse}

\begin{transliteration}
tataḥ śvetairhayairyukte mahati syandane sthitau, \\
mādhavaḥ pāṇḍavaścaiva divyau śaṅkhau pradadhmatuḥ.
\end{transliteration}

14. Then, also Madhava and the son of Pandu, seated in their magnificent
chariot yoked with white horses, blew their divine conches.

\begin{gitaverse}
पाञ्चजन्यं हृषीकेशो देवदत्तं धनञ्जयः । \\
पौण्ड्रं दध्मौ महाशङ्खं भीमकर्मा वृकोदरः ॥१५॥
\end{gitaverse}

\begin{transliteration}
pāñcajanyaṁ hṛṣīkeśo devadattaṁ dhanañjayaḥ, \\
pauṇḍraṁ dadhmau mahāśaṅkhaṁ bhīmakarmā vṛkodaraḥ.
\end{transliteration}

15. Hrishikesha blew the Panchajanya and Dhananjaya (Arjuna) blew the Devadatta
and Vrikodara (Bhima), the doer of terrible deeds, blew the great conch, named
Paundra.

\begin{gitaverse}
अनन्तविजयं राजा कुन्तीपुत्रो युधिष्ठिरः । \\
नकुलः सहदेवश्च सुघोषमणिपुष्पकौ ॥१६॥
\end{gitaverse}

\begin{transliteration}
anantavijayaṁ rājā kuntīputro yudhiṣṭhiraḥ, \\
nakulaḥ sahadevaśca sughoṣamaṇipuṣpakau.
\end{transliteration}

16. King Yudhisthira, the son of Kunti, blew the Anantavijaya; Nakula and
Sahadeva blew the Sughosha and the Manipushpaka.

\begin{gitaverse}
काश्यश्च परमेष्वासः शिखण्डी च महारथः । \\
धृष्टद्युम्नो विराटश्च सात्यकिश्चापराजितः ॥१७॥
\end{gitaverse}

\begin{transliteration}
kāśyaśca parameṣvāsaḥ śikhaṇḍī ca mahārathaḥ, \\
dhṛṣṭadyumno virāṭaśca sātyakiścāparājitaḥ.
\end{transliteration}

17. The king of Kashi, an excellent archer, Shikhandi, the mighty commander of
eleven thousand archers, Dhristadyumna and Virata and Satyaki, the unconquered;

\begin{gitaverse}
द्रुपदो द्रौपदेयाश्च सर्वशः पृथिवीपते । \\
सौभद्रश्च महाबाहुः शङ्खान्दध्मुः पृथक्पृथक् ॥१८॥
\end{gitaverse}

\begin{transliteration}
drupado draupadeyāśca sarvaśaḥ pṛthivīpate, \\
saubhadraśca mahābāhuḥ śaṅkhāndadhmuḥ pṛthak pṛthak.
\end{transliteration}

18. Drupada and the sons of Draupadi, O Lord of the Earth, and the son of
Subhadra, the mighty armed, blew their respective conches.

\begin{gitaverse}
स घोषो धार्तराष्ट्राणां हृदयानि व्यदारयत् । \\
नभश्च पृथिवीं चैव तुमुलो व्यनुनादयन् ॥१९॥
\end{gitaverse}

\begin{transliteration}
sa ghoṣo dhārtarāṣṭrāṇāṁ hṛdayāni vyadārayat, \\
nabhaśca pṛthivīṁ caiva tumulo vyanunādayan.
\end{transliteration}

19. That tumultuous sound rent the hearts of (the people of) Dhritarashtra's
party and made both heaven and earth reverberate.

\begin{gitaverse}
अथ व्यवस्थितान्दृष्ट्वा धार्तराष्ट्रान् कपिध्वजः । \\
प्रवृत्ते शस्त्रसम्पाते धनुरुद्यम्य पाण्डवः ॥२०॥ \\
हृषीकेशं तदा वाक्यमिदमाह महीपते ।
\end{gitaverse}

\begin{transliteration}
atha vyavasthitāndṛṣṭvā dhārtarāṣṭrān kapidhvajaḥ, \\
pravṛtte śastrasampāte dhanurudyamya pāṇḍavaḥ. \\
hṛṣīkeśaṃ tadā vākyamidamāha mahīpate
\end{transliteration}

20--21. Then, seeing the people of Dhritarashtra's party standing arrayed and the
discharge of weapons about to begin, Arjuna, the son of Pandu, whose ensign was
a monkey, took up his bow and said these words to Krishna (Hrishikesha), O Lord
of the Earth!

\begin{gitaverse}
अर्जुन उवाच \\
सेनयोरुभयोर्मध्ये रथं स्थापय मेऽच्युत ॥२१॥
\end{gitaverse}

\begin{transliteration}
arjuna uvāca \\
senayorubhayormadhye rathaṁ sthāpaya me'cyuta.
\end{transliteration}

Arjuna said: \\
21. In the midst of the two armies, place my chariot, O Achyuta.

\begin{gitaverse}
यावदेतान्निरीक्षेऽहं योद्धुकामानवस्थितान् । \\
कैर्मया सह योद्धव्यमस्मिन्रणसमुद्यमे ॥२२॥
\end{gitaverse}

\begin{transliteration}
yāvadetānnirīkṣe'haṁ yoddhukāmānavasthitān, \\
kairmayā saha yoddhavyamasminraṇasamudyame.
\end{transliteration}

22. That I may behold those who stand here desirous of fighting and, on the eve
of this battle, let me know with whom I must fight.

\begin{gitaverse}
योत्स्यमानानवेक्षेऽहं य एतेऽत्र समागताः । \\
धार्तराष्ट्रस्य दुर्बुद्धेर्युद्धे प्रियचिकीर्षवः ॥२३॥
\end{gitaverse}

\begin{transliteration}
yotsyamānānavekṣe'haṁ ya ete'tra samāgatāḥ, \\
dhārtarāṣṭrasya durbuddheryuddhe priyacikīrṣavaḥ.
\end{transliteration}

23. For I desire to observe those who are assembled here for the fight, wishing
to please in battle, the evil-minded sons of Dhritarashtra.

\begin{gitaverse}
सञ्जय उवाच \\
एवमुक्तो हृषीकेशो गुडाकेशेन भारत । \\
सेनयोरुभयोर्मध्ये स्थापयित्वा रथोत्तमम् ॥२४॥
\end{gitaverse}

\begin{transliteration}
sañjaya uvāca \\
evamukto hṛṣīkeśo guḍākeśena bhārata, \\
senayorubhayormadhye sthāpayitvā rathottamam.
\end{transliteration}

Sanjaya said: \\
24. Thus addressed by Gudakesha, O Bharata, Hrishikesha, having stationed the
best of chariots between the two armies;

\begin{gitaverse}
भीष्मद्रोणप्रमुखतः सर्वेषां च महीक्षिताम् । \\
उवाच पार्थ पश्यैतान्समवेतान्कुरूनिति ॥२५॥
\end{gitaverse}

\begin{transliteration}
bhīṣmadroṇapramukhataḥ sarveṣāṁ ca mahīkṣitām, \\
uvāca pārtha paśyaitānsamavetānkurūniti.
\end{transliteration}

25. In front of Bhishma and Drona, and all the rulers of the earth, he said,
``O Partha, behold these Kurus gathered together''.

\begin{gitaverse}
तत्रापश्यत्स्थितान् पार्थः पितॄनथ पितामहान् । \\
आचार्यान्मातुलान्भ्रातॄन्पुत्रान्पौत्रान्सखींस्तथा ॥२६॥
\end{gitaverse}

\begin{transliteration}
tatrāpaśyatsthitān pārthaḥ pitṝnatha pitāmahān, \\
ācāryānmātulānbhrātṝnputrānpautrānsakhīṁstathā.
\end{transliteration}

26. Then Partha saw stationed there in both the armies, fathers, grandfathers,
teachers, maternal uncles, brothers, sons, grandsons and friends too.

\begin{gitaverse}
श्वशुरान्सुहृदश्चैव सेनयोरुभयोरपि । \\
तान्समीक्ष्य स कौन्तेयः सर्वान्बन्धूनवस्थितान् ॥२७॥
\end{gitaverse}

\begin{transliteration}
śvaśurānsuhṛdaścaiva senayorubhayorapi, \\
tānsamīkṣya sa kaunteyaḥ sarvānbandhūnavasthitān.
\end{transliteration}

27. (He saw) Fathers-in-law and friends also in both the armies. Then the son
of Kunti, seeing all these kinsmen thus standing arrayed, spoke thus
sorrowfully, filled with deep pity.

\begin{gitaverse}
अर्जुन उवाच \\
दृष्ट्वेमं स्वजनं कृष्ण युयुत्सुं समुपस्थितम् ॥२८॥
\end{gitaverse}

\begin{transliteration}
arjuna uvāca \\
dṛṣṭvemaṁ svajanaṁ kṛṣṇa yuyutsuṁ samupasthitam.
\end{transliteration}

Arjuna said: \\
28. Seeing these my kinsmen, O Krishna, arrayed, eager to fight,

\begin{gitaverse}
सीदन्ति मम गात्राणि मुखं च परिशुष्यति । \\
वेपथुश्च शरीरे मे रोमहर्षश्च जायते ॥२९॥
\end{gitaverse}

\begin{transliteration}
sīdanti mama gātrāṇi mukhaṁ ca pariśuṣyati, \\
vepathuśca śarīre me romaharṣaśca jāyate.
\end{transliteration}

29. My limbs fail and my mouth is parched, my body quivers and my hair stand on
end.

\begin{gitaverse}
गाण्डीवं स्रंसते हस्तात्त्वक्चैव परिदह्यते । \\
न च शक्नोम्यवस्थातुं भ्रमतीव च मे मनः ॥३०॥
\end{gitaverse}

\begin{transliteration}
gāṇḍīvaṁ sraṁsate hastāttvakcaiva paridahyate, \\
na ca śaknomyavasthātuṁ bhramatīva ca me manaḥ.
\end{transliteration}

30. The Gandiva-bow slips from my hand and my skin burns all over; I am also
unable to stand and my mind is whirling round, as it were.

\begin{gitaverse}
निमित्तानि च पश्यामि विपरीतानि केशव । \\
न च श्रेयोऽनुपश्यामि हत्वा स्वजनमाहवे ॥३१॥
\end{gitaverse}

\begin{transliteration}
nimittāni ca paśyāmi viparītāni keśava, \\
na ca śreyo'nupaśyāmi hatvā svajanamāhave.
\end{transliteration}

31. And I see adverse omens, O Keshava. Nor do I see any good in killing my
kinsmen in battle.

\begin{gitaverse}
न काङ्क्षे विजयं कृष्ण न च राज्यं सुखानि च । \\
किं नो राज्येन गोविन्द किं भोगैर्जीवितेन वा ॥३२॥
\end{gitaverse}

\begin{transliteration}
na kāṅkṣe vijayaṁ kṛṣṇa na ca rājyaṁ sukhāni ca, \\
kiṁ no rājyena govinda kiṁ bhogairjīvitena vā.
\end{transliteration}

32. For, I desire not victory, O Krishna, nor kingdom, nor pleasures. Of what
avail, is dominion to us, O Govinda? Of what avail are pleasures or even life
itself?

\begin{gitaverse}
येषामर्थे काङ्क्षितं नो राज्यं भोगाः सुखानि च । \\
त इमेऽवस्थिता युद्धे प्राणांस्त्यक्त्वा धनानि च ॥३३॥
\end{gitaverse}

\begin{transliteration}
yeṣāmarthe kāṅkṣitaṁ no rājyaṁ bhogāḥ sukhāni ca, \\
ta ime'vasthitā yuddhe prāṇāṁstyaktvā dhanāni ca.
\end{transliteration}

33. They for whose sake we desire kingdom, enjoyment and pleasures stand here
in battle, having renounced life and wealth.

\begin{gitaverse}
आचार्याः पितरः पुत्रास्तथैव च पितामहाः । \\
मातुलाः श्वशुराः पौत्राः श्यालाः सम्बन्धिनस्तथा ॥३४॥
\end{gitaverse}

\begin{transliteration}
ācāryāḥ pitaraḥ putrāstathaiva ca pitāmahāḥ, \\
mātulāḥ śvaśurāḥ pautrāḥ śyālāḥ sambandhinastathā.
\end{transliteration}

34. Teachers, fathers, sons and also grandfathers, maternal uncles,
fathers-in-law, grandsons, brothers-in-law and other relatives.

\begin{gitaverse}
एतान्न हन्तुमिच्छामि घ्नतोऽपि मधुसूदन । \\
अपि त्रैलोक्यराज्यस्य हेतोः किं नु महीकृते ॥३५॥
\end{gitaverse}

\begin{transliteration}
etānna hantumicchāmi ghnato'pi madhusūdana, \\
api trailokyarājyasya hetoḥ kiṁ nu mahīkṛte.
\end{transliteration}

35. These I do not wish to kill, though they may kill me, O Madhusudana, even
for the sake of dominion over the three worlds; how much less for the sake of
the earth.

\begin{gitaverse}
निहत्य धार्तराष्ट्रान्नः का प्रीतिः स्याज्जनार्दन । \\
पापमेवाश्रयेदस्मान्हत्वैतानाततायिनः ॥३६॥
\end{gitaverse}

\begin{transliteration}
nihatya dhārtarāṣṭrānnaḥ kā prītiḥ syājjanārdana, \\
pāpamevāśrayedasmānhatvaitānātatāyinaḥ.
\end{transliteration}

36. Killing these sons of Dhritarashtra, what pleasure can be ours, O
Janardana? Sin alone will be our gain by killing these felons.

\begin{gitaverse}
तस्मान्नार्हा वयं हन्तुं धार्तराष्ट्रान्स्वबान्धवान् । \\
स्वजनं हि कथं हत्वा सुखिनः स्याम माधव ॥३७॥
\end{gitaverse}

\begin{transliteration}
tasmānnārhā vayaṁ hantuṁ dhārtarāṣṭrānsvabāndhavān, \\
svajanaṁ hi kathaṁ hatvā sukhinaḥ syāma mādhava.
\end{transliteration}

37. Therefore, we shall not kill the sons of Dhritarashtra, our relatives; for
how can we be happy by killing our own people, O Madhava?

\begin{gitaverse}
यद्यप्येते न पश्यन्ति लोभोपहतचेतसः । \\
कुलक्षयकृतं दोषं मित्रद्रोहे च पातकम् ॥३८॥
\end{gitaverse}

\begin{transliteration}
yadyapyete na paśyanti lobhopahatacetasaḥ, \\
kulakṣayakṛtaṁ doṣaṁ mitradrohe ca pātakam.
\end{transliteration}

38. Though these, with their intelligence clouded by greed, see no evil in the
destruction of the families in the society, and no sin in their cruelty to
friends\ldots

\begin{gitaverse}
कथं न ज्ञेयमस्माभिः पापादस्मान्निवर्तितुम् । \\
कुलक्षयकृतं दोषं प्रपश्यद्भिर्जनार्दन ॥३९॥
\end{gitaverse}

\begin{transliteration}
kathaṁ na jñeyamasmābhiḥ pāpādasmānnivartitum, \\
kulakṣayakṛtaṁ doṣaṁ prapaśyadbhirjanārdana.
\end{transliteration}

39. Why should not we, who clearly see evil in the destruction of the
family-units, learn to turn away from this sin, O Janardana?

\begin{gitaverse}
कुलक्षये प्रणश्यन्ति कुलधर्माः सनातनाः । \\
धर्मे नष्टे कुलं कृत्स्नमधर्मोऽभिभवत्युत ॥४०॥
\end{gitaverse}

\begin{transliteration}
kulakṣaye praṇaśyanti kuladharmāḥ sanātanāḥ, \\
dharme naṣṭe kulaṁ kṛtsnamadharmo'bhibhavatyuta.
\end{transliteration}

40. In the destruction of a family, the immemorial religious rites of that
family perish; on the destruction of spirituality, impiety overcomes the whole
family.

\begin{gitaverse}
अधर्माभिभवात्कृष्ण प्रदुष्यन्ति कुलस्त्रियः । \\
स्त्रीषु दुष्टासु वार्ष्णेय जायते वर्णसङ्करः ॥४१॥
\end{gitaverse}

\begin{transliteration}
adharmābhibhavātkṛṣṇa praduṣyanti kulastriyaḥ, \\
strīṣu duṣṭāsu vārṣṇeya jāyate varṇasaṅkaraḥ.
\end{transliteration}

41. By the prevalence of impiety, O Krishna, the women of the family become
corrupt; and women being corrupted, O descendent of the Vrishni-clan, there
arises `intermingling of castes' (VARNA-SAMKARA).

\begin{gitaverse}
सङ्करो नरकायैव कुलघ्नानां कुलस्य च । \\
पतन्ति पितरो ह्येषां लुप्तपिण्डोदकक्रियाः ॥४२॥
\end{gitaverse}

\begin{transliteration}
saṅkaro narakāyaiva kulaghnānāṁ kulasya ca, \\
patanti pitaro hyeṣāṁ luptapiṇḍodakakriyāḥ.
\end{transliteration}

42. `Confusion of castes' leads the slayer of the family to hell; for their
forefathers fall, deprived of the offerings of PINDA (riceball) and water
(libations).

\begin{gitaverse}
दोषैरेतैः कुलघ्नानां वर्णसङ्करकारकैः । \\
उत्साद्यन्ते जातिधर्माः कुलधर्माश्च शाश्वताः ॥४३॥
\end{gitaverse}

\begin{transliteration}
doṣairetaiḥ kulaghnānāṁ varṇasaṅkarakārakaiḥ, \\
utsādyante jātidharmāḥ kuladharmāśca śāśvatāḥ.
\end{transliteration}

43. By these evil deeds of the `destroyers of the family', which cause
confusion of castes, the eternal religious rites of the caste and the family
are destroyed.

\begin{gitaverse}
उत्सन्नकुलधर्माणां मनुष्याणां जनार्दन । \\
नरकेऽनियतं वासो भवतीत्यनुशुश्रुम ॥४४॥
\end{gitaverse}

\begin{transliteration}
utsannakuladharmāṇāṁ manuṣyāṇāṁ janārdana, \\
narake'niyataṁ vāso bhavatītyanuśuśruma.
\end{transliteration}

44. We have heard, O Janardana, that it is inevitable for those men, in whose
families the religious practices have been destroyed, to dwell in hell for an
unknown period of time.

\begin{gitaverse}
अहो बत महत्पापं कर्तुं व्यवसिता वयम् । \\
यद्राज्यसुखलोभेन हन्तुं स्वजनमुद्यताः ॥४५॥
\end{gitaverse}

\begin{transliteration}
aho bata mahatpāpaṁ kartuṁ vyavasitā vayam, \\
yadrājyasukhalobhena hantuṁ svajanamudyatāḥ.
\end{transliteration}

45. Alas! We are involved in a great sin, in that we are prepared to kill our
kinsmen, from greed for the pleasures of the kingdom.

\begin{gitaverse}
यदि मामप्रतीकारमशस्त्रं शस्त्रपाणयः । \\
धार्तराष्ट्रा रणे हन्युस्तन्मे क्षेमतरं भवेत् ॥४६॥
\end{gitaverse}

\begin{transliteration}
yadi māmapratīkāramaśastraṁ śastrapāṇayaḥ, \\
dhārtarāṣṭrā raṇe hanyustanme kṣemataraṁ bhavet.
\end{transliteration}

46. If the sons of Dhritarashrta weapons-in-hand, slay me in battle,
unresisting and unarmed, that would be better for me.

\begin{gitaverse}
सञ्जय उवाच \\
एवमुक्त्वार्जुनः सड्ख़्ये रथोपस्थ उपाविशत् । \\
विसृज्य सशरं चापं शोकसंविग्नमानसः ॥४७॥
\end{gitaverse}

\begin{transliteration}
sañjaya uvāca \\
evamuktvārjunaḥ saṅkhye rathopastha upāviśat, \\
visṛjya saśaraṁ cāpaṁ śokasaṁvignamānasaḥ.
\end{transliteration}

Sanjaya said: \\
47. Having thus spoken in the midst of the battlefield, Arjuna sat down on the
seat of the chariot, casting away his bow and arrow, with a mind distressed
with sorrow.

\begin{gitaverse}
ॐ तत्सदिति श्रीमद् भगवद् गीतासूपनिषत्सु ब्रह्मविद्यायां \\
योगशास्त्रे श्रीकृष्णार्जुनसंवादे अर्जुनविषादयोगो नाम \\
प्रथमोऽध्यायः
\end{gitaverse}

\begin{transliteration}
oṁ tatsaditi śrīmad bhagavad gītāsūpaniṣatsu brahmavidyāyāṁ \\
yogaśāstre śrīkṛṣṇārjunasaṁvāde arjunaviṣādayogo nāma \\
prathamo'dhyāyaḥ.
\end{transliteration}

Thus, in the UPANISHADS of the glorious Bhagawad Geeta, in the Science of the
Eternal, in the scripture of YOGA, in the dialogue between Sri Krishna and
Arjuna, the first discourse ends entitled: The Yoga of the Arjuna-Grief
