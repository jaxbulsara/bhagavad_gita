\chapterdrop

\begin{center}
\headersanskrit{अथ सप्तमोऽध्यायः}

\headerspace
\headertransliteration{Atha Saptamo'dhyāyaḥ}

\section{Chapter 7}

\headerspace
\headersanskrit{ज्ञानविज्ञानयोगः}

\headerspace
\headertransliteration{Jñāna Vijñāna Yogah}

\headerspace
\headertranslation{The Yoga of Knowledge and Wisdom}

\headerspace
\end{center}

\begin{gitaverse}
श्रीभगवानुवाच \\
मय्यासक्तमनाः पार्थ योगं युञ्जन्मदाश्रयः । \\
असंशयं समग्रं मां यथा ज्ञास्यसि तच्छृणु ॥१॥
\end{gitaverse}

\begin{transliteration}
śrībhagavānuvāca \\
mayyāsaktamanāḥ pārtha yogaṁ yuñjanmadāśrayaḥ, \\
asaṁśayaṁ samagraṁ māṁ yathā jñāsyasi tacchṛṇu.
\end{transliteration}

The Blessed Lord said: \\
1. With the mind intent on Me, Partha, practising YOGA and taking refuge in Me,
how thou shalt, without doubt, know Me fully, that do thou hear.

\begin{gitaverse}
ज्ञानं तेऽहं सविज्ञानमिदं वक्ष्याम्यशेषतः । \\
यज्ज्ञात्वा नेह भूयोऽन्यज्ज्ञातव्यमवशिष्यते ॥२॥
\end{gitaverse}

\begin{transliteration}
jñānaṁ te'haṁ savijñānamidaṁ vakṣyāmyaśeṣataḥ, \\
yajjñātva neha bhūyo'nyajjñātavyamavaśiṣyate.
\end{transliteration}

2. I shall declare to thee in full this knowledge combined with realisation,
which being known, nothing more here remains to be known.

\begin{gitaverse}
मनुष्याणां सहस्रेषु कश्चिद्यतति सिद्धये । \\
यततामपि सिद्धानां कश्चिन्मां वेत्ति तत्त्वतः ॥३॥
\end{gitaverse}

\begin{transliteration}
manuṣyāṇāṁ sahasreṣu kaṣcidyatati siddhaye, \\
yatatāmapi siddhānāṁ kaścinmāṁ vetti tattvataḥ.
\end{transliteration}

3. Among thousands of men, one perchance strives for perfection; even among
those successful strivers, only one perchance knows Me in essence.

\begin{gitaverse}
भूमिरापोऽनलो वायुः खं मनो बुद्धिरेव च । \\
अहङ्कार इतीयं मे भिन्ना प्रकृतिरष्टधा ॥४॥
\end{gitaverse}

\begin{transliteration}
bhūmirāpo'nalo vāyuḥ khaṁ mano buddhireva ca, \\
ahaṅkāra itīyaṁ me bhinnā prakṛtiraṣṭadhā.
\end{transliteration}

4. Earth, water, fire, air, ether, mind, intellect, egoism---these are My
eightfold PRAKRITI.\@

\begin{gitaverse}
अपरेयमितस्त्वन्यां प्रकृतिं विद्धि मे पराम् । \\
जीवभूतां महाबाहो ययेदं धार्यते जगत् ॥५॥
\end{gitaverse}

\begin{transliteration}
apareyamitastvanyāṁ prakṛtiṁ viddhi me parām, \\
jīvabhūtāṁ mahābāho yayedaṁ dhāryate jagat.
\end{transliteration}

5. This is the `lower' PRAKRITI;\@ different from it, know thou, O
mighty-armed, My `Higher' PRAKRITI, the very Lifeelement, by which this world
is upheld.

\begin{gitaverse}
एतद्योनीनि भूतानि सर्वाणीत्युपधारय । \\
अहं कृत्स्नस्य जगतः प्रभवः प्रलयस्तथा ॥६॥
\end{gitaverse}

\begin{transliteration}
etadyonīni bhūtāni sarvāṇītyupadhāraya, \\
ahaṁ kṛtsnasya jagataḥ prabhavaḥ pralayastathā.
\end{transliteration}

6. Know that these (two PRAKRITIS), are the womb of all beings. So I am the
source and dissolution of the whole universe.

\begin{gitaverse}
मत्तः परतरं नान्यत्किञ्चिदस्ति धनञ्जय । \\
मयि सर्वमिदं प्रोतं सूत्रे मणिगणा इव ॥७॥
\end{gitaverse}

\begin{transliteration}
mattaḥ parataraṁ nānyatkiñcidasti dhanañjaya, \\
mayi sarvamidaṁ protaṁ sūtre maṇigaṇā iva.
\end{transliteration}

7. There is nothing whatsoever higher than Me, O Dhananjaya. All this is strung
in Me, as clusters of gems on a string.

\begin{gitaverse}
रसोऽहमप्सु कौन्तेय प्रभास्मि शशिसूर्ययोः । \\
प्रणवः सर्ववेदेषु शब्दः खे पौरुषं नृषु ॥८॥
\end{gitaverse}

\begin{transliteration}
raso'hamapsu kaunteya prabhāsmi śaśisūryayoḥ, \\
praṇavaḥ sarvavedeṣu śabadaḥ khe pauruṣaṁ nṛṣu.
\end{transliteration}

8. I am the sapidity in water, O son of Kunti, I am the light in the moon and
the sun; I am the syllable OM in all the VEDAS, sound in ether, and virility in
men;

\begin{gitaverse}
पुण्यो गन्धः पृथिव्यां च तेजश्चास्मि विभावसौ । \\
जीवनं सर्वभूतेषु तपश्चास्मि तपस्विषु ॥९॥
\end{gitaverse}

\begin{transliteration}
puṇyo gandhaḥ pṛthivyāṁ ca tejaścāsmi vibhāvasau, \\
jīvanaṁ sarvabhūteṣu tapaścasmi tapasviṣu.
\end{transliteration}

9. I am the sweet fragrance in earth and the brilliance in fire, the life in
all beings, and I am austerity in the austere.

\begin{gitaverse}
बीजं मां सर्वभूतानां विद्धि पार्थ सनातनम् । \\
बुद्धिर्बुद्धिमतामस्मि तेजस्तेजस्विनामहम् ॥१०॥
\end{gitaverse}

\begin{transliteration}
bījaṁ māṁ sarvabhūtānāṁ viddhi pārtha sanātanam, \\
buddhirbuddhimatāmasmi tejastejasvināmaham.
\end{transliteration}

10. Know Me, O Partha, as the eternal seed of all beings; I am the intelligence
of the intelligent. The splendour of the splendid (things and beings), am I.\@

\begin{gitaverse}
बलं बलवतां चाहं कामरागविवर्जितम् । \\
धर्माविरुद्धो भूतेषु कामोऽस्मि भरतर्षभ ॥११॥
\end{gitaverse}

\begin{transliteration}
balaṁ balavatāṁ cāhaṁ kāmarāgavivarjitam, \\
dharmāviruddho bhūteṣu kāmo'smi bharatarṣabha.
\end{transliteration}

11. Of the strong, I am the strength---devoid of desire and attachment, and in
(all) beings, I am the desire---unopposed to DHARMA, O best among the Bharatas.

\begin{gitaverse}
ये चैव सात्त्विका भावा राजसास्तामसाश्च ये । \\
मत्त एवेति तान्विद्धि न त्वहं तेषु ते मयि ॥१२॥
\end{gitaverse}

\begin{transliteration}
ye caiva sāttvikā bhāvā rājasāstāmasāśca ye, \\
matta eveti tānviddhi na tvahaṁ teṣu te mayi.
\end{transliteration}

12. Whatever beings (and objects) that are pure, active and inert, know them to
proceed from Me; yet, I am not in them, they are in Me.

\begin{gitaverse}
त्रिभिर्गुणमयैर्भावैरेभिः सर्वमिदं जगत् । \\
मोहितं नाभिजानाति मामेभ्यः परमव्ययम् ॥१३॥
\end{gitaverse}

\begin{transliteration}
tribhirguṇamayairbhāvairebhiḥ sarvamidaṁ jagat, \\
mohitaṁ nābhijānāti māmebhyaḥ paramavyayam.
\end{transliteration}

13. Deluded by these natures (states or things) composed of the three GUNAS (of
PRAKRITI) all the world knows Me not as Immutable and distinct from them.

\begin{gitaverse}
दैवी ह्येषा गुणमयी मम माया दुरत्यया । \\
मामेव ये प्रपद्यन्ते मायामेतां तरन्ति ते ॥१४॥
\end{gitaverse}

\begin{transliteration}
daivī hyeṣā guṇamayī mama māyā duratyayā, \\
māmeva ye prapadyante māyāmetāṁ taranti te.
\end{transliteration}

14. Verily, this divine illusion of Mine, made up of GUNAS (caused by the
qualities) is difficult to cross over; those who take refuge in Me, they alone
cross over this illusion.

\begin{gitaverse}
न मां दुष्कृतिनो मूढाः प्रपद्यन्ते नराधमाः । \\
माययापहृतज्ञाना आसुरं भावमाश्रिताः ॥१५॥
\end{gitaverse}

\begin{transliteration}
na māṁ duṣkṛtino mūḍhāḥ prapadyante narādhamāḥ, \\
māyayāpahṛtajñānā āsuraṁ bhāvamāśritāḥ.
\end{transliteration}

15. The evil-doers, the deluded, the lowest of men, do not seek Me; they, whose
discrimination has been destroyed by their own delusions, follow the ways of
the demons.

\begin{gitaverse}
चतुर्विधा भजन्ते मां जनाः सुकृतिनोऽर्जुन । \\
आर्तो जिज्ञासुरर्थार्थी ज्ञानी च भरतर्षभ ॥१६॥
\end{gitaverse}

\begin{transliteration}
caturvidhā bhajante māṁ janāḥ sukṛtino'rjuna, \\
ārto jijñsurarthārthī jñānī ca bharatarṣabha.
\end{transliteration}

16. Four kinds of virtuous men worship Me, O Arjuna, the dissatisfied, the
seeker of (systematised) knowledge, the seeker of wealth, and the wise, O best
among the Bharatas.

\begin{gitaverse}
तेषां ज्ञानी नित्ययुक्त एकभक्तिर्विशिष्यते । \\
प्रियो हि ज्ञानिनोऽत्यर्थमहं स च मम प्रियः ॥१७॥
\end{gitaverse}

\begin{transliteration}
teṣāṁ jñānī nityayukta ekabhaktirviśiṣyate, \\
priyo hi jñānino'tyarthamahaṁ sa ca mama priyaḥ.
\end{transliteration}

17. Of them the wise, ever steadfast and devoted to the One, excels; for, I am
exceedingly dear to the wise, and he is dear to Me.

\begin{gitaverse}
उदाराः सर्व एवैते ज्ञानी त्वात्मैव मे मतम् । \\
आस्थितः स हि युक्तात्मा मामेवानुत्तमां गतिम् ॥१८॥
\end{gitaverse}

\begin{transliteration}
udārāḥ sarva evaite jñānī tvātmaiva me matam, \\
āsthitaḥ sa hi yuktātmā māmevānuttamāṁ gatim.
\end{transliteration}

18. Noble indeed are all these, but the wise man, I deem, as My very Self; for,
steadfast in mind he is established in Me alone as the Supreme Goal.

\begin{gitaverse}
बहूनां जन्मनामन्ते ज्ञानवान्मां प्रपद्यते । \\
वासुदेवः सर्वमिति स महात्मा सुदुर्लभः ॥१९॥
\end{gitaverse}

\begin{transliteration}
bahūnāṁ janmanāmante jñānavānmāṁ prapadyate, \\
vāsudevaḥ sarvamiti sa mahātmā sudurlabhaḥ.
\end{transliteration}

19. At the end of many births the wise man comes to Me, realising that all this
is Vasudeva (the innermost Self); such a great soul (MAHATMA) is very hard to
find.

\begin{gitaverse}
कामैस्तैस्तैर्हृतज्ञानाः प्रपद्यन्तेऽन्यदेवताः । \\
तं तं नियममास्थाय प्रकृत्या नियताः स्वया ॥२०॥
\end{gitaverse}

\begin{transliteration}
kāmaistaistairhṛtajñānāḥ prapadyante'nyadevatāḥ, \\
taṁ taṁ niyamamāsthāya prakṛtyā niyatāḥ svayā.
\end{transliteration}

20. Those whose wisdom has been looted away by this or that desire go to other
gods, following this or that rite, led by their own nature.

\begin{gitaverse}
यो यो यां यां तनुं भक्तः श्रद्धयार्चितुमिच्छति । \\
तस्य तस्याचलां श्रद्धां तामेव विदधाम्यहम् ॥२१॥
\end{gitaverse}

\begin{transliteration}
yo yo yāṁ yāṁ tanuṁ bhaktaḥ śraddhayārcitumicchati, \\
tasya tasyācalāṁ śraddhāṁ tāmeva vidadhāmyaham.
\end{transliteration}

21. Whatsoever form any devotee desires to worship with faith---that (same)
faith of his I make (firm and) unflinching.

\begin{gitaverse}
स तया श्रद्धया युक्तस्तस्याराधनमीहते । \\
लभते च ततः कामान्मयैव विहितान्हि तान् ॥२२॥
\end{gitaverse}

\begin{transliteration}
sa tayā śraddhayā yuktastasyārādhanamīhate, \\
labhate ca tataḥ kāmānmayaiva vihitānhi tān.
\end{transliteration}

22. Endowed with that faith, he engages in the worship of that `DEVATA' and
from it he obtains his desire-fulfilments; all these being ordained, indeed, by
Me (alone).

\begin{gitaverse}
अन्तवत्तु फलं तेषां तद्भवत्यल्पमेधसाम् । \\
देवान्देवयजो यान्ति मद्भक्ता यान्ति मामपि ॥२३॥
\end{gitaverse}

\begin{transliteration}
antavattu phalaṁ teṣāṁ tadbhavatyalpamedhasām, \\
devāndevayajo yānti madbhaktā yānti māmapi.
\end{transliteration}

23. Verily the `fruit' that accrues to those men of little intelligence is
finite. The worshippers of the DEVAS go to the DEVAS but My devotees come to
Me.

\begin{gitaverse}
अव्यक्तं व्यक्तिमापन्नं मन्यन्ते मामबुद्धयः । \\
परं भावमजानन्तो ममाव्ययमनुत्तमम् ॥२४॥
\end{gitaverse}

\begin{transliteration}
avyaktaṁ vyaktimāpannaṁ manyante māmabuddhayaḥ, \\
paraṁ bhāvamajānanto mamāvyayamanuttamam.
\end{transliteration}

24. The foolish think of Me, the Unmanifest, as having come to manifestation,
not knowing My higher, immutable and peerless nature.

\begin{gitaverse}
नाहं प्रकाशः सर्वस्य योगमायासमावृतः । \\
मूढोऽयं नाभिजानाति लोको मामजमव्ययम् ॥२५॥
\end{gitaverse}

\begin{transliteration}
nāhaṁ prakāśaḥ sarvasya yogamāyāsamāvṛtaḥ, \\
mūḍho'yaṁ nābhijānāti loko māmajamavyayam.
\end{transliteration}

25. I am not manifest to all (in My Real Nature) veiled by Divine---`MAYA'. This
deluded world knows not Me, the Unborn, the Imperishable.

\begin{gitaverse}
वेदाहं समतीतानि वर्तमानानि चार्जुन । \\
भविष्याणि च भूतानि मां तु वेद न कश्चन ॥२६॥
\end{gitaverse}

\begin{transliteration}
vedāhaṁ samatītāni vartamānāni cārjuna, \\
bhaviṣyāṇi ca bhūtāni māṁ tu veda na kaścana.
\end{transliteration}

26. I know, O Arjuna, the beings of the past, and present and the future, but
no one knows Me.

\begin{gitaverse}
इच्छाद्वेषसमुत्थेन द्वन्द्वमोहेन भारत । \\
सर्वभूतानि सम्मोहं सर्गे यान्ति परन्तप ॥२७॥
\end{gitaverse}

\begin{transliteration}
icchādveṣasamutthena dvandvamohena bhārata, \\
sarvabhūtāni sammohaṁ sarge yānti parantapa.
\end{transliteration}

27. By the delusion of the pairs-of-opposites arising from desire and aversion,
O Bharata, all beings are subject to delusion at birth, O Parantapa (scorcher
of foes).

\begin{gitaverse}
येषां त्वन्तगतं पापं जनानां पुण्यकर्मणाम् । \\
ते द्वन्द्वमोहनिर्मुक्ता भजन्ते मां दृढव्रताः ॥२८॥
\end{gitaverse}

\begin{transliteration}
yeṣaṁ tvantagataṁ pāpaṁ janānāṁ puṇyakarmaṇām, \\
te dvandvamohanirmuktā bhajante māṁ dṛḍhavratāḥ.
\end{transliteration}

28. But those men of virtuous deeds whose sins have come to an end, who are
freed from the delusion of the pairs-of-opposites and steadfast in vows,
worship Me.

\begin{gitaverse}
जरामरणमोक्षाय मामाश्रित्य यतन्ति ये । \\
ते ब्रह्म तद्विदुः कृत्स्नमध्यात्मं कर्म चाखिलम् ॥२९॥
\end{gitaverse}

\begin{transliteration}
jarāmaraṇamokṣāya māmāśritya yatanti ye, \\
te brahma tadviduḥ kṛtsnamadhyātmaṁ karma cākhilam.
\end{transliteration}

29. Those who strive for liberation from old age and death, taking refuge in
Me---They realise in full that BRAHMAN, the whole knowledge of the Self and all
action.

\begin{gitaverse}
साधिभूताधिदैवं मां साधियज्ञं च ये विदुः । \\
प्रयाणकालेऽपि च मां ते विदुर्युक्तचेतसः ॥३०॥
\end{gitaverse}

\begin{transliteration}
sādhibhūtādhidaivaṁ māṁ sādhiyajñaṁ ca ye viduḥ, \\
prayāṇakāle'pi ca māṁ te viduryuktacetasaḥ.
\end{transliteration}

30. Those who know Me with the ADHIBHUTA (pertaining to elements; the
world-of-objects), ADHIDAIVA (pertaining to the gods; the sense-organs) and the
ADHIYAJNA (pertaining to the sacrifice; all perceptions), even at the time of
death, steadfast in mind, know Me.

\begin{gitaverse}
ॐ तत्सदिति श्रीमद् भगवद् गीतासूपनिषत्सु ब्रह्मविद्यायां \\
योगशास्त्रे श्रीकृष्णार्जुनसंवादे ज्ञानविज्ञानयोगो नाम \\
सप्तमोऽध्यायः
\end{gitaverse}

\begin{transliteration}
oṁ tatsaditi śrīmad bhagavad gītāsūpaniṣatsu brahmavidyāyāṁ \\
yogaśāstre śrī kṛṣṇārjuna saṁvāde jñānavijñānayogo nāma \\
saptamo'dhyāyaḥ.
\end{transliteration}

Thus, in the UPANISHADS of the glorious Bhagawad Geeta, in the Science of the
Eternal, in the Scripture of YOGA, in the dialogue between Shri Krishna and
Arjuna, the seventh discourse ends entitled: The Yoga of Knowledge and Wisdom
