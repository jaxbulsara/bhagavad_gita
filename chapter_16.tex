\chapterdrop

\begin{center}
\headersanskrit{अथ षोडशोऽध्यायः}

\headerspace
\headertransliteration{Atha Ṣodaśo'dhyāyaḥ}

\section{Chapter 16}

\headerspace
\headersanskrit{दैवासुरसम्पद्विभागयोगः}

\headerspace
\headertransliteration{Daivāsura Sampadvibhāga Yogah}

\headerspace
\headertranslation{The Divine and the Devilish Estates}

\headerspace
\end{center}

\begin{gitaverse}
श्रीभगवानुवाच \\
अभयं सत्त्वसंशुद्धिर्ज्ञानयोगव्यवस्थितिः । \\
दानं दमश्च यज्ञश्च स्वाध्यायस्तप आर्जवम् ॥१॥
\end{gitaverse}

\begin{transliteration}
śrībhagavānuvāca \\
abhayaṁ satva-saṁśuddhir-jñāna-yoga-vyavasthitiḥ, \\
dānaṁ damaśca yajñaśca svādhyāyastapa ārjavam.
\end{transliteration}

The Blessed Lord said: \\
1. Fearlessness, purity of heart, steadfastness in the YOGA-of-Knowledge,
alms-giving, control of the senses, sacrifice, study of the SHASTRAS, and
straightforwardness\ldots

\begin{gitaverse}
अहिंसा सत्यमक्रोधस्त्यागः शान्तिरपैशुनम् । \\
दया भूतेष्वलोलुप्त्वं मार्दवं ह्रीरचापलम् ॥२॥
\end{gitaverse}

\begin{transliteration}
ahiṁsā satyam-akrodhas-tyāgaḥ śāntir-apaiśunam, \\
dayā bhūteṣvaloluptvaṁ mārdavaṁ hrīr-acāpalam.
\end{transliteration}

2. Harmlessness, truth, absence of anger, renunciation, peacefulness, absence
of crookedness, compassion to beings, noncovetousness, gentleness, modesty,
absence of fickleness\ldots

\begin{gitaverse}
तेजः क्षमा धृतिः शौचमद्रोहो नातिमानिता । \\
भवन्ति सम्पदं दैवीमभिजातस्य भारत ॥३॥
\end{gitaverse}

\begin{transliteration}
tejaḥ kṣamā dhṛtiḥ śaucam-adroho nātimānitā, \\
bhavanti sampadaṁ daivīm-abhijātasya bhārata.
\end{transliteration}

3. Vigour, forgiveness, fortitude, purity, absence of hatred, absence of
pride---these belong to the one born for the Divine Estate, O Bharata.

\begin{gitaverse}
दम्भो दर्पोऽभिमानश्च क्रोधः पारुष्यमेव च । \\
अज्ञानं चाभिजातस्य पार्थ सम्पदमासुरीम् ॥४॥
\end{gitaverse}

\begin{transliteration}
dambho darpo'bhimānaśca krodhaḥ pāruṣyam-eva ca, \\
ajnānaṁ cābhijātasya pārtha sampadamāsurīm.
\end{transliteration}

4. Hypocrisy, arrogance and self-conceit, anger and also harshness and
ignorance, belong to one who is born, O Partha, for a demoniac-Estate.

\begin{gitaverse}
दैवी सम्पद्विमोक्षाय निबन्धायासुरी मता । \\
मा शुचः सम्पदं दैवीमभिजातोऽसि पाण्डव ॥५॥
\end{gitaverse}

\begin{transliteration}
daivī sampad-vimokṣāya nibandhāyāsurī matā, \\
mā śucaḥ sampadaṁ daivīm-abhijāto'si pāṇḍava.
\end{transliteration}

5. The divine nature is deemed for liberation, the demoniacal for bondage;
grieve not, O Pandava, you are born with divine qualities.

\begin{gitaverse}
द्वौ भूतसर्गौ लोकेऽस्मिन्दैव आसुर एव च । \\
दैवो विस्तरशः प्रोक्त आसुरं पार्थ मे श्रृणु ॥६॥
\end{gitaverse}

\begin{transliteration}
dvau bhūtasargau loke'smin-daiva āsura eva ca, \\
daivo vistaraśaḥ prokta āsuraṁ pārtha me śṛṇu.
\end{transliteration}

6. There are two types of beings, in this world, the `divine' and the
`demoniacal;' the divine have been described at length; hear from Me, O Partha,
of the demoniacal.

\begin{gitaverse}
प्रवृत्तिं च निवृत्तिं च जना न विदुरासुराः । \\
न शौचं नापि चाचारो न सत्यं तेषु विद्यते ॥७॥
\end{gitaverse}

\begin{transliteration}
pravṛttiṁ ca nivṛttiṁ ca janā na vidur-āsurāḥ, \\
na śaucaṁ nāpi cācāro na satyaṁ teṣu vidyate.
\end{transliteration}

7. The demoniac know not what to do and what to refrain from; neither purity,
nor right conduct, nor truth is found in them.

\begin{gitaverse}
असत्यमप्रतिष्ठं ते जगदाहुरनीश्वरम् । \\
अपरस्परसम्भूतं किमन्यत्कामहैतुकम् ॥८॥
\end{gitaverse}

\begin{transliteration}
asatyam-apratiṣṭhaṁ te jagad-āhur-anīśvaram, \\
aparasparasambhūtaṁ kim-anyat-kāmahaitukam.
\end{transliteration}

8. They say, ``the universe is without truth, without (moral) basis, without a
God; not brought about by any regular causal sequence, with lust for its cause;
what else?''

\begin{gitaverse}
एतां दृष्टिमवष्टभ्य नष्टात्मानोऽल्पबुद्धयः । \\
प्रभवन्त्युग्रकर्माणः क्षयाय जगतोऽहिताः ॥९॥
\end{gitaverse}

\begin{transliteration}
etāṁ dṛṣṭim-avaṣṭabhya naṣṭātmāno'lpabuddhayaḥ, \\
prabhavantyugrakarmāṇaḥ kṣayāya jagato'hitāḥ.
\end{transliteration}

9. Holding this view, these ruined souls of small intellect and fierce deeds,
come forth as the enemies of the world, for its destruction.

\begin{gitaverse}
काममाश्रित्य दुष्पूरं दम्भमानमदान्विताः । \\
मोहाद्गृहीत्वासद्ग्राहान्प्रवर्तन्तेऽशुचिव्रताः ॥१०॥
\end{gitaverse}

\begin{transliteration}
kāmam-āśritya duṣpūraṁ dambhamānamadānvitāḥ, \\
mohād-gṛhītvāsadgrāhān-pravartante'śucivratāḥ.
\end{transliteration}

10. Filled with insatiable desires, full of hypocrisy, pride and arrogance,
holding evil ideas through delusion, they work with impure resolves.

\begin{gitaverse}
चिन्तामपरिमेयां च प्रलयान्तामुपाश्रिताः । \\
कामोपभोगपरमा एतावदिति निश्चिताः ॥११॥
\end{gitaverse}

\begin{transliteration}
cintām-aparimeyāṁ ca pralayāntām-upāśritāḥ, \\
kāmopabhogaparamā etāvad-iti niścitāḥ.
\end{transliteration}

11. Giving themselves over to immeasurable cares ending only with death,
regarding gratification of lust as their highest aim, and feeling sure that,
that is all (that matters).

\begin{gitaverse}
आशापाशशतैर्बद्धाः कामक्रोधपरायणाः । \\
ईहन्ते कामभोगार्थमन्यायेनार्थसञ्चयान् ॥१२॥
\end{gitaverse}

\begin{transliteration}
āśā-pāśa-śatair-baddhāḥ kāma-krodha-parāyanāḥ, \\
īhante kāmabhogārtham-anyāyenārthasañcayān.
\end{transliteration}

12. Bound by a hundred ties of hope, given to lust and anger, they do strive to
obtain, by unlawful means, hoards of wealth for sensual enjoyments.

\begin{gitaverse}
इदमद्य मया लब्धमिमं प्राप्स्ये मनोरथम् । \\
इदमस्तीदमपि मे भविष्यति पुनर्धनम् ॥१३॥
\end{gitaverse}

\begin{transliteration}
idam-adya mayā labdham-imaṁ prāpsye manoratham, \\
idam-astīdam-api me bhaviṣyati punar-dhanam.
\end{transliteration}

13. ``This has to-day been gained by me''---``this desire I shall
obtain''---``this is mine''---and ``this wealth shall also be mine in future''.

\begin{gitaverse}
असौ मया हतः शत्रुर्हनिष्ये चापरानपि । \\
ईश्वरोऽहमहं भोगी सिद्धोऽहं बलवान्सुखी ॥१४॥
\end{gitaverse}

\begin{transliteration}
asau mayā hataḥ śatrur-haniṣye cāparān-api, \\
īśvaro'ham-ahaṁ bhogī siddho'haṁ balavān-sukhi.
\end{transliteration}

14. ``That enemy has been slain by me''---``and others also shall I
destroy''---``I am the Lord''---``I am the enjoyer''---``I am perfect, powerful
and happy''.

\begin{gitaverse}
आढ्योऽभिजनवानस्मि कोऽन्योऽस्ति सदृशो मया । \\
यक्ष्ये दास्यामि मोदिष्य इत्यज्ञानविमोहिताः ॥१५॥
\end{gitaverse}

\begin{transliteration}
āḍhyo'bhijanavān-asmi ko'nyo'sti sadṛśo mayā, \\
yakṣye dāsyāmi modiṣya ityajñānavimohitāḥ.
\end{transliteration}

15. ``I am rich and well-born''---``who else is equal to me?''---``I will give
(alms, money)''---``I will rejoice''. Thus are they, deluded by `ignorance'.

\begin{gitaverse}
अनेकचित्तविभ्रान्ता मोहजालसमावृताः । \\
प्रसक्ताः कामभोगेषु पतन्ति नरकेऽशुचौ ॥१६॥
\end{gitaverse}

\begin{transliteration}
aneka-citta-vibhrāntā moha-jāla-samāvṛtāḥ, \\
prasaktāḥ kāma-bhogeṣu patanti narake'śucau.
\end{transliteration}

16. Bewildered by many a fancy, entangled in the snare of delusion, addicted to
the gratification of lust, they fall into a foul hell.

\begin{gitaverse}
आत्मसम्भाविताः स्तब्धा धनमानमदान्विताः । \\
यजन्ते नामयज्ञैस्ते दम्भेनाविधिपूर्वकम् ॥१७॥
\end{gitaverse}

\begin{transliteration}
ātma-sambhāvitāḥ stabdhā dhanamāna-madānvitāḥ, \\
yajante nāmayajñaiste dambhenāvidhipūrvakam.
\end{transliteration}

17. Self-conceited, stubborn, filled with pride and drunk with wealth, they
perform sacrifices in name (only) out of ostentation, contrary to scriptural
ordinance.

\begin{gitaverse}
अहङ्कारं बलं दर्पं कामं क्रोधं च संश्रिताः । \\
मामात्मपरदेहेषु प्रद्विषन्तोऽभ्यसूयकाः ॥१८॥
\end{gitaverse}

\begin{transliteration}
ahaṅkāraṁ balaṁ darpaṁ kāmaṁ krodhaṁ ca samśritāḥ, \\
mām-ātma-para-deheṣu pradviṣanto'bhyasūyakāh.
\end{transliteration}

18. Given to egoism, power, haughtiness, lust and anger, these malicious people
hate Me in their own bodies, and in those of others.

\begin{gitaverse}
तानहं द्विषतः क्रूरान्संसारेषु नराधमान् । \\
क्षिपाम्यजस्रमशुभानासुरीष्वेव योनिषु ॥१९॥
\end{gitaverse}

\begin{transliteration}
tān-ahaṁ dviṣataḥ krūrān-saṁsāreṣu narādhamān, \\
kṣipāmyajasram-aśubhānāsurīṣveva yoniṣu.
\end{transliteration}

19. These cruel haters, worst among men in the world, I hurl these evil-doers
for ever into the wombs of the demons only.

\begin{gitaverse}
आसुरीं योनिमापन्ना मूढा जन्मनि जन्मनि । \\
मामप्राप्यैव कौन्तेय ततो यान्त्यधमां गतिम् ॥२०॥
\end{gitaverse}

\begin{transliteration}
āsurīṁ yoni-māpannā mūḍhā janmani janmani, \\
mām-aprāyaiva kaunteya tato yāntyadhamāṁ gatim.
\end{transliteration}

20. Entering into demoniacal wombs, and deluded, not attaining to Me, birth
after birth, they thus fall, O Kaunteya, into a condition still lower than
that.

\begin{gitaverse}
त्रिविधं नरकस्येदं द्वारं नाशनमात्मनः । \\
कामः क्रोधस्तथा लोभस्तस्मादेतत्त्रयं त्यजेत् ॥२१॥
\end{gitaverse}

\begin{transliteration}
trividhaṁ narakasyedaṁ dvāraṁ nāśanam-ātmanaḥ, \\
kāmaḥ krodhas-tathā lobhas-tasmād-etat-trayaṁ tyajet.
\end{transliteration}

21. These three are the gates of hell, destructive of the Self---lust, anger
and greed; therefore, one should abandon these three.

\begin{gitaverse}
एतैर्विमुक्तः कौन्तेय तमोद्वारैस्त्रिभिर्नरः । \\
आचरत्यात्मनः श्रेयस्ततो याति परां गतिम् ॥२२॥
\end{gitaverse}

\begin{transliteration}
etair-vimuktaḥ kaunteya tamo-dvārais-tribhir-naraḥ, \\
ācaratyātmanaḥ śreyas-tato yāti parāṁ gatim.
\end{transliteration}

22. A man who is liberated from these three gates to darkness, O Kaunteya,
practises what is good for him and thus goes to the Supreme Goal.

\begin{gitaverse}
यः शास्त्रविधिमुत्सृज्य वर्तते कामकारतः । \\
न स सिद्धिमवाप्नोति न सुखं न परां गतिम् ॥२३॥
\end{gitaverse}

\begin{transliteration}
yaḥ śāstra-vidhim-utsṛjya vartate kāmakārataḥ, \\
na sa siddhim-avāpnoti na sukhaṁ na parāṁ gatim.
\end{transliteration}

23. He who, having cast aside the ordinance of the scriptures, acts under the
impulse of desire, attains neither perfection, nor happiness, nor the Supreme
Goal.

\begin{gitaverse}
तस्माच्छास्त्रं प्रमाणं ते कार्याकार्यव्यवस्थितौ । \\
ज्ञात्वा शास्त्रविधानोक्तं कर्म कर्तुमिहार्हसि ॥२४॥
\end{gitaverse}

\begin{transliteration}
tasmācchāstraṁ pramāṇaṁ te kāryākārya-vyavasthitau, \\
jñātvā śāstra-vidhānoktaṁ karma kartum-ihārhasi.
\end{transliteration}

24. Therefore, let the Scriptures be your authority, in determining what ought
to be done and what ought not to be done. Having known what is said in the
commandments of the Scripture, you should act here (in this world).

\begin{gitaverse}
ॐ तत्सदिति श्रीमद् भगवद् गीतासूपनिषत्सु ब्रह्मविद्यायां \\
योगशास्त्रे श्रीकृष्णार्जुनसंवादे दैवासुर सम्पद्विभाग योगो नाम \\
षोडशोऽध्यायः
\end{gitaverse}

\begin{transliteration}
oṁ tatsaditi śrīmad bhagavad gītāsūpaniṣatsu brahmavidyāyāṁ \\
yogaśāstre śrīkṛṣṇārjunasaṁvāde daivāsura sampadvibhāga yogo nāma \\
ṣodaśo'dhyāyaḥ
\end{transliteration}

Thus, in the UPANISHADS of the glorious Bhagawad Geeta, in the Science of the
Eternal, in the scripture of YOGA, in the dialogue between Sri Krishna and
Arjuna, the sixteenth discourse ends entitled:* The Divine and the Devilish
Estates
